\chapter{Further Results on Finite Elements}
\label{auxlabel202}
\section*{Introducci\'on al cap\'itulo}

Probamos aqu\'i estimaciones anis\'otropas de error de interpolaci\'on local
para el operador $\bcurl$--conforme de orden arbitrario sobre elementos
prism\'aticos. 
Con t\'ecnicas similares a las del Cap\'itulo~\ref{auxlabel421}
podemos realizar el mismo 
an\'alisis para estos elementos $H(\bcurl)$--conformes en prismas
que el que fue hecho para los elementos $H(\Div)$--conformes.

Con respecto a las aplicaciones, 
teniendo en cuenta nuestros resultados para el espacio $H(\Div)$ 
del Cap\'itulo~\cite{auxlabel421} m\'as los que presentaremos
ahora para el espacio $H(\bcurl)$, se podr\'ia intentar realizar
un an\'alisis para las ecuaciones de Maxwell arm\'onicas en tiempo
como el que se encuentra en~\cite{buffaCostabelDauge}, pero
ahora usando mallas que contengan prismas alargados 
--en~\cite{buffaCostabelDauge} los autores consideran
mallas exclusivamente de tetraedros--, y tambi\'en 
para el problema arm\'onico en tiempo en una cavidad,
para el problema del resonador de cavidad, para el problemma
de la dispersi\'on de un objeto acotado y para el problema
de la dispersi\'on de un objeto enterrado, todos los cuales
son algunos de los principales problemas de contorno
para las ecuaciones de Maxwell
(ver la Secci\'on 1.4 de~\cite{monk}), 
y con la misma consideraci\'on
reci\'en hecha acerca de las mallas.

En~\cite{MR1860445} se encuentran estimaciones anis\'otropas
para el caso de grado m\'as bajo de edge--elements en prismas
(y para tetraedros resultantes de la subdivisi\'on de cada prisma),
aplicados al mallado de dominios cil\'indricos.
Nuevamente, algunas de las estimaciones halladas 
en este cap\'itulo extienden a las anteriores a grado arbitrario,
y adem\'as podr\'iamos intentar aplicarlas a dominios poliedrales 
generales, como hicimos con nuestras estimaciones $H(\Div)$.

Recordar la Notaci\'on~\ref{auxlabel200} para los espacios polinomiales. 
Tambi\'en nos sujetaremos a la notaci\'on e indexaci\'on de la  
Tabla~\ref{prismNotationTableEdges}.

Desde la Secci\'on~\ref{label999} en adelante definimos
familias de elementos finitos 
$\bcurl$--conformes and $\Div$--conformes en pir\'amides. Estos elementos
se encuentran en~\cite{gh99}. En ese art\'iculo los autores
realizan una construcci\'on de formas diferenciales discretas
mediante la resoluci\'on de un problema de interpolaci\'on local
en cierta pir\'amide de referencia. El elemento finito en una
pir\'amide arbitraria de una malla es obtenida haciendo \emph{push--forward}
los campos vectoriales \emph{proxies} de dichas formas discretas.
A continuaci\'on probamos estimaciones anis\'otropas de error de interpolaci\'on
local para los correspondientes operadores de orden m\'as bajo.

\section*{Introduction to the chapter}

We prove anisotropic local interpolation error estimates for the
$\bcurl$--conforming operator of arbitrary order over a prismatic element.
With techniques similar to those of Chapter~\ref{auxlabel421} we can
perform the same analysis for these $H(\bcurl)$--conforming elements
as the one performed for the $H(\Div)$ elements.

With respect to the applications,
taking into account our results 
from Chapter~\ref{auxlabel421}
for the $H(\Div)$ space together with the ones we are presenting now
for $H(\bcurl)$, we could try an analysis for the
time--harmonic Maxwell's equations analogue to the one found
in~\cite{buffaCostabelDauge}, but now using meshes including
elongated prisms --in~\cite{buffaCostabelDauge} the authors consider
meshes of tetrahedra--, and also for the time--harmonic cavity problem,
for the cavity resonator problem, for the problem of the scattering from a 
bounded object, and for the problem of the scattering from a
buried object, all of which are some of the principal boundary value
problems for the Maxwell's equations (see Section 1.4 of~\cite{monk}),
and with the same consideration as above concerning the meshes.

In~\cite{MR1860445} there are anisotropic estimates for 
the least order case of the edge--elements on prisms
(and for the tetrahedra resulting from dividing each prism into three),
applied to cylindrical domains.
Again, some of the estimates found in this chapter extend those estimates
to abitrary degree, and moreover we could try to apply them
to general polihedral domains, as we did we our $H(\Div)$ estimates.

Please recall 
the notation for the polynomial spaces given in Notation~\ref{auxlabel200}. We 
will alse stick to the notation and indices of 
Table~\ref{prismNotationTableEdges}.

Later, from Section~\ref{label999} on, we define families of
$\bcurl$--conforming and $\Div$--conforming finite elements on pyramids.
The finite elements defined here 
are the ones found in~\cite{gh99}. There the authors
perform a
construction of discrete differential
forms by solving a local interpolation problem on a reference pyramid. The
finite element in an arbitrary mesh pyramid is obtained by pushing forward
the vector proxies of those discrete forms.
Then we prove local anisotropic interpolation error estimates for the
corresponding interpolation operators of lowest order.

\section{An\-iso\-tropic Stability Estimates for $H(\bcurl)$ Conforming Finite
Elements on Prisms}
\label{stab_edge_prism}
We are writing the following three Lemmas which state some special
behavior of the interpolation operator, mostly related to the preservation of
null components of the fields, and whose proofs consist in a smart use of the
degrees of freedom and the very definition of the operator. These Lemmas, 
although just with technical purpose, exhibit
nice properties of the interpolators.

In the present subsection $\hat\bu$ is an element
in $W^{1,p}(\hat{E})$ for $p>2$ which is a space whose elements have well
defined tangential traces on each edge of the prism $\hat{E}$.
Another possibility, as stated in Lemma $5.38$ in the page $134$ of~\cite{monk},
is to assume there are 
a positive $\delta$ and a $p>2$ such that 
$\hat\bu$ belongs to $H^{1/2+\delta}(\hat{E})^3$ and
${\bf curl}\,\bu$ belongs to $L^p(\hat{E})^3$.
For the whole section, $\hat\bw_k$ will be the $k$--th order edge 
interpolation operator on the reference
Prism determined by the element of
\emph{Definition}~\ref{edgeelement}.

\begin{lemma}\label{lema_PIu3_k_cualquiera} 
$(\wku)_3$ is linearly and univocally 
determined by $\hat{u}_3$.
\end{lemma}
\begin{proof} If we pay attention to the directions of the unit
tangents and normals to the edges and faces, respectively, of $\hat E$,
we realize that
the degrees of freedom which involve $(\wku)_3$ give rise only to the 
following linear equations
\begin{IEEEeqnarray}{rCrc}
\varphi_{\hat{\be}_i,p}\,(\wku) & = & \varphi_{\hat{\be}_i,p}\,(\hat{\bu}) &\quad\mbox{as in~(\ref{momentos1hcurl}) for $i$ = 3, 6, 7,}\\
\varphi_{f_1,\bq}\,(\wku) & = & \varphi_{f_1,\boldsymbol{q}}\,(\hat{\bu})
  &\quad\mbox{as in~(\ref{momentos3hcurl})}\\
\varphi_{f_2,\bq}\,(\wku) & = & \varphi_{f_2,\boldsymbol{q}}\,(\hat{\bu})
  &\quad\mbox{as in~(\ref{momentos4hcurl})}  \\
\varphi_{f_5,\bq}\,(\wku) & = & \varphi_{f_5,\boldsymbol{q}}\,(\hat{\bu})
  &\quad\mbox{as in~(\ref{momentos5hcurl})}  \\
\varphi_{\boldsymbol{r}}\,(\wku) & = & \varphi_{\boldsymbol{r}}\,(\hat{\bu})
  &\quad\mbox{as in~(\ref{momentos6hcurl})}.
\end{IEEEeqnarray}
These are 
$3k$+$3k(k-1)$+$k(k-1)(k-2)/2 = k(k+1)(k+2)/2$ equations,
just the dimension of $P_k(\hat{T})\otimes P_{k-1}(\hat{I})$, 
which is the space $(\wku)_3$ belongs to by definition.
%$\frac{k(k+1)(k+2)}{2}$. 
%libertad en los que no des\-a\-pa\-re\-ce $(\wku)_3$ son \'unicamente:
Now set all those equations equal to zero (that is, pick $u_3 = 0$) and see that
the unique solution is $(\wku)_3 = 0$.
A little more explicitly, we have:
\begin{IEEEeqnarray}{lCll}
  \label{aristas} \int_{\hat\be_i} (\wku)_3\,\hat q \, d\alpha 
  & = & 0 &\qquad \mbox{for $i$ = 3, 6 and 7, }q\in P_{k-1}(\hat\be_i)\\[5pt]
  \label{caras} \iint_{\hat f_j} (\wku)_3\,\hat q \,d\hat S
  & = & 0 &\qquad \mbox{for $j$ = 1, 2, and 5, } \hat q\in Q_{k-2,k-1}(\hat f_j)\\[5pt]
  \label{enK} \int_{\hat{E}} (\wku)_3\,\hat q_3 \, d\bx 
  & = & 0 &\qquad \mbox{for }\hat q_3\in P_{k-3,k-1}.
\end{IEEEeqnarray}
Start considering the face $\hat f_2$.
The restriction of $(\wku)_3$ to $\be_3$
is itself 
an element in $P_{k-1}({\be_3})$, 
and the same holds for $\be_6$,  
so equations~(\ref{aristas}), for $i = 3$, $6$, say that $(\wku)_3$
is identically null on those edges by which
the restriction 
$(\wku)_3|_{f_2}$, which is an element of $P_k(\hat x_1)\otimes P_{k-1}(\hat x_3)$
may be se factorized
as
$(\wku)_3|_{f_2}(\hat x_1,0,\hat x_3) = \hat x_1\,(1-\hat x_1)\,w_0(\hat x_1,\hat x_3)$,
with $w_0$ equal to some polynomial in $P_{k-2}(\hat x_1) \otimes P_{k-1}(\hat x_3)$.
Now choose $\hat q = w_0$ in the degrees of freedom~(\ref{caras}) for the face
$\hat f_2$ and it holds
\begin{IEEEeqnarray}{lClc}
	\iint_{\hat f_2} \hat{x}_1(1-\hat{x}_1)w_0(\hat{x}_1,\hat{x}_3)^2\,d\hat S & = & 0.
\end{IEEEeqnarray}
But $\hat{x}_1\,(1-\hat{x}_1)$ is almost everywhere positive over the closure
of $\hat f_2$, so 
$(\wku)_3|_{\hat f_2}$ vanishes identically.
By a completely symmetric observation we can prove 
$(\wku)_3|_{\hat f_1} \equiv 0$.\\
One more time, if we use~(\ref{aristas}) for $i =$ 6, 7,
$\hat x_1\,(1-\hat x_1)$ divides the restriction $(\wku)_3|_{\hat f_5}$, so that
there is a 
$w_1 \in P_{k-2}(\hat x_1)\otimes P_{k-1}(\hat x_3)$ for which 
$(\wku)_3|_{\hat f_5}(\hat x_1, \hat x_2, \hat x_3) = \hat{x}_1\,\displaystyle{(1-\hat{x}_1)}\,w_1(\hat{x}_1,\hat{x}_3)$.
Now equality~(\ref{caras}) for $j = 5$ implies
$(\wku)_3|_{f_5} \equiv 0$.\\
Next, since $(\wku)_3$ vanishes when restricted $\hat f_1$, $\hat f_2$ and $\hat f_5$
we get to factorize it on $\hat E$ as 
\begin{IEEEeqnarray*}{rCl}
	(\wku)_3(\hat x_1, \hat x_2, \hat x_3) 	& = 	& \hat{x}_1\,\hat{x}_2\,(1-\hat{x}_1-\hat{x}_2)\,w_3(\hat{x}_1,\hat{x}_2,\hat{x}_3),\\
									w_3		& \in 	& P_{k-3}(\hat x_1)\otimes P_{k-1}(\hat x_3).
\end{IEEEeqnarray*}
And now we evaluate the degrees of freedom~(\ref{enK}) choosing
$\hat q_3 \equiv w_3$ and conclude immediately
$(\wku)_3 \equiv 0$ on $\hat E$. 
\end{proof}
\label{auxlabel500}
\begin{lemma}\label{auxlabel501}
\begin{itemize}
	\item []
	\item [(a)] If $\hat\bu(\hat x_1,\hat x_2,\hat x_3) = (0, \hat u_2(\hat x_2,\hat x_3), 0)'$,
	then 
  \[
  \wku\xyz = (0, \hat\xi_2(\hat x_2,\hat x_3) ,0)'
  \]
  for some 
	$\hat\xi_2 \in P_{k-1}(\hat{x}_2) \otimes P_k(\hat{x}_3)$.
	\item [(b)]\label{piu1_k_in_N} If $\hat\bu(\hat x_1,\hat x_2,\hat x_3) = (\hat u_1(\hat x_1,\hat x_3), 0, 0)'$
	then
  \[
  \wku\xyz = (\hat\xi_1(\hat x_1,\hat x_3), 0 ,0)'
  \]
  for some
    $\hat\xi_1\in P_{k-1}(\hat{x}_1) \otimes P_k(\hat{x}_3)$.
\end{itemize}
\end{lemma}
\begin{proof} We will prove the first inequality, as the second follows
with the same ideas. In Subsection~\ref{sub:defEdgeElement} we found 
expression~(\ref{elemento_P_k}) which states
\begin{IEEEeqnarray*}{rCl}
  \wku\xyz  & = & (p_1\xyz, p_2\xyz, p_3\xyz)^t\\[4pt]
  			    & = & \begin{pmatrix}
  					        \xi_1\xyz + \hat{x}_2\,h\xyz\\
                    \yesnumber\label{expr_wku}\xi_2\xyz - \hat{x}_1\,h\xyz \\
  					        \xi_3\xyz
  				        \end{pmatrix}
\end{IEEEeqnarray*}
for
$\xi_1$ and $\xi_2$ in $P_{k-1}(\hat{f}_3) \otimes P_k(\hat{x}_3)$,
$\xi_3$ in $P_{k}(\hat{f}_3) \otimes P_{k-1}(\hat{x}_3)$,
and $h$ in $\tilde{P}_{k-1}(\hat{f}_3) \otimes P_k(\hat{x}_3)$.
Thanks to Lemma~\ref{lema_PIu3_k_cualquiera} we already know that $\xi_3 \equiv 0$,
so we are going to show
that $h \equiv 0$, $\xi_1 \equiv 0$ and that $\xi_2$ does not depend 
on $\hat{x}_1$. First, if $\hat{f}$ is either $\hat{f}_3$ o $\hat{f}_4$, then
with a direct calculation we see that $({\bf curl}\,\wku)_3 |_{\hat f}$ belongs to
$P_{k-1}(\hat{f})$. By the commutative diagram property expressed
in~(\ref{curl_commutativity}), the definition of the degrees of freedom~(\ref{momentos1hdiv})
and the interpolation operator $\br_{\hat E}$
in Definition~\ref{defi_face_element}, it holds that, if $\hat{f}$ is 
either $\hat{f}_3$ or $\hat{f}_4$, then for every $q \in P_{k-1}(\hat{f})$,
\begin{IEEEeqnarray*}{rCCCl}
  \hat\rho_{\hat{f},q}\,({\bf curl\,}\wku)
  & = & \hat\rho_{\hat{f},q} (\br_{\hat E}\,{\bf curl\,}\hat\bu) &&\\[5pt]
  & = & \iint_{\hat f} ({\bf curl\,}\hat \bu)_3\,q \,d\hat S & = & 0.	
\end{IEEEeqnarray*}
As is for time being expected, the choice $q = ({\bf curl}\,\wku)_3 |_{\hat{f}}$
yields 
\[
  ({\bf curl}\,\wku)_3 |_{\hat{f}} \equiv 0
\]
so we may write again $(\curl \wku)_3 = \hat{x}_3\,(\hat{x}_3-1)\,\hat\psi$ for
a $\hat\psi\in P_{k-1}(\hat{f}) \otimes P_{k-2}(\hat{x}_3)$.
We choose now $q=\hat\psi$ in the degrees of freedom~(\ref{momentos4hdiv}).
By the commutative diagram property
and the definition of $\br_{\hat E}$, we have 
\begin{IEEEeqnarray*}{rCCCl}
	\int_{\hat{E}} \hat{x}_3\,(\hat{x}_3-1)\,\hat\psi^2\,d\hat{\bx}
  & = &\int_{\hat{E}} (\curl\wku)_3\,\hat\psi\,d\hat{\bx}&&\\
  & = &\int_{\hat{E}} (\br_{\hat E}\curl\hat{\bu})_3\,\hat{\psi}\,d\hat{\bx}&&\\
  & = &\int_{\hat{E}} (\curl\hat{\bu})_3\,\hat{\psi}\,d\hat{\bx} & = & 0\\
\end{IEEEeqnarray*}
and it follows that
\begin{IEEEeqnarray}{rCl}
	\label{rot_3_es_0} (\curl\wku)_3 &\equiv& 0.
\end{IEEEeqnarray}
Now if we explore $({\bf curl}\,\wku)_3$ taking derivatives in  
expression~(\ref{expr_wku}) we get 
\begin{IEEEeqnarray*}{rCl}
  (\curl\wku)_3 & = & 
  \dfrac{\partial}{\partial \hat x_1}(\wku)_2 - \dfrac{\partial}{\partial \hat x_2}(\wku)_1\\[5pt]
  \label{expre_h} \yesnumber & = & -(2\,h + \hat{x}_2\,\dfrac{\partial h}{\partial \hat{x}_2} + 
	\hat{x}_1\,\dfrac{\partial h}{\partial \hat{x}_1}) + 
	\dfrac{\partial\hat\xi_2}{\partial \hat{x}_1} - \dfrac{\partial\hat\xi_1}{\partial \hat{x}_2}.
\end{IEEEeqnarray*}
Observing the degrees in each term, there hold
\begin{enumerate}
  \item 
  $g\,:=\,2\,h + \hat{x}_2\,\dfrac{\partial h}{\partial \hat{x}_2} + 
  \hat{x}_1\,\dfrac{\partial h}{\partial \hat{x}_1}
  \mbox{ belongs to } \tilde{P}_{k-1}(\hat{f}_3) \otimes P_k(\hat x_3)$
  \item 
  $\dfrac{\partial\hat\xi_2}{\partial \hat{x}_1} -
  \dfrac{\partial\hat\xi_1}{\partial \hat{x}_2}
  \mbox{ belongs to } P_{k-2}(\hat{f}_3) \otimes P_k(\hat x_3)\mbox{,}$
\end{enumerate}
but from this it follows necessarily that $g \equiv 0$. Now, how do the terms
of $g$ look like? Let us put
\begin{IEEEeqnarray*}{rCl}
	h\xyz &=& \sum_{\stackrel{i+j\,=\,k-1}{l\,\leqslant\,k}} \alpha_{_{i,j,l}}\,\hat{x}_1^i \hat{x}_2^j \hat{x}_3^l.
\end{IEEEeqnarray*}
Then
\begin{IEEEeqnarray*}{rCl}
  g\xyz & = & \sum_{\stackrel{i+j\,=\,k-1}{l\,\leqslant\,k}} 
  (2\alpha_{_{i,j,l}} + j\,\alpha_{_{i,j,l}} + i\,\alpha_{_{i,j,l}}) \hat{x}_1^i \hat{x}_2^j \hat{x}_3^l\\
  \yesnumber\label{h_is_zero} & = &(k+1)\,h\xyz\,=\,0,
\end{IEEEeqnarray*}
so $h \equiv 0$ too and, for now, 
\begin{IEEEeqnarray*}{rCl}
\wku\xyz &=& 
(\hat\xi_1\xyz, \hat\xi_2\xyz, 0)'. 
\end{IEEEeqnarray*}
The second--to--last task is to see that $\hat\xi_1$ vanishes identically.
We turn back to de edge degrees of freedom.
Set $\be$ equal to $\hat\be_1$ or $\hat\be_4$. Then the restriction
$\hat\xi_1|_{\be}$ belongs to $P_{k-1}(\be)$, so letting $\hat q = \hat\xi_1|_{\be}$ in
~(\ref{momentos1hcurl}) we obtain
\begin{IEEEeqnarray*}{rCCCCCl}
	0 &=& \hat\varphi_{\be,\,\hat\xi_1}\,(\hat\bu) &=&
	\hat\varphi_{\be,\,\hat\xi_1}\,(\wku) &=& \int_{\be} (\hat\xi_1)^2\,d\alpha\textrm{,}
\end{IEEEeqnarray*}
so, for some $\hat{p} \in P_{k-1}(\hat x_1)\otimes P_{k-2}(\hat x_3)$ we have
\[
  \hat\xi_1|_{\hat f_2}(\hat x_1,\hat x_3) = \hat{x}_3\,(\hat{x}_3-1)\,\hat{p}(\hat x_1,\hat x_3).
\]
Next choose $\hat{f} = \hat{f}_2$ and $\boldsymbol{q}=(0,0,\hat{p})'$ in~(\ref{momentos4hcurl}).
\begin{IEEEeqnarray*}{rCCCCCl} 
  0 & = & \hat\varphi_{\hat{f}_2,\bq}\,(\hat\bu) 
    & = & \hat\varphi_{\hat{f}_2,\bq}\,(\wku) 
    & = & \iint_{\hat{f}_2} \hat{x}_3\,(\hat{x}_3-1)\hat{p}^2\,d\hat S.
\end{IEEEeqnarray*}
It follows that $\hat\xi_1|_{\hat f_2}\equiv 0$, by which we know it exists
certain $\zeta \in P_{k-2}(\hat{f}_3)\otimes P_k(\hat{x}_3)$ satisfying
\[
\hat\xi_1\xyz = \hat{x}_2\,\zeta\xyz.
\]
Now we switch to the faces $\hat{f} = \hat{f}_3$ or $\hat{f}_4$. 
Take $\hat{\boldsymbol{q}} = (\zeta|_{\hat f},0,0)'$ in~(\ref{momentos2hcurl})
\begin{IEEEeqnarray*}{rCCCCCl}
  0 & = & \varphi_{\hat f,\hat{\boldsymbol{q}}}\,(\hat\bu) 
    & = & \varphi_{\hat f,\hat{\boldsymbol{q}}}\,(\wku) 
    & = & \iint_{\hat f} \hat{x}_2\zeta^2\,d\hat S\textrm{,}
\end{IEEEeqnarray*}
and it follows that
$\hat{x}_3\,(\hat{x}_3-1)$ divides $\zeta$. So putting this together
with the previous factorization, there is some
$r \in P_{k-2}(\hat{f}_3)\otimes P_{k-2}(\hat{x}_3)$ which satisfies.
\begin{IEEEeqnarray*}{rCl}
    \hat\xi_1\xyz &=& \hat{x}_2\,\hat{x}_3\,(\hat{x}_3-1)\,r\xyz.
\end{IEEEeqnarray*}
It remains to use the volume degrees of freedom. We could choose
 $\hat\br := (r,0,0)'$ in degree of freedom~(\ref{momentos6hcurl})
to get
\begin{IEEEeqnarray*}{rCCCCCl}
  0 & = & \hat\varphi_{\hat\br}\,(\hat{\bu}) & = & \hat\varphi_{\hat\br}\,(\wku)
    & = & \int_{\hat{E}} \hat{x}_2\,\hat{x}_3\,(\hat{x}_3-1)\,r\xyz^2\,d\hat\bx\textrm{,} 
\end{IEEEeqnarray*}
which yields, over all $\hat{E}$, $\hat\xi_1  \equiv  0$. Finally, if we
combine this last property with~(\ref{rot_3_es_0}) we prove that $\hat{\xi}_2$
does not depend on $\hat{x}_2$.
\end{proof}
\begin{lemma}\label{pi00u3} 
If $\hat{\bu}\xyz=(0,0, \hat{u}_3\xyz)^t$, then
$\wku\xyz = (0,0,\hat\xi_3\xyz)^t$ for some
$\hat\xi_3 \in {P}_k(\hat{f}_3)\otimes
{P}_{k-1}(\hat{x}_3)$.
\end{lemma}
\begin{proof} We will work again with
expression~(\ref{expr_wku}).
%but we will write with less details than in the previous two Lemmas
By expression~(\ref{expre_h}) for $(\curl\wku)_3$
and the commutativity in equation~(\ref{curl_commutativity}),
if we apply degrees of freedom~(\ref{momentos1hdiv})
to $\curl\hat\bu$ we obtain that
$(\curl\wku)_3$ vanishes on any of the horizontal faces $\hat{f}_3$ or $\hat{f}_4$ in
Table~\ref{prismNotationTableFaces}.
In other words, $(\curl\wku)_3
= \hat{x}_3\,(\hat{x}_3-1)\,\hat\psi$, 
($\hat\psi\in P_{k-1}(\hat{f}_3)\otimes P_{k-2}(\hat{x}_3)$)
and if we set $\hat{\br} := (0,0,\hat\psi)'$ in the
$H(\mbox{div})$ degrees of freedom~(\ref{momentos4hdiv})
we have
\begin{IEEEeqnarray*}{rCCCCCCCl}
0 & = & \int_{\hat{E}} (\curl\hat{\bu})_3\,\hat{\psi}
  & = & \hat\rho_{\br} (\curl\hat{\bu})
  & = & \hat\rho_{\br} (\hat{\br}_k\curl\hat{\bu})
  & = & \hat\rho_{\br} (\curl\wku)\\[4pt]
  &&&&&&& = & \int_{\hat{E}} \hat{x}_3(1-\hat{x}_3)\hat{\psi}^2\,d\bx
\end{IEEEeqnarray*}
yielding that $\hat\psi$ is identically zero, and also
$(\curl\wku)_3$ is identically zero.

From this point, if
we copy the argument in the proof of
Lemma~\ref{auxlabel500} starting with equation~(\ref{expre_h}) we arrive at
$h\equiv 0$, so we may rewrite~(\ref{expr_wku}) for the present case as
\begin{IEEEeqnarray}{rCl}
  \label{expre_pi00u3_} \wku &=&
  (\hat\xi_1,\hat\xi_2,\hat\xi_3)^t.
\end{IEEEeqnarray}
We claim that $\hat{\xi}_1\equiv\hat{\xi}_2\equiv0$.
To see this, first observe that the evaluation of the degrees of freedom
for the edges $\hat\be_1$ and $\hat\be_2$ yields
$\hat\xi_1|_{\hat\be_1} \equiv \hat\xi_2|_{\hat\be_2} \equiv 0$,
hence, evaluating the degree of freedom~(\ref{momentos2hcurl})
tangent to the face $\hat{f}_3$ two times we have
$\hat\xi_1|_{\hat{f}_3}  \equiv  \hat\xi_2|_{\hat{f}_3}  \equiv  0$.
In equal manner, if we pick $\hat\be_4$ and $\hat\be_5$, and then the 
degree of freedom tangent to $\hat{f}_4$ we obtain
$\hat\xi_1|_{\hat{f}_3} \equiv \hat\xi_2|_{\hat{f}_3} \equiv  0$.
So we proved there are polynomials $p_1$ and $p_2$ in
$P_{k-1}(\hat{f}_3)\otimes P_{k-2}(\hat{x}_3)$ which allow us to write
\begin{IEEEeqnarray*}{rCl}
  \hat\xi_1\xyz & = & \hat{x}_3(1-\hat{x}_3)p_1\xyz\\[4pt]
  \hat\xi_2\xyz & = & \hat{x}_3(1-\hat{x}_3)p_2\xyz.
\end{IEEEeqnarray*}
Take $\hat\bq := (0,0, \hat{q}_2|_{\hat{f}_2})'$ and 
evaluate the degree of freedom~(\ref{momentos3hcurl}). We have
\begin{IEEEeqnarray*}{rCCCCCl}
  0 & = & \hat\varphi_{\hat{f}_1,\hat{\bq}}\,(\hat\bu) 
    & = & \hat\varphi_{\hat{f}_1,\hat{\bq}}\,(\wku) 
    & = & \iint_{\hat{f}_1} \hat{x}_3(1-\hat{x}_3)\hat{q}_2^2\,d\hat S.
\end{IEEEeqnarray*}
Hence, there is some $\hat{r}_2\in P_{k-2}(\hat{f_3})\otimes P_{k-2}(\hat{x}_3)$
such that $\hat\xi_2 = \hat{x}_1\hat{x}_3(1-\hat{x}_3)\hat{r}_2$.
Now choose $\br = (0,\hat{r}_2,0)'$ and use degree of freedom~(\ref{momentos6hcurl})
to obtain $\int_{\hat{E}}\hat{x}_1\hat{x}_3(1-\hat{x}_3)\hat{r}_2^2\,d\hat\bx=0.$
Since 
$\hat{x}_1\hat{x}_3(1-\hat{x}_3)\hat{r}_2^2$ is almost everywhere greater than zero,
this implies
$\hat{\xi}_2 = 0$.
With the simmetric procedure starting with face $\hat f_2$ we get to prove
$\hat{\xi}_1 = 0$.
\end{proof}
Now here is our first important result.
\begin{theorem}\label{thm_stab_edge}
Given $p > 2$, $\hat{\bu} \in \wpcurl{\hat{E}}$,
\begin{IEEEeqnarray}{rCl}
\label{teorema_1} \norm{(\wku)_1}_{L^{\infty}(\hat{E})} & 
	\lesssim & \|\hat{u}_1\|_{W^{1,p}(\hat{E})} + 
	\|(\curl\hat{\bu})_3\|_{W^{1,1}(\hat{E})} \\	
\label{teorema_2} \norm{(\wku)_2}_{L^{\infty}(\hat{E})} & 
	\lesssim & \|\hat{u}_2\|_{W^{1,p}(\hat{E})} + 
	\|(\curl\hat{\bu})_3\|_{W^{1,1}(\hat{E})} \\	
\label{teorema_3} \norm{(\wku)_3}_{L^{\infty}(\hat{E})} & 
	\lesssim & \|\hat{u}_3\|_{W^{1,p}(\hat{E})}
\end{IEEEeqnarray}
where the constants in the inequalities depend only on $\hat{E}$.
\end{theorem}
\begin{proof}
The proof will rely on the three previous Lemmas, 
the triangular inequality applied on each component of 
expression~(\ref{edge_interp_explicit}) and traces inequalities or,
more precisely, the proof
of Lemma $5.38$ in the page $134$ of~\cite{monk}
and Theorem $3.9$ (\emph{Trace Theorem})
in page $43$ of
the same book.
First we will take a smooth field $\hat{\bu}$ defined on $\hat{E}$
and, by Proposition~\ref{density_wpcurl}, we will conclude the Theorem 
with a density argumentation.\\[4pt]
To prove~(\ref{teorema_1}) the idea will be to take another function
$\hat{\bw}$ such that its interpolate has the same first component
as the one of $\hat\bu$ and such that its degrees of freedom are
more easily bounded in terms of $\hat{u}_1$ and $\curl(\hat{\bu})_3$.

Let us define, for a given $\hat{\bu} \in C^\infty(\bar{\hat{E}})^3$,
$\hat{\bv}\,:\,\hat{E}\to\mathbb{R}^3$ with
\begin{IEEEeqnarray}{rCl} \label{auxlabel201}
  \hat{\bv}\xyz &=& (\hat{u}_1\xyz, \hat{u}_2\xyz - \hat{u}_2(0,\hat{x}_2,\hat{x}_3), 0)'.
\end{IEEEeqnarray}
Thanks to the Lemmas~\ref{auxlabel500} and~\ref{pi00u3} it holds
\begin{IEEEeqnarray*}{rCl}
	(\hat{\bw}_{\hat E}\hat{\bv})_1 & = & (\wku)_1 - 
	\hat{\bw}_{\hat E}(0, \hat{u}_2(0,\hat{x}_2,\hat{x}_3), 0)_1 -
	\hat{\bw}_{\hat E}(0, 0, \hat{u}_3)_1\\
						& = & (\wku)_1\mbox{,}
\end{IEEEeqnarray*}
and we also have $(\curl\hat{\bu})_3 = (\curl\hat{\bv})_3$.
Now let us explore one by one the degrees of freedom that define
$\hat{\bw}_{\hat E}\hat{\bv}$. The only edge degrees
that do not vanish directly or depend explicitly just on 
$\hat{u}_1$ are
\begin{IEEEeqnarray*}{rCl}
	\int_{\hat{\be}_8} q\,\hat{\bv}\cdot d\hat\balpha & = &
	\tfrac{1}{\sqrt{2}} \int_{\hat{\be}_8} (\hat{v}_1 - \hat{v}_2)\,q\,d\alpha\\
	\int_{\hat{\be}_9} q\,\hat{\bv}\cdot d\hat\balpha & = &
	\tfrac{1}{\sqrt{2}} \int_{\hat{\be}_9} (\hat{v}_1 - \hat{v}_2)\,q\,d\alpha
\end{IEEEeqnarray*}
for $q$ in $\pazocal{P}_{k-1}(\hat{\be}_8)$ or $\pazocal{P}_{k-1}(\hat{\be}_9)$ 
respectively. 
Pick a polynomial $q \in P_{k-1}(\hat{\be}_8)$. Since on
$\hat{\be}_8$ it is $\hat{x}_1 = 1 - \hat{x}_2$, we evaluate $q$ as
$q(\hat{x}_2)$, with $0\leqslant\hat{x}_2 \leqslant 1$. Integration
by parts over the face $\hat{f}_4$ yields
\begin{IEEEeqnarray*}{rCl}
  \iint_{\hat{f}_4} (\curl\hat{\bv})_3\,q\,d\hat S
	& = & -\iint_{\hat{f}_4} \left(\hat{v}_2\,\partial_{\hat{x}_1}q - \hat{v}_1\,
  \partial_{\hat{x}_2}q\right)\,d\hat{S}
		+ \int_{\partial \hat{f}_4} \left(\hat{v}_2\,\hat{\nu}_1 
    - \hat{v}_1\,\hat{\nu}_2\right)\,q\,d\hat\alpha\\
	& = & \iint_{\hat{f}_4} \hat{v}_1\,\partial_{\hat{x}_2}q\,d\hat S
		+ \int_{\hat{\be}_8} \left(\hat{v}_2 - \hat{v}_1\right)\,q\,d\hat\alpha + 
			\int_{\hat{\be}_4} \hat{v}_1\,q\,d\hat\alpha\mbox{,}
\end{IEEEeqnarray*}
hence
\begin{IEEEeqnarray*}{rCl}
	\hat\varphi_{\hat{\be}_8,\,q}(\hat\bv) & = &
  \dfrac{1}{\sqrt{2}} \int_{\hat{\be}_8} (\hat{v}_1 - \hat{v}_2)\,q\,d\hat\alpha\\
    & = &\tfrac{1}{\sqrt{2}} \int_{\hat{\be}_4} \hat{u}_1\,q\,d\hat\alpha - 
    \tfrac{1}{\sqrt{2}} \iint_{\hat{f}_4} (\curl\hat{\bu})_3\,q\,d\hat{S}
    + \iint_{\hat{f}_3} \hat{u}_1\,\partial_{\hat{x}_2}q\,d\hat{S}.\\
	\yesnumber\label{momentosWaristas}
    &&
\end{IEEEeqnarray*}
In a similar manner if we integrate over $\hat{f}_3 \subseteq \{ \hat{x}_3 = 0 \}$
we get
\begin{IEEEeqnarray*}{rCl}
	\hat\varphi_{\hat{\be}_9,\,q}(\hat\bv) & = & \tfrac{1}{\sqrt{2}} 
  \int_{\hat{\be}_9} (\hat{v}_1 - \hat{v}_2)\,q\,d\hat\alpha \\
     &=&\tfrac{1}{\sqrt{2}} \int_{\hat{\be}_1} \hat u_1\,q\,d\hat\alpha -
     \tfrac{1}{\sqrt{2}} \iint_{\hat{f}_3} (\curl\hat{\bu})_3\,q\,d\hat{S}
     + \iint_{\hat{f}_3} \hat{u}_1\,\partial_{\hat{x}_2}q\,d\hat{S}.\\
	\yesnumber\label{momentosWaristas2}
     &&
\end{IEEEeqnarray*}
If we evaluate now the face degrees of freedom, we only have to bound
those corresponding to $\hat{f}_3$, $\hat{f}_4$ and $\hat{f}_5$.
Take $\hat{q}_1$, $\hat{q}_2 \in P_{k-2}(\hat{f}_3)$ and consider $\hat{\bq} := (\hat{q}_1, \hat{q}_2, 0)$.
\begin{IEEEeqnarray}{rCl}
 	\label{cotaf3}\iint_{\hat{f}_3} \hat{\bv} \times \hat\bn \cdot \hat\bq\,d\hat{S}
 		& = & \iint_{\hat{f}_3} \hat u_1\,\hat q_2\,d\hat{S} -
    \iint_{\hat f_3} \hat{v}_2\,\hat q_1\,d\hat{S}.
\end{IEEEeqnarray}
Observe that $\hat{v}_2$ vanishes over the face $\hat{f}_1\subseteq\{\hat{x}_1=0\}$.
Now we need a polynomial $\hat\zeta \in P_{k-1}(\hat{f}_3) $ such that 
$\partial_{\hat{x}_1} \hat\zeta = \hat{q}_1$ and
$\hat\zeta |_{\hat{\be}_9} = 0$; take for instance
$\hat\zeta(\hat{x}_1,\hat{x}_2) = -\int_{\hat{x}_1}^{1-\hat{x}_2} \hat q_1(t,\hat{x}_2)\,dt$. Then
\begin{IEEEeqnarray*}{rCl}
	\iint_{\hat{f}_3} (\curl\hat{\bv})_3\,\hat\zeta\,d\hat{S} & = & 
 -\iint_{\hat{f}_3} \left(\hat{v}_2\,\hat q_1 - 
  \hat{v}_1\,\partial_{\hat{x}_2}\hat\zeta\right)\,d\hat{S}
		-\int_{\hat{\be}_1} \hat{v}_1\,\hat n_2\,\hat\zeta\,d\hat\alpha,
\end{IEEEeqnarray*}
which, together with~(\ref{cotaf3}) implies
\begin{IEEEeqnarray}{rCl}
  \nonumber  
  \hat\varphi_{\hat{f}_3,\,\hat{\bq}}(\hat{\bv})
    & = & \iint_{\hat{f}_3} \hat{u}_1\,\hat q_2\,d\hat{S} +
    \iint_{\hat{f}_3} (\curl\hat{\bu})_3\,\hat\zeta\,d\hat{S}\\[4pt]
\label{momentosWcaras}
  &&\,- \iint_{\hat{f}_3} \hat u_1\,\partial_{\hat{x}_2}\hat\zeta\,d\hat{S} +
        \int_{\hat{\be}_1} \hat{u}_1\,\hat n_2\,\hat\zeta\,d\hat\alpha.
\end{IEEEeqnarray}
If we repeated the procedure for the degree of freedom on $\hat{f}_4$, for a given 
$\hat\bp = (\hat p_1, \hat p_2, 0) \in P_{k-2}(\hat{f}_4)^2\times \{0\}$ we would set 
$\hat\psi (\hat{x}_1,\hat{x}_2) = \int_{1-\hat{x}_2}^{\hat{x}_1} 
\hat p_1 (t,\hat{x}_2)\,dt$
and had
\begin{IEEEeqnarray}{rCl}
\nonumber
  \hat\varphi_{\hat{f}_4,\,\hat{\bp}}(\hat{\bv})
    & = & - \iint_{\hat{f}_4} \hat{u}_1\,\hat{p}_2\,d\hat{S} -
    \iint_{\hat{f}_4} (\curl\hat{\bu})_3\,\hat\psi\,d\hat{S} \\[4pt]
\label{momentosWcaras2} && \,+ 
    \iint_{\hat{f}_4} \hat{u}_1\,\partial_{\hat{x}_2}\hat\psi\,d\hat{S}	-
    \int_{\hat{\be}_4} \hat{u}_1\,\hat{n}_2\,\hat\psi\,d\hat{\alpha}.
\end{IEEEeqnarray}
For the degree of freedom~(\ref{momentos5hcurl}) corresponding to $\hat{f}_5$, given
$\hat\bq = (0, \hat q_3, \hat q_1) \in \{ 0 \} \times Q_{k-2,k-1} \times Q_{k-1,k-2}$
observe
\begin{IEEEeqnarray}{rCl}\label{momentosWcaras3}
  \iint_{\hat{f}_5} \hat{\bv} \times \bn \cdot \hat{\bq}\,d\hat{S}
    & = & \iint_{\hat{f}_5} (\hat{v}_1 - \hat{v}_2)\,\hat{q}_1\,d\hat{S}
\end{IEEEeqnarray}
Now, if $\hat{q}$ is the extension of $\hat{q}_1$ to the whole prism, then
\begin{IEEEeqnarray*}{rCl}
  \iint_{\hat{f}_5} \hat{v}_2\,\hat{q}_1\,d\hat{S} & = &
  \sqrt{2} \iint\limits_{[0,1]^2}\hat{v}_2(1-\hat{x}_2,\hat{x}_2,\hat{x}_3)\hat{q}_1(\hat{x}_2,\hat{x}_3)\,d\hat{x}_2d\hat{x}_3\\[5pt]
  &=&\sqrt{2} \iint\limits_{[0,1]^2}\int_{0}^{1-\hat{x}_2}\tfrac{\partial\hat{v}_2}{\partial{\hat{x}_1}}
  (\hat{t},\hat{x}_2,\hat{x}_3)\hat{q}(\hat{t}, \hat{x}_2,\hat{x}_3)\,d\hat{t}d\hat{x}_2d\hat{x}_3\\[5pt]
  \yesnumber\label{momentosWcaras3_}
  &=&\sqrt{2}\int_{\hat{E}} (\curl\hat\bv)_3\hat{q}\,d\hat{\bx} + 
  \sqrt{2}\int_{\hat{E}} \tfrac{\partial\hat{v}_1}{\partial{\hat{x}_2}}\hat{q}\,d\hat{\bx}.
\end{IEEEeqnarray*}
Joining~(\ref{momentosWcaras3}) and~(\ref{momentosWcaras3_}) and using the
expression for $\hat{\bv}$ in~(\ref{auxlabel201}) we get 
\begin{IEEEeqnarray}{rCl}
  \nonumber
  \varphi_{\hat{f}_5,\,\hat{\bq}}(\hat{\bv})
  & = & \iint_{\hat{f}_5} \hat{u}_1\,\hat{q}_1\,d\hat{S}
  -\sqrt{2}\int_{\hat{E}} \tfrac{\partial\hat{u}_1}{\partial{\hat{x}_2}}\hat{q}\,d\hat{\bx}
  -\sqrt{2}\int_{\hat{E}} (\curl\hat\bu)_3\,\hat{q}\,d\hat{\bx}. \\
  & & \label{momentosWcaras3__}
\end{IEEEeqnarray}
At last, we study the volume degrees of freedom. Pick
$\hat\br = (\hat r_1, \hat r_2, \hat r_3)'$ belonging to the space
\[
 (P_{k-2}(\hat{f}_3) \otimes P_{k-2}(\hat{x}_3))^{2}
\times P_{k-3}(\hat{f}_3) \otimes
P_{k-1}(\hat{x}_3)
\]
(cfr. degree of freedom~(\ref{momentos6hcurl}))
and let $\hat\varphi_2$ be defined in such a way that
$\varphi_2\xyz = \int_{1-\hat{x}_2}^{\hat{x}_1} 
\hat{r}_2(\hat{t},\hat{x}_2,\hat{x}_3)\,d\hat{t}$.
%, para el cual
%vale 
%$\partial_x\varphi_2 = \hat{r}_2 $
%y $\varphi_2|_{\hat{f}_5} \equiv 0$.
Green's Theorem and the fact that $\varphi_2|_{\hat{f}_5} \equiv 0$
give 
\begin{IEEEeqnarray}{rCl}
  \nonumber\int_{\hat{E}} \hat{\bv} \cdot \hat\br\,d\hat\bx
  & = & \int_{\hat{E}} \hat{u}_1\,\hat{r}_1\,d\hat\bx 
  - \int_{\hat{E}} (\curl\hat{\bu})_3\,
  \hat\varphi_2\,d\hat\bx
  - \int_{\hat{E}}
  \tfrac{\partial\hat{u}_1}{\partial\hat{x}_2}\,
  \hat\varphi_2\,d\hat\bx.\\
\label{momentosWvolumen}
  &&
\end{IEEEeqnarray}
%,~(\ref{momentosWcaras2}),~(\ref{momentosWcaras3})
Now we collect what has been said so far.
%%%%%%%%%%%%%%%%%%%%%%%%% equalities,,~(\ref{momentosWcaras}),~(\ref{momentosWcaras2}),~(\ref{momentosWcaras3}) and~(\ref{momentosWvolumen}).
For the edge degrees of freedom we use an inequality in page $135$ of~\cite{monk},
in the proof of Lemma 5.38, which states, for $\hat\bu$ in the present conditions,
\begin{IEEEeqnarray}{rCl}\label{edgeTrace}
  \left|\int_{\hat\be} \hat\bu\cdot\hat\btau\,q\,d\hat\alpha\right| 
  & \leqslant & C(q) \,\{\, \|\curl\hat\bu\|_{L^p(\hat{E})^3}
    + \|\mbox{Tr}\,\hat\bu\|_{L^p(\partial\hat{E})^3} \}.
\end{IEEEeqnarray}
The details needed to the proof of inequality~(\ref{edgeTrace}) can be completed
from Theorem 3.14 of~\cite{A-2001}.

Now if we put the field $(\hat{u}_1,0,0)'$ in inequality~(\ref{edgeTrace}) then
by, H\"older's Inequality and standard traces inequalities, equation~(\ref{momentosWaristas})
and~(\ref{momentosWaristas2}) yield, for $i=8$ and $9$,
\begin{IEEEeqnarray*}{rCl}
  \left|\varphi_{\hat{\be}_i,\,\hat{q}}(\hat\bv)\right| & \leqslant & c(\hat{q})\,
  \{\,\|\hat{u}_1\|_{W^{1,p}(\hat{E})} + \|\mbox{Tr}\,(\curl{\hat{\bu}})_3\|_{L^1(\partial\hat{E})}
  +\|\mbox{Tr}\,\hat{u}_1\|_{L^p(\partial\hat{E})}\,\}\\[5pt]
  \yesnumber\label{traceE8}
  & \leqslant & c(\hat{q})\,\{\,\|\hat{u}_1\|_{W^{1,p}(\hat{E})} + 
  \|(\curl\hat{\bu})_3\|_{W^{1,1}(\hat{E})}\,\}.
\end{IEEEeqnarray*}
If we repeat the argument for the line integral
terms in~(\ref{momentosWcaras}) and~(\ref{momentosWcaras2}) we get, for $j=3$ and $4$,
\begin{IEEEeqnarray}{rCl}
\nonumber
  \left|\varphi_{\hat{f}_j,\,\hat{\bq}}(\hat\bv)\right| & \leqslant &
  c(\hat{\bq})\,\{\,
    \|\mbox{Tr}\,\hat{u}_1\|_{L^p(\partial\hat{E})} +
    \|\mbox{Tr}\,(\curl{\hat{\bu}})_3\|_{L^1(\partial\hat{E})} +
    |\hat{u}_1|_{W^{1,p}(\hat{E})}\,\}.\\
\label{traceF3}&&
\end{IEEEeqnarray}
And finally, by estimates~(\ref{momentosWcaras3})--(\ref{traceF3})
and one more time H\"older's and traces inequalities,
\begin{IEEEeqnarray*}{rCCCl}
	\|(\wku)_1\|_{L^\infty(\hat{E})} & = & \|(\hat{\bw}_{\hat E}\hat{\bv})_1\|_{L^\infty(\hat{E})}
  &\lesssim&
  \sum_{i=8,9,\,\hat p} |\hat\varphi_{\hat{\be}_i,\hat p}(\hat{\bv})|\,\|(\hat{\bv}_{\hat{\be}_i,\hat p})_1\|_{L^\infty(\hat{E})}
  \\[4pt]\IEEEeqnarraymulticol{5}{C}{\,+
  \sum_{j=3,4,\,\hat\bq} |\hat\varphi_{\hat{f}_j,\hat\bq}(\hat{\bv})|\,
  \|(\hat{\bv}_{\hat f_j,\hat\bq})_1\|_{L^\infty(\hat{E})} +
  \sum_{\hat\br} |\hat\varphi_{\hat\br}(\hat{\bv})|\,\|(\hat{\bv}_{\hat\br})_1\|_{L^\infty(\hat{E})}
  }\\[4pt]
	& \leqslant & c(\hat{E})\,\{\,\|\hat{u}_1\|_{W^{1,p}(\hat{E})} & + &
		\|(\curl\hat{\bu})_3\|_{W^{1,1}(\hat{E})}\,\}
\end{IEEEeqnarray*}
which is the bound we wanted to prove. The summation indices with polynomials
$\hat p$, $\hat\bq$ and $\hat\br$ mean that we use the way of writing
the interpolator stated in~(\ref{edge_interp_explicit}). The same proving procedure applies to 
inequality~(\ref{teorema_2}).\\[7pt]
For inequality~(\ref{teorema_3}), given $\hat{\bu} \in W^{1,p}(\hat{E})^3$, define
$\hat{\bv}  =  (0,0, \hat{u}_3)'.$
Thanks to Lemma~\ref{lema_PIu3_k_cualquiera} we have 
$(\hat{\bw}_{\hat E}\hat{\bv})_3 = (\hat{\bw}_{\hat E}\hat{\bu})_3 - (\hat{\bw}_{\hat E}(\hat{u}_1, \hat{u}_2, 0))_3 = (\hat{\bw}_{\hat E}\hat{\bu})_3.$
By expression~(\ref{edge_interp_explicit}), taking another look at 
the unit tangent vector of the edges and unit normal vectors to the
faces, we have
\begin{IEEEeqnarray*}{rCl}
  (\hat{\bw}_{\hat E}\hat{\bv})_3 & = &
  \sum_{j=3,6,7;\,\hat{\bp}}
  \int_{\hat e_j} \hat{u}_3 \hat{p}_3\,d\alpha\,(\hat{\bv}_{\hat{\be}_j,\hat{\bp}})_3 +
  \sum_{i=1,2,4;\,\hat q}
  \int_{\hat f_i} \hat{u}_3 \hat q\,d\hat S \,(\hat{\bv}_{\hat{f}_i,\hat q})_3\\
  &&\,+\sum_{\hat \br}
  \int_{\hat E} \hat{u}_3 \hat r_3\,d\hat\bx\,(\hat{\bv}_{\hat\br})_3.
\end{IEEEeqnarray*}
This implies, by traces inequalities and~(\ref{edgeTrace}), that
\begin{IEEEeqnarray*}{rCl}
  \norm{(\hat{\bw}_{\hat E}\hat{\bu})_3}_{L^{\infty}(\hat{E})}
  & \leqslant & C(\hat{E}) \big\{
  \sum_{j=3,6,7; \hat{\bp}}
  \left|\int_{\hat e_j} \hat{u}_3\,\hat{p}_3\,d\alpha\right| \\[4pt]
  &&\,+
  \sum_{i=1,2,4}
  \iint_{\hat f_i} |\hat{u}_3|^p\,d\hat S
  + \int_{\hat E} |\hat{u}_3|^p\,d\hat\bx  \big\}\\[4pt]
  &\leqslant& C(\hat{E}) \|\hat{u}_3\|_{W^{1,p}(\hat{E})}.
\end{IEEEeqnarray*}
The constants in the three inequalities of this Theorem depend only
on the choice of the bases of the test polynomials for the degrees of freedom.
\end{proof}
As in the div--conforming case, the next step is to estimate the stability in 
an anisotropically rescaled prism. Consider again the element $\tilde{E}$ defined in~(\ref{tilde_prism}).
Given a natural number $k$, denote with ${\bw}_{\tilde{E}}$ the $k$--th order 
$\bcurl$--conforming interpolation
operator over $\tilde{E}$ defined as in Corollary~\ref{aux_label26}. 
For the rest of the Subsection, $\tilde\bu$ will be an element
with a well defined $\bcurl$--conforming interpolate.
%, namely of
%$H({\bf curl},\tilde{E})\cap H^{1/2+\delta}(\tilde{E})^3$ for 
%a positive $\delta$ with 
%${\bf curl}\,\tilde{\bu}\in L^p(\tilde{E})^3$
%for some
%$p>2$.
Write the diameter of $\tilde{E}$ as $h_{\tilde{E}}$ and as
$\tilde{x}_i,\,1\leqslant i\leqslant 3$, the coordinates along the axes
in $\mathbb{R}^3$.
\begin{lemma}\label{estabLinf} There exists a positive $C$, independent
of $h_i,\,1\leqslant i\leqslant 3$ such that for all $p > 2$ and 
$\tilde{\bu}\in\wpcurl{\tilde{E}}$
\begin{IEEEeqnarray*}{rCl}
    \left\| \wkutilde \right\|_{L^\infty(\tilde{E})^3}
    & \leqslant & C \left[ |\tilde{E}|^{-\nicefrac{1}{p}} \left( \left\| \tilde{\bu} 
    \right\|_{L^p(\tilde{E})^3} +
        \sum_{i=1}^3 h_i \left\| \partial_{\tilde{x}_i}\tilde{\bu} 
        \right\|_{L^p(\tilde{E})^3} \right)\right.\\
\IEEEeqnarraymulticol{3}{c}
{\left.\:+\; (h_1+h_2)\, |\tilde{E}|^{-1} \left( \left\|(\curl\,\tilde{\bu})_3 
    \right\|_{L^1(\tilde{E})} + 
    \sum_{i=1}^3 h_i \left\| \partial_{\tilde{x}_i}(\curl\,\tilde{\bu})_3 
    \right\|_{L^1(\tilde{E})}\right)
    \right].}
\end{IEEEeqnarray*}
%============
%{\color{BrickRed} ver las cuentas donde dice $h_1 + h_2$}
%============
% tal vez esto no haga falta
%para transformar $\hat{E} $ en $\tilde{E} $ v\'ia
%
%in this particular case ...
%\begin{IEEEeqnarray*}{rCl}
%    \hat{\pi}_i & = & h_i\tilde{\pi}_i \\
%    (\textbf{curl}\,\hat{\bu})_3 & = & h_1h_2(\textbf{curl}\,\tilde{\bu})_3.
%\end{IEEEeqnarray*}
%============
\end{lemma}
\begin{proof}
The proof of this estimate will be made componentwise
using the inequalities of 
Theorem~\ref{thm_stab_edge} and the vectorial bound will hold immediately.
Bounds for $(\wkutilde)_1$ and $(\wkutilde)_3$
will be established, as the bounding for $(\wkutilde)_2$ is the same as the first one.
Pulling $\wkutilde$ back to $\hat{E}$ we get by~(\ref{piTransformado}) 
that $(\wkutilde)_i = 
\nicefrac{1}{h_i} (\wku)_i,\,1\leqslant i\leqslant 3$. By inequality~(\ref{teorema_1}) and a suitable, though elementary,
change of variables dictated by~(\ref{change_var}) we do
\begin{IEEEeqnarray*}{rCl}
  \left\| (\wkutilde)_1 \right\|_{L^\infty(\tilde{E})} & = &
    \frac{1}{h_1} \left\| (\wku)_1 \right\|_{L^\infty(\hat{E})}\\
    & \leqslant & \frac{c(\hat{E})}{h_1} \left[\|\hat{u}_1\|_{W^{1,p}(\hat{E})} + 
        \|(\curl\,\hat{\bu})_3\|_{W^{1,1}(\hat{E})}\right] \\
    & \leqslant & c(\hat{E})
  \left[
    |\tilde{E}|^{\nicefrac{-1}{p}}
    \big\{
    \|\tilde{u}_1\|_{L^p(\tilde{E})} + \sum_{i=1}^3 h_i
    \|\tfrac{\partial\tilde{u}_1}{\partial\tilde{x}_i}\|_{L^p(\tilde{E})}
    \big\}
  \right.\\
\IEEEeqnarraymulticol{3}{r}{+
  \left.
    h_2|\tilde{E}|^{-1}
    \big\{
    \|(\curl\,\tilde{\bu})_3\|_{L^1(\tilde{E})} + 
        \sum_{i=1}^3 h_i \|\partial_{\tilde{x}_i}(\curl\,\tilde{\bu})_3\|_{L^1(\tilde{E})}
    \big\}
  \right].}
  \\&&\yesnumber\label{number1}
\end{IEEEeqnarray*}
With respect to component number three, from~(\ref{teorema_3}) we write
\begin{IEEEeqnarray}{rCl}\label{number2}
  \left\| (\wkutilde)_3 \right\|_{L^\infty(\tilde{E})}
  & \leqslant & C|\tilde{E}|^{-\nicefrac{1}{p}}
  \left(
    \|\tilde{u}_3\|_{L^p(\tilde{E})} + \sum_{i=1}^3 h_i \|\partial_{\tilde{x}_i}\tilde{u}_3\|_{L^p(\tilde{E})}
  \right).
\end{IEEEeqnarray}
\end{proof}
\noindent With the previous bound we deduce the following
anisotropic stability estimate for the rescaled prismatic element $\tilde{E}$.
\begin{theorem} \label{aux_label27}
There is a $C > 0$ independent of $h_i$ such that for all
$\tilde{\bu}\in\wpcurl{\tilde{E}}$ and $p>2$.
\begin{IEEEeqnarray*}{rCl}
    \|\wkutilde\|_{L^p(\tilde{E})}
    & \leqslant & C \left[ \left\| \tilde{\bu} \right\|_{L^p(\tilde{E})}
    + \sum_{i=1}^3 h_i \left\| \partial_{\tilde{x}_i}\tilde{\bu} \right\|_{L^p(\tilde{E})}\right.
\\\IEEEeqnarraymulticol{3}{r}{
\left.
    \:+\;(h_1+h_2)\left(\left\|(\curl\,\tilde{\bu})_3 \right\|_{L^p(\tilde{E})}
     + \sum_{i=1}^3 h_i
     \left\| \partial_{\tilde{x}_i}(\curl\,\tilde{\bu})_3 \right\|_{L^p(\tilde{E})}\right)
  \right].
}
\end{IEEEeqnarray*}
\end{theorem}
\begin{proof}
    \noindent From Lemma~\ref{estabLinf}, since $|\tilde{E}|$ is finite measured,
    the H\"older inequality tells us that, for any real $q \geqslant 1$,
    \begin{IEEEeqnarray*}{rCl}
        \|(\curl\tilde{\bu})_3\|_{L^1(\tilde{E})} &\leqslant&
         |\tilde{E}|^{1-\frac{1}{q}}\,\|(\curl\,\tilde{\bu})_3\|_{L^q(\tilde{E})}\\
        \|\partial_{\tilde{x}_i}(\curl\,\tilde{\bu})_3\|_{L^1(\tilde{E})} &\leqslant&
         |\tilde{E}|^{1-\frac{1}{q}}\,\|\partial_{\tilde{x}_i}(\curl\,\tilde{\bu})_3\|_{L^q(\tilde{E})}.
    \end{IEEEeqnarray*}
    So we get to
    \begin{IEEEeqnarray*}{rCl}
    \left\| (\tilde{\bw}_{\tilde E}\tilde{{\bu}})_1 \right\|_{L^p(\tilde{E})}
      & \leqslant & |\tilde{E}|^{\nicefrac{1}{p}}\left\| (\tilde{\bw}_{\tilde E}\tilde{{\bu}})_1 \right\|_{L^\infty(\tilde{E})}\\
      \mbox{(by~(\ref{number1}))\hspace{.6cm}}   & \leqslant & C
      \left[
        \|\tilde{u}_1\|_{L^p(\tilde{E})} + \sum_{i=1}^3 h_i \|\tfrac{\partial\tilde{u}_1}{\partial\tilde{x}_i}\|_{L^p(\tilde{E})}
        \right.\\
          & & \:\:+
        \left.
            h_2
            \left(
            \|(\curl\tilde{\bu})_3\|_{L^p(\tilde{E})} + 
                \sum_{i=1}^3 h_i \|\partial_{\tilde{x}_i}(\curl\tilde{\bu})_3\|_{L^p(\tilde{E})}
            \right)
        \right].
    \end{IEEEeqnarray*}
    Now combine this with an entirely analogous argument for component two and with~(\ref{number2}) and
    the Theorem follows.
\end{proof}
\section{Local Interpolation Estimates for $H(\bcurl)$ Conforming Prismatic Elements}
\begin{theorem} \label{aux_label32} Let $k\in\mathbb{N}$ and $p>2$ and let $E$ be a prism whose triangular
faces have greatest angle less than $c_0$.
There exists $C > 0$ and three edges $\be_i$ of $E$ incident to a common vertex
$\bx_E$ such that for all $\bu\in W^{m + 1,p}(E)^3$
and $m\leqslant k-1$, %with $\bcurl \bu\in W^{m,p}(E)^3$
\begin{IEEEeqnarray*}{rCl}\label{aux_label55}
  \|\bu-\bw_E \bu\|_{L^p(E)} & \leqslant & C
  \left\{
    \sum_{|{\balpha}|=m+1}\bh^{\balpha} \|\partial^{\balpha} \bu\|_{L^p(E)} +\right.\\[4pt]
  \yesnumber\label{auxlabel5}
   &&\qquad\left. h_E\sum_{|{\balpha}|=m}\bh^{\balpha}\|\partial^{\balpha} 
    (\curl \bu)_3 \|_{L^p(E)}
  \right\}.
\end{IEEEeqnarray*} 
$C$ depends only on $c_0$.
$C$ can be chosen so that, if $M_E$ is the matrix made with
$\xi_i$ as columns, then $\|M\|_\infty\leqslant C$ and $\|M^{-1}\|_\infty\leqslant C$ 
and $\det M_E \geqslant C$
\end{theorem}
Notice the an\-iso\-tropic character of the inequality in~\eqref{auxlabel5}. Notice only the component
of the $\curl$ corresponding to the direction that is orthogonal to the 
triangular faces.
%[Proof of Theorem~\ref{aux_label32}]
\begin{proof}[Proof of Theorem~\ref{aux_label32}] %%% TODO: {\color{BrickRed}\#\#\#\#\#\#\#\# Ariel, por favor mirar si es correcta la manera en que lo digo.}\\\\
Since $W^{m+1,p}(E)\hookrightarrow W^{1,p}(E)$ and $p$ is greater than $2$,
the edge interpolator $\bw_E$ is well--defined via Corollary~\ref{aux_label26}.
Consider the prism $\tilde E$ as in~(\ref{tilde_prism}). There is an affine map
$\tilde \bx \mapsto \bx = M_E\,\tilde\bx+\bx_E = F_E\,\tilde\bx$ from $\tilde E$ onto $E$, such that 
$\|M_E\|$, $\|M_E\|^{-1}\leqslant C$. The matrix $M_E$ is made up of vectors 
$\xi_i$, $i = 1$, $2$, $3$ as its columns. First we take $\bq := \Qbb_{m,E}\,\bu$ and
do%where $\xi_i$ are the unitary vectors in the directions of three edges $\ell_i$  of E of lengths $h_i$ sharing the vertex $\bx_E$.
\begin{IEEEeqnarray*}{rCl}
  \|\bu-\bw_E\bu\|_{L^p(E)} & \leqslant & \|\bu-\bq\|_{L^p(E)}
    +\|\bw_E(\bu-\bq)\|_{L^p(E)}
\end{IEEEeqnarray*}
For the first term we may simply do, by Remark~\ref{aux_label28} and the
transformation~(\ref{transfHcurl}),
\begin{IEEEeqnarray}{rCl}
\nonumber
  \|\bu-\bq\|_{L^p(E)} & = & \|M_E^{-t}(\tilde{\bu}-\tilde{\bq})\circ F_E^{-1}\|_{L^p(E)} \\[5pt]
\label{aux_label37}
  &\leqslant& \|M^{-1}\||\det M_E|^{\nicefrac1p}\|\tilde{\bu}-\tilde{\bq}\|_{L^p(\tilde E)}.
\end{IEEEeqnarray}
With regard to the second term, by the commutativity property~(\ref{piTransformado})
and again the coordinate transformation,
\begin{IEEEeqnarray*}{rCl}
  \|\bw_E(\bu-\bq)\|_{L^p(E)}&\leqslant&
    |M|^{\nicefrac1p}\|M_E^{-1}\|  
      \|\tilde{\bw}_{\tilde E}(\tilde\bu-\tilde\bq)\|_{L^p(\tilde E)}.
\end{IEEEeqnarray*}
Theorem~\ref{aux_label27} implies
\begin{IEEEeqnarray*}{rCl}
    \|\bw_E(\bu-\bq)\|_{L^p(E)}
    & \leqslant & \\
\IEEEeqnarraymulticol{3}{r}{
\begin{IEEEeqnarraybox*}{rl}
  \qquad & C\|M^{-1}\||\det M_E|^{\nicefrac1p}
\left[ \| \tilde\bu-\tilde\bq \|_{L^p(\tilde{E})}
    + \sum_{i=1}^3 h_i \| \partial_{\tilde{x}_i}(\tilde\bu-\tilde\bq) \|_{L^p(\tilde{E})}\right.\\
&
    \left.
    \:+\;h\left(\left\|(\curl(\tilde\bu-\tilde\bq))_3 \right\|_{L^p(\tilde{E})}
     + \sum_{i=1}^3 h_i
     \left\| \partial_{\tilde{x}_i}(\curl(\tilde\bu-\tilde\bq))_3 \right\|_{L^p(\tilde{E})}\right)
  \right].
\end{IEEEeqnarraybox*}
}\\[4pt]
&&\yesnumber\label{aux_label34}
\end{IEEEeqnarray*}
By expressions~(\ref{aux_label30}),~(\ref{aux_label24}),~(\ref{aux_label25})
the last expression is bounded by a constant times
$\|M_E^{-1}\||\det M_E|^{\nicefrac1p}$
times the following sum
\begin{IEEEeqnarray}{rCl}
\nonumber
\sum_{i+j+k=m+1} h_1^ih_2^jh_3^k \left\| \frac{\partial^{m+1}\tilde\bu}
    {\partial\tilde x_1^i\partial\tilde x_2^j\partial\tilde x_3^k}
    \right\|_{0,\tilde E} &+&
h \sum_{j+k+l=m}  h_1^jh_2^kh_3^l
  \left\|\frac{\tilde\partial^m(\curl\tilde\bu)_3}{\partial\tilde x_1^j\partial\tilde x_2^k\partial\tilde x_3^l}
  \right\|_{0,\tilde E}
\\[7pt]
\IEEEeqnarraymulticol{3}{r}{
\nonumber
+\,h \sum_{i=1}^3 h_i\sum_{j+k+l=m-1}  h_1^jh_2^kh_3^l
        \left\|\frac{\tilde\partial^{m-1}\tilde\partial(\tilde\curl\tilde\bu)_3}
               {\partial\tilde x_1^j\partial\tilde x_2^k\partial\tilde x_3^l\partial\tilde x_j}
       \right\|_{0,\tilde E}
}\\[7pt]
\IEEEeqnarraymulticol{3}{r}{\nonumber
\lesssim
\sum_{i+j+k=m+1} h_1^ih_2^jh_3^k \left\| \frac{\partial^{m+1}\tilde\bu}
    {\partial\tilde x_1^i\partial\tilde x_2^j\partial\tilde x_3^k}
    \right\|_{0,\tilde E}
+  h \sum_{j+k+l=m}  h_1^jh_2^kh_3^l
  \left\|\frac{\tilde\partial^m(\curl\tilde\bu)_3}{\partial\tilde x_1^j\partial\tilde x_2^k\partial\tilde x_3^l}
  \right\|_{0,\tilde E}.}\\[4pt]&&
\label{aux_label33}
\end{IEEEeqnarray}
From equality~(\ref{aux_label29}), for every $\balpha$ of order
$m+1$ it holds
\begin{IEEEeqnarray}{rCl}\label{aux_label36}
  \|\tilde{\partial}^{\balpha}\tilde\bu\|_{L^p(\tilde{E})} & \leqslant & 
  \|M_E\|\,|\det M_E|^{-\nicefrac1p} \|\partial^{\balpha}\bu\|_{L^p(E)}.
\end{IEEEeqnarray} %%{\color{BrickRed}\#\#\#\#\#\#\#\# esto de la matriz realmente hace falta?.}
Lastly, adapting Lemma 3.57 in page 77 of~\cite{monk}, we observe
\begin{IEEEeqnarray*}{rCl}
  \begin{pmatrix}
    0 & -(\tilde\curl\tilde\bu)_3 & 0 \\
    (\tilde\curl\tilde\bu)_3 & 0 & 0 \\
    0 & 0 & 0 
  \end{pmatrix}M_E^{-1}
  & = & M_E^{t}
  \begin{pmatrix}
    0 & -(\curl\bu)_3 & 0 \\
    (\curl\bu)_3 & 0 & 0 \\
    0 & 0 & 0 
  \end{pmatrix}\circ F_E
\end{IEEEeqnarray*}
which implies, for every $\balpha$ of order $m$,
\begin{IEEEeqnarray}{rCl} \label{aux_label35}
  \|\tilde{\partial}^{\balpha}(\tilde{\curl}\tilde\bu)_3\|_{L^p(\tilde E)}
  & \leqslant & C |\det M_E|^{-\nicefrac1p}\,\|M_E\|^{2} 
  \|\partial^{\balpha}(\curl\bu)_3\|_{L^p(E)}.
\end{IEEEeqnarray}   %%{\color{blue}\#\#\#\#\#\#\#\# De donde sale el $(2+m)$ de (6.15) en~\cite{ariel}?.}
Now combine~(\ref{aux_label33}),~(\ref{aux_label36}) and~(\ref{aux_label35}) 
with~(\ref{aux_label34}) and~(\ref{aux_label37}) to obtain the
Theorem.
\end{proof}
\section{Pyramidal Finite Elements}\label{label999}
Here we state and prove least order an\-is\-otrop\-ic stability inequalities
and an\-is\-otrop\-ic local interpolation
inequalities for the finite elements on pyramids defined in~\cite{gh99} 
and~\cite{Nigam-2012}. This estimates could be used to build a variant
of our method presented in this Thesis, using finite elements in all the types
of elements. As we said in the introduction of the Thesis, one of the ideas of 
our method was the combination of finite elements in prisms and tetrahedra with 
virtual elements on pyramids.
\facesOfPyramid
\edgesOfPyramid
\begin{figure}[!h]
\centering
  \unitTangentsPyramid
  \caption{Directions of the positive unit tangents (cfr. Table~\ref{pyramidNotationTableEdges}).}
  \label{reference_pyramid}
\end{figure}

\subsection{$H(\bcurl)$--Conforming Element on Pyramids} % (fold)
\label{sub:edge}
\begin{defi}\label{aux_label50}
  The following items define a least order $\bcurl$--conforming finite element
  on the reference Pyramid.
  \begin{enumerate}
    \item $\hat E$ is the reference Pyramid in Definition~\ref{defi_of_ref_pyr}. 
    \item The rational space $P_{\hat E}$ is the span of
    $\{\hat{\bgamma}_1,\,\ldots,\,\hat{\bgamma}_8\}$ with $\hat{\bgamma}_i$
    as in Table~\ref{shape_edge_table}.
    \item The degrees of freedom are the line integrals
      \begin{IEEEeqnarray*}{c}
        \int_{\hat\be_j}\hat\bu\cdot\,d\hat\balpha
      \end{IEEEeqnarray*}
      for every edge $\hat\be_j$ of $\hat E$, $1\leqslant j\leqslant 8$.
  \end{enumerate}
\end{defi}
\edgeShapeTable
A direct computation yields the following Lemma.
\begin{lemma}
  For $1\leqslant i,j\leqslant 8$,
  $\int_{\hat\be_j}\hat\bgamma_i\cdot d\hat\balpha = \delta_{ij}$ which
  implies immediately that the finite element in Definition~\ref{aux_label50}
  is unisolvent in $\hat E$. %$H(\bcurl)$--conforming and 
\end{lemma}
\begin{lemma}
  $P_0(\hat E)^3 \subseteq P_{\hat E}$.  
\end{lemma}
\begin{proof}
  Cfr. Lemma 7.3 of~\cite{Nigam-2012}.
\end{proof}
% subsection edge (end)
\subsection{$H(\Div)$--Conforming Element on Pyramids} % (fold)
\label{sub:face}
\begin{defi}\label{aux_label71}
The following items define a least order $\Div$--conforming finite element
  on the reference Pyramid.
\begin{enumerate}
  \item $\hat{E}$ is the reference pyramid of Figure~\ref{reference_pyramid}.
  \item The space $P_{\hat{E}}$ is the span of 
  $\{\hat{\bz}_1,\,\ldots,\,\hat{\bz}_5\}$ with $\hat{\bz}_i$
    as in Table~\ref{shape_face_table}.
  \item The degrees of freedom are the surface integrals
  \begin{IEEEeqnarray*}{c}
    \label{dofsdivpyramid} \iint_{\hat{f}_j} \hat\bv\cdot\hat\bn\,d\hat S
  \end{IEEEeqnarray*}
  for every face $\hat{f}_j$ of $\hat E$, $1\leqslant j\leqslant 5$.
\end{enumerate}
\end{defi}
\faceShapeTable
A direct computation yields the following Lemma.
\begin{lemma}
  For $1\leqslant i,j\leqslant 5$,
  $\iint_{f_j}\hat\bz_i\cdot\hat\bn\,d\hat S = \delta_{ij}$ which
  implies immediately that the finite element in Definition~\ref{aux_label71}
  is unisolvent in $\hat E$. %$H(\bcurl)$--conforming and 
\end{lemma}
\begin{lemma}
  $P_0(\hat E)^3 \subseteq P_{\hat E}$.
\end{lemma}
\begin{proof}
  Cfr. Lemma 7.2 of~\cite{Nigam-2012}.
\end{proof}
% subsection face (end)
\section{Stability Estimates for Pyramidal Finite Elements} % (fold)
\label{sec:pyramidal_finite_elements}
$\hat{E}$ will be the reference pyramid  in Figure~\ref{reference_pyramid}.
Anisotropic interpolation error estimates for pyramidal $\bcurl$--conforming
and div--conforming finite elements of least order will we established.
\subsection{Anisotropic Stability Estimates for $H(\bcurl)$--Conforming 
Elements on Pyramids} % (fold)
\label{sub:edge_elements}
Here we state an anisotropic stability bound for
the least order $\bcurl$ -- conforming operator on pyramids.
As always, we write it componentwise to make the anisotropy of the 
estimate clearer.
\begin{theorem} \label{aux_label53}
Let $\hat E$ be the reference pyramid and let $p>2$. Let $\bw_{\hat E}(\cdot)$ denote
the interpolation operator determined by the degrees of freedom in Definition~\ref{aux_label50}.
There is $C>0$ such that,
for all $\hat\bu\in W^{1,p}(\hat E)$ with first derivatives in $W^{1,1}(\hat E)$, there hold
\begin{IEEEeqnarray}{rCl}
  \nonumber\|(\wku)_1\|_{L^\infty(\hat E)}&\lesssim&
  \|\hat u_1\|_{\scriptscriptstyle W^{1,p}(\hat E)} +
  \|(\nabla\times\hat\bu)_2\|_{\scriptscriptstyle W^{1,1}(\hat E)} +  
  \|(\nabla\times\hat\bu)_3\|_{\scriptscriptstyle W^{1,1}(\hat E)}\\[4pt]
  \label{auxlabel203}
  &&\, +  \left\|\tfrac{\partial \hat u_1}{\partial\hat x_2}\right\|_{\scriptscriptstyle W^{1,1}(\hat E)} +
          \left\|\tfrac{\partial^2 \hat u_3}{\partial\hat x_2\partial\hat x_1}\right\|_{\scriptscriptstyle L^{1}(\hat E)}.\\[12pt]
  \nonumber\|(\wku)_2\|_{L^\infty(\hat E)}&\lesssim&
\|\hat u_2\|_{\scriptscriptstyle W^{1,p}(\hat E)} +
  \|(\nabla\times\hat\bu)_1\|_{\scriptscriptstyle W^{1,1}(\hat E)} +  
  \|(\nabla\times\hat\bu)_3\|_{\scriptscriptstyle W^{1,1}(\hat E)}\\[4pt]
  &&\, +  \left\|\tfrac{\partial \hat u_2}{\partial\hat x_1}\right\|_{\scriptscriptstyle W^{1,1}(\hat E)} +
          \left\|\tfrac{\partial^2 \hat u_3}{\partial\hat x_2\partial\hat x_1}\right\|_{\scriptscriptstyle L^{1}(\hat E)}.\\[12pt]
  \nonumber\|(\wku)_3\|_{L^\infty(\hat E)}&\lesssim&
  \|\hat u_3\|_{\scriptscriptstyle W^{1,p}(\hat E)} +
  \|(\nabla\times\hat\bu)_2\|_{\scriptscriptstyle W^{1,1}(\hat E)} +  
  \|(\nabla\times\hat\bu)_1\|_{\scriptscriptstyle W^{1,1}(\hat E)}\\[5pt]
  \IEEEeqnarraymulticol{3}{r}{\label{auxlabel209}
  +\left\|\tfrac{\partial \hat u_3}{\partial\hat x_1}\right\|_{\scriptscriptstyle W^{1,1}(\hat E)} +
        \left\|\tfrac{\partial \hat u_2}{\partial\hat x_1}\right\|_{\scriptscriptstyle W^{1,1}(\hat E)} +
        \left\|\tfrac{\partial \hat u_1}{\partial\hat x_2}\right\|_{\scriptscriptstyle W^{1,1}(\hat E)} +
        \left\|\tfrac{\partial^2 \hat u_3}{\partial\hat x_2\partial\hat x_1}\right\|_{\scriptscriptstyle L^{1}(\hat E)}.}
\end{IEEEeqnarray}
\end{theorem}
\begin{proof}
Take an element $\hat\bu$ of $W^{1,p}(\hat{E})$ for a $p > 2$.
Let us recall the shape funtions in Table~\ref{shape_edge_table}.
For the variables of the shape functions in upcoming computations we will write $x$,$y$ and $z$ instead of
$\hat x_i$ to get a cleaner reading. Start with $\hat\bu$ of the form $(\hat u_1,0,0)'$. After calculating we have
\begin{IEEEeqnarray*}{rCl} %%\nabla\times\hat\bu &=& (0, \tfrac{{\s\partial} \hat u_1}{{\s\partial} \hat x_3},-\tfrac{{\s\partial} \hat u_1}{{\s\partial} \hat x_2})'.\\[6pt]
	\wku	&=& [{\s\int_{\hat{\be}_1}\hat\bu\cdot d\hat\balpha_1}]\hat\bgamma_1 +
				    [{\s\int_{\hat{\be}_3}\hat\bu\cdot d\hat\balpha_3}]\hat\bgamma_3 + 
				    [{\s\int_{\hat{\be}_6}\hat\bu\cdot d\hat\balpha_6}]\hat\bgamma_6 + 
				    [{\s\int_{\hat{\be}_8}\hat\bu\cdot d\hat\balpha_8}]\hat\bgamma_8\\[5pt]
			&=:& \varphi_1(\hat\bu)\hat\bgamma_1 + 
				   \varphi_3(\hat\bu)\hat\bgamma_3 + 
				   \varphi_6(\hat\bu)\hat\bgamma_6 + 
				   \varphi_8(\hat\bu)\hat\bgamma_8.
\end{IEEEeqnarray*}
\begin{IEEEeqnarray*}{rCl}
  (\wku)_1(x,y,z) 
    &  = & \varphi_1(\hat\bu)(1-z-y)+ 
	  \varphi_3(\hat\bu)y+ 
	  \varphi_6(\hat\bu)(-z+\frac{yz}{1-z})\\[4pt]
    &&\,+\,  \varphi_8(\hat\bu)(-\frac{yz}{1-z})\\[4pt]
	& = & \varphi_1(\hat\bu) - (\varphi_1 + \varphi_6)(\hat\bu)\,z+ 
	  (\varphi_3 - \varphi_1)(\hat\bu)\,y\\[4pt]
  &&\, +\, (\varphi_6-\varphi_8)(\hat\bu)\,\frac{yz}{1-z}.
\end{IEEEeqnarray*}
Now we explore the new coefficients separately. As the tangential component of $\hat\bu$
along $\hat\be_5$ equals zero, and this is an argument we are using repeatedly in the forthcoming
computations, we may write, by Stokes' Theorem,
\begin{IEEEeqnarray*}{rCl}
  (\varphi_1+\varphi_6)(\hat\bu)
  	& = & \int_{\hat{\be}_1}\hat\bu\cdot d\hat{\balpha}_1  
        +	\int_{\hat{\be}_6}\hat\bu\cdot d\hat\balpha_6 -
  		  	\int_{\hat{\be}_5}\hat\bu\cdot d\hat\balpha_5 \\[5pt]
  	& = & \iint_{\hat{f}_1} \nabla\times\hat\bu\cdot\hat\bn\,d\hat S \\[5pt]
  	& = & -\iint_{\hat{f}_1} \tfrac{{\partial} \hat u_1}{{\partial}\hat x_3}\,d\hat S.
\end{IEEEeqnarray*}
Next,
\begin{IEEEeqnarray*}{rCl}
	(\varphi_3-\varphi_1)(\hat\bu) & = & (\varphi_3-\varphi_2-\varphi_1+\varphi_4)(\hat\bu)\\
	& = & - \int_{\partial\hat{f}_5}\hat{\bu}\cdot d\hat{\balpha}\\[4pt]
	&=& -\iint_{\hat{f}_5}\nabla\times\hat{\bu}\cdot\hat{\bn}_5\,d\hat S\\
	&=&	 \iint_{\hat{f}_5}\tfrac{{\partial} \hat{u}_1}{{\partial} \hat{x}_2}\,d\hat S.
\end{IEEEeqnarray*}
And
\begin{IEEEeqnarray*}{rCl}
  (\varphi_6-\varphi_8)(\hat\bu) & = & \int_{\hat\be_6}\hat\bu\cdot d\hat\balpha_6 -
    \int_{\hat\be_8}\hat\bu\cdot\hat\btau_6\,d\hat s-
	\int_{\hat\be_2}\hat\bu\cdot\hat\btau_6\,d\hat s\\[5pt]
	& = &-\int_{\partial\hat{f}_3}\hat\bu\cdot\hat\btau\,d\hat s  
	  =  -\iint_{\hat{f}_3}\nabla\times\hat\bu\cdot\hat\bn_3\,d\hat S
	  =   \tfrac{1}{\sqrt{2}}\iint_{\hat{f}_3}\tfrac{{\s\partial} \hat u_1}{{\s\partial} \hat x_2}\,d\hat S.
\end{IEEEeqnarray*}
So in this case in which $\hat\bu$ has null first and second components, it holds
\begin{IEEEeqnarray}{rCl}\label{first_a}
	\nonumber
  (\wku)_1 & = & \int_{\hat\be_1}\hat u_1\,d\hat\alpha_1 + 
                z\iint_{\hat f_1} \tfrac{{\s\partial}\hat u_1}{{\s\partial} \hat x_3}\,d\hat S +
                y\iint_{\hat f_5} \tfrac{{\s\partial}\hat u_1}{{\s\partial} \hat x_2}\,d\hat S\\
           &&\,+\frac{yz}{1-z}\,2^{-\nicefrac12}\iint_{\hat f_3} \tfrac{{\s\partial}\hat u_1}{{\s\partial} \hat x_2}\,d\hat S.
\end{IEEEeqnarray}
By exactly the last computation,
\begin{IEEEeqnarray}{rCl}\label{second_a}
  (\wku)_2 & = & (\varphi_6-\varphi_8)(\hat\bu)\frac{xz}{1-z}
  = \iint_{\hat{f}_3}\tfrac{{\s\partial}\hat u_1}{{\s\partial} \hat x_2}\,d\hat S\,\frac{xz}{1-z}.
\end{IEEEeqnarray}
Next,
\begin{IEEEeqnarray*}{rCl}
	(\wku)_3 & = &     \varphi_1(\hat\bu)\left(x-\frac{xy}{1-z}\right) + \varphi_3(\hat\bu)\frac{xy}{1-z}\\[6pt]
			 &   &\,+\,\varphi_6(\hat\bu)\left(x-\frac{xy}{1-z}+\frac{xyz}{(1-z)^2}\right)
		            +  \varphi_8(\hat\bu)\left(\frac{xy}{1-z}-\frac{xyz}{(1-z)^2}\right)\\[6pt]
			 & = &  (\varphi_1 + \varphi_6)(\hat\bu)\,x +
			 		(\varphi_3-\varphi_1+\varphi_8-\varphi_6)(\hat\bu)\,\frac{xy}{1-z}\\[6pt]
			 &   &\,+ (\varphi_6-\varphi_8)(\hat\bu)\,\frac{xyz}{(1-z)^2}.
\end{IEEEeqnarray*}
As $\hat\bu$ has zero tangential component along $\hat\be_5$ and $\hat\be_7$,
\begin{IEEEeqnarray*}{rCl}
  (\varphi_3-\varphi_1+\varphi_8-\varphi_6)(\hat\bu)&=&
  (\varphi_3-\varphi_7+\varphi_8)(\hat\bu)+(\varphi_5-\varphi_6-\varphi_1)(\hat\bu)\\[8pt]
  &=&-\int_{\partial\hat{f}_4}\hat\bu\cdot d\hat\balpha
   -\int_{\partial\hat{f}_1}\hat\bu\cdot d\hat\balpha\\[8pt]
  &=&-\iint_{\hat{f}_1}\nabla\times\hat\bu\cdot\hat\bn\,d\hat S
   -\iint_{\hat{f}_4}\nabla\times\hat\bu\cdot\hat\bn\,d\hat S\\[8pt]
%\yesnumber\label{pyr_edge_one}
  &=&\iint_{\hat{f}_1}(\nabla\times\hat\bu)_2\,d\hat S\\[8pt]
  & &\quad-2^{-\nicefrac{1}{2}}\iint_{\hat{f}_4}[(\nabla\times\hat\bu)_2 + (\nabla\times\hat\bu)_3]\,d\hat S.\\[8pt]
  &=&\iint_{\hat{f}_1}\tfrac{\partial\hat{u}_1}{\partial\hat{x}_3}\,d\hat S
  -2^{-\nicefrac{1}{2}}\iint_{\hat{f}_4}[\tfrac{\partial\hat{u}_1}{\partial\hat{x}_3}
   + \tfrac{\partial\hat{u}_1}{\partial\hat{x}_2}]\,d\hat S.
\end{IEEEeqnarray*}
We write down this component:
\begin{IEEEeqnarray}{rCl}\label{third_a}
	\nonumber
  (\wku)_3 & = & 
    -x\,\iint_{\hat{f}_1} \tfrac{{\s\partial} \hat u_1}{{\s\partial} \hat x_3}\,d\hat S
    +\frac{xyz}{(1-z)^2}\,2^{-\nicefrac12}\iint_{\hat{f}_3}\tfrac{{\s\partial} \hat u_1}{{\s\partial} \hat x_2}\,d\hat S\\[8pt]
    &&\,+\frac{xy}{1-z}
     \left\{\iint_{\hat{f}_1}\tfrac{\partial\hat{u}_1}{\partial\hat{x}_3}\,d\hat S
      -2^{-\nicefrac{1}{2}}\iint_{\hat{f}_4}[\tfrac{\partial\hat{u}_1}{\partial\hat{x}_3}
     +\tfrac{\partial\hat{u}_1}{\partial\hat{x}_2}]\,d\hat S\right\}
\end{IEEEeqnarray}
\noindent Now, as expected, we switch to $\hat\bu = (0,\hat u_2,0)'$. In this case we have
\begin{IEEEeqnarray*}{rCl}
  \wku     & = & \varphi_2(\hat{\bu})\,\hat\bgamma_2 +
	\varphi_4(\hat{\bu})\,\hat\bgamma_4+ \varphi_7(\hat{\bu})\,\hat\bgamma_7+\varphi_8(\hat{\bu})\,\hat\bgamma_8.\\[4pt]
  (\wku)_1 & = &(\varphi_7-\varphi_8)(\hat{\bu})\,\frac{yz}{1-z}\\[4pt]
  		   & = &(\varphi_7-\varphi_8 - \varphi_3)(\hat{\bu})\,\frac{yz}{1-z}\\[4pt]
  		   & = &\int_{\partial\hat{f}_4}\hat{\bu}\cdot d\hat\balpha\,\frac{yz}{1-z}\\[4pt]
  		   \yesnumber\label{first_b}
  		   & = &\iint_{\hat{f}_4} \nabla\times\hat\bu\cdot\hat\bn_4\,d\hat S\,\frac{yz}{1-z}
  		  \, = \,2^{-\nicefrac12} \iint_{\hat{f}_4} \tfrac{\partial\hat{u}_2}{\partial\hat{x}_1}\,d\hat S\,\frac{yz}{1-z}.
\end{IEEEeqnarray*}
For the next component,
\begin{IEEEeqnarray*}{rCl}
	(\wku)_2 & = &\varphi_4(\hat\bu) + (\varphi_2-\varphi_4)(\hat\bu)\,x -
	(\varphi_4+\varphi_7)(\hat\bu)\,z\\[4pt]
  &&\,+\, (\varphi_7-\varphi_8)(\hat\bu)\,\frac{xz}{1-z}.\\[4pt]
	(\varphi_2-\varphi_4)(\hat\bu) & = & (\varphi_2-\varphi_3-\varphi_4+\varphi_1)(\hat\bu)\\[4pt]
  &=&-\int_{\partial\hat{f}_5}\hat\bu\cdot d\hat\balpha\\[4pt]
  &=&-\iint_{\hat{f}_5}\nabla\times\hat{\bu}\cdot\hat\bn_5\,d\hat S
   =  \iint_{\hat{f}_5}\tfrac{\partial\hat{u}_2}{\partial\hat{x}_1}\,d\hat S.\\[4pt]
  (\varphi_4+\varphi_7)(\hat\bu) & = & 
  (\varphi_4+\varphi_7-\varphi_5)(\hat\bu)\\[4pt]
  &=& - \int_{\partial\hat{f}_2} \hat\bu\cdot d\hat\balpha\\[4pt]
  &=& -\iint_{\hat{f}_2}\nabla\times\hat\bu\cdot\hat\bn\,d\hat S~=~
      -\iint_{\hat{f}_2}\tfrac{{\s\partial} \hat u_2}{{\s\partial} \hat x_3}\,d\hat S\mbox{,}
\end{IEEEeqnarray*}
and we write down this second component of the interpolate
\begin{IEEEeqnarray}{rCl}
  \nonumber
  (\wku)_2& = & \int_{\hat\be_4}\hat u_2\,d\hat\alpha_4
               +x\iint_{\hat{f}_5}\tfrac{\partial\hat{u}_2}{\partial\hat{x}_1}\,d\hat S.
               +z\iint_{\hat{f}_2}\tfrac{{\s\partial} \hat u_2}{{\s\partial} \hat x_3}\,d\hat S\\
\label{second_b}
&&\,+\frac{xz}{1-z}\,2^{-\nicefrac12} \iint_{\hat{f}_4} \tfrac{\partial\hat{u}_2}{\partial\hat{x}_1}\,d\hat S.
\end{IEEEeqnarray}
And for the third one,
\begin{IEEEeqnarray*}{rCl}
	(\wku)_3&=&(\varphi_4+\varphi_7)(\hat\bu)\,y + (\varphi_2-\varphi_4-\varphi_7+\varphi_8)(\hat\bu)\,\frac{xy}{1-z}\\[4pt]
	& &\,+\,(\varphi_7-\varphi_8)(\hat\bu)\,\frac{xyz}{(1-z)^2}.\\[8pt]
	&=&(\varphi_2 - \varphi_4)(\hat\bu)\,\frac{xy}{1-z} + (\varphi_4+\varphi_7)(\hat\bu)\,y\\[8pt]
	& &-(\varphi_7-\varphi_8)(\hat\bu)\,\frac{xy}{(1-z)^2}.
\end{IEEEeqnarray*}
But  expressions for $(\varphi_2 - \varphi_4)(\cdot)$, $(\varphi_4+\varphi_7)(\cdot)$ and
$(\varphi_7-\varphi_8)(\cdot)$ were already stated above, so we have
\begin{IEEEeqnarray}{rCl}
  \nonumber
  (\wku)_3 & = & 
\frac{xy}{1-z}\,\iint_{\hat{f}_5}\tfrac{\partial\hat{u}_2}{\partial\hat{x}_1}\,d\hat S 
-y\,\iint_{\hat{f}_2}\tfrac{{\s\partial} \hat u_2}{{\s\partial} \hat x_3}\,d\hat S\\[8pt]
  \label{third_b}
  & &-2^{-\nicefrac12} \iint_{\hat{f}_4} \tfrac{\partial\hat{u}_2}{\partial\hat{x}_1}\,d\hat S\,\frac{xy}{(1-z)^2}.
\end{IEEEeqnarray}
\noindent{Finally for $\hat\bu = (0,0,\hat u_3)'$} it is
$\wku = \varphi_5(\hat\bu)\hat\bgamma_5 + 
		\varphi_6(\hat\bu)\hat\bgamma_6 + 
		\varphi_7(\hat\bu)\hat\bgamma_7 +
		\varphi_8(\hat\bu)\hat\bgamma_8$ and $\nabla\times\hat\bu = (\partial_2\hat{u}_3,
		-\partial_1\hat{u}_3,0)'.$\\[7pt]
First component of the interpolate:
\begin{IEEEeqnarray*}{rCl}
	(\wku)_1 & = &(\varphi_5-\varphi_6)(\hat\bu)z+
		(-\varphi_5+\varphi_6+\varphi_7-\varphi_8)(\hat\bu)\,\frac{yz}{1-z}.
\end{IEEEeqnarray*}
On one hand,
\begin{IEEEeqnarray*}{rCl}
	(\varphi_5-\varphi_6)(\hat\bu) & = & (\varphi_5-\varphi_6-\varphi_1)(\hat\bu) \\
	&=&-\iint_{\hat f_1}\nabla\times\hat\bu\cdot\hat\bn\,d\hat S
\end{IEEEeqnarray*}
On the other hand and analogously
\begin{IEEEeqnarray*}{rCl} 	
	(\varphi_7-\varphi_8)(\hat\bu) & = &	\iint_{\hat{f}_4}\nabla\times\hat\bu\cdot\hat\bn\,d\hat S
\end{IEEEeqnarray*}
% &=&-\iint_{\hat f_1}\partial_1 \hat{u}_3\,d\hat S.
so it follows   %% \noindent{\color{blue} cambia lo de region tipo I y eso? estoy haciendo $0\leqslant t_2 \leqslant 1; 0\leqslant t_1\leqslant 1-t_2$} 
\begin{IEEEeqnarray*}{rCl}
  (-\varphi_5+\varphi_6+\varphi_7-\varphi_8)(\hat\bu) & = & 
  \iint_{\hat{f}_1}\nabla\times\hat\bu\cdot\hat\bn\,d\hat S
  +\iint_{\hat{f}_4}\nabla\times\hat\bu\cdot\hat\bn\,d\hat S\\[4pt]
\IEEEeqnarraymulticol{3}{R}{
\begin{IEEEeqnarraybox*}{rCl}
\qquad&=&
  \int_{\mathbb{D}_{\hat f_1}}\tfrac{{\s\partial} \hat u_3}{{\s\partial} \hat x_1}
  (\Phi_{\hat f_1}(t_1,t_2))\,dt_1dt_2
  -\int_{\mathbb{D}_{\hat f_4}}\tfrac{{\s\partial} \hat u_3}{{\s\partial} \hat x_1}
  (\Phi_{\hat f_4}(t_1,t_2))\,dt_1dt_2\\[4pt]
&=&
  \int_0^1\int_0^{1-t_2} 
  \left[\tfrac{{\s\partial} \hat u_3}{{\s\partial} \hat x_1}(t_1,0,t_2)
    - \tfrac{{\s\partial} \hat u_3}{{\s\partial} \hat x_1}(t_1,1-t_2,t_2)\right]
  \,dt_1dt_2\\[4pt]
&=&
  -\int_0^1\int_0^{1-t_2}\int_0^{1-t_2} 
  \tfrac{{\s\partial^2} \hat u_3}{{\s\partial} \hat x_2{\s\partial} \hat x_1}(t_1,s,t_2)
  \,dsdt_1dt_2\\[6pt]
&=&-\int_{\hat{E}}\tfrac{{\s\partial}^2\hat u_3}{{\s\partial} \hat x_2{\s\partial} \hat x_1}
\,d\hat\bx.
\end{IEEEeqnarraybox*}
}
\end{IEEEeqnarray*}
For now we obtained
\begin{IEEEeqnarray}{rCl}\label{first_c}
	(\wku)_1 & = & -z\iint_{\hat f_1}\tfrac{{\s\partial} \hat u_3}{{\s\partial} \hat x_1}\,d\hat S
	-\frac{yz}{1-z}\int_{\hat{E}}
		\tfrac{{\s\partial}^2\hat u_3}{{\s\partial} \hat x_2{\s\partial} \hat x_1}\,d\hat\bx.
\end{IEEEeqnarray}
Regarding the second component, it is the symmetrical case, so we write
\begin{IEEEeqnarray*}{rCl}
  (\wku)_2& = &(\varphi_5-\varphi_7)(\hat\bu)z
		+(-\varphi_5+\varphi_6+\varphi_7-\varphi_8)(\hat\bu)\frac{xz}{1-z}\\
	& = &z\iint_{\hat f_2}\nabla\times\hat\bu\cdot\hat\bn\,d\hat S
	-\frac{xz}{1-z}\int_{\hat{E}}
	\tfrac{{\s\partial}^2\hat u_3}{{\s\partial} \hat x_2{\s\partial} \hat x_1}\,d\hat\bx\\
	\yesnumber\label{second_c}
	& = &-z\iint_{\hat f_2}\tfrac{{\s\partial} \hat u_3}{{\s\partial} \hat{x_2}}\,d\hat S
		-\frac{xz}{1-z}\int_{\hat{E}}
	\tfrac{{\s\partial}^2\hat u_3}{{\s\partial} \hat x_2{\s\partial} \hat x_1}\,d\hat\bx.
\end{IEEEeqnarray*}
For the third component let us denote $\xi(x,y,z) = 
  \frac{xyz}{(1-z)^2}-\frac{xz}{1-z}$. Then
\begin{IEEEeqnarray*}{rCl}
  (\wku)_3& = & \varphi_5(\hat\bu) + x\,(\varphi_6-\varphi_5)(\hat\bu)+
  y\,(\varphi_7-\varphi_5)(\hat\bu)\\[5pt]
  && \,+\,\xi(x,y,z)\,(\varphi_6-\varphi_5+\varphi_7-\varphi_8)(\hat\bu)\\[5pt]
  &=& \int_{\hat\be_5}\hat\bu\cdot d\hat\balpha_5 + 
  y\,\iint_{\hat{f}_2}\tfrac{{\s\partial} \hat u_3}{{\s\partial} {x_2}}\,d\hat S +
  x\,\iint_{\hat{f}_1}\tfrac{{\s\partial} \hat u_3}{{\s\partial} {x_1}}\,d\hat S
  \\[5pt]&&\,-\,\xi(x,y,z)\int_{\hat{E}}
    \tfrac{{\s\partial}^2\hat u_3}{{\s\partial} \hat x_2{\s\partial} \hat x_1}\,d\hat\bx.
  \yesnumber\label{third_c}
\end{IEEEeqnarray*}
%\begin{IEEEeqnarray*}{rCl}
%	(\wku)_3 & = & \int_{0}^1u_3(0,0,t)\,dt
%				+ x \int_{0}^{1}\int_{0}^{1-t}
%						\tfrac{{\s\partial} \hat u_3}{{\s\partial} \hat x_1} (s,0,t) \,d\hat s\,dt
%				+ y \int_{0}^{1}\int_{0}^{1-t}
%						\tfrac{{\s\partial} \hat u_3}{{\s\partial} \hat x_2}(0,s,t) \,d\hat s\,dt\\
%				&&\,\xi\int_{\hat{E}}
%				\tfrac{{\s\partial}^2\hat u_3}{{\s\partial} \hat x_2{\s\partial} \hat x_1}\,d\hat\bx.
%\end{IEEEeqnarray*}
All together, for a $\hat\bu=(\hat u_1,\hat u_2,\hat u_3)'$, if we combine 
what was obtained in~(\ref{first_a})--(\ref{third_c}) 
then it holds
\begin{IEEEeqnarray*}{rCl}
  (\wku)_1 & = & 
    \int_{\hat\be_1}\hat u_1\,d\hat\alpha + 
  z \iint_{\hat f_1}(\nabla\times\hat\bu)_2\,d\hat S +
  y \iint_{{\hat f_5}}\tfrac{{\s\partial} \hat u_1}{{\s\partial} \hat x_2}\,d\hat S\\[6pt]
    &&\,
+\frac{yz}{1-z} \iint_{\hat f_3} \tfrac{{\s\partial} \hat u_1}{{\s\partial} \hat x_2}\,d\hat S +
 \frac{yz}{1-z} \iint_{\hat f_4} \tfrac{{\s\partial} \hat u_2}{{\s\partial} \hat x_1}\,d\hat S\\[6pt]
    &&\,
-\frac{yz}{1-z} \int_{\hat{E}}\tfrac{{\s\partial}^2\hat u_3}{{\s\partial} \hat x_2{\s\partial} \hat x_1}\,d\hat\bx.\\[12pt]
    (\wku)_2 & = & \int_{\hat\be_4}\hat u_2\,d\hat\alpha - 
    z \iint_{\hat f_2}(\nabla\times\hat\bu)_1\,d\hat S +
    x \iint_{\hat f_5}\tfrac{{\s\partial} \hat u_2}{{\s\partial} \hat x_1}\,d\hat S\\
    &&\,+\frac{xz}{1-z} \iint_{\hat f_3}
    \tfrac{{\s\partial} \hat u_1}{{\s\partial} \hat x_2}\,d\hat S +
    \frac{xz}{1-z} \iint_{\hat f_4}
    \tfrac{{\s\partial} \hat u_2}{{\s\partial} \hat x_1}\,d\hat S\\
    &&\,+\frac{xz}{1-z} \int_{\hat{E}}
    \tfrac{{\s\partial}^2\hat u_3}{{\s\partial} \hat x_2{\s\partial} \hat x_1}\,d\hat\bx.\\[12pt]
  (\wku)_3 & = & \int_{\hat\be_5}\hat u_3\,d\hat\alpha - 
    x \iint_{\hat{f}_1} (\nabla\times\hat\bu)_2\,d\hat S +
    y \iint_{\hat{f}_2} (\nabla\times\hat\bu)_1\,d\hat S\\[8pt]
  &&\,+\frac{xy}{1-z}
\left\{
  \iint_{\hat{f}_5}\tfrac{{\s\partial} \hat u_2}{{\s\partial} \hat x_1}\,d\hat S+
  \iint_{\hat{f}_1}\tfrac{{\s\partial} \hat u_1}{{\s\partial} \hat x_3}\,d\hat S-
  2^{-\nicefrac12}\iint_{\hat{f}_4}\tfrac{{\s\partial} \hat u_1}{{\s\partial} \hat x_3}\,d\hat S
\right.\\[8pt]
  &&\,-
\left.
  2^{-\nicefrac12}\iint_{\hat{f}_4}\tfrac{{\s\partial} \hat u_1}{{\s\partial} \hat x_2}\,d\hat S
\right\}-
\frac{xy}{(1-z)^2}
2^{-\nicefrac12}\iint_{\hat{f}_4}\tfrac{{\s\partial} \hat u_2}{{\s\partial} \hat x_1}\,d\hat S\\[8pt]
\yesnumber\label{aux_label42}
&&\,+
\frac{xyz}{(1-z)^2}
2^{-\nicefrac12}\iint_{\hat{f}_3}\tfrac{{\s\partial} \hat u_1}{{\s\partial} \hat x_2}\,d\hat S+
\xi(x,y,z)\,
\int_{\hat{E}}
  \tfrac{{\s\partial}^2 u_3}{{\s\partial} \hat x_1{\s\partial} \hat x_2}\,d\hat\bx.
\end{IEEEeqnarray*}
From here we apply Lemma~\ref{auxlabel350} and the result follows.
\end{proof}
We continue with the local interpolation error estimate.
\rescaledPyramidTikz
\begin{theorem} \label{auxlabel211}
  Let $E$ be any pyramid which is
  a non degenerate affine image 
  of the reference pyramid $\hat{E}$. We fix a positively oriented local system of 
  coordinates $(\bxi_1, \bxi_2, \bxi_3)$
  with origin in a vertex $\bx_E$ of the parallelogram basis, for which $(\bxi_1, \bxi_2)$
  correspond to the two basis edges incident to $\bx_E$ and $\bxi_3$ is parallel to the 
  edge joining $\bx_E$ with the top of the pyramid. Let $h_1, h_2, h_3$ be the corresponding 
  edge lengths. With $\partial^{\balpha}$ we denote 
  $\tfrac{\partial^{|\balpha|}}{\partial_{\bxi_1}^{\alpha_1}\partial_{\bxi_2}^{\alpha_2}\partial_{\bxi_3}^{\alpha_3}}$.
  Suppose that
  $h_3 \geqslant \min \{h_1, h_2\}$ and let  $p>2$.
  For all $\bu\in W^{2,p}(E)^3$
\begin{IEEEeqnarray*}{rCl}\label{aux_label55}
  \|\bu-\bw_E \bu\|_{L^p(E)} & \lesssim &
    \sum_{|{\balpha}|=1}\bh^{\balpha} \|\partial^{\balpha} \bu\|_{L^p(E)} +\\[4pt]
   &&\,+\,h_E\big\{\|\curl\bu\|_{\scriptscriptstyle L^p(E)} + 
   \sum_{|{\balpha}|=1}\bh^{\balpha}\|\partial^{\balpha} \curl \bu\|_{\scriptscriptstyle L^p(E)}\big\}\\
  & &\,+\,h_E^2 |\bu|_{2,p,E} \\[5pt]
  & &\,+\,\max \{h_{1}, h_2\} \big\{ \|\partial_{\tilde{x}_1}\tilde{u}_2\|_{\scriptscriptstyle L^p(\tilde E)}
   + \|\partial_{\tilde{x}_2}\tilde{u}_1\|_{\scriptscriptstyle L^p(\tilde E)}\big\}.
\end{IEEEeqnarray*} 
\end{theorem}
\begin{proof}
  Consider the matrix $M_{\tilde{E}}$ with coefficients $h_i\delta_{i,j}$, 
  $1\leqslant i,j\leqslant 3$ and take $\tilde{E}$ as the rescaled reference
  pyramid, that is, $\tilde{E} = M_{\tilde{E}}\hat{E}$. In 
  Figure~\ref{rescaled_pyramid} we have illustrated the scaling.
  Let us start with a stability estimate in $\tilde{E}$. Given a field $\tilde{\bu}$
  in $\tilde{E}$, pulling $\tilde{\bu}$ back to $\hat{E}$, using~\eqref{auxlabel203}
  and pushing forward to $\tilde{E}$ we get
  \begin{IEEEeqnarray*}{rCl}
    \|(\bw_{\tilde{E}}\tilde{\bu})_1\|_{\scriptscriptstyle L^\infty(\tilde{E})} &\lesssim&
      |\tilde{E}|^{-1/p}
      \big\{
        \|\tilde u_1\|_{\scriptscriptstyle L^p(\tilde E)} + 
          \sum_{i=1}^3 h_i \|\partial_{\tilde{x}_i}\tilde{u}_1\|_{\scriptscriptstyle L^p(\tilde{E})}
      \big\} \\[5pt]
    \IEEEeqnarraymulticol{3}{r}{+\,
      |\tilde{E}|^{\scriptscriptstyle -1} h_2
      \big\{
        \|(\nabla\times\tilde{\bu})_3\|_{\scriptscriptstyle L^1(\tilde{E})} + 
        \sum_{i=1}^3h_i(\|\partial_{\tilde{x}_i}(\nabla\times\tilde{\bu})_3\|_{\scriptscriptstyle L^1(\tilde{E})} +
                     \|\tfrac{\partial^2\tilde{u}_1}{\partial\tilde{x}_i\partial\tilde{x}_2}\|_{\scriptscriptstyle L^1(\tilde{E})})
      \big\}} \\[5pt]
    &&\,+\,
      |\tilde{E}|^{-1} h_3
      \big\{
        \|(\nabla\times\tilde{\bu})_2\|_{\scriptscriptstyle L^1(\tilde{E})} + 
             \sum_{i=1}^3h_i(\|\partial_{\tilde{x}_i}(\nabla\times\tilde{\bu})_2\|_{\scriptscriptstyle L^1(\tilde{E})}
      \big\} \\[5pt]
    &&\,+\,
      |\tilde{E}|^{-1} h_2h_3 \|\tfrac{\partial^2\tilde{u}_3}{\partial\tilde x_1\partial\tilde x_2}\|_{L^1(\tilde E)}.
  \end{IEEEeqnarray*}
  Estimate for component number two yields the analogue and now we write
  something similar to the third component. Note that in some cases we group terms
  using 
  \begin{IEEEeqnarray}{rCl}\label{auxlabel213}
  |\tilde E|^{-\tfrac{1}{q}}\|g\|_{\scriptscriptstyle L^q(\tilde E)} &\leqslant &
  |\tilde E|^{-\tfrac{1}{p}}\|g\|_{\scriptscriptstyle L^p(\tilde E)}\mbox{,}
  \end{IEEEeqnarray}
  whenever $q<p$, for scalar 
  functions
  in $L^p$. From~\eqref{auxlabel209}
  \begin{IEEEeqnarray*}{rCl}
    \|(\tilde\bw_{\tilde{E}}\tilde{\bu})_3\|_{L^\infty(\tilde{E})}&\lesssim&
    |\tilde{E}|^{-1/p}
    \big\{ 
      \|\tilde{u}_3\|_{\scriptscriptstyle L^p(\tilde{E})} + 
      \sum_{i=1}^3 h_i \|\tfrac{\partial\tilde{u}_3}{\partial\tilde{x}_i}\|_{\scriptscriptstyle L^p(\tilde{E})}
    \big\}\\[5pt]
    &&\,+\,|\tilde{E}|^{-1}h_1
    \big\{
      \|(\nabla\times\tilde{\bu})_2\|_{\scriptscriptstyle L^1(\tilde{E})}+
      \\[5pt]
    &&\,+
      \sum_{i=1}^3h_i
      (
        \|\partial_{\tilde{x}_i}(\nabla\times\tilde{\bu})_2\|_{\scriptscriptstyle L^1(\tilde{E})}+
        \|\tfrac{\partial^2\tilde{u}_3}{\partial\tilde{x}_i\partial\tilde{x}_1}\|_{\scriptscriptstyle L^1(\tilde{E})}
      )
    \big\}\\[5pt]
    \IEEEeqnarraymulticol{3}{r}{\,+\,|\tilde{E}|^{-1}h_2
        \big\{
        \|(\nabla\times\tilde{\bu})_1\|_{\scriptscriptstyle L^1(\tilde{E})} + 
                 \sum_{i=1}^3h_i(\|\partial_{\tilde{x}_i}(\nabla\times\tilde{\bu})_1\|_{\scriptscriptstyle L^1(\tilde{E})}
        \big\}}\\[5pt]
    &&  \,+\,|\tilde{E}|^{-1}\tfrac{h_1h_2}{h_3}
    \big\{
      \|\partial_{\tilde{x}_1}\tilde{u}_2\|_{\scriptscriptstyle L^1(\tilde{E})} + 
      \|\partial_{\tilde{x}_2}\tilde{u}_1\|_{\scriptscriptstyle L^1(\tilde{E})}\\[5pt]
    &&\,+ 
\sum_{i=1}^3 h_i
      (
        \|\tfrac{\partial^2\tilde{u}_1}{\partial\tilde{x}_i\partial\tilde{x}_2}\|_{\scriptscriptstyle L^1(\tilde{E})}+
        \|\tfrac{\partial^2\tilde{u}_2}{\partial\tilde{x}_i\partial\tilde{x}_1}\|_{\scriptscriptstyle L^1(\tilde{E})}
      )
    \big\}
    \\[5pt]
    &&\,+\,|\tilde{E}|^{-1}h_1h_2\|\tfrac{\partial^2\tilde{u}_3}{\partial\tilde{x}_1\partial\tilde{x}_2}\|_{\scriptscriptstyle L^1(\tilde{E})}.
  \end{IEEEeqnarray*}
Proceeding as in the proof of Theorem~\ref{aux_label27}
we obtain the following vectorial stability estimate in $\tilde E$:
\begin{IEEEeqnarray*}{rCl}
  \| \bw_{\tilde E}\tilde{\bu} \|_{\scriptscriptstyle L^p(\tilde E)}
  & \leqslant & \|\tilde{\bu}\|_{\scriptscriptstyle L^p(\tilde E)}
     + \sum_{i=1}^3 h_i\|\partial_{\tilde{x_i}}\tilde{\bu}\|_{\scriptscriptstyle L^p(\tilde E)} \\[5pt]  
  & &\,+\, \max \{h_{i}\} \left( \|\nabla\times\tilde{\bu}\|_{\scriptscriptstyle L^p(\tilde E)} + 
   \sum_{i=1}^3 h_i \|\partial_{\tilde{x}_i}\nabla\times\tilde{\bu}\|_{\scriptscriptstyle L^p(\tilde E)} \right) \\[5pt]
  & &\,+\, \max \{h_{i}\}^2 |\tilde{\bu}|_{2,p,\tilde{E}} \\[5pt]
  & &\,+\, \max \{h_{1}, h_2\} \big( \|\partial_{\tilde{x}_1}\tilde{u}_2\|_{\scriptscriptstyle L^p(\tilde E)}
   + \|\partial_{\tilde{x}_2}\tilde{u}_1\|_{\scriptscriptstyle L^p(\tilde E)} \big)
\end{IEEEeqnarray*}
And now we proceed as in the proof of Theorem~\ref{aux_label32}. First we 
transform from
a physical pyramidal element $E$ to $\tilde{E}$. In Section 5 of~\cite{gh99}
the approximation property of the finite element is stated and then we add the 
estimate~\eqref{aux_label30} for the rescaled pyramid $\tilde E$ in the case with 
multi--indices of order two,
to use in the corresponding terms of the averaged Taylor polynomial approximation
and the result follows.
\end{proof}













%%==========================================================================
%{\color{brown}
%    \begin{IEEEeqnarray*}{rCl}
%        (\wku)_1 & = & \int_{0}^{1}u_1(t,0,0)\,dt + 
%        z \int_0^1\int_0^{1-t_1}
%        \tfrac{{\s\partial} \hat u_1}{{\s\partial} \hat x_3}(t_1,0,t_2)\,dt_2dt_1 +
%        y \int_{{\hat f_5}}
%        \tfrac{{\s\partial} \hat u_1}{{\s\partial} \hat x_2}\,d\hat S\\
%        &&\,+\frac{yz}{1-z} \int_0^1\int_0^{1-t}
%        \tfrac{{\s\partial} \hat u_1}{{\s\partial} \hat x_2}(1-t,s,t)\,d\hat sdt +
%        \frac{yz}{1-z} \int_0^1\int_0^{1-t}
%        \tfrac{{\s\partial} \hat u_2}{{\s\partial} \hat x_1}(s,1-t,t)\,d\hat sdt\\
%        &&\,-z\int_0^1\int_0^{1-t_1}
%        \tfrac{{\s\partial} \hat u_3}{{\s\partial} \hat x_1}(t_1,0,t_2)\,dt_2dt_1 -
%        \frac{yz}{1-z} \int_{\hat{E}}
%        \tfrac{{\s\partial}^2\hat u_3}{{\s\partial} \hat x_2{\s\partial} \hat x_1}\,dV.
%    \end{IEEEeqnarray*}
%}
%%=========================================================================

%%===================================================================
%{\color{brown}
%\begin{IEEEeqnarray*}{rCl}
%    (\wku)_2 & = & \int_{0}^{1}u_2(0,t,0)\,dt + 
%    z \int_0^1\int_0^{1-t}
%    \tfrac{{\s\partial} \hat u_2}{{\s\partial} \hat x_3}(0,t,s)\,d\hat sdt +
%    x \int_0^1\int_0^{1}
%    \tfrac{{\s\partial} \hat u_2}{{\s\partial} \hat x_1}(s,t,0)\,d\hat sdt\\
%    &&\,+\frac{xz}{1-z} \int_0^1\int_0^{1-t}
%    \tfrac{{\s\partial} \hat u_1}{{\s\partial} \hat x_2}(1-t,s,t)\,d\hat sdt +
%    \frac{xz}{1-z} \int_0^1\int_0^{1-t}
%    \tfrac{{\s\partial} \hat u_2}{{\s\partial} \hat x_1}(s,1-t,t)\,d\hat sdt\\
%    &&\,-z\int_0^1\int_0^{1-t}
%    \tfrac{{\s\partial} \hat u_3}{{\s\partial} \hat x_2}(0,s,t)\,d\hat sdt +
%    \frac{xz}{1-z} \int_{\hat{P}}
%    \tfrac{{\s\partial}^2\hat u_3}{{\s\partial} \hat x_2{\s\partial} \hat x_1}\,dV.
%\end{IEEEeqnarray*}
%}
%%==================================================================

%%===========================================================================
%{\color{brown}
%\begin{IEEEeqnarray*}{rCl}
%        (\wku)_3 & = & \int_{0}^{1}u_3(0,0,t)\,dt + 
%        x \int_0^1\int_0^{1-t}
%        \tfrac{{\s\partial} \hat u_3}{{\s\partial} \hat x_1}(s,0,t)\,d\hat sdt -
%        x \int_0^1\int_0^{1-t}
%        \tfrac{{\s\partial} \hat u_1}{{\s\partial} \hat x_3}(t,0,s)\,d\hat sdt\\
%        &&\,+y \int_0^1\int_0^{1-t}
%        \tfrac{{\s\partial} \hat u_3}{{\s\partial} \hat x_2}(0,s,t)\,d\hat sdt -
%        y \int_0^1\int_0^{1-t}
%        \tfrac{{\s\partial} \hat u_2}{{\s\partial} \hat x_3}(0,t,s)\,d\hat sdt\\
%        &&\,+\frac{xy}{1-z} \int_0^1\int_t^{1}
%        \tfrac{{\s\partial} \hat u_1}{{\s\partial} \hat x_2}(t,s,0)\,d\hat sdt +
%        \frac{xy}{1-z} \int_0^1\int_0^{t}
%        \tfrac{{\s\partial} \hat u_2}{{\s\partial} \hat x_1}(t,s,1-t)\,d\hat sdt\\
%        &&\,+\frac{xyz}{(1-z)^2} \int_0^1\int_0^{1-t}
%        \tfrac{{\s\partial} \hat u_2}{{\s\partial} \hat x_1}(s,1-t,t)\,d\hat sdt
%        +\frac{xyz}{(1-z)^2} \int_0^1\int_0^{1-t}
%        \tfrac{{\s\partial} \hat u_1}{{\s\partial} \hat x_2}(1-t,s,t)\,d\hat sdt\\
%        &&\,-\frac{xy}{1-z}
%        \int_{0}^{1}
%        \int_{0}^{t}
%        \int_{0}^{1-t}
%        \tfrac{{\s\partial}^2u_1}{{\s\partial}x_2{\s\partial}x_3}(t,s,r)\,dr\,d\hat s\,dt
%        -\frac{xy}{1-z}
%        \int_{0}^{1}
%        \int_{0}^{t}
%        \int_{0}^{t}
%        \tfrac{{\s\partial}^2u_2}{{\s\partial}x_1{\s\partial}x_3}(r,s,1-t)\,dr\,d\hat s\,dt\\
%        &&\,
%        +\frac{xyz}{(1-z)^2} \int_{\hat{P}}
%        \tfrac{{\s\partial}^2 u_3}{{\s\partial} \hat x_1{\s\partial} \hat x_2}\,dV
%        -\frac{xz}{1-z} \int_{\hat{P}}
%        \tfrac{{\s\partial}^2 u_3}{{\s\partial} \hat x_1{\s\partial} \hat x_2}\,dV
%    \end{IEEEeqnarray*}
%    }
%%======================================================================================

% &=&-\int_0^1\int_0^{1-t}\tfrac{{\s\partial} \hat u_3}{{\s\partial} \hat x_1}(s,0,t)\,d\hat sdt.

%	(\pi\bu)_2 & = &\varphi_4 + (\varphi_2-\varphi_4)x -
%	(\varphi_4+\varphi_7)z + (\varphi_7-\varphi_8)\frac{xz}{1-z}.\\
%	\varphi_4 = \int_{0}^{1}u_2(0,t,0)\,dt\\
%	\varphi_2-\varphi_4 & = & \int_{0}^{1} u_2(1,t,0)-u_2(0,t,0)\,dt\\
%		&=&\int_{0}^{1}\int_{0}^{1}\tfrac{{\s\partial} \hat u_2}{{\s\partial} \hat x_1}(s,t,0)\,d\hat sdt\\
%	\varphi_4+\varphi_7 & = & \int_0^1 u_2(0,t,0)-u_2(0,t,1-t)\,dt\\
%		& = & \int_0^1\int_0^{1-t}\tfrac{{\s\partial} \hat u_2}{{\s\partial} \hat x_3}(0,t,s)\,d\hat sdt.\\
%	\varphi_7-\varphi_8&=&\int_{0}^{1} u_2(1-t,1-t,t)-u_2(0,1-t,t)\,dt\\
%		&=&\int_{0}^{1}\int_{0}^{1-t}\tfrac{{\s\partial} \hat u_2}{{\s\partial} \hat x_1}(s,1-t,t)\,d\hat sdt.

% subsection edge_elements (end)
\subsection{Anisotropic Stability Estimates for $H(\text{div})$--Conforming 
Elements on Pyramids} % (fold)
\label{sub:face_elements}
Here we will work on the div--conforming analogue of
Theorem~\ref{aux_label53}.
\begin{theorem} \label{aux_label54}
\begin{IEEEeqnarray*}{rCl}
  \|(\rku)_1\|_{\scriptscriptstyle{L^\infty(\hat{E})}}
  &\lesssim& \|\hat u_1\|_{\scriptscriptstyle{W^{1,p}(\hat{E})}} +
    \|\dv \hat\bu\|_{\scriptscriptstyle{L^p}(\hat{E})} + 
    \left\|\hat{u}_3\right\|_{\scriptscriptstyle{W^{1,p}}(\hat{E})}\\[12pt]
  \|(\rku)_2\|_{\scriptscriptstyle{L^\infty(\hat{E})}}
  &\lesssim& \|\hat u_2\|_{\scriptscriptstyle{W^{1,p}(\hat{E})}} +
    \|\dv \hat\bu\|_{\scriptscriptstyle{L^p}(\hat{E})} + 
    \left\|\hat{u}_3\right\|_{\scriptscriptstyle{W^{1,p}}(\hat{E})}\\[12pt]
  \|(\rku)_3\|_{\scriptscriptstyle{L^\infty(\hat{E})}} & \lesssim & 
    \|\hat u_3\|_{\scriptscriptstyle{W^{1,p}(\hat{E})}} +
    \|\dv \hat\bu\|_{\scriptscriptstyle{L^p}(\hat{E})}.
\end{IEEEeqnarray*}
\end{theorem}
\begin{proof}
We will use the notation of Table~\ref{shape_face_table} for the 
shape functions and Tables~\ref{pyramidNotationTableFaces} and~\ref{pyramidNotationTableEdges}
for the boundary of the reference pyramid. This proof is based on explicit computation as well.
The variables 
in the local coordinate system of $\hat E$ for the shape functions $\hat\bz_i$ 
are $x$, $y$ and $z$ instead
of $\hat x_1$, $\hat x_2$ and $\hat x_3$.\\[5pt]
Consider the case $\hat{\bu} = (\hat{u}_1,0,0)'$ to start with and compute it's 
interpolate. 
\begin{IEEEeqnarray*}{rCl}
  \rku & = & \{{\scriptstyle\iint_{\hat{f}_2} \hat\bu \cdot \hat\bn_2\,d\hat S}\}\,\hat\bz_2 + 
             \{{\scriptstyle\iint_{\hat{f}_3} \hat\bu \cdot \hat\bn_3\,d\hat S}\}\,\hat\bz_3\\[4pt]
       & =: & \rho_2(\hat\bu)\,\hat\bz_2 + \rho_3(\hat\bu)\,\hat\bz_3.
\end{IEEEeqnarray*}
Then for the first two components of the interpolate it holds
\begin{IEEEeqnarray*}{rCl}
  (\rku)_1(x,y,z) & = & -2\rho_2(\hat\bu) + 
    \{\rho_2(\hat\bu)+\rho_3(\hat\bu)\}\,\tfrac{2x-xz}{1-z}\\[4pt]
    & = & -2{\iint_{\hat{f}_2} \hat{\bu} \cdot \hat\bn_2\,d\hat S}\\[4pt]
    &&\, +\,\left\{
          {\iint_{\hat{f}_2} \hat{\bu} \cdot \hat\bn_2\,d\hat S}+
                  {\iint_{\hat{f}_3} \hat{\bu} \cdot \hat\bn_3\,d\hat S}\right\}
                  \tfrac{2x-xz}{1-z}\\[4pt]
    & = & -2{\iint_{\hat{f}_2} \hat{\bu} \cdot \hat\bn_2\,d\hat S} + 
          {\iint_{\partial\hat{E}} \hat{\bu} \cdot \hat\bn\,d\hat S}\,\tfrac{2x-xz}{1-z}\\[4pt]
    & = & -2{\iint_{\hat{f}_2} \hat{\bu} \cdot \hat\bn_2\,d\hat S} + 
            {\int_{\hat{E}} \dv\hat{\bu} \,d\hat{\boldsymbol{x}}}\,\tfrac{2x-xz}{1-z}
\end{IEEEeqnarray*}
and
\begin{IEEEeqnarray*}{rCl}
  (\rku)_2\xyz & = & -(\rho_2(\hat\bu)+\rho_3(\hat\bu))\,\tfrac{yz}{1-z}\\[4pt]
    & = & -{\int_{\hat{E}} \dv\hat{\bu} \,d\hat{\boldsymbol{x}}}\,\tfrac{yz}{1-z}.
\end{IEEEeqnarray*}
Switch to $\hat{\bu}$ of the form $(0,\hat{u}_2,0)'$.
\begin{IEEEeqnarray*}{rCl}
  \rku & = & ({\scriptstyle\iint_{\hat{f}_1} \hat\bu \cdot \hat\bn_1\,d\hat S})\,\hat\bz_1 + 
         ({\scriptstyle\iint_{\hat{f}_4} \hat\bu \cdot \hat\bn_4\,d\hat S})\,\hat\bz_4\\[4pt]
       & = & \rho_1(\hat\bu)\,\hat\bz_1 + \rho_4(\hat\bu)\,\hat\bz_4.
\end{IEEEeqnarray*}
Then summing up yields, for now,
\begin{IEEEeqnarray*}{rCl}
  (\rku)_1(x,y,z) & = & -(\rho_1(\hat\bu)+\rho_4(\hat\bu))\,\tfrac{xz}{1-z}\\[4pt]
    & = & -{\int_{\hat{E}} \dv\hat{\bu} \,d\hat{\boldsymbol{x}}}\,\tfrac{xz}{1-z}.\\[8pt]
  (\rku)_2(x,y,z) & = & -2\rho_1(\hat\bu) + 
  (\rho_1(\hat\bu)+\rho_4(\hat\bu))\,\tfrac{2y-yz}{1-z}\\[4pt]
    & = & -2{\iint_{\hat{f}_1} \hat{\bu} \cdot \hat\bn_1\,d\hat S} + 
            {\iint_{\partial\hat{E}} \hat{\bu} \cdot \hat\bn\,d\hat S}\,\tfrac{2y-yz}{1-z}\\[4pt]
    & = & -2{\iint_{\hat{f}_1} \hat{\bu} \cdot \hat\bn_1\,d\hat S} + 
            {\int_{\hat{E}} \dv\hat{\bu} \,d\hat{\boldsymbol{x}}}\,\tfrac{2y-yz}{1-z}.\\[8pt]
\end{IEEEeqnarray*}
Now continue with $\hat{\bu}$ of the form $(0,0,\hat{u}_3)'$.
\begin{IEEEeqnarray*}{rCl}
  \rku & = & ({\scriptstyle\iint_{\hat{f}_3} \hat\bu \cdot \hat\bn_3\,d\hat S})\,\hat\bz_3 + 
         ({\scriptstyle\iint_{\hat{f}_4} \hat\bu \cdot \hat\bn_4\,d\hat S})\,\hat\bz_4 + 
         ({\scriptstyle\iint_{\hat{f}_5} \hat\bu \cdot \hat\bn_5\,d\hat S})\,\hat\bz_5\\[4pt]
       & =: & \rho_3(\hat\bu)\,\hat\bz_3 + \rho_4(\hat\bu)\,\hat\bz_4
       + \rho_5(\hat\bu)\,\hat\bz_5.
\end{IEEEeqnarray*}
Then
\begin{IEEEeqnarray*}{rCl}
  (\rku)_1(x,y,z) & = & \{\rho_3(\hat\bu) + \rho_5(\hat\bu)\}\,x
  + \rho_3(\hat\bu) \tfrac{x}{1-z} - \rho_4\tfrac{xz}{1-z}.
\end{IEEEeqnarray*}
Now observe that
\begin{IEEEeqnarray*}{rCl}
  (\rho_3 + \rho_5)(\hat\bu) & = & 
    {\iint_{\partial\hat{E}} \hat{\bu} \cdot \hat\bn\,d\hat S} - 
      {\iint_{\hat{f}_4} \hat{\bu} \cdot \hat\bn_4\,d\hat S} \\[4pt]
  & = & {\int_{\hat{E}} \dv\hat{\bu}\,d\hat{\boldsymbol{x}}} - 
        \rho_4(\hat{\bu})
\end{IEEEeqnarray*}
and, on the other hand,
\begin{IEEEeqnarray*}{rCl}
  (\rho_3-\rho_4)(\hat\bu) & = & 
  {\iint_{\hat{f}_3} \hat\bu \cdot \hat\bn_3\,d\hat S} - 
  {\iint_{\hat{f}_4} \hat\bu \cdot \hat\bn_4\,d\hat S} \\[4pt]
  & = & \int_{0}^{1}\int_{0}^{x} \hat{u}_3(x,y,1-x)\,dydx - 
        \int_{0}^{1}\int_{0}^{y} \hat{u}_3(x,y,1-y)\,dxdy\mbox{,}
\end{IEEEeqnarray*}
so
\begin{IEEEeqnarray*}{rCl}
  (\rku)_1(x,y,z) & = & {x\int_{\hat{E}} \dv\hat{\bu}\,d\hat{\boldsymbol{x}}}\,+\\[4pt]
  \IEEEeqnarraymulticol{3}{r}{
    \qquad\left\{\int_{0}^{1}\int_{0}^{x} \hat{u}_3(x,y,1-x)\,dydx - 
    \int_{0}^{1}\int_{0}^{y} \hat{u}_3(x,y,1-y)\,dxdy\right\}\,
    \tfrac{x}{1-z}.}
\end{IEEEeqnarray*}
In a completely similar fashion we arrive at
\begin{IEEEeqnarray*}{rCl}
  (\rku)_2(x,y,z) & = & y\,{\int_{\hat{E}} \dv\hat{\bu}\,d\hat{\boldsymbol{x}}}\,+
  \left\{{\iint_{\hat{f}_3} \hat\bu \cdot \hat\bn_3\,d\hat S} - 
   {\iint_{\hat{f}_4} \hat\bu \cdot \hat\bn_4\,d\hat S}\right\}\,\tfrac{y}{1-z}.\\[4pt]
               & = & y\,{\int_{\hat{E}} \dv\hat{\bu}\,d\hat{\boldsymbol{x}}}\,+\\[4pt]
  \IEEEeqnarraymulticol{3}{r}{
    \qquad\left\{\int_{0}^{1}\int_{0}^{x} \hat{u}_3(x,y,1-x)\,dydx - 
    \int_{0}^{1}\int_{0}^{y} \hat{u}_3(x,y,1-y)\,dxdy\right\}
    \tfrac{y}{1-z}.}
\end{IEEEeqnarray*}
We collect every term obtained so far for the first and second components in
Table~\ref{terms_table}.
\begin{table}[!h]
    \centering  
    \caption{Terms\\[4pt]$q(s,t) = \tfrac{2s-st}{1-t},\,r(s,t) = \tfrac{st}{1-t}$}
    \label{terms_table}
    \begin{IEEEeqnarraybox*}
    [\IEEEeqnarraystrutmode
    \IEEEeqnarraystrutsizeadd{2pt}{12pt}]{v/c/v/c/v/c/v/}
        \IEEEeqnarrayrulerow\\
        \IEEEeqnarrayseprow[5pt]\\
        & & & (\rku)_1 & & (\rku)_2 & \\
        \IEEEeqnarrayrulerow\\
        \IEEEeqnarrayseprow[5pt]\\
        & (\hat{u}_1,0,0)' & &
          \begin{IEEEeqnarraybox*}{l}
            -2{\iint_{\hat{f}_2} \hat{\bu} \cdot \hat\bn_2\,d\hat S}\\ + 
            {q(x,z)\int_{\hat{E}} \dv\hat{\bu} \,d\hat{\boldsymbol{x}}}
          \end{IEEEeqnarraybox*}
        & &
          -r(y,z){\int_{\hat{E}} \dv\hat{\bu} \,d\hat{\boldsymbol{x}}} &\\
        \IEEEeqnarrayrulerow\\
        \IEEEeqnarrayseprow[5pt]\\
        & (0,\hat{u}_2,0)' & & 
          -r(x,z){\int_{\hat{E}} \dv\hat{\bu} \,d\hat{\boldsymbol{x}}} 
        & & 
          \begin{IEEEeqnarraybox*}{l}
            -2{\iint_{\hat{f}_1} \hat{\bu} \cdot \hat\bn_1\,d\hat S}\\ + 
            {q(x,z)\int_{\hat{E}} \dv\hat{\bu} \,d\hat{\boldsymbol{x}}}
          \end{IEEEeqnarraybox*}
        &\\
        \IEEEeqnarrayrulerow\\
        \IEEEeqnarrayseprow[5pt]\\
        & (0,0,\hat{u}_3)' & & 
          \begin{IEEEeqnarraybox*}{l}
            x\int_{\hat{E}} \dv\hat{\bu}\,d\hat{\boldsymbol{x}} \\[5pt] +\, 
            \left\{\iint_{\hat{f}_3} \hat\bu \cdot \hat\bn_3\,d\hat S\right.
             \\[5pt] 
            \left. -\iint_{\hat{f}_4} \hat\bu \cdot \hat\bn_4\,d\hat S\right\}r(x,z)
          \end{IEEEeqnarraybox*}
         & & 
          \begin{IEEEeqnarraybox*}{l}
            y\int_{\hat{E}} \dv\hat{\bu}\,d\hat{\boldsymbol{x}}\\[5pt] +\, 
              \left\{\iint_{\hat{f}_4} \hat\bu \cdot \hat\bn_4\,d\hat S\right.
             \\[5pt] 
             \left.-\iint_{\hat{f}_3} \hat\bu \cdot \hat\bn_3\,d\hat S\right\}r(y,z)
          \end{IEEEeqnarraybox*}
        &\\\IEEEeqnarrayrulerow
    \end{IEEEeqnarraybox*}
\end{table}

Lastly, the third component of $\rku$ can be treated at once for any 
field $\hat\bu$ as follows:
\begin{IEEEeqnarray*}{rCl}
    (\rku)_3(x,y,z) & = &   z\sum_{i=1}^4\iint_{\hat{f}_i} \hat{\bu}\cdot\hat{\bn}_i\,d\hat S
                      + (z-1) \iint_{\hat{f}_5}\hat{\bu}\cdot\hat{\bn}_5\,d\hat S\\[5pt]
                    & = & z\iint_{\partial\hat{E}} \hat{\bu}\cdot\hat{\bn} - \iint_{f_5}
                 \hat{\bu}\cdot\hat{\bn}_5\,d\hat S\\[5pt]
    \yesnumber\label{term_rk3}
                    & = &\hat{x}_3\int_{\hat{E}} \mbox{div}\,\hat{\bu}\,d\hat{\bx} 
     + \iint_{\hat{f}_5} \hat{u}_3\,d\hat{S}.
\end{IEEEeqnarray*}
Now we bound each term in Table~\ref{terms_table} and in expression~(\ref{term_rk3}).
\begin{IEEEeqnarray*}{rCl}
  (\rku)_1 & = & (\br_{\hat{E}}(\hat{u}_1,0,0)')_1 + 
                 (\br_{\hat{E}}(0,\hat{u}_2,0)')_1 + 
                 (\br_{\hat{E}}(0,0,\hat{u}_3)')_1\\[5pt]
           & = & -2\iint_{\hat{f}_2}\hat{u}_1\,d\hat S +
  \tfrac{2x}{1-z}\int_{\hat{E}} \tfrac{\partial\hat{u}_1}{\partial\hat{x}_1}\,d\hat{\bx} -
  \tfrac{xz}{1-z}\int_{\hat{E}} \tfrac{\partial\hat{u}_1}{\partial\hat{x}_1}\,d\hat{\bx} \\[5pt]
  & & \,- \tfrac{xz}{1-z}\int_{\hat{E}} \tfrac{\partial\hat{u}_2}{\partial\hat{x}_2}\,d\hat{\bx}
  + \left( x + \tfrac{xz}{1-z}-\tfrac{xz}{1-z} \right)
  \int_{\hat{E}} \tfrac{\partial\hat{u}_3}{\partial\hat{x}_3}\,d\hat{\bx}\\[5pt]
  &   &\, + \left({\iint_{\hat{f}_3} \hat{u}_3\,d\hat S}
        - {\iint_{\hat{f}_4} \hat{u}_3\,d\hat S}\right)\tfrac{xz}{1-z}\\[5pt]
           & = & -2\iint_{\hat{f}_2}\hat{u}_1\,d\hat S +
  \tfrac{2x}{1-z}\int_{\hat{E}}\tfrac{\partial\hat{u}_1}{\partial\hat{x}_1}\,d\hat{\bx} -
  \tfrac{xz}{1-z}\int_{\hat{E}}\dv\hat{\bu}\,d\hat{\bx}\\[5pt]
  \yesnumber\label{face_integrals}
  &  & \,+\tfrac{x}{1-z}\int_{\hat{E}}\tfrac{\partial\hat{u}_3}{\partial\hat{x}_3}\,d\hat{\bx}
  + \left({\iint_{\hat{f}_3} \hat{u}_3\,d\hat S}
        - {\iint_{\hat{f}_4} \hat{u}_3\,d\hat S}\right)\tfrac{xz}{1-z}.
\end{IEEEeqnarray*}
For the surface integrals in~(\ref{face_integrals}), by Lemma 5.15 in~\cite{monk}, page 120,
\begin{IEEEeqnarray*}{rCl}
\left|\iint_{\hat{f}_2} \hat{u}_1\,d\hat S\right| 
  & \leqslant & C\,\|\hat{u}_1\|_{H^{1}(\hat{E})}
\end{IEEEeqnarray*}
and similarly
\begin{IEEEeqnarray*}{rCl}
  \left|\iint_{\hat{f}_3} \hat{u}_3\,d\hat S - \iint_{\hat{f}_4} \hat{u}_3\,d\hat S\right| 
  & \leqslant & C\,\|\hat{u}_3\|_{H^{1}(\hat{E})}
\end{IEEEeqnarray*}
all of which leads to
\begin{IEEEeqnarray*}{rCl}
  \|(\rku)_1\|_{L^{\infty}(\hat{E})} & \leqslant & C_{\hat{E}} 
  \left[ 
    \|\hat{u}_1\|_{H^{1}(\hat{E})} + 
    \|\dv\hat{\bu}\|_{L^{2}(\hat{E})} + 
    \|\hat{u}_3\|_{H^{1}(\hat{E})}
  \right].
\end{IEEEeqnarray*}
Copying the argument for the second component
\begin{IEEEeqnarray*}{rCl}
  \|(\rku)_2\|_{L^{\infty}(\hat{E})} & \leqslant & C_{\hat{E}} 
  \left[ 
    \|\hat{u}_2\|_{H^{1}(\hat{E})} + 
    \|\dv\hat{\bu}\|_{L^{2}(\hat{E})} + 
    \|\hat{u}_3\|_{H^{1}(\hat{E})}
  \right].
\end{IEEEeqnarray*}
Finally from~(\ref{term_rk3}) we deduce
\begin{IEEEeqnarray*}{rCl}
  \|(\rku)_3\|_{\scriptscriptstyle{L^\infty(\hat{E})}} & \leqslant & C_{\hat{E}}
    \left[\|\hat u_3\|_{\scriptscriptstyle{H^{1}(\hat{E})}} +
    \|\dv \hat\bu\|_{\scriptscriptstyle{L^2}(\hat{E})}\right].
\end{IEEEeqnarray*}
The quantity $C_{\hat{E}}$ depends only on the supremum of the (fixed)
basis shape functions of Table~\ref{shape_face_table} over the pyramid.
\end{proof}
% subsection face_elements (end)

%% ============================================================================
%% TODO: ver si esto finalmente va
%% \subsection{Local Interpolation Estimates for Pyramidal Finite Elements} % (fold)
%% \label{sub:local_interpolation_estimates_for_pyramidal_elements}
%% decir que permitimos pirámides elongadas perpendicularmente a la base
%% $h_3\geqslant C\min\{h_1,h_2\}$
%% Verificar si es esto o $h3 >= max (h1, h2)$\\
%% poner tres dibujos con casos $h1=h2<h3$; $h1<h2<h3$; $h2<h1<h3$
% subsection local_interpolation_estimates_for_pyramidal_elements (end)
%% ============================================================================


% section pyramidal_finite_elements (end)