\section{Pyramidal Finite Elements} % (fold)
\label{sec:pyramidal_finite_elements}
$\hat{E}$ will be the reference pyramid  in Figure~\ref{reference_pyramid}.
Anisotropic interpolation error estimates for pyramidal $\bcurl$--conforming
and div--conforming finite elements of least order will we established.
\subsection{Anisotropic Stability Estimates for $H(\bcurl)$--Conforming 
Elements on Pyramids} % (fold)
\label{sub:edge_elements}
The proof is based on mere calculation.
Recall the shape funtions in Table~(\ref{shape_edge_table}).
Start with $\bu$ of the form $(u_1,0,0)'$. After computations we have
\begin{IEEEeqnarray*}{rCl}
	\nabla\times\bu &=& (0, \frac{{\s\partial} u_1}{{\s\partial} x_3},-\frac{{\s\partial} u_1}{{\s\partial} x_2})\\[5pt]
	\wku	&=& [{\s\int_{\hat{\be}_1}\bu\cdot\btau\, ds}]\bgamma_1 +
				[{\s\int_{\hat{\be}_3}\bu\cdot\btau\, ds}]\bgamma_3 + 
				[{\s\int_{\hat{\be}_6}\bu\cdot\btau\, ds}]\bgamma_6 + 
				[{\s\int_{\hat{\be}_8}\bu\cdot\btau\, ds}]\bgamma_8\\[5pt]
			&=& \alpha_1(\hat\bu)\hat\bgamma_1 + 
				\alpha_3(\hat\bu)\hat\bgamma_3 + 
				\alpha_6(\hat\bu)\hat\bgamma_6 + 
				\alpha_8(\hat\bu)\hat\bgamma_8
\end{IEEEeqnarray*}
Now, $\hat\bu$ has zero tangential component along $\hat\be_2$, so then
\begin{IEEEeqnarray*}{rCl}
	(\wku)_2 		  & = &\left(\alpha_6(\hat\bu)-\alpha_8(\hat\bu)\right)\frac{xz}{1-z} \\[5pt]
					  & = &\left(\int_{\hat\be_6}\hat\bu\cdot\hat\btau_6\,ds-
					  			 \int_{\hat\be_8}\hat\bu\cdot\hat\btau_6\,ds-
					  			 \int_{\hat\be_2}\hat\bu\cdot\hat\btau_6\,ds\right)\frac{xz}{1-z} \\[5pt]
					  & = &-\int_{\partial\hat{f}_3}\hat\bu\cdot\hat\btau\,ds\,\frac{xz}{1-z} \\[5pt]
					  & = &-\int_{\hat{f}_3}\nabla\times\hat\bu\cdot\bn\,d\gamma\,\frac{xz}{1-z} \\[5pt]
					  & = &\int_{\hat{f}_3}\frac{{\s\partial} \hat u_1}{{\s\partial} x_2}\,d\gamma\,\frac{xz}{1-z}
\end{IEEEeqnarray*}
by Stoke's theorem. Next,
\begin{IEEEeqnarray*}{rCl}
	(\wku)_3 & = &     \alpha_1(\hat\bu)\left(x-\frac{xy}{1-z}\right) + \alpha_3(\hat\bu)\frac{xy}{1-z}\\[6pt]
			 &   &\,+\,\alpha_6(\hat\bu)\left(x-\frac{xy}{1-z}+\frac{xyz}{(1-z)^2}\right)
		            +  \alpha_8(\hat\bu)\left(\frac{xy}{1-z}-\frac{xyz}{(1-z)^2}\right)\\[6pt]
			 & = &  \left(\alpha_1(\hat\bu) + \alpha_6(\hat\bu)\right)x +
			 		\left(\alpha_3(\hat\bu)-\alpha_1(\hat\bu)+\alpha_8(\hat\bu)-\alpha_6(\hat\bu)\right)\frac{xy}{1-z}\\[6pt]
			 &   &\,+ \left(\alpha_6(\hat\bu)-\alpha_8(\hat\bu)\right) \frac{xyz}{(1-z)^2}.
\end{IEEEeqnarray*}
Now we explore the new coefficients separately. Because $\hat\bu\cdot\hat\be_5$ equals zero,
\begin{IEEEeqnarray*}{rCl}
  \alpha_1(\hat\bu)+\alpha_6(\hat\bu) & = & \int_{\hat{\be}_1}\hat\bu\cdot\hat\btau_1\, ds +
  											\int_{\hat{\be}_6}\hat\bu\cdot\hat\btau_6\, ds -
  											\int_{\hat{\be}_5}\hat\bu\cdot\hat\btau_5\, ds \\[5pt]
  									  & = & \int_{\hat{f}_1} \nabla\times\hat\bu\cdot\hat\bn\,d\gamma \\[5pt]
  									  & = & -\int_{\hat{f}_1} \frac{{\s\partial} u_1}{{\s\partial} x_3}\,d\gamma.
\end{IEEEeqnarray*}
{\color{blue}\#\#\#\#\#\#\#\# continue here.}
\begin{IEEEeqnarray*}{rCl}
	\alpha_1 &=& \int\limits_{0}^{1}u_1(t,0,0)\,dt\\
	\alpha_3-\alpha_1+\alpha_8-\alpha_6&=&
	\int\limits_0^1u_1(t,1,0)\,dt-\int\limits_0^1u_1(t,0,0)\,dt-
	\int\limits_0^1u_1(1-t,1-t,t)\,dt+\int\limits_0^1u_1(1-t,0,t)\,dt\\
	&=&\int\limits_0^1\int\limits_0^1\frac{{\s\partial} u_1}{{\s\partial} x_2}(t,s,0)\,dsdt-
	\int\limits_0^1\int\limits_0^{1-t}\frac{{\s\partial} u_1}{{\s\partial} x_2}(1-t,s,t)\,dsdt\\
	&=&\int\limits_0^1\int\limits_0^1\frac{{\s\partial} u_1}{{\s\partial} x_2}(t,s,0)\,dsdt-
	\int\limits_0^1\int\limits_0^{t}\frac{{\s\partial} u_1}{{\s\partial} x_2}(t,s,1-t)\,dsdt\\
	&=&-\int\limits_0^1\int\limits_0^t\int\limits_0^{1-t}
		\partial^2_{x_3,x_2}u_1(t,s,r)\,drdsdt + 
	\int\limits_0^1\int\limits_t^{1}\frac{{\s\partial} u_1}{{\s\partial} x_2}(t,s,0)\,dsdt
\end{IEEEeqnarray*}

\begin{IEEEeqnarray*}{rCl}
	(\pi\bu)_1 & = & \alpha_1(1-z-y)+ 
			  		\alpha_3y+ 
			  		\alpha_6(-z+\frac{yz}{1-z})+ 
			  		\alpha_8(-\frac{yz}{1-z})\\
			  & = & \alpha_1 - (\alpha_1 + \alpha_6)z+ 
			  		(\alpha_3 - \alpha_1)y + (\alpha_6-\alpha_8)\frac{yz}{1-z}.
\end{IEEEeqnarray*}
\begin{IEEEeqnarray*}{rCl}
	\alpha_3-\alpha_1&=& \int\limits_{0}^{1}\int\limits_{0}^{1}\frac{{\s\partial} u_1}{{\s\partial} x_2}(t,s,0)\,dsdt
\end{IEEEeqnarray*}
\centerline{\rule{100pt}{0.5pt}}\newpage
\begin{IEEEeqnarray*}{rCl}
	\bu & = & (0,u_2,0)\\
	\nabla\times\bu &=& (-\frac{{\s\partial} u_2}{{\s\partial} x_3},0,\frac{{\s\partial} u_2}{{\s\partial} x_1})\\
	\pi\bu &=& \alpha_2\gamma_2 + \alpha_4\gamma_4+ \alpha_7\gamma_7+\alpha_8\gamma_8\\
	(\pi\bu)_1 & = &(\alpha_7-\alpha_8)\frac{yz}{1-z}\\
	\alpha_7-\alpha_8 & = & -\int\limits_0^1 u_2(0,1-t,t)\,dt
							+\int\limits_0^1 u_2(1-t,1-t,t)\,dt\\
					& = & \int\limits_0^1\int\limits_0^{1-t}\frac{{\s\partial} u_2}{{\s\partial} x_1}(s,1-t,t)\,dsdt.\\
	(\pi\bu)_2 & = &\alpha_4 + (\alpha_2-\alpha_4)x -
	(\alpha_4+\alpha_7)z + (\alpha_7-\alpha_8)\frac{xz}{1-z}.\\
	\alpha_4 & = & \int\limits_{0}^{1}u_2(0,t,0)\,dt\\
	\alpha_2-\alpha_4 & = & \int\limits_{0}^{1} u_2(1,t,0)-u_2(0,t,0)\,dt\\
		&=&\int\limits_{0}^{1}\int\limits_{0}^{1}\frac{{\s\partial} u_2}{{\s\partial} x_1}(s,t,0)\,dsdt\\
	\alpha_4+\alpha_7 & = & \int\limits_0^1 u_2(0,t,0)-u_2(0,t,1-t)\,dt\\
		& = & \int\limits_0^1\int\limits_0^{1-t}\frac{{\s\partial} u_2}{{\s\partial} x_3}(0,t,s)\,dsdt.\\
	\alpha_7-\alpha_8&=&\int\limits_{0}^{1} u_2(1-t,1-t,t)-u_2(0,1-t,t)\,dt\\
		&=&\int\limits_{0}^{1}\int\limits_{0}^{1-t}\frac{{\s\partial} u_2}{{\s\partial} x_1}(s,1-t,t)\,dsdt.
\end{IEEEeqnarray*}
\begin{IEEEeqnarray*}{rCl}
	(\pi\bu)_3&=&(\alpha_4+\alpha_7)y + (\alpha_2-\alpha_4-\alpha_7+\alpha_8)\frac{xy}{1-z}
	+(\alpha_7-\alpha_8)\frac{xyz}{(1-z)^2}.\\
	\alpha_4+\alpha_7 &=&-\int\limits_{0}^{1}\int\limits_{0}^{1-t}\frac{{\s\partial} u_2}{{\s\partial} x_3}(0,t,s)\,dsdt\\
	\alpha_7-\alpha_8 &=&\int\limits_{0}^{1}\int\limits_{0}^{1-t}\frac{{\s\partial} u_2}{{\s\partial} x_1}(s,1-t,t)\,dsdt.
\end{IEEEeqnarray*}
\begin{IEEEeqnarray*}{rCl}
	\alpha_2+\alpha_8&=&\int\limits_{e_2}\bu\cdot\btau+\int\limits_{e_8}\bu\cdot\btau
	+\int\limits_{e_6}\bu\cdot\btau\\
	&=&\iint\limits_{[a_2,a_4,a_5]}\nabla\times\bu\cdot\bn\,dS\\
	&=&\int\limits_{0}^{1}\int\limits_{0}^{t}-\frac{{\s\partial} u_2}{{\s\partial} x_3}(t,s,1-t) + \frac{{\s\partial} u_2}{{\s\partial} x_1}(t,s,1-t)\,dsdt.
\end{IEEEeqnarray*}
On the other hand, since $\bu$ is orthogonal to the tangent to $e_5$:
\begin{IEEEeqnarray*}{rCl}
	\alpha_4+\alpha_7&=&
		\int\limits_{e_4}\bu\cdot\btau+
		\int\limits_{e_7}\bu\cdot\btau+
		\int\limits_{e_5}\bu\cdot\btau\\[5pt]
		&=&\iint\limits_{[a_1,a_3,a_5]}\nabla\times\bu\cdot\bn\,dS\\[5pt]
		&=&-\int\limits_{0}^{1}\int\limits_{0}^{1-t}\frac{{\s\partial} u_2}{{\s\partial} x_3}(0,t,s)\,dsdt\\[5pt]
		&=&-\int\limits_{0}^{1}\int\limits_{t}^{1}\frac{{\s\partial} u_2}{{\s\partial} x_3}(0,t,1-s)\,dsdt\\[5pt]
		&=&-\int\limits_{0}^{1}\int\limits_{0}^{s}\frac{{\s\partial} u_2}{{\s\partial} x_3}(0,t,1-s)\,dtds\\[5pt]
		&=&-\int\limits_{0}^{1}\int\limits_{0}^{t}\frac{{\s\partial} u_2}{{\s\partial} x_3}(0,s,1-t)\,dsdt
\end{IEEEeqnarray*}
So
\begin{IEEEeqnarray*}{rCl}
	\alpha_2-\alpha_4-\alpha_7+\alpha_8 & = &
	\int\limits_{0}^{1}\int\limits_{0}^{t}
	-\frac{{\s\partial} u_2}{{\s\partial} x_3}(t,s,1-t) + \frac{{\s\partial} u_2}{{\s\partial} x_3}(0,s,1-t)\,dsdt\\
	&&\,+\int\limits_{0}^{1}\int\limits_{0}^{t}\frac{{\s\partial} u_2}{{\s\partial} x_1}(t,s,1-t)\,dsdt\\
	&=&\int\limits_{0}^{1}
	\int\limits_{0}^{t}
	\int\limits_{0}^{t}\frac{\partial^2u_2}{\partial x_1\partial x_3}(r,s,1-t)
	\,dr\,ds\,dt+
	\int\limits_{0}^{1}\int\limits_{0}^{t}\frac{{\s\partial} u_2}{{\s\partial} x_1}(t,s,1-t)\,ds\,dt.
\end{IEEEeqnarray*}
\centerline{\rule{100pt}{0.5pt}}
\begin{IEEEeqnarray*}{rCl}
	\bu & = & (0,0,u_3)\\
	\nabla\times\bu &=& (\frac{{\s\partial} u_3}{{\s\partial} x_2},-\frac{{\s\partial} u_3}{{\s\partial} x_1},0)\\
	\pi\bu &=& \alpha_5\gamma_5 + \alpha_6\gamma_6+ \alpha_7\gamma_7+\alpha_8\gamma_8\\
	(\pi\bu)_1 & = &(\alpha_5-\alpha_6)z+
		(-\alpha_5+\alpha_6+\alpha_7-\alpha_8)\frac{yz}{1-z}.\\
	\alpha_5-\alpha_6 & = &-\int\limits_{\mathcal{C}_{125}}\bu\cdot\btau\,d\sigma\\
	&=&-\iint\limits_{[a_1,a_2,a_5]}\nabla\times\bu\cdot\bn\,dS
\end{IEEEeqnarray*}
($\bn$ is the outer normal $(0,-1,0)$)
Similarly:
\begin{IEEEeqnarray*}{rCl} 	
	\alpha_7-\alpha_8 & = & 
	\iint\limits_{[a_3,a_5,a_4]}\nabla\times\bu\cdot\bn\,dS
\end{IEEEeqnarray*}
so {\color{red}($\nabla\times\bu\cdot(0,1,1)/\sqrt{2} = -\frac{{\s\partial} u_3}{{\s\partial} x_1}$)}
\begin{IEEEeqnarray*}{rCl}
	-\alpha_5+\alpha_6+\alpha_7-\alpha_8 & = & 
	\iint\limits_{[a_1,a_2,a_5]}\nabla\times\bu\cdot\bn\,dS
	+\iint\limits_{[a_3,a_5,a_4]}\nabla\times\bu\cdot\bn\,dS\\
	&=&\int\limits_{0}^{1}
	\int\limits_{0}^{1-t}\frac{{\s\partial} u_3}{{\s\partial} x_1} (s,0,t) \,ds\,dt
	-\int\limits_{0}^{1}
	 \int\limits_{0}^{1-t}\frac{{\s\partial} u_3}{{\s\partial} x_1} (s,1-t,t) \,ds\,dt\\
	&=&-\int\limits_{0}^{1}
	\int\limits_{0}^{1-t}
	\int\limits_{0}^{1-t}\frac{{\s\partial}^2u_3}{{\s\partial} x_2{\s\partial} x_1}(s,r,t)\,dr\,ds\,dt\\
	&=&-\iiint\limits_{\hat{P}}\frac{{\s\partial}^2u_3}{{\s\partial} x_2{\s\partial} x_1}\,dV.
\end{IEEEeqnarray*}
For now
\begin{IEEEeqnarray*}{rCl}
	(\pi\bu)_1 & = & -z\int\limits_{0}^{1-t}\frac{{\s\partial} u_3}{{\s\partial} x_1}\,ds\,dt
	-\frac{yz}{1-z}\iiint\limits_{\hat{P}}
		\frac{{\s\partial}^2u_3}{{\s\partial} x_2{\s\partial} x_1}\,dV.
\end{IEEEeqnarray*}
\begin{IEEEeqnarray*}{rCl}
	(\pi\bu)_2& = &(\alpha_5-\alpha_7)z
				+(-\alpha_5+\alpha_6+\alpha_7-\alpha_8)\frac{xz}{1-z}\\[5pt]
\end{IEEEeqnarray*}
\begin{IEEEeqnarray*}{rCl}
	\alpha_5-\alpha_7& = & \iint\limits_{[a_1,a_5,a_3]}\nabla\times\bu\cdot\bn\,dS\\
	& = &-\int\limits_{0}^{1}\int\limits_{0}^{1-t} \frac{{\s\partial} u_3}{{\s\partial} x_2}(0,s,t)\,ds\,dt
\end{IEEEeqnarray*}
\begin{IEEEeqnarray*}{rCl}
	(\pi\bu)_2& = &
	-z\int\limits_{0}^{1}\int\limits_{0}^{1-t}\frac{{\s\partial} u_3}{{\s\partial} x_2}(0,s,t)\,ds\,dt
	+\frac{xz}{1-z}\iiint\limits_{\hat{P}}
	\frac{{\s\partial}^2u_3}{{\s\partial} x_2{\s\partial} x_1}\,dV.
\end{IEEEeqnarray*}
\begin{IEEEeqnarray*}{rCl}
	(\pi\bu)_3&=& \alpha_5 + x(-\alpha_5+\alpha_6)+y(-\alpha_5+\alpha_7)\\[5pt]
	&&\,+(\alpha_5-\alpha_6-\alpha_7+\alpha_8)\frac{xy}{1-z}
	+(-\alpha_5+\alpha_6+\alpha_7-\alpha_8)\frac{xyz}{(1-z)^2}.
\end{IEEEeqnarray*}
\begin{IEEEeqnarray*}{rCl}
	\alpha_5&=&\int\limits_{0}^1u_3(0,0,t)\,dt.
\end{IEEEeqnarray*}
Joinig everything
\begin{IEEEeqnarray*}{rCl}
	(\pi\bu)_3 & = & \int\limits_{0}^1u_3(0,0,t)\,dt
				+ x \int\limits_{0}^{1}\int\limits_{0}^{1-t}
						\frac{{\s\partial} u_3}{{\s\partial} x_1} (s,0,t) \,ds\,dt
				+ y \int\limits_{0}^{1}\int\limits_{0}^{1-t}
						\frac{{\s\partial} u_3}{{\s\partial} x_2}(0,s,t) \,ds\,dt\\
				&&\,+ \frac{xyz}{(1-z)^2}\iiint\limits_{\hat{P}}
				\frac{{\s\partial}^2u_3}{{\s\partial} x_2{\s\partial} x_1}\,dV.
				-\frac{xz}{1-z}\iiint\limits_{\hat{P}}
				\frac{{\s\partial}^2u_3}{{\s\partial} x_2{\s\partial} x_1}\,dV.
\end{IEEEeqnarray*}
All together, for an $\bu=(u_1,u_2,u_3)$.
\begin{IEEEeqnarray*}{rCl}
	(\pi\bu)_1 & = & \int\limits_{0}^{1}u_1(t,0,0)\,dt + 
	z \int\limits_0^1\int\limits_0^{1-t}
	\frac{{\s\partial} u_1}{{\s\partial} x_3}(t,0,s)\,dsdt +
	y \int\limits_0^1\int\limits_0^{1}
	\frac{{\s\partial} u_1}{{\s\partial} x_2}(t,s,0)\,dsdt\\
	&&\,+\frac{yz}{1-z} \int\limits_0^1\int\limits_0^{1-t}
	\frac{{\s\partial} u_1}{{\s\partial} x_2}(1-t,s,t)\,dsdt +
	\frac{yz}{1-z} \int\limits_0^1\int\limits_0^{1-t}
	\frac{{\s\partial} u_2}{{\s\partial} x_1}(s,1-t,t)\,dsdt\\
	&&\,-z\int\limits_0^1\int\limits_0^{1-t}
	\frac{{\s\partial} u_3}{{\s\partial} x_1}(s,0,t)\,dsdt +
	\frac{yz}{1-z} \iiint\limits_{\hat{P}}
	\frac{{\s\partial}^2u_3}{{\s\partial} x_2{\s\partial} x_1}\,dV.
\end{IEEEeqnarray*}
\begin{IEEEeqnarray*}{rCl}
	(\pi\bu)_2 & = & \int\limits_{0}^{1}u_2(0,t,0)\,dt + 
	z \int\limits_0^1\int\limits_0^{1-t}
	\frac{{\s\partial} u_2}{{\s\partial} x_3}(0,t,s)\,dsdt +
	x \int\limits_0^1\int\limits_0^{1}
	\frac{{\s\partial} u_2}{{\s\partial} x_1}(s,t,0)\,dsdt\\
	&&\,+\frac{xz}{1-z} \int\limits_0^1\int\limits_0^{1-t}
	\frac{{\s\partial} u_1}{{\s\partial} x_2}(1-t,s,t)\,dsdt +
	\frac{xz}{1-z} \int\limits_0^1\int\limits_0^{1-t}
	\frac{{\s\partial} u_2}{{\s\partial} x_1}(s,1-t,t)\,dsdt\\
	&&\,-z\int\limits_0^1\int\limits_0^{1-t}
	\frac{{\s\partial} u_3}{{\s\partial} x_2}(0,s,t)\,dsdt +
	\frac{xz}{1-z} \iiint\limits_{\hat{P}}
	\frac{{\s\partial}^2u_3}{{\s\partial} x_2{\s\partial} x_1}\,dV.
\end{IEEEeqnarray*}
\begin{IEEEeqnarray*}{rCl}
	(\pi\bu)_3 & = & \int\limits_{0}^{1}u_3(0,0,t)\,dt + 
	x \int\limits_0^1\int\limits_0^{1-t}
	\frac{{\s\partial} u_3}{{\s\partial} x_1}(s,0,t)\,dsdt -
	x \int\limits_0^1\int\limits_0^{1-t}
	\frac{{\s\partial} u_1}{{\s\partial} x_3}(t,0,s)\,dsdt\\
	&&\,+y \int\limits_0^1\int\limits_0^{1-t}
	\frac{{\s\partial} u_3}{{\s\partial} x_2}(0,s,t)\,dsdt -
	y \int\limits_0^1\int\limits_0^{1-t}
	\frac{{\s\partial} u_2}{{\s\partial} x_3}(0,t,s)\,dsdt\\
	&&\,+\frac{xy}{1-z} \int\limits_0^1\int\limits_t^{1}
	\frac{{\s\partial} u_1}{{\s\partial} x_2}(t,s,0)\,dsdt +
	\frac{xy}{1-z} \int\limits_0^1\int\limits_0^{t}
	\frac{{\s\partial} u_2}{{\s\partial} x_1}(t,s,1-t)\,dsdt\\
	&&\,+\frac{xyz}{(1-z)^2} \int\limits_0^1\int\limits_0^{1-t}
	\frac{{\s\partial} u_2}{{\s\partial} x_1}(s,1-t,t)\,dsdt
	+\frac{xyz}{(1-z)^2} \int\limits_0^1\int\limits_0^{1-t}
	\frac{{\s\partial} u_1}{{\s\partial} x_2}(1-t,s,t)\,dsdt\\
	&&\,-\frac{xy}{1-z}
	\int\limits_{0}^{1}
	\int\limits_{0}^{t}
	\int\limits_{0}^{1-t}
	\frac{{\s\partial}^2u_1}{{\s\partial}x_2{\s\partial}x_3}(t,s,r)\,dr\,ds\,dt
	-\frac{xy}{1-z}
	\int\limits_{0}^{1}
	\int\limits_{0}^{t}
	\int\limits_{0}^{t}
	\frac{{\s\partial}^2u_2}{{\s\partial}x_1{\s\partial}x_3}(r,s,1-t)\,dr\,ds\,dt\\
	&&\,
	+\frac{xyz}{(1-z)^2} \iiint\limits_{\hat{P}}
	\frac{{\s\partial}^2 u_3}{{\s\partial} x_1{\s\partial} x_2}\,dV
	-\frac{xz}{1-z} \iiint\limits_{\hat{P}}
	\frac{{\s\partial}^2 u_3}{{\s\partial} x_1{\s\partial} x_2}\,dV
\end{IEEEeqnarray*}
{\color{red}controlar contra el papel y contra los signos
según las normales exteriores}

	% &=&-\int\limits_0^1\int\limits_0^{1-t}\frac{{\s\partial} u_3}{{\s\partial} x_1}(s,0,t)\,dsdt.


%	(\pi\bu)_2 & = &\alpha_4 + (\alpha_2-\alpha_4)x -
%	(\alpha_4+\alpha_7)z + (\alpha_7-\alpha_8)\frac{xz}{1-z}.\\
%	\alpha_4 = \int\limits_{0}^{1}u_2(0,t,0)\,dt\\
%	\alpha_2-\alpha_4 & = & \int\limits_{0}^{1} u_2(1,t,0)-u_2(0,t,0)\,dt\\
%		&=&\int\limits_{0}^{1}\int\limits_{0}^{1}\frac{{\s\partial} u_2}{{\s\partial} x_1}(s,t,0)\,dsdt\\
%	\alpha_4+\alpha_7 & = & \int\limits_0^1 u_2(0,t,0)-u_2(0,t,1-t)\,dt\\
%		& = & \int\limits_0^1\int\limits_0^{1-t}\frac{{\s\partial} u_2}{{\s\partial} x_3}(0,t,s)\,dsdt.\\
%	\alpha_7-\alpha_8&=&\int\limits_{0}^{1} u_2(1-t,1-t,t)-u_2(0,1-t,t)\,dt\\
%		&=&\int\limits_{0}^{1}\int\limits_{0}^{1-t}\frac{{\s\partial} u_2}{{\s\partial} x_1}(s,1-t,t)\,dsdt.

% subsection edge_elements (end)
\subsection{Anisotropic Stability Estimates for $H(\Div)$--Conforming 
Elements on Pyramids} % (fold)
\label{sub:face_elements}

\paragraph{Stability $\hat{E}$} 
\label{par:stability_hat}
\begin{theorem} \label{aux_label54}
\noindent{\color{Orange}\#\#\#\#\#\#\# cambiar H1 por W1p a la derecha.
corregir en las estimaciones de la demostración porque
está todo con H1} 
\begin{IEEEeqnarray*}{rCl}
  \|(\rku)_1\|_{\scriptscriptstyle{L^p(\hat{E})}}
  &\lesssim& \|\hat u_1\|_{\scriptscriptstyle{W^{1,p}(\hat{E})}} +
    \|\dv \bu\|_{\scriptscriptstyle{L^p}(\hat{E})} + 
    \left\|\hat{u}_3\right\|_{\scriptscriptstyle{H^1}(\hat{E})}\\[12pt]
  \|(\rku)_2\|_{\scriptscriptstyle{L^p(\hat{E})}}
  &\lesssim& \|\hat u_2\|_{\scriptscriptstyle{W^{1,p}(\hat{E})}} +
    \|\dv \bu\|_{\scriptscriptstyle{L^p}(\hat{E})} + 
    \left\|\hat{u}_3\right\|_{\scriptscriptstyle{H^1}(\hat{E})}\\[12pt]
  \|(\rku)_3\|_{\scriptscriptstyle{L^p(\hat{E})}} & \lesssim & 
    \|u_3\|_{\scriptscriptstyle{W^{1,p}(\hat{E})}} +
    \|\dv \bu\|_{\scriptscriptstyle{L^p}(\hat{E})}.
\end{IEEEeqnarray*}
\end{theorem}

\begin{proof}
We will use the notation of Table~\ref{shape_edge_table} for the 
shape functions and Tables~\ref{pyramidNotationTableFaces} and
~\ref{pyramidNotationTableEdges} for the border of the pyramid. The variables 
in the local coordinate system of $\hat E$ are $x,y$ and $z$.\\[5pt]
Consider the case $\hat{\bu} = (\hat{u}_1,0,0)'$ and compute it's interpolate. 
\begin{IEEEeqnarray*}{rCl}
  \rku & = & ({\scriptstyle\iint_{\hat{f}_2} \bu \cdot \hat\bn_2\,d\gamma})\,\bz_2 + 
         \iint_{\hat{f}_3} \bu \cdot \hat\bn_3\,d\gamma\,\bz_3\\[4pt]
       & = & \alpha_2(\hat\bu)\,\bz_2 + \alpha_3(\hat\bu)\,\bz_3.
\end{IEEEeqnarray*}
Then 
\begin{IEEEeqnarray*}{rCl}
  (\rku)_1\xyz & = & -2\alpha_2(\hat\bu) + 
    (\alpha_2(\hat\bu)+\alpha_3(\hat\bu))\,\frac{2x-xz}{1-z}\\[4pt]
    & = & -2{\iint_{\hat{f}_2} \hat{\bu} \cdot \hat\bn_2\,d\gamma} + 
          ({\iint_{\hat{f}_2} \hat{\bu} \cdot \hat\bn_2\,d\gamma}+
                  {\iint_{\hat{f}_3} \hat{\bu} \cdot \hat\bn_3\,d\gamma})\frac{2x-xz}{1-z}\\[4pt]
    & = & -2{\iint_{\hat{f}_2} \hat{\bu} \cdot \hat\bn_2\,d\gamma} + 
          {\iint_{\partial\hat{E}} \hat{\bu} \cdot \hat\bn\,d\gamma}\frac{2x-xz}{1-z}\\[4pt]
    & = & -2{\iint_{\hat{f}_2} \hat{\bu} \cdot \hat\bn_2\,d\gamma} + 
            {\int_{\hat{E}} \dv\hat{\bu} \,d\hat{\boldsymbol{x}}}\frac{2x-xz}{1-z}.\\[8pt]
  (\rku)_2\xyz & = & -(\alpha_2(\hat\bu)+\alpha_3(\hat\bu))\,\frac{yz}{1-z}\\[4pt]
    & = & -{\int_{\hat{E}} \dv\hat{\bu} \,d\hat{\boldsymbol{x}}}\frac{yz}{1-z}.
\end{IEEEeqnarray*}
Switch to $\hat{\bu} = (0,\hat{u}_2,0)'$.
\begin{IEEEeqnarray*}{rCl}
  \rku & = & ({\scriptstyle\iint_{\hat{f}_1} \bu \cdot \hat\bn_1\,d\gamma})\,\bz_1 + 
         \iint_{\hat{f}_4} \bu \cdot \hat\bn_4\,d\gamma\,\bz_4\\[4pt]
       & = & \alpha_1(\hat\bu)\,\bz_1 + \alpha_4(\hat\bu)\,\bz_4.
\end{IEEEeqnarray*}
Then
\begin{IEEEeqnarray*}{rCl}
  (\rku)_1\xyz & = & -(\alpha_1(\hat\bu)+\alpha_4(\hat\bu))\,\frac{xz}{1-z}\\[4pt]
    & = & -{\int_{\hat{E}} \dv\hat{\bu} \,d\hat{\boldsymbol{x}}}\frac{xz}{1-z}.\\[8pt]
  (\rku)_2\xyz & = & -2\alpha_1(\hat\bu) + 
  (\alpha_1(\hat\bu)+\alpha_4(\hat\bu))\,\frac{2y-yz}{1-z}\\[4pt]
    & = & -2{\iint_{\hat{f}_1} \hat{\bu} \cdot \hat\bn_1\,d\gamma} + 
            {\iint_{\partial\hat{E}} \hat{\bu} \cdot \hat\bn\,d\gamma}\frac{2y-yz}{1-z}\\[4pt]
    & = & -2{\iint_{\hat{f}_1} \hat{\bu} \cdot \hat\bn_1\,d\gamma} + 
            {\int_{\hat{E}} \dv\hat{\bu} \,d\hat{\boldsymbol{x}}}\frac{2y-yz}{1-z}.\\[8pt]
\end{IEEEeqnarray*}
Switch to $\hat{\bu} = (0,0,\hat{u}_3)'$.
\begin{IEEEeqnarray*}{rCl}
  \rku & = & ({\scriptstyle\iint_{\hat{f}_3} \bu \cdot \hat\bn_3\,d\gamma})\,\bz_3 + 
         \iint_{\hat{f}_4} \bu \cdot \hat\bn_4\,d\gamma\,\bz_4 + 
         \iint_{\hat{f}_5} \bu \cdot \hat\bn_5\,d\gamma\,\bz_5\\[4pt]
       & = & \alpha_3(\hat\bu)\,\bz_3 + \alpha_4(\hat\bu)\,\bz_4
       + \alpha_5(\hat\bu)\,\bz_5.
\end{IEEEeqnarray*}
Then
\begin{IEEEeqnarray*}{rCl}
  (\rku)_1\xyz & = & (\alpha_3(\hat\bu)+\alpha_5(\hat\bu))x
  + \alpha_3(\hat\bu) \frac{x}{1-z} - \alpha_4\frac{xz}{1-z}.
\end{IEEEeqnarray*}
Now observe
\begin{IEEEeqnarray*}{rCl}
  (\alpha_3(\hat\bu)+\alpha_5(\hat\bu)) & = & 
    {\iint_{\partial\hat{E}} \hat{\bu} \cdot \hat\bn\,d\gamma} - 
      {\iint_{\hat{f}_4} \hat{\bu} \cdot \hat\bn_4\,d\gamma} \\[4pt]
  & = & {\int_{\hat{E}} \dv\hat{\bu}\,d\hat{\boldsymbol{x}}} - 
        \alpha_4(\hat{\bu})
\end{IEEEeqnarray*}
and
\begin{IEEEeqnarray*}{rCl}
  \alpha_3(\hat\bu)-\alpha_4(\hat\bu) & = & 
  {\iint_{\hat{f}_3} \bu \cdot \hat\bn_3\,d\gamma} - 
  {\iint_{\hat{f}_4} \bu \cdot \hat\bn_4\,d\gamma} \\[4pt]
  & = & \int_{0}^{1}\int_{0}^{x} \hat{u}_3(x,y,1-x)\,dydx - 
        \int_{0}^{1}\int_{0}^{y} \hat{u}_3(x,y,1-y)\,dxdy\mbox{,}
\end{IEEEeqnarray*}
so
\begin{IEEEeqnarray*}{rCl}
  (\rku)_1\xyz & = & {\int_{\hat{E}} \dv\hat{\bu}\,d\hat{\boldsymbol{x}}}\,x +\\[4pt]
  \IEEEeqnarraymulticol{3}{r}{\qquad(\int_{0}^{1}\int_{0}^{x} \hat{u}_3(x,y,1-x)\,dydx - 
                          \int_{0}^{1}\int_{0}^{y} \hat{u}_3(x,y,1-y)\,dxdy)
                        \frac{x}{1-z}.}
\end{IEEEeqnarray*}
In a completely similar fashion
\begin{IEEEeqnarray*}{rCl}
  (\rku)_2\xyz & = & {\int_{\hat{E}} \dv\hat{\bu}\,d\hat{\boldsymbol{x}}}\,y\,+
  ({\iint_{\hat{f}_3} \bu \cdot \hat\bn_3\,d\gamma} - 
   {\iint_{\hat{f}_4} \bu \cdot \hat\bn_4\,d\gamma})\frac{y}{1-z}.\\[4pt]
               & = & {\int_{\hat{E}} \dv\hat{\bu}\,d\hat{\boldsymbol{x}}}\,y\,+\\[4pt]
  \IEEEeqnarraymulticol{3}{r}{\qquad(\int_{0}^{1}\int_{0}^{x} \hat{u}_3(x,y,1-x)\,dydx - 
                          \int_{0}^{1}\int_{0}^{y} \hat{u}_3(x,y,1-y)\,dxdy)
                        \frac{y}{1-z}.}
\end{IEEEeqnarray*}
We collect every term obtained so far for the first and second components in
Table~\ref{terms_table}.
\begin{table}[!h]
    \centering  
    \caption{Terms\\[4pt]$q(s,t) = \frac{2s-st}{1-t},\,r(s,t) = \frac{st}{1-t}$}
    \label{terms_table}
    \begin{IEEEeqnarraybox*}
    [\IEEEeqnarraystrutmode
    \IEEEeqnarraystrutsizeadd{2pt}{12pt}]{v/c/v/c/v/c/v/}
        \IEEEeqnarrayrulerow\\
        \IEEEeqnarrayseprow[5pt]\\
        & & & (\rku)_1 & & (\rku)_2 & \\
        \IEEEeqnarrayrulerow\\
        \IEEEeqnarrayseprow[5pt]\\
        & (\hat{u}_1,0,0)' & &
          \begin{IEEEeqnarraybox*}{l}
            -2{\iint_{\hat{f}_2} \hat{\bu} \cdot \hat\bn_2\,d\gamma}\\ + 
            {q(x,z)\int_{\hat{E}} \dv\hat{\bu} \,d\hat{\boldsymbol{x}}}
          \end{IEEEeqnarraybox*}
        & &
          -r(y,z){\int_{\hat{E}} \dv\hat{\bu} \,d\hat{\boldsymbol{x}}} &\\
        \IEEEeqnarrayrulerow\\
        \IEEEeqnarrayseprow[5pt]\\
        & (0,\hat{u}_2,0)' & & 
          -r(x,z){\int_{\hat{E}} \dv\hat{\bu} \,d\hat{\boldsymbol{x}}} 
        & & 
          \begin{IEEEeqnarraybox*}{l}
            -2{\iint_{\hat{f}_1} \hat{\bu} \cdot \hat\bn_1\,d\gamma}\\ + 
            {q(x,z)\int_{\hat{E}} \dv\hat{\bu} \,d\hat{\boldsymbol{x}}}
          \end{IEEEeqnarraybox*}
        &\\
        \IEEEeqnarrayrulerow\\
        \IEEEeqnarrayseprow[5pt]\\
        & (0,0,\hat{u}_3)' & & 
          \begin{IEEEeqnarraybox*}{l}
            x\int_{\hat{E}} \dv\hat{\bu}\,d\hat{\boldsymbol{x}} \\[5pt] +\, 
            ({\iint_{\hat{f}_3} \bu \cdot \hat\bn_3\,d\gamma}
             \\[5pt] 
             -{\iint_{\hat{f}_4} \bu \cdot \hat\bn_4\,d\gamma})r(x,z)
          \end{IEEEeqnarraybox*}
         & & 
          \begin{IEEEeqnarraybox*}{l}
            y\int_{\hat{E}} \dv\hat{\bu}\,d\hat{\boldsymbol{x}}\\[5pt] +\, 
              ({\iint_{\hat{f}_4} \bu \cdot \hat\bn_4\,d\gamma}
             \\[5pt] 
             -{\iint_{\hat{f}_3} \bu \cdot \hat\bn_3\,d\gamma})r(y,z)
          \end{IEEEeqnarraybox*}
        &\\\IEEEeqnarrayrulerow
    \end{IEEEeqnarraybox*}
\end{table}
Lastly, for any $\hat\bu$
\begin{IEEEeqnarray*}{rCl}
    (\rku)_3\xyz & = &   z\sum_{i=1}^4\iint_{\hat{f}_i} \hat{\bu}\cdot\hat{\bn}_i\,d\gamma
                      + (z-1) \iint_{\hat{f}_5}\hat{\bu}\cdot\hat{\bn}_5\,d\gamma\\[5pt]
                 & = & z\iint_{\partial\hat{E}} \hat{\bu}\cdot\hat{\bn} - \iint_{f_5}
                 \hat{\bu}\cdot\hat{\bn}_5\,d\gamma\\[5pt]
   \yesnumber\label{term_rk3}
                 & = &\hat{x}_3\int\limits_{\hat{E}} \mbox{div}\,\hat{\bu}\,d\hat{\bx} 
     + \iint\limits_{\hat{f}_5} \hat{u}_3\,d\hat{\gamma}
\end{IEEEeqnarray*}
Now we bound each term 
\begin{IEEEeqnarray*}{rCl}
  (\rku)_1 & = & (\br_{\hat{E}}(\hat{u}_1,0,0)')_1 + 
                 (\br_{\hat{E}}(0,\hat{u}_2,0)')_1 + 
                 (\br_{\hat{E}}(0,0,\hat{u}_3)')_1\\[5pt]
  & = & -2\iint_{\hat{f}_2}\hat{u}_1\,d\gamma +
  \frac{2x}{1-z}\int_{\hat{E}} \frac{\partial\hat{u}_1}{\partial\hat{x}_1}\,d\hat{\bx} -
  \frac{xz}{1-z}\int_{\hat{E}} \frac{\partial\hat{u}_1}{\partial\hat{x}_1}\,d\hat{\bx} \\[5pt]
  & & \,- \frac{xz}{1-z}\int_{\hat{E}} \frac{\partial\hat{u}_2}{\partial\hat{x}_2}\,d\hat{\bx}
  + \left( x + \frac{xz}{1-z}-\frac{xz}{1-z} \right)
  \int_{\hat{E}} \frac{\partial\hat{u}_3}{\partial\hat{x}_3}\,d\hat{\bx}\\[5pt]
  &   &\, + \left({\iint_{\hat{f}_3} \hat{u}_3\,d\gamma}
        - {\iint_{\hat{f}_4} \hat{u}_3\,d\gamma}\right)\frac{xz}{1-z}\\[5pt]
  & = & -2\iint_{\hat{f}_2}\hat{u}_1\,d\gamma +
  \frac{2x}{1-z}\int_{\hat{E}}\frac{\partial\hat{u}_1}{\partial\hat{x}_1}\,d\hat{\bx} -
  \frac{xz}{1-z}\int_{\hat{E}}\dv\hat{\bu}\,d\hat{\bx}\\[5pt]
  \yesnumber\label{face_integrals}
  &  & \,+\frac{x}{1-z}\int_{\hat{E}}\frac{\partial\hat{u}_3}{\partial\hat{x}_3}\,d\hat{\bx}
  + \left({\iint_{\hat{f}_3} \hat{u}_3\,d\gamma}
        - {\iint_{\hat{f}_4} \hat{u}_3\,d\gamma}\right)\frac{xz}{1-z}
\end{IEEEeqnarray*}
For the surface integrals in~(\ref{face_integrals}), by Theorem 3.9 in~\cite{monk}, page 43,
\begin{IEEEeqnarray*}{rCl}
  \left|\iint_{\hat{f}_2} \hat{u}_1\,d\gamma\right| 
  & \leqslant & 2^{-1/4}\,\|\mbox{Tr}_{\hat{f}_2}\hat{u}_1\|_{L^2(\hat{f}_3)} \\[5pt]
  & \leqslant & C\,\|\mbox{Tr\,}\hat{u}_1\|_{H^{\delta}(\partial\hat{E})} \\[5pt]
  & \leqslant & C\,\|\hat{u}_1\|_{H^{1/2+\delta}(\hat{E})}. \\[5pt]
  & \leqslant & C\,\|\hat{u}_1\|_{H^{1}(\hat{E})}.
\end{IEEEeqnarray*}
and similarly
\begin{IEEEeqnarray*}{rCl}
  \left|\iint_{\hat{f}_3} \hat{u}_3\,d\gamma - \iint_{\hat{f}_4} \hat{u}_3\,d\gamma\right| 
  & \leqslant & C\,\|\hat{u}_3\|_{H^{1}(\hat{E})}
\end{IEEEeqnarray*}
all of which leads to
\begin{IEEEeqnarray*}{rCl}
  \|(\rku)_1\|_{L^{\infty}(\hat{E})} & \leqslant & C_{\hat{E}} 
  \left[ 
    \|\hat{u}_1\|_{H^{1}(\hat{E})} + 
    \|\dv\hat{\bu}\|_{L^{2}(\hat{E})} + 
    \|\hat{u}_3\|_{H^{1}(\hat{E})}
  \right].
\end{IEEEeqnarray*}
Copying the argument for the second component
\begin{IEEEeqnarray*}{rCl}
  \|(\rku)_1\|_{L^{\infty}(\hat{E})} & \leqslant & C_{\hat{E}} 
  \left[ 
    \|\hat{u}_1\|_{H^{1}(\hat{E})} + 
    \|\dv\hat{\bu}\|_{L^{2}(\hat{E})} + 
    \|\hat{u}_3\|_{H^{1}(\hat{E})}
  \right].
\end{IEEEeqnarray*}
From~(\ref{term_rk3}) we deduce
\begin{IEEEeqnarray*}{rCl}
  \|(\rku)_3\|_{\scriptscriptstyle{L^\infty(\hat{E})}} & \leqslant & C_{\hat{E}}
    \left[\|u_3\|_{\scriptscriptstyle{H^{1}(\hat{E})}} +
    \|\dv \bu\|_{\scriptscriptstyle{L^2}(\hat{E})}\right].
\end{IEEEeqnarray*}
The quantity $C_{\hat{E}}$ depends only on the supremum of the (fixed)
basis shape functions of Table~\ref{shape_face_table} over the pyramid.
\end{proof}
% subsection face_elements (end)

%% ============================================================================
%% TODO: ver si esto finalmente va
%% \subsection{Local Interpolation Estimates for Pyramidal Finite Elements} % (fold)
%% \label{sub:local_interpolation_estimates_for_pyramidal_elements}
%% decir que permitimos pirámides elongadas perpendicularmente a la base
%% $h_3\geqslant C\min\{h_1,h_2\}$
%% Verificar si es esto o $h3 >= max (h1, h2)$\\
%% poner tres dibujos con casos $h1=h2<h3$; $h1<h2<h3$; $h2<h1<h3$
% subsection local_interpolation_estimates_for_pyramidal_elements (end)
%% ============================================================================


% section pyramidal_finite_elements (end)