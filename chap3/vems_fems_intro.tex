\chapter{vems--fems--intro}
--$E\in\mathcal T_h$\\ 
--local space
\begin{eqnarray*}
V_h(E)&=&\Big\{\bv\in H(\mbox{div},E)\cap H(\textbf{curl},E)\,:\,\\
&&\qquad \bv\cdot\bn|_f\in \mathcal P_0(f) \,\,\mbox{for all face $f$ of }E, \\
&& \qquad\mbox{div\,}\bv\in \mathcal P_0(E) \mbox{ and } 
\mbox{{\bf curl}\,}\bv = 0 \Big\}.
\end{eqnarray*}
$-->$ global space
\begin{eqnarray*}
V_h&=&\left\{\bv\in H(\dv,\Omega): \bv|_E\in V_h(E), \mbox{for all element }
E\in\mathcal T_h\right\}
\end{eqnarray*}


--------------------------------------------------------------------------------
\subsection{Definition of the $H(\text{div})$-conforming element on prisms. 
Generalization of the Raviart-Thomas Finite Element} % (fold)
\label{sub:definition_of_the_h_div_element_on_prisms}

Notaci\'on:{\color{red} Controlar si est\'a bien usada la notaci\'on de prod. tensorial cuando
es un vector de dos coords. por un escalar, y Controlar si es el mismo $h(x,y,z)$ en las coords. 1 y 2 (el 
que es un homog\'eneo en $x,y$ por uno en $z$.)}
\begin{IEEEeqnarray*}{rCl}
	D_k & = & P_{k-1}^2(x,y) \oplus \tilde{P}_{k-1}(x,y) \textbf{x},\\
	\textbf{x} & = & (x,y).
\end{IEEEeqnarray*}
\begin{defi}\label{defi_h_div_conforme} Definimos aqu\'{\i} un elemento 
$H(\text{div})$--conforme. 
\begin{enumerate}
	\item $\hat{K}$ es el prisma de referencia.
	\item El espacio $P_{\hat{K}}$ de polinomios es
		\begin{IEEEeqnarray*}{rCl}
		 	P_{\hat{K}} & = & \{ \boldsymbol{v} = (v_1,v_2,v_3):\,(v_1,v_2)\in D_k\otimes P_{k-1}(z),\\ 
						& 	& v_3\in P_{k-1}(x,y)\otimes P_k(z) \}.
		 \end{IEEEeqnarray*} 
	\item Los grados de libertad son:
\begin{IEEEeqnarray}{lll}
	\label{momentos1hdiv} \int\limits_{f} (\boldsymbol{v}\cdot\boldsymbol{\nu})q\,d\gamma 
		& q \in P_{k-1}(x,y)\textrm{,} & \textrm{ para las dos caras horizontales; } \\
	\label{momentos2hdiv} \int\limits_{f} (\boldsymbol{v}\cdot\boldsymbol{\nu})q\,d\gamma 
		& q \in Q_{k-1, k-1, k-1}\textrm{,} & \textrm{ para las tres caras verticales; } \\
	\label{momentos3hdiv} \int\limits_{\hat{K}} (v_1q_1 + v_2q_2)\,d\textbf{x} 
		& \IEEEeqnarraymulticol{2}{c}{q_1\textrm{, } q_2 \in P_{k-2}(x,y) \otimes P_{k-1}(z);}\\
	\label{momentos4hdiv} \int\limits_{\hat{K}} v_3q_3\,d\textbf{x} 
		& \IEEEeqnarraymulticol{2}{c}{ q_3\in P_{k-1}(x,y) \otimes P_{k-2}(z).} 
\end{IEEEeqnarray}
\end{enumerate}
\end{defi}
---------------------------------------------------------
% h div low order reference prism
\paragraph{$k=1$} (lowest order basis elements)\\[5pt] % (fold)
\label{par:_k_0_}
\begin{IEEEeqnarraybox*}{rCl}
	v_1&=&\left(
		\begin{array}{c}
			-1+x_1\\
			x_2\\[3pt]
			0\\
		\end{array}
	\right)
\end{IEEEeqnarraybox*}
\begin{IEEEeqnarraybox*}{rCl}
	v_2&=&\left(
		\begin{array}{c}
			x_1\\
			-1+x_2\\[3pt]
			0\\
		\end{array}
	\right)
\end{IEEEeqnarraybox*}
% paragraph _k_0_ (end)\\
---------------------------------------------------------

\subsection{Definition of the $H(\text{div})$-conforming element on tetrahedra. 
Generalization of the Raviart-Thomas Finite Element} % (fold)
\label{sub:definition_of_the_h_div_element_on_tetrahedra}

{\color{red} TODO: completar esto }
\begin{IEEEeqnarray*}{rCl}
	P_{\hat{K}} & = & P_k^3(\hat{K}) + P_{k-1}(\hat{K})\,\textbf{x}. \\[5pt]
\end{IEEEeqnarray*}

% subsection definition_of_the_h_div_conforming_element_on_tetrahedra_generalization_of_the_raviart_thomas_finite_element (end)