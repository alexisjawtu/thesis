\documentclass{article}
\usepackage{amssymb}
\usepackage{amsmath}
\usepackage{amsfonts}

\topmargin -2.5cm
\textheight 25cm
\def\bcurl{\textbf{curl}}
\newcommand\dv{\mbox{div\,}}
\date{}
\begin{document}

\title{A mixed method with hybrid meshes for elliptic problems
in polihedral domains}

\maketitle
\vspace{4cm}
\section*{Abstract}
In this Thesis we introduce a combined Finite and Virtual Element Method in dimension three
for the mixed approximation of a 
model elliptic problem for the Laplace operator on an 
arbitrary polyhedron. 
The 
method is fully analysed when the meshes are made up of triangularly
right prisms, pyramids and tetrahedra. The local discrete 
spaces coincide with the lowest order Raviart-Thomas 
spaces on tetrahedra and prisms, and 
extend them to pyramidal elements. The discrete scheme 
is well posed and optimal error estimates are proved on meshes which 
allow for anisotropic elements. In particular, local 
interpolation error estimates for the discrete element space are 
optimal and anisotropic on anisotropic right prisms.
The motivation to work with anisotropic elements is that 
in several situations in mixed finite element approximations the use of meshes 
with narrow elements is needed. 
This is the case for instance when dealing with the Poisson equation 
in a polyhedron $\Omega$ with concave edges or vertices, which, in mixed form
can be written as
\begin{equation*}\label{mf} \left\{\begin{array}{rcl}
\boldsymbol{u}&=&-\nabla p\qquad \mbox{in }\Omega\\
\dv\boldsymbol{u}&=&f\qquad \mbox{in }\Omega\\
p&=&0\qquad\mbox{on }\partial\Omega.\end{array}\right.
\end{equation*}
In this case the vectorial variable of the solution, $\boldsymbol{u}$, is in 
general not in $H^1(\Omega)$ due to vertex and edges 
singularities. In particular, close to concave edges, 
$\boldsymbol{u}$ is expected to be more regular in its direction than transversally 
to it, and consequently the mesh has to be accordingly refined in order to 
recover optimal order of convergence with respect to the number of degrees of 
freedom. Those meshes contain elements that are arbitrarily elongated
in the direction of concave edges.

Likewise, we propose a meshing process to construct a mesh that
allows us to  obtain optimal global 
approximation error estimates when the 
solution has edge or vertex singularities as the mesh results, by construction,
suitably graded and adapted to the singularities.

Furthermore, in the present Thesis we obtained
local anisotropic stability
and interpolation error estimates for arbitrary order Prismatic
Finite Elements in both
the $\bcurl$--conforming and div--conforming classes of elements, which are 
additional results that extend some theoretical facts we proved for
the main problem of the Thesis.

Moreover, we present local anisotropic stability
and interpolation error estimates for lowest order Pyramidal
Finite Elements constructed in the literature for both
the $\bcurl$--conforming and div--conforming classes of elements. This result
is included to show a variant to our main method, that is one 
with only Finite Elements. Regarding this variant, as we show
explicitly in the Thesis, the shape functions 
spanning the pyramidal Finite Element spaces are rational functions and are 
singular, yet bounded,
in the reference pyramid. This is a reason why we considered to move
to our combined FE--VE approach better, because we are avoiding the evaluation
of functions with those properties in computer implementations. 
    
\end{document}
