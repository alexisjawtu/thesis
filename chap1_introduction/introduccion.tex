\chapter*{Introducci\'on}
\addcontentsline{toc}{chapter}{Introducci\'on} \markboth{INTRODUCCION}{}
Un objeto f\'isico se dice anis\'otropo si exhibe 
propiedades f\'isicas vectoriales con magnitudes
distintas cuando se las mide en direcciones distintas. En
el caso de un m\'etodo num\'erico diremos que la familia
de mallas usada es anis\'otropa si contiene sucesiones 
de elementos tales que el orden de decrecimiento de
sus tamaños a lo largo de una direcci\'on es superior
al correspondiente a otras direcciones independientes.
Formalizaremos e ilustraremos esto en el 
Cap\'itulo~\ref{auxlabel207}.

En varias situaciones en aproximaciones por elementos 
finitos mixtos es necesario el uso de mallas anis\'otropas.
Este es el caso, por ejemplo, cuando se trata con la 
ecuaci\'on de Poisson en un poliedro 
$\Omega\subseteq\mathbb{R}^3$ con aristas y v\'ertices
c\'oncavos, la cual, introduciendo la variable vectorial
$\bu=\nabla$ puede escribirse en forma mixta como
\begin{equation}\label{mfSpanish} 
\left\{\begin{IEEEeqnarraybox*}[][c]{,r/c/c/c/l,}
	&&&&\\
	\bu     & = & \nabla p   & \qquad & \mbox{in }\Omega\\
	-\dv\bu & = &        f   & \qquad & \mbox{in }\Omega\qquad(f\in L^2(\Omega))\\
	p       & = & 0          & \qquad & \mbox{on }\partial\Omega.\\
	&&&&%
	\end{IEEEeqnarraybox*}\right.
\end{equation}
La formulaci\'on variacional mixta est\'andar es hallar
$\bu\in H(\Div,\Omega)$ and $p\in L^2(\Omega)$ 
tales que
\begin{equation}\label{PSpanish}
	\begin{IEEEeqnarraybox*}[][c]{l/c/r/C/l}
	\forall \bv\in H(\Div,\Omega)  & \qquad & a(\bu,\bv) + b(\bv,p)   & = & 0    \\
	\forall q\in   L^2(\Omega)	   & \qquad &    		   b(\bu,q)   & = & (-f,q)
	\end{IEEEeqnarraybox*}
\end{equation}
en donde
\[
a(\bv,\bw)=\int_\Omega\bv\cdot\bw\,d\bx,\qquad b(\bv,q)=\int_\Omega q\,\mbox{div\,}\bv\,d\bx.
\]
Dado que $\Omega$ no es convexo, la solución $\bu$ 
de~\ref{PSpanish} no pertenece a $H^1(\Omega)$ en
el caso general por tener singularidades de arista y 
v\'ertice. En particular, cerca de aristas c\'oncavas,
se espera que la soluci\'on sea m\'as regular a lo largo
de la direcci\'on de esa arista que transversalmente a ella
(ver~\cite{apelNicaise}), y es por eso que las mallas
tienen que ser adecuadamente graduadas y refinadas
para recuperar el orden \'optimo de convergencia de 
la sucesión de soluciones numéricas con respecto al 
número de grados de libertad (ver~\cite{alw,apelNicaise}).
Ese tipo de mallas contiene elementos que son tan
estrechos como se quiera en direcci\'on ortogonal
a las aristas c\'oncavas, con tal de tomarlos lo 
suficientemente cerca de ellas. Si tom\'aramos 
una sucesi\'on de mallas como esta hecha únicamente con
tetraedros, incurriríamos en el uso de subfamilias de 
estos que no verifican ciertas condiciones que son 
necesarias para el an\'alisis anisótropo, y es aquí donde
nuestro m\'etodo propuesto  con elementos de geometr\'ias
diferentes cobra sentido.

Para aclara lo \'ultimo y el principal resultado de esta
Tesis (cfr.~\cite{alexisAriel}) ponemos las siguientes definiciones y resultados
previos relacionados.

Primero, $T$ satisface la \emph{propiedad del
vértice regular}
con una 
constante $\bar{c} > 0$ (escrito $T\in \pazocal{RVP}(\bar c)$) si
$T$ tiene un vértice $\bx_T$ tal que,
tomando $M_T$ como la  matriz cuyas columnas
son los vectores unitarios en las direcciones
de las aristas que comparten a $\bx_T$, entonces
$|\det M_T| > \bar{c}$.

Una propiedad geométrica menos restrictiva  es
la siguiente. 
Diremos que un tetraedro $T$ satisface la
\emph{propiedad del ángulo máximo}
con par\'ametro $\bar\alpha$
(escrito $T\in\pazocal{MAC}(\bar\alpha$))
si los ángulos de las caras de $T$
y entre caras son menores a $\bar\alpha$. 

Con estas dos nociones en mente, citamos el siguiente
resultado de~\cite{aadl}. Si $T$ es un tetraedro en 
$\pazocal{RVP}(\bar c)$ y $\br_{\sss k,T}$ 
es el interpolador de Raviart-Thomas
(ver~\cite{nedelec2, MR0483555}), entonces existe una
$C(\bar c)>0$ tal que para toda  
$\bu\in H^1(T)^3$
\begin{IEEEeqnarray}{rCl}\label{rvp}
  \|\bu-\br_{\sss k,T}\bu\|_{\sss L^2(T)^3}& \leqslant & C 
    \left\{\sum_{1\leqslant i\leqslant 3} h_i \|{\s\partial_{\xi_i}}\bu\|_{\sss L^2(T)^3}
	  + h_T \|\dv\bu\|_{\sss L^2(T)}\right\}
\end{IEEEeqnarray}
donde escribimos
$\xi_i$ para las coordenadas locales con origen
en el vértice regular
y $h_i$ para las longitudes
de las aristas incidentes a \'el y 
$h_T$ para el diámetro de $T$.

Por otro lado, si $T\in\pazocal{MAC}(\bar\alpha)$ entonces
existe una $C(\bar\alpha)>0$
tal que 
para toda $\bu\in H^1(T)^3$
vale
\begin{IEEEeqnarray}{rCl}\label{mac}
  \|\bu-\br_{\sss k,T}\bu\|_{\sss L^2(T)^3}& \leqslant & C h_T \sum_{1\leqslant i\leqslant 3}
  \|{\s\partial_{\xi_i}}\bu\|_{\sss L^2(T)^3}
\end{IEEEeqnarray}
($\xi_i$ son coordenadas locales a partir de
un vértice de $T$, pero
no necesariamente tenemos uno regular).

Observamos que la desigualdad~\eqref{mac} es estrictamente
más débil que~\eqref{rvp}, dado que hay elementos
que satisfacen la condición $\pazocal{MAC}(\bar\alpha)$
para $\bar\alpha$ fijo, pero con parámetro $\pazocal{RVP}$
arbitrariamente peque\~no, haciendo que se degenere la
cantidad $C(\bar c)$ en~\eqref{rvp}. Ponemos un ejemplo
en la Figura~\ref{fig:tetraedros}. Adem\'as, como est\'a
dicho en~\cite{aadl} mediante un contraejemplo, 
la desigualdad~\eqref{rvp} no puede ser probada bajo
condici\'on de \'angulo m\'aximo solamente.

Volviendo al segundo p\'arrafo, lo que dec\'iamos es que 
es posible construir mallas graduadas anisótropas para
problemas elípticos en dominios con singularidades que
consistan solamente de tetraedros que satisfacen 
condic\'on  $\pazocal{MAC}(\bar\alpha)$ para $\alpha<\pi$,
pero estos elementos no satisfacen la condici\'on de $\pazocal{RVP}(\bar c)$
para ningún parámetro uniforme positivo $\bar c$.
En consecuencia, la estimaci\'on 
del error de interpolaci\'on~\eqref{rvp} no puede ser 
usada de manera global y en cambio~\eqref{mac} debe ser tomada,
y por esto las propiedades de anisotrop\'ia de las mallas
pueden no traer ninguna ventaja. En otras palabras,
en la segunda desigualdad una \emph{derivada grande} en
la dirección $\xi_i$ no estaría necesariamente compensada
por un $h_i$ peque\~no, así que tendríamos que hacer
peque\~no al diámetro $h_T$ y refinar las mallas en
todas las direcciones, que es lo que queremos evitar, y
perder\'iamos orden $\pazocal{O}(\textit{h})$ en la 
convergencia del error numérico con la relación 
asintótica $\textit{h}\sim N_{\Th}^{-\nicefrac13}$
 (aquí $\Th$ es una malla y $N_{\Th}$ es su 
cardinal; el significado concreto meaning 
del parámetro de mallado $\textit{h}$, que no es
el diámetro de sus elementos, como sí lo es en el caso
de las mallas uniformes, se hará claro en 
la Subsección~\ref{meshes} cuando propongamos nuestro proceso
de mallado).

\tetsTikz

Una idea para sobrellevar la mencionada dificultad, para
el caso en que $\Omega$ es un dominio cil\'indrico 
poliedral, fue propuesto en~\cite{MR1866274}. En este caso,
cuando $f$ está en $L^2(\Omega)$, la solución puede
exhibir solamente singularidades a lo largo de aristas
cóncavas. Los autores proponen un m\'etodo mixto de 
Raviart--Thomas en mallas graduadas de prismas triangulares
y probaron estimaciones óptimas de error por medio de 
resultados de interpolación con anisotropía y de este modo
los tetraedros que no satisfacen una propiedad 
$\pazocal{RVP}$ uniforme son evitados. Tambi\'en proponen
un m\'etodo similar con mallas de tetraedros anisótropos
 graduados obtenido subdividiendo cada prisma entre tres
 tetraedros. Por supuesto, estas mallas contienen a los 
tetraedros malosm, y para obtener estimaciones
de error óptimas el precio pagado es el de requerir m\'as
regularidad al dato $f$, más precisamente, debe pertenecer
a un espacio de Sobolev con pesos.

Uno de los resultados presentados en esta Tesis extiende
los resultados de~\cite{MR1866274} puesto que nuestro
m\'etodo fue buscado con el propósito de poder
lidiar con la aproximación mixta de~\eqref{mfSpanish}
para $f\in L^2(\Omega)$, y tambi\'en para un poliedro
cualquiera $\Omega$. Como estos dominios no necesariamente
admiten una partición en términos de prismas rectos y como
también nosotros quisiéramos evitar pedir
más regularidad al dato $f$, proponemos una discretización
basada en mallas híbridas consistentes en prismas combinados
con tetraedros, con brechas interelementales rellenadas
con pirámides. Obtenemos la estimación del error de
aproximación
\begin{equation}\label{auxlabel410}
 \|\bu-\bu_{\textit h}\|_{L^2(\Omega)^3} + \|p-p_{\textit h}\|_{L^2(\Omega)} 
 \leqslant C\,\textit{h}\|f\|_{L^2(\Omega)}{\mbox{,}}
\end{equation}
en donde $\bu_{\textit h}$ y $p_{\textit h}$ son las
aproximaciones de las soluciones $\bu$ y $p$, por medio
de estimaciones con normas de Sobolev con pesos 
(para $\bu$) tratando las singularidades de manera 
localizada. Primero introducimos y analizamos por completo
un nuevo
método combinado de Elementos Finitos y Virtuales
en un poliedro cuando las mallas son construidas con
los mencionados poliedros elementales. Las estimaciones
de interpolación locales obtenidas 
para los espacios discretos son
óptimas y anisótropas en prismas anisótropos, lo que 
nos permite obtener estimaciones óptimas de error de
aproximación cuando la solución tiene tanto singularidades
de aristas como de vértices, usando
mallas graduadas adecuadamente que incluyen
elementos anisótropos, y que solamente incluyen tetraedros
con un parámetro de condición $\pazocal{RVP}$ uniforme positivo.

No solamente probamos el resultado para una familia abstracta
de tales mallas con las mencionadas propiedades,
 sino que también mostramos un proceso general de mallado
 que comienza aislando y clasificando las singularidades
del dominio (cfr. Teorema~\ref{thm_regularity}), y
con eso probamos la existencia de una familia de mallas
como la requerimos mediante su construcción.

Los espacios discretos $V_{\textit h}$ correspondientes
a las mallas propuestas fueron obtenidos satisfaciendo
las siguientes condiciones, adecuadas para una discretización
en $\mathbb{R}^3$.
\begin{enumerate}
	\item \label{auxlabel411} Conformidad.
	\item \label{auxlabel412} Propiedades de aproximación anisótropas y óptimas.
	\item \label{auxlabel413} Generalidad de dominio.
\end{enumerate}
Tal como es sugerido en~\cite{bfm} para el caso $2D$,
presentamos $V_{\textit h}$ como un espacio de elementos 
virtuales que coincide localmente con el espacio original
$3D$ de Raviart--Thomas en tetraedros y prismas, y lo
extiende de manera natural a elementos piramidales. 
En particular, las componentes normales de las
funciones discretas son constantes sobre las caras de
los elementos, coincidiendo de manera conforme a través 
de caras comunes a elementos de distinta geometría. De
este modo el requerimiento~\ref{auxlabel411} fue satisfecho.
Una ventaja de esta presentación es que la definición
de los espacios locales es independiente de la geometría 
del elemento, ver Sección~\ref{auxlabel290}. Como ya comentamos,
el \'item~\ref{auxlabel413} es verificado mediante la 
incorporación de pirámides de base paralelogramo.
Con respecto al punto~\ref{auxlabel412}, presentamos
estimaciones óptimas de estabilidad y de interpolación
anisótropas en varias partes aunque no todas ellas
son usadas en la aplicación final, en algunos casos
porque los espacios funcionales para los cuales son usadas
no se ven involucrados en el problema 
modelo~\eqref{PSpanish} y en otros casos porque
nuestro esquema de mallado funciona sin requerir
que los elementos de todos los tipos presenten 
anisotropía.

También podrían ser consideradas mallas con elementos
de geometría más general. De hecho esta es una 
de las principales propiedades de los métodos
de elementos virtuales. Nosotros decidimos
restringirnos a unas pocas geometrías elementales
pues nuestro principal objetivo fue admitir 
mallas anisótropas, y con estimaciones anisótropas
uniformemente válidas. Una dificultad que aparece
cuando se consideran otras geometrías elementales 
(como prismas oblícuos) es que los espacios virtuales
locales no necesariamente se ve preservado 
por transformaciones \emph{push--forward} basadas 
en aplicaciones afines (nosotros 
usamos una propiedad de $\bcurl$ nulo que no
es invariante en esta situación), con
la consecuencia de que los argumentos de
reescalado estándar son difíciles de usar
(ver~\ref{auxlabel290}). Sin embargo, esto no deviene
en una limitación en los dominios poliedrales que
podemos tratar con nuestro método, pues podemos
restringirnos a usar prismas rectos duplicando, en el
peor de los casos, el número de elementos de la
primera malla, y así aumentando el número de elementos
en una malla cualquiera posterior en un factor constante.

Nuestro método presenta distinta cantidad de grados
de libertad a través de la malla, pues prismas, tetraedros
y pirámides tienen distinta cantidad de caras, y también
admite distintas geometrías para los grados de libertad pues
estamos trabajando con caras triangulares y cuadriláteras
al mismo tiempo. Como consecuencia de esto, nuestro método
es en cierto sentido una generalización de los
métodos de elementos virtuales clásicos, en los cuales todos
los elementos tienen la misma geometría arbitraria, y de 
cada elemento de toma el mismo número de grados de libertad, todos
del mismo tipo (por ejemplo, evaluación en los mismos puntos en
el borde de cada elemento).

Las mallas híbridas que incluyen tetraedros y prismas (y hasta
hexaedros) pueden ser necesarias para satisfaces las demandas
de la geometría específica de un problema (regiones no triviales)
o para alcanzar cálculos eficientes. Si se requiere que estas mallas
eviten nodos colgantes entonces a menudo incluirán pirámides; ver
por ejemplo~\cite{owenSaigal}. Varios art\'iculos contienen
la construcción de elementos finitos conformes en pirámides,
algunos de los cuales son~\cite{bergot} 
para elementos $H^1$ y~\cite{gh99, Nigam-2012}
para elementos $H(\mbox{div})$ y $H({\bf curl})$,
el primero para orden bajo y el segundo para orden arbitrario.
En~\cite{Nigam-2012} los autores prueban que no es posible
construir elementos finitos $H^1$ útiles en pirámides
usando solamente funciones polinomiales y muestran que, 
en el caso $H(\mbox{div})$,
todos los espacios construidos en la literatura contienen
funciones no polinomiales.

Queremos finalizar esta introducción mencionando otros
resultados que obtuvimos, presentados como resultados
adicionales porque no fueron usados para el problema principal 
de la Tesis, y que pueden ser vistos como extensiones 
de ciertos resultados teóricos.

Probamos estimaciones locales anisótropas 
de estabilidad y de error 
de interpolación para elementos finitos prismáticos
tanto para las clases de elementos 
$\bcurl$--conformes como para las div--conformes.
Nuestro método usa solo el caso de orden bajo de las 
estimaciones en $H(\Div)$ y dejamos una versión 
de orden alto del método para investigaciones futuras.

Finalmente, obtuvimos y presentamos estimaciones locales anisótropas 
de estabilidad y de error 
de interpolación para los elementos finitos piramidales
introducidos en~\cite{gh99, Nigam-2012} 
tanto para las clases de elementos 
$\bcurl$--conformes como para las div--conformes.
Como mostramos en el Capítulo~\ref{auxlabel202}, las funciones
de forma que generan estos espacios son racionales y singulares,
aunque acotadas, en la pirámide de referencia. Por esta 
razón consideramos que nuestro método combinado FE--VE
presenta una ventaja, la de evitar las evaluaciones de 
funciones con esas propiedades en implementaciones en
computadoras.
% section intro (end)