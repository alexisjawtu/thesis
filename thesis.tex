\documentclass[12pt,a4paper,openany,oneside]{book}
\usepackage{geometry}\geometry{top=5cm,bottom=2cm,left=3cm,right=3cm}
%\usepackage{a4wide}

\usepackage[utf8x]{inputenc}
%\usepackage{babel-spanish}
%\usepackage[spanish]{babel}
%\usepackage{graphicx}
%\usepackage{txfonts}
\usepackage{cite}
\usepackage{amsmath}
\usepackage{algorithm,algorithmicx}
\usepackage[noend]{algpseudocode}
\usepackage{amssymb}
\usepackage{amsfonts}
\usepackage{amsthm}
\usepackage{mathtools}
\usepackage{calrsfs}
%\usepackage{helvet}
\usepackage{nicefrac}
\usepackage{palatino}
%\usepackage{charter}
\usepackage[usenames,dvipsnames]{xcolor}
\usepackage{tikz}
\usepackage[retainorgcmds]{IEEEtrantools}
\usepackage{subfig}
\usepackage[format=hang,margin=10pt,font=small,labelfont=bf,labelsep=endash]{caption}
\usepackage{tocbibind}
\usepackage[pdftex,colorlinks=true]{hyperref}

\def\referenceTriangleTikz
{
\begin{tikzpicture}[scale=2]
  \draw (0,0,0) -- (1.2,0,0);
	\draw (0,0,0) -- (0,1.2,0); 
  \draw (1,0,0) -- (0,1,0);
  \draw[white] (1.2,0,0) -- (1.5,0,0);
  \draw[white] (0,0,0) -- (0,0,.7);
\end{tikzpicture}
}

\def\referencePrismTikz
{
\begin{tikzpicture}
  \draw (0, 0, 0) -- (0, 0, 2.4); 
  \draw (0, 0, 0) -- (2.2, 0, 0);
  \draw (0, 0, 0) -- (0, 2.2, 0);
	\draw (0, 0, 2) -- (2, 0, 0);
	\draw (0, 2, 2) -- (2, 2, 0);
	\draw (2, 0, 0) -- (2, 2, 0);
	\draw (0, 0, 2) -- (0, 2, 2);
	\draw (0, 2, 2) -- (0,2,0);
	\draw (0, 2, 0) -- (2,2,0);
\end{tikzpicture}
}

\def\referencePyramidTikz
{
\begin{tikzpicture}[scale=2]
  \coordinate (o) at (0,0,0);
  \coordinate (e1) at (0,0,1);
  \coordinate (e2) at (1,0,0);
  \coordinate (e3) at (0,1,0);

  \draw (o) -- (e1) -- ($(e1)+(e2)$) -- (e2) -- (e3) -- 
      (o)-- (e2);
  \draw (e3) -- (e1);
  \draw (e3) -- ($(e1)+(e2)$);
  \draw[white] (e1) -- ($1.4*(e1)$);
  \draw[white] (e2) -- ($1.2*(e2)$);
\end{tikzpicture}
}

\def\referenceTetrahedronTikz
{
\begin{tikzpicture}[scale=1.1]
  \coordinate (o) at (0,0,0);
  \coordinate (e1) at (0,0,1);
  \coordinate (e2) at (1,0,0);
  \coordinate (e3) at (0,1,0);

  \draw (o) -- (e1) -- (e2) -- (o) -- (e3) -- (e2);
  \draw (e3) -- (e1);

\end{tikzpicture}
}

\def\macroRegularity{  
\begin{figure}[!h]
  \centering
  \subfloat[Tetrahedral Macro\-ele\-ment.]
  {
    \label{macro_tetra_reg}
    \begin{tikzpicture}[scale=1.5]
      \draw[white] (-.8,0,0) -- (1.3,0,0);
      \draw (0,0,0) -- node[midway] {\tiny{\color{black}$\xi_1$}} (0,1,0);
      \draw (0,0,0) -- node[midway] {\tiny{\color{black}$\xi_2$}} (1,0,0);
      \draw (0,1,0) -- (1,0,0);
      \draw[orange] (0,0,0) -- node[midway] {\tiny{\color{black}$\xi_3$}} (0,0,1);
      \draw[white] (0,0,1) -- (0,0,1.5);
      \draw (0,1,0) -- (0,0,1);
      \draw (1,0,0) -- (0,0,1);
      \fill[red] (0,0,1) circle (1pt);
    \end{tikzpicture}
  }\hspace{1.5cm}
  \subfloat[Prismatic Macro\-ele\-ment.]
  {
    \label{macro_prism_reg}
    \begin{tikzpicture}[scale=1.5]
      \fill[black] (0,0,0) circle (1pt);
      \draw (0,0,0) -- node[midway] {\tiny{\color{black}$\xi_1$}} (0,1,0);
      \draw (0,0,0) -- node[midway] {\tiny{\color{black}$\xi_2$}} (1,0,0);
      \draw (0,1,0) -- (1,0,0) -- (1,0,1) -- (0,0,1);
      \draw (0,1,0) -- (0,1,1) -- (0,0,1);
      \draw (1,0,1) -- (0,1,1);
      \draw[orange] (0,0,0) -- node[midway] {\tiny{\color{black}$\xi_3$}} (0,0,1);
      \draw[white] (0,0,1) -- (0,0,1.5);
      \draw[white] (1,0,0) -- (1.3,0,0);
    \end{tikzpicture}
  }
  \caption{Macroelements.}
\end{figure}
}

\def\unitTangentsPyramid
{
\begin{tikzpicture}[scale=2,post/.style={->, shorten >=1pt, >=stealth', semithick}]
  \coordinate (o) at (0,0,0);
  \coordinate (e1) at (0,0,1);
  \coordinate (e2) at (1,0,0);
  \coordinate (e3) at (0,1,0);
  \coordinate (e4) at ($(e1)+(e2)$);

  \coordinate (a1start) at ($.2*(e1)$);
  \coordinate (a1end)   at ($.5*(e1)$);
  \coordinate (a2start)   at ($.8*(e1)+.2*(e4)$);
  \coordinate (a2end)   at ($.5*(e1)+.5*(e4)$);
  \coordinate (a3start)   at ($.8*(e2)+.2*(e4)$);
  \coordinate (a3end)   at ($.5*(e2)+.5*(e4)$);
  \coordinate (a4start) at ($.17*(e2)$);
  \coordinate (a4end)   at ($.4*(e2)$);
  \coordinate (a5start) at ($.2*(e3)$);
  \coordinate (a5end)   at ($.5*(e3)$);
  \coordinate (a6start) at ($.2*(e3)+.8*(e1)$);
  \coordinate (a6end)   at ($.5*(e3)+.5*(e1)$);
  \coordinate (a7start) at ($.2*(e3)+.8*(e2)$);
  \coordinate (a7end)   at ($.5*(e3)+.5*(e2)$);
  \coordinate (a8start) at ($.2*(e3)+.8*(e4)$);
  \coordinate (a8end)   at ($.5*(e3)+.5*(e4)$);

  \draw (o) -- (e1) -- ($(e1)+(e2)$) -- (e2) -- (e3) -- 
      (o)-- (e2);
  \draw (e3) -- (e1);
  \draw (e3) -- (e4);
  \foreach \i in {1,...,8}{
    \draw [post,very thick] (a\i start) -- (a\i end);
  }

\end{tikzpicture}
}

\def\unitTangentsPrism
{
\begin{tikzpicture}[scale=2,post/.style={->, shorten >=1pt, >=stealth', semithick}]
  \coordinate (o) at (0,0,0);
  \coordinate (e1) at (0,0,1);
  \coordinate (e2) at (1,0,0);
  \coordinate (e3) at (0,1,0);
  \coordinate (e4) at ($(e1)+(e3)$);
  \coordinate (e5) at ($(e2)+(e3)$);

  \coordinate (a1start) at ($.2*(e1)$);
  \coordinate (a1end)   at ($.5*(e1)$);
  \coordinate (a2start)   at ($.2*(e2)$);
  \coordinate (a2end)   at ($.5*(e2)$);

  \coordinate (a3start)   at ($.2*(e3)$);
  \coordinate (a3end)   at ($.5*(e3)$);

  \coordinate (a4start) at ($.83*(e3)+.17*(e4)$);
  \coordinate (a4end)   at ($.5*(e3)+.5*(e4)$);
  
  \coordinate (a5start) at ($.8*(e3)+.2*(e5)$);
  \coordinate (a5end)   at ($.5*(e3)+.5*(e5)$);
  
  \coordinate (a6start) at ($.2*(e4)+.8*(e1)$);
  \coordinate (a6end)   at ($.5*(e4)+.5*(e1)$);
  \coordinate (a7start) at ($.2*(e5)+.8*(e2)$);
  \coordinate (a7end)   at ($.5*(e5)+.5*(e2)$);
  \coordinate (a8start) at ($.2*(e4)+.8*(e5)$);
  \coordinate (a8end)   at ($.5*(e4)+.5*(e5)$);
  \coordinate (a9start) at ($.2*(e1)+.8*(e2)$);
  \coordinate (a9end)   at ($.5*(e1)+.5*(e2)$);

  \draw (o) -- (e1) -- (e2) -- (o) -- (e3) -- (e4) -- (e5) -- (e3);
  \draw (e4) -- (e1);
  \draw (e5) -- (e2);

  \foreach \i in {1,...,9}{
    \draw [post, very thick] (a\i start) -- (a\i end);
  }
\end{tikzpicture}
}

\newcommand{\polygon}[6]{
\def\m{#1} % piso y tapa como m-'agonos
\def\n{#2} % (n + 1) puntos en cada arista del tri'angulo
\def\mu{#3} % graduaci'on
\def\init{#4} % hacia d'onde queremos que apunte la boca

\begin{scope}[shift={(0,#5,0)}]   %% --->>> perspectiva
\coordinate (point0) at (0,0,0);

\pgfmathparse{\m + \init - 2} \let\last\pgfmathresult

%%% TODO perspectiva: hacer unos defs con los canonicos y el entero 1, 2, 3 de entrada

\foreach \r in {\init,...,\last} {
  \coordinate (point1) at ({cos((\r-2)*\twoPi/\m)},0,{sin((\r-2)*\twoPi/\m)});  %% -->> perspectiva
  \coordinate (point2) at ({cos((\r-1)*\twoPi/\m)},0,{sin((\r-1)*\twoPi/\m)});
  \draw[color=#6] (point1) -- (point2);

  \pgfmathparse{\n-1} \let\to\pgfmathresult
  \foreach \i in {0,...,\to} {  
    \pgfmathparse{pow(\i/\n,1/\mu)} \let\rad\pgfmathresult
    \draw[color=#6] ($\rad*(point1)$) -- ($\rad*(point2)$);
    
    \pgfmathparse{\i/\n}    \let\alfa\pgfmathresult
    \pgfmathparse{(\n-\i)/\n}   \let\beta\pgfmathresult
    
    \pgfmathparse{\n-\i-1} \let\ni\pgfmathresult
    \foreach \j in {0,...,\ni} {
      \pgfmathparse{(\i/\n)*pow((\i+\j)/\n,1/\mu-1)} \let\alfa\pgfmathresult
      \pgfmathparse{(\j/\n)*pow((\i+\j)/\n,1/\mu-1)} \let\beta\pgfmathresult

      \pgfmathparse{(\i/\n)*pow((\i+(\j+1))/\n,1/\mu-1)}    \let\alfaSuc\pgfmathresult
      \pgfmathparse{((\j+1)/\n)*pow((\i+(\j+1))/\n,1/\mu-1)}  \let\betaSuc\pgfmathresult
      
      \draw[color=#6] ($\alfa*(point1) + \beta*(point2)$) -- 
          ($\alfaSuc*(point1) + \betaSuc*(point2)$);
      \draw[color=#6] ($\alfa*(point2) + \beta*(point1)$) -- 
          ($\alfaSuc*(point2) + \betaSuc*(point1)$);
    };
  }
}
\end{scope}
}

\newcommand{\prismaticMacroelement}[5]{
\def\m{#1} % piso y tapa como m-'agonos
\def\n{#2} % (n + 1) puntos en cada arista del tri'angulo
\def\mu{#3} % graduaci'on
\def\init{#4} % hacia d'onde queremos que apunte la boca

\foreach \l in {0,...,\n} {
  \pgfmathparse{-\l*2/\n} \let\L\pgfmathresult
  %\pgfmathparse{\m + \init - 2} \let\last\pgfmathresult
  %%% TODO perspectiva: hacer unos defs con los canonicos y el entero 1, 2, 3 de entrada
  %\pgfmathparse{\n + 1} \let\N\pgfmathresult
  \begin{scope}[shift={(0,\L,0)}]   %% --->>> altura
    \coordinate (point1) at ({cos((\init-2)*\twoPi/\m)},0,{sin((\init-2)*\twoPi/\m)});  %% -->> perspectiva
    \coordinate (point2) at ({cos((\init-1)*\twoPi/\m)},0,{sin((\init-1)*\twoPi/\m)});
    \draw[color=#5] (point1) -- (point2);

    \pgfmathparse{\n-1} \let\to\pgfmathresult
    \foreach \i in {0,...,\to} {  
      \pgfmathparse{pow(\i/\n,1/\mu)} \let\rad\pgfmathresult
      \draw[color=#5] ($\rad*(point1)$) -- ($\rad*(point2)$);
      
      \pgfmathparse{\n-\i-1} \let\ni\pgfmathresult
      \foreach \j in {0,...,\ni} {
        \pgfmathparse{(\i/\n)*pow((\i+\j)/\n,1/\mu-1)} \let\alfa\pgfmathresult
        \pgfmathparse{(\j/\n)*pow((\i+\j)/\n,1/\mu-1)} \let\beta\pgfmathresult

        \pgfmathparse{(\i/\n)*pow((\i+(\j+1))/\n,1/\mu-1)}    \let\alfaSuc\pgfmathresult
        \pgfmathparse{((\j+1)/\n)*pow((\i+(\j+1))/\n,1/\mu-1)}  \let\betaSuc\pgfmathresult
        
        \draw[color=#5] ($\alfa*(point1) + \beta*(point2)$) -- 
            ($\alfaSuc*(point1) + \betaSuc*(point2)$);
        \draw[color=#5] ($\alfa*(point2) + \beta*(point1)$) -- 
            ($\alfaSuc*(point2) + \betaSuc*(point1)$);
      };
    }
  \end{scope}
  }
  \coordinate (point1) at ({cos((\init-2)*\twoPi/\m)},0,{sin((\init-2)*\twoPi/\m)});
  \coordinate (point2) at ({cos((\init-1)*\twoPi/\m)},0,{sin((\init-1)*\twoPi/\m)}); 
  \pgfmathparse{\n} \let\to\pgfmathresult
  \foreach \i in {0,...,\to} {  
    \pgfmathparse{\n-\i} \let\ni\pgfmathresult
    \foreach \j in {0,...,\ni} {
      \pgfmathparse{(\i/\n)*pow((\i+\j)/\n,1/\mu-1)} \let\alfa\pgfmathresult
      \pgfmathparse{(\j/\n)*pow((\i+\j)/\n,1/\mu-1)} \let\beta\pgfmathresult
      \pgfmathparse{(\i/\n)*pow((\i+(\j+1))/\n,1/\mu-1)}    \let\alfaSuc\pgfmathresult
      \pgfmathparse{((\j+1)/\n)*pow((\i+(\j+1))/\n,1/\mu-1)}  \let\betaSuc\pgfmathresult        
      \draw[color=#5] ($\alfa*(point1) + \beta*(point2)$) -- 
        ($\alfa*(point1)+ \beta*(point2) + (0,-2,0)$); 
   };
  };
  \draw[red] (0,0,0) -- (0,-2,0);
}

\def\tableErrorsUniformCylinder{
\begin{table}[!ht]
    \centering
    \caption{Octogonal $\omega = 3\pi/2$, $\gamma = 1$}
    \label{table_errors_1}
    \begin{IEEEeqnarraybox}[\IEEEeqnarraystrutmode\IEEEeqnarraystrutsizeadd{3pt}{1pt}]
    {v / c / v / c / v / c / v / r / v / c / v / c / v}
        \IEEEeqnarrayrulerow\\
        & n && \text{Nel} && \boldsymbol{u} && p  
            && \frac{\Delta_i \log(e_{\boldsymbol{u}})}{- \Delta_i \log(n)}
            && \frac{\Delta_i \log(e_{p})}{- \Delta_i \log(n)} & 
        \IEEEeqnarraystrutsizeadd{7pt}{7pt}\\\IEEEeqnarrayrulerow\\
        & 6  &&1512      && 0.632043 &&0.205469  &&                  &&                & \\ \IEEEeqnarrayrulerow\\
        & 10 &&7000      && 0.324654 &&0.123666  && 1.3041582077     && 0.993902349971 & \\ \IEEEeqnarrayrulerow\\
        & 20 &&56000     && 0.243846 &&0.064251  && 0.4129326777     && 0.94465809513  & \\ \IEEEeqnarrayrulerow\\
        & 30 &&189000    && 0.130700 &&0.042238  && 1.5380663968     && 1.03455757271  & \\ \IEEEeqnarrayrulerow\\
        & 40 &&448000    && 0.108696 &&0.031693  && 0.6408102671     && 0.998409289236 & \\ \IEEEeqnarrayrulerow\\
        & 50 &&875000    && 0.091443 &&0.025360  && 0.7745649227     && 0.999010302685 & \\ \IEEEeqnarrayrulerow\\
        & 60 &&1512000   && 0.079951 &&0.021128  && 0.7366209607     && 1.00138435512  & \\ \IEEEeqnarrayrulerow
    \end{IEEEeqnarraybox}
\end{table}
}

\def\tableErrorsAnisoCylinder{
    \begin{table}[!ht]
    \centering
    \caption{Octogonal $\omega = 3\pi/2$, $\gamma = 1.5$}
    \label{table_errors_2}
    \begin{IEEEeqnarraybox}[\IEEEeqnarraystrutmode\IEEEeqnarraystrutsizeadd{3pt}{1pt}]
    {v / c / v / c / v / c / v / r / v / c / v / c / v}
        \IEEEeqnarrayrulerow\\
        & n && \text{Nel} && \boldsymbol{u} && p  
            && \frac{\Delta_i \log(e_{\boldsymbol{u}})}{- \Delta_i \log(n)}
            && \frac{\Delta_i \log(e_{p})}{- \Delta_i \log(n)} & 
        \IEEEeqnarraystrutsizeadd{7pt}{7pt}\\\IEEEeqnarrayrulerow\\
        & 6  && 1512     && 0.612728631191887  && 0.206872796008702  &&                &&                    & \\ \IEEEeqnarrayrulerow\\
        & 10 && 7000     && 0.292898549721497  && 0.124441257916811  && 1.44490763867  && 0.994997682366827  & \\ \IEEEeqnarrayrulerow\\
        & 20 && 56000    && 0.218083938394525  && 0.064388585160060  && 0.42551752854  && 0.950588029132187  & \\ \IEEEeqnarrayrulerow\\
        & 30 && 189000   && 0.094069683304253  && 0.042391308867588  && 2.07377698103  && 1.030897599534560  & \\ \IEEEeqnarrayrulerow\\
        & 40 && 448000   && 0.074568571275959  && 0.031808637905069  && 0.80754666264  && 0.998343324300149  & \\ \IEEEeqnarrayrulerow\\
        & 50 && 875000   && 0.057800258295080  && 0.025453468142382  && 1.14153367162  && 0.998845260486221  & \\ \IEEEeqnarrayrulerow\\
        & 60 && 1512000  && 0.047035601433131  && 0.021210116487177  && 1.13035702302  && 1.000286247305515  & \\ \IEEEeqnarrayrulerow\\
        & 70 && 2401000  && 0.040214794437822  && 0.018179235090025  && 1.01634221582  && 1.000308576087509  & \\ \IEEEeqnarrayrulerow
    \end{IEEEeqnarraybox}
\end{table}
}

\def\facesOfPrism{
\begin{table}[!ht]
    \centering  
    \caption{Notation for the faces and positive normals of the reference prism.}
    \label{prismNotationTableFaces}
    \begin{IEEEeqnarraybox*}
      [\IEEEeqnarraystrutmode
      \IEEEeqnarraystrutsizeadd{2pt}{6pt}]{v/c/x/c/x/c/x/v/x/c/x/c/x/c/v/}
        \IEEEeqnarrayrulerow\\
        \IEEEeqnarrayseprow[5pt]\\
          & \hat f_1 && \subseteq &&  \{\hat x_1 = 0 \}            && && \hat{\bn}_1 && = && (-1,0,0)' & \\
        \IEEEeqnarrayrulerow\\
        \IEEEeqnarrayseprow[5pt]\\
          & \hat f_2 && \subseteq &&  \{\hat x_2 = 0 \}            && && \hat{\bn}_2 && = && (0,-1,0)' &\\
        \IEEEeqnarrayrulerow\\
        \IEEEeqnarrayseprow[5pt]\\
          & \hat f_3 && \subseteq &&  \{\hat x_3 = 0 \} && && \hat{\bn}_3 && = && (0,0,-1)' &\\
        \IEEEeqnarrayrulerow\\
        \IEEEeqnarrayseprow[5pt]\\
          & \hat f_4 && \subseteq &&  \{\hat x_3 = 1 \} && && \hat{\bn}_4 && = && (0,0,1)' &\\
        \IEEEeqnarrayrulerow\\
        \IEEEeqnarrayseprow[5pt]\\
          & \hat f_5 && \subseteq &&  \{\hat x_1+\hat x_2 = 1\} && && \hat{\bn}_5 && = && 2^{-\nicefrac{1}{2}}(1,1,0)' &\\
        \IEEEeqnarrayrulerow
    \end{IEEEeqnarraybox*}
\end{table}
}

\def\edgesOfPrism{
\begin{table}[!ht]
    \centering  
    \caption{Notation for the edges and positive tangents of the reference prism.}
    \label{prismNotationTableEdges}
    \begin{IEEEeqnarraybox*}
      [\IEEEeqnarraystrutmode
      \IEEEeqnarraystrutsizeadd{2pt}{6pt}]{v/c/x/c/x/c/x/v/x/c/x/c/x/c/v/}
        \IEEEeqnarrayrulerow\\
        \IEEEeqnarrayseprow[5pt]\\
   & \hat \be_1 && = && \{(\hat x_1,0,0)^t\,:\,0\leqslant\hat x_1\leqslant 1\} && && \hat \btau_1 && = && (1,0,0)' & \\
        \IEEEeqnarrayrulerow\\
        \IEEEeqnarrayseprow[5pt]\\
   & \hat \be_2 && = && \{(0,\hat x_2,0)^t\,:\,0\leqslant\hat x_2\leqslant 1\} && && \hat \btau_2 && = && (0,1,0)' & \\
        \IEEEeqnarrayrulerow\\
        \IEEEeqnarrayseprow[5pt]\\
   & \hat \be_3 && = && \{(0,0,\hat x_3)^t\,:\,0\leqslant\hat x_3\leqslant 1\} && && \hat \btau_3 && = && (0,0,1)' & \\
        \IEEEeqnarrayrulerow\\
        \IEEEeqnarrayseprow[5pt]\\
   & \hat \be_4 && = && \{(\hat x_1,0,1)^t\,:\,0\leqslant\hat x_1\leqslant 1\} && && \hat \btau_4 && = && (0,0,1)' & \\
        \IEEEeqnarrayrulerow\\
        \IEEEeqnarrayseprow[5pt]\\
   & \hat \be_5 && = && \{(0,\hat x_2,1)^t\,:\,0\leqslant\hat x_2\leqslant 1\} && && \hat \btau_5 && = && (0,0,1)' & \\
        \IEEEeqnarrayrulerow\\
        \IEEEeqnarrayseprow[5pt]\\
   & \hat \be_6 && = && \{(1,0,\hat x_3)^t\,:\,0\leqslant\hat x_3\leqslant 1\} && && \hat \btau_6 && = && (0,0,1)' & \\
        \IEEEeqnarrayrulerow\\
        \IEEEeqnarrayseprow[5pt]\\
   & \hat \be_7 && = && \{(0,1,\hat x_3)^t\,:\,0\leqslant\hat x_3\leqslant 1\} && && \hat \btau_7 && = && (0,0,1)' & \\
        \IEEEeqnarrayrulerow\\
        \IEEEeqnarrayseprow[5pt]\\
   & \hat \be_8 && = && \{(\hat x_1,1-\hat x_1,1)^t\,:\,0\leqslant\hat x_1\leqslant 1\} && && \hat \btau_8 && = && 2^{-\nicefrac{1}{2}}(1,-1,0)' & \\
        \IEEEeqnarrayrulerow\\
        \IEEEeqnarrayseprow[5pt]\\
   & \hat \be_9 && = && \{(\hat x_1,1-\hat x_1,0)^t\,:\,0\leqslant\hat x_1\leqslant 1\} && && \hat \btau_9 && = && 2^{-\nicefrac{1}{2}}(1,-1,0)' & \\
        \IEEEeqnarrayrulerow  
    \end{IEEEeqnarraybox*}
\end{table}
}

\def\edgeShapeTable{\begin{table}[!ht]
    \centering  
    \caption{{Edge shape functions} on the reference pyramid}
    \label{shape_edge_table}
    \begin{IEEEeqnarraybox*}
    [\IEEEeqnarraystrutmode
    \IEEEeqnarraystrutsizeadd{2pt}{25pt}]{x/c/x/c/x/c/x}
        \IEEEeqnarrayseprow[5pt]\\
        &\IEEEeqnarraymulticol{6}{c}{
                {%\tiny
                {\scriptstyle\hat\bgamma_1} = 
                \left(
                    \begin{array}{c}
                        {1-z-y} \\[5pt]             
                        0 \\[5pt]
                        x-\frac{xy}{1-z}               
                    \end{array}
                \right)}\;\;
                {%\tiny
                {\scriptstyle\hat\bgamma_2} = 
                \left(
                    \begin{array}{c}
                        0 \\[5pt]
                        x \\[5pt]                
                        \frac{xy}{1-z}               
                    \end{array}
                \right)}\;\;
                {%\tiny
                {\scriptstyle\hat\bgamma_3} = 
                \left(
                    \begin{array}{c}
                        y \\[5pt]
                        0 \\[5pt]                
                        \frac{xy}{1-z}               
                    \end{array}
                \right)}\;\;
        {%\tiny
        {\scriptstyle\hat\bgamma_4} = 
        \left(
                    \begin{array}{c}
                        0 \\[8pt]                
                        {1-z-x} \\[8pt]
                        y-\frac{xy}{1-z}  
                    \end{array}
                \right)}}\\
        \IEEEeqnarrayseprow[5pt]\\
        &\IEEEeqnarraymulticol{6}{c}{
        {%\tiny
        {\scriptstyle\hat\bgamma_5} = 
                \left(
                        \begin{array}{c}
                            z-\frac{yz}{1-z} \\[8pt]               
                            z-\frac{xz}{1-z} \\[8pt]               
                            1-x-y+\frac{xy}{1-z}-\frac{xyz}{(1-z)^2}               
                        \end{array}
               \right)}\;\;
                {%\tiny
                {\scriptstyle\hat\bgamma_6} = 
                \left(
                    \begin{array}{c}
                        -z+\frac{yz}{1-z} \\[8pt]               
                        \frac{xz}{1-z} \\[8pt]               
                        x-\frac{xy}{1-z}+\frac{xyz}{(1-z)^2}               
                    \end{array}
                        \right)}}\\
        \IEEEeqnarrayseprow[5pt]\\
        &\IEEEeqnarraymulticol{6}{c}{
        {%\tiny
        {\scriptstyle\hat\bgamma_7} = 
                \left(
                        \begin{array}{c}
                            \frac{yz}{1-z} \\[8pt]               
                            -z+\frac{xz}{1-z} \\[8pt]               
                            y-\frac{xy}{1-z}+\frac{xyz}{(1-z)^2}               
                        \end{array}
               \right)}\;\;
        {%\tiny
                {\scriptstyle\hat\bgamma_8} = 
                \left(
                    \begin{array}{c}
                        -\frac{yz}{1-z} \\[8pt]               
                        -\frac{xz}{1-z} \\[8pt]               
                        \frac{xy}{1-z}-\frac{xyz}{(1-z)^2}               
                    \end{array}
                        \right)}}
    \end{IEEEeqnarraybox*}
\end{table}}

\def\faceShapeTable{
 \begin{table}[!ht]
    \centering  
    \caption{{Face shape functions} on the reference pyramid}
    \label{shape_face_table}
    \begin{IEEEeqnarraybox*}
    [\IEEEeqnarraystrutmode
    \IEEEeqnarraystrutsizeadd{2pt}{25pt}]{x/c/x/c/x/c/x}
        \IEEEeqnarrayseprow[5pt]\\
        &\IEEEeqnarraymulticol{5}{c}{
                \raisebox{-15pt}{}\;\;
                {%\tiny
                {\scriptstyle\hat\bz_1} = 
                \left(
                        \begin{array}{c}
                            -\frac{xz}{1-z} \\[8pt]             
                            y-2+\frac{y}{1-z} \\[8pt]
                            z               
                        \end{array}
                        \right)}\;\;
                {%\tiny
                {\scriptstyle\hat\bz_2} = 
                \left(
                            \begin{array}{c}
                                x-2+\frac{x}{1-z} \\[8pt]
                                -\frac{yz}{1-z} \\[8pt]                
                                z               
                            \end{array}
                        \right)}}&\\
        \IEEEeqnarrayseprow[10pt]\\
        &
        {%\tiny
        {\scriptstyle\hat\bz_3} = 
        \left(
                    \begin{array}{c}
                        x+\frac{x}{1-z} \\[8pt]
                        -\frac{yz}{1-z} \\[8pt]                
                        z               
                    \end{array}
                \right)}&&
        {%\tiny
        {\scriptstyle\hat\bz_4} = 
        \left(
                    \begin{array}{c}
                        -\frac{xz}{1-z} \\[8pt]                
                        y+\frac{y}{1-z} \\[8pt]
                        z               
                    \end{array}
                \right)}&&
        {%\tiny
        {\scriptstyle\hat\bz_5} = 
        \left(
                    \begin{array}{c}
                        x \\[8pt]               
                        y \\[8pt]
                        z-1                 
                    \end{array}
                \right)}&\\
        \IEEEeqnarrayseprow[10pt]
    \end{IEEEeqnarraybox*}
\end{table}}

\def\facesOfPyramid{
\begin{table}[!ht]
    \centering  
    \caption{Notation for the faces and positive normals of the
    reference pyramid.}
    \label{pyramidNotationTableFaces}
    \begin{IEEEeqnarraybox*}
      [\IEEEeqnarraystrutmode
      \IEEEeqnarraystrutsizeadd{2pt}{6pt}]{v/c/x/c/x/c/x/v/x/c/x/c/x/c/v/}
        \IEEEeqnarrayrulerow\\
        \IEEEeqnarrayseprow[5pt]\\
          & \hat f_1 && \subseteq &&  \{\hat x_2 = 0 \}            && && \hat{\bn}_1 && = && (0,-1,0)' & \\
        \IEEEeqnarrayrulerow\\
        \IEEEeqnarrayseprow[5pt]\\
          & \hat f_2 && \subseteq &&  \{\hat x_1 = 0 \}            && && \hat{\bn}_2 && = && (-1,0,0)' &\\
        \IEEEeqnarrayrulerow\\
        \IEEEeqnarrayseprow[5pt]\\
          & \hat f_3 && \subseteq &&  \{\hat x_1 + \hat x_3 = 1 \} && && \hat{\bn}_3 && = && 2^{-\nicefrac{1}{2}}(1,0,1)' &\\
        \IEEEeqnarrayrulerow\\
        \IEEEeqnarrayseprow[5pt]\\
          & \hat f_4 && \subseteq &&  \{\hat x_2 + \hat x_3 = 1 \} && && \hat{\bn}_4 && = && 2^{-\nicefrac{1}{2}}(0,1,1)' &\\
        \IEEEeqnarrayrulerow\\
        \IEEEeqnarrayseprow[5pt]\\
          & \hat f_5 && \subseteq &&  \{\hat x_3 = 0\}             && && \hat{\bn}_5 && = && (0,0,-1)' &\\
        \IEEEeqnarrayrulerow
    \end{IEEEeqnarraybox*}
\end{table}}

\def\edgesOfPyramid{
\begin{table}[!ht]
    \centering  
    \caption{Notation for the edges and positive tangents of the
    reference pyramid.}
    \label{pyramidNotationTableEdges}
    \begin{IEEEeqnarraybox*}
      [\IEEEeqnarraystrutmode
      \IEEEeqnarraystrutsizeadd{2pt}{6pt}]{v/c/x/c/x/c/x/v/x/c/x/c/x/c/v/}
        \IEEEeqnarrayrulerow\\
        \IEEEeqnarrayseprow[5pt]\\
   & \hat \be_1 && = && \{(\hat x_1,0,0)^t\,:\,0\leqslant\hat x_1\leqslant 1\} && && \hat \btau_1 && = && (1,0,0)' & \\
        \IEEEeqnarrayrulerow\\
        \IEEEeqnarrayseprow[5pt]\\
   & \hat \be_2 && = && \{(1,\hat x_2,0)^t\,:\,0\leqslant\hat x_2\leqslant 1\} && && \hat \btau_2 && = && (0,1,0)' & \\
        \IEEEeqnarrayrulerow\\
        \IEEEeqnarrayseprow[5pt]\\
   & \hat \be_3 && = && \{(\hat x_1,1,0)^t\,:\,0\leqslant\hat x_1\leqslant 1\} && && \hat \btau_3 && = && (-1,0,0)' & \\
        \IEEEeqnarrayrulerow\\
        \IEEEeqnarrayseprow[5pt]\\
   & \hat \be_4 && = && \{(0,\hat x_2,0)^t\,:\,0\leqslant\hat x_2\leqslant 1\} && && \hat \btau_4 && = && (0,1,0)' & \\
        \IEEEeqnarrayrulerow\\
        \IEEEeqnarrayseprow[5pt]\\
   & \hat \be_5 && = && \{(0,0,\hat x_3)^t\,:\,0\leqslant\hat x_3\leqslant 1\} && && \hat \btau_5 && = && (0,0,1)' & \\
        \IEEEeqnarrayrulerow\\
        \IEEEeqnarrayseprow[5pt]\\
   & \hat \be_6 && = && \{(1-\hat x_3,0,\hat x_3)^t\,:\,0\leqslant\hat x_3\leqslant 1\} && && \hat \btau_6 && = && 2^{-\nicefrac{1}{2}}(-1,0,1)' & \\
        \IEEEeqnarrayrulerow\\
        \IEEEeqnarrayseprow[5pt]\\
   & \hat \be_7 && = && \{(0,1-\hat x_3,\hat x_3)^t\,:\,0\leqslant\hat x_3\leqslant 1\} && && \hat \btau_7 && = && 2^{-\nicefrac{1}{2}}(0,-1,1)' & \\
        \IEEEeqnarrayrulerow\\
        \IEEEeqnarrayseprow[5pt]\\
   & \hat \be_8 && = && \{(1-\hat x_3,1-\hat x_3,\hat x_3)^t\,:\,0\leqslant\hat x_3\leqslant 1\} && && \hat \btau_8 && = && 3^{-\nicefrac{1}{2}}(-1,-1,1) & \\
        \IEEEeqnarrayrulerow
    \end{IEEEeqnarraybox*}
\end{table}}
\def\rescaledPrismTikz
{
\begin{figure}[!h]
	\centering
		\begin{tikzpicture}[scale=.4]
		  \draw[->] (0, 0) -- (0, 0, 3);	
		  \draw[->] (0, 0) -- (3,0);
		  \draw[->] (0, 0) --  (0, 5, 0);
		  \draw (0, 0, 1) -- (2, 0, 0); 
		  \draw (0, 3.5, 1) -- (2, 3.5, 0); 
		  \draw (2, 0, 0) -- node[label={right:$_{_{h_3}}$}] { } (2, 3.5, 0);
		  \draw (0, 0, 1) -- (0, 3.5, 1); 
		  \draw (0,3.5,1) -- node[label={left:$_{_{h_1}}$}] { }  (0,3.5,0) ;
		  \draw (0,3.5,0) -- node[label={above:$_{_{h_2}}$}] { } (2,3.5,0);
		\end{tikzpicture}
	\caption{Rescaled Prism}
	\label{rescaled_prism}		
\end{figure}
}

\def\tetrahedralmacroelementA{
\begin{tikzpicture}[scale=1.8]	\draw (0.0,1.0,0.0);
	\draw (0.0,0.0,0.0) -- (0.0,1.0,0.0);
	\draw (0.0,0.0,0.0) -- (1.0,0.0,0.0);
	\draw (1.0,0.0,0.0);
	\draw (0.0,0.0,0.0);
	\draw (0.0,1.0,0.0) -- (1.0,0.0,0.0);
	\fill[red] (0.0,0.0,1.0) circle (1pt);
	\draw[orange] (0.0,0.0,0.0) -- (0.0,0.0,1.0);
	\draw (0.0,1.0,0.0) -- (0.0,0.0,1.0);
	\draw (1.0,0.0,0.0) -- (0.0,0.0,1.0);
\end{tikzpicture}}

\def\tetrahedralmacroelementB{
\begin{tikzpicture}[scale=1.8]	\draw (0.0,0.3869,0.0) -- (0.0,0.0,0.0);
	\draw (0.0,0.0,0.0) -- (0.0,0.3869,0.0) -- (0.0,1.0,0.0);
	\draw (0.0,0.0,0.0) -- (0.3869,0.0,0.0) -- (0.0,0.0,0.0);
	\draw (0.0,0.3869,0.0) -- (0.5,0.5,0.0);
	\draw (0.3869,0.0,0.0) -- (0.5,0.5,0.0);
	\draw (0.0,0.0,0.0) -- (0.3869,0.0,0.0) -- (1.0,0.0,0.0);
	\draw (1.0,0.0,0.0);
	\fill[red] (0.0,0.0,1.0) circle (1pt);
	\draw (0.0,0.3869,0.0) -- (0.3869,0.0,0.0);
	\draw (0.0,1.0,0.0) -- (0.5,0.5,0.0) -- (1.0,0.0,0.0);
	\draw (0.0,0.0,0.6130) -- (0.0,0.0,0.0);
	\draw (0.0,0.3869,0.6130);
	\draw (0.0,0.0,0.6130) -- (0.0,0.3869,0.6130);
	\draw (0.0,0.0,0.6130) -- (0.3869,0.0,0.6130);
	\draw (0.3869,0.0,0.6130);
	\draw (0.0,0.0,0.6130);
	\draw (0.0,0.3869,0.6130) -- (0.3869,0.0,0.6130);
	\draw[orange] (0.0,0.0,0.0) -- (0.0,0.0,0.6130) -- (0.0,0.0,1.0);
	\draw (0.0,0.3869,0.0) -- (0.0,0.3869,0.6130);
	\draw (0.0,1.0,0.0) -- (0.0,0.3869,0.6130) -- (0.0,0.0,1.0);
	\draw (1.0,0.0,0.0) -- (0.3869,0.0,0.6130) -- (0.0,0.0,1.0);
	\draw (0.3869,0.0,0.0) -- (0.3869,0.0,0.6130);
	\draw (0.5,0.5,0.0) -- (0.3869,0.0,0.6130);
	\draw (0.5,0.5,0.0) -- (0.0,0.3869,0.6130);
\end{tikzpicture}}

\def\tetrahedralmacroelementC{
\begin{tikzpicture}[scale=1.8]	\draw (0.0,0.2220,0.0) -- (0.0,0.0,0.0) -- (0.0,0.0,0.0);
	\draw (0.0,0.5738,0.0) -- (0.0,0.0,0.0);
	\draw (0.0,0.0,0.0) -- (0.0,0.2220,0.0) -- (0.0,0.5738,0.0) -- (0.0,1.0,0.0);
	\draw (0.0,0.0,0.0) -- (0.2220,0.0,0.0) -- (0.0,0.0,0.0) -- (0.0,0.0,0.0);
	\draw (0.0,0.2220,0.0) -- (0.2869,0.2869,0.0) -- (0.0,0.0,0.0);
	\draw (0.0,0.5738,0.0) -- (0.3333,0.6666,0.0);
	\draw (0.2220,0.0,0.0) -- (0.2869,0.2869,0.0) -- (0.3333,0.6666,0.0);
	\draw (0.0,0.0,0.0) -- (0.2220,0.0,0.0) -- (0.5738,0.0,0.0) -- (0.0,0.0,0.0);
	\draw (0.0,0.2220,0.0) -- (0.2869,0.2869,0.0) -- (0.6666,0.3333,0.0);
	\draw (0.5738,0.0,0.0) -- (0.6666,0.3333,0.0);
	\draw (0.0,0.0,0.0) -- (0.2220,0.0,0.0) -- (0.5738,0.0,0.0) -- (1.0,0.0,0.0);
	\draw (0.0,0.2220,0.0) -- (0.2220,0.0,0.0);
	\draw (0.0,0.5738,0.0) -- (0.2869,0.2869,0.0) -- (0.5738,0.0,0.0);
	\draw (0.0,1.0,0.0) -- (0.3333,0.6666,0.0) -- (0.6666,0.3333,0.0) -- (1.0,0.0,0.0);
	\draw (0.0,0.0,0.4261) -- (0.0,0.0,0.0) -- (0.0,0.0,0.0);
	\draw (0.0,0.2220,0.4261) -- (0.0,0.0,0.0);
	\draw (0.0,0.0,0.4261) -- (0.0,0.2220,0.4261) -- (0.0,0.5738,0.4261);
	\draw (0.0,0.0,0.4261) -- (0.2220,0.0,0.4261) -- (0.0,0.0,0.0);
	\draw (0.0,0.2220,0.4261) -- (0.2869,0.2869,0.4261);
	\draw (0.2220,0.0,0.4261) -- (0.2869,0.2869,0.4261);
	\draw (0.0,0.0,0.4261) -- (0.2220,0.0,0.4261) -- (0.5738,0.0,0.4261);
	\draw (0.0,0.2220,0.4261) -- (0.2220,0.0,0.4261);
	\draw (0.0,0.5738,0.4261) -- (0.2869,0.2869,0.4261) -- (0.5738,0.0,0.4261);
	\draw (0.0,0.0,0.7779) -- (0.0,0.0,0.0);
	\draw (0.0,0.0,0.7779) -- (0.0,0.2220,0.7779);
	\draw (0.0,0.0,0.7779) -- (0.2220,0.0,0.7779);
	\draw (0.0,0.2220,0.7779) -- (0.2220,0.0,0.7779);
	\draw[orange] (0.0,0.0,0.0) -- (0.0,0.0,0.4261) -- (0.0,0.0,0.7779) -- (0.0,0.0,1.0);
	\draw (0.0,0.2220,0.0) -- (0.0,0.2220,0.4261) -- (0.0,0.2220,0.7779);
	\draw (0.0,0.5738,0.0) -- (0.0,0.5738,0.4261);
	\draw (0.0,1.0,0.0) -- (0.0,0.5738,0.4261) -- (0.0,0.2220,0.7779) -- (0.0,0.0,1.0);
	\draw (1.0,0.0,0.0) -- (0.5738,0.0,0.4261) -- (0.2220,0.0,0.7779) -- (0.0,0.0,1.0);
	\draw (0.2220,0.0,0.0) -- (0.2220,0.0,0.4261) -- (0.2220,0.0,0.7779);
	\draw (0.2869,0.2869,0.0) -- (0.2869,0.2869,0.4261);
	\draw (0.3333,0.6666,0.0) -- (0.2869,0.2869,0.4261) -- (0.2220,0.0,0.7779);
	\draw (0.6666,0.3333,0.0) -- (0.2869,0.2869,0.4261) -- (0.0,0.2220,0.7779);
	\draw (0.5738,0.0,0.0) -- (0.5738,0.0,0.4261);
	\draw (0.6666,0.3333,0.0) -- (0.5738,0.0,0.4261);
	\draw (0.3333,0.6666,0.0) -- (0.0,0.5738,0.4261);
	\fill[red] (0,0,1) circle (1pt);
\end{tikzpicture}}

\def\tetrahedralmacroelementD{
\begin{tikzpicture}[scale=1.8]	\fill[red] (0,0,1) circle (1pt);
	\draw (0.0,0.1497,0.0) -- (0.0,0.0,0.0) -- (0.0,0.0,0.0) -- (0.0,0.0,0.0);
	\draw (0.0,0.3869,0.0) -- (0.0,0.0,0.0) -- (0.0,0.0,0.0);
	\draw (0.0,0.6742,0.0) -- (0.0,0.0,0.0);
	\draw (0.0,1.0,0.0);
	\draw (0.0,0.0,0.0) -- (0.0,0.1497,0.0) -- (0.0,0.3869,0.0) -- (0.0,0.6742,0.0) -- (0.0,1.0,0.0);
	\draw (0.0,0.0,0.0) -- (0.1497,0.0,0.0) -- (0.0,0.0,0.0) -- (0.0,0.0,0.0) -- (0.0,0.0,0.0);
	\draw (0.0,0.1497,0.0) -- (0.1934,0.1934,0.0) -- (0.0,0.0,0.0) -- (0.0,0.0,0.0);
	\draw (0.0,0.3869,0.0) -- (0.2247,0.4495,0.0) -- (0.0,0.0,0.0);
	\draw (0.0,0.6742,0.0) -- (0.25,0.75,0.0);
	\draw (0.1497,0.0,0.0) -- (0.1934,0.1934,0.0) -- (0.2247,0.4495,0.0) -- (0.25,0.75,0.0);
	\draw (0.0,0.0,0.0) -- (0.1497,0.0,0.0) -- (0.3869,0.0,0.0) -- (0.0,0.0,0.0) -- (0.0,0.0,0.0);
	\draw (0.0,0.1497,0.0) -- (0.1934,0.1934,0.0) -- (0.4495,0.2247,0.0) -- (0.0,0.0,0.0);
	\draw (0.0,0.3869,0.0) -- (0.2247,0.4495,0.0) -- (0.5,0.5,0.0);
	\draw (0.3869,0.0,0.0) -- (0.4495,0.2247,0.0) -- (0.5,0.5,0.0);
	\draw (0.0,0.0,0.0) -- (0.1497,0.0,0.0) -- (0.3869,0.0,0.0) -- (0.6742,0.0,0.0) -- (0.0,0.0,0.0);
	\draw (0.0,0.1497,0.0) -- (0.1934,0.1934,0.0) -- (0.4495,0.2247,0.0) -- (0.75,0.25,0.0);
	\draw (0.6742,0.0,0.0) -- (0.75,0.25,0.0);
	\draw (0.0,0.0,0.0) -- (0.1497,0.0,0.0) -- (0.3869,0.0,0.0) -- (0.6742,0.0,0.0) -- (1.0,0.0,0.0);
	\draw (1.0,0.0,0.0);
	\draw (0.0,0.0,0.0);
	\draw (0.0,0.1497,0.0) -- (0.1497,0.0,0.0);
	\draw (0.0,0.3869,0.0) -- (0.1934,0.1934,0.0) -- (0.3869,0.0,0.0);
	\draw (0.0,0.6742,0.0) -- (0.2247,0.4495,0.0) -- (0.4495,0.2247,0.0) -- (0.6742,0.0,0.0);
	\draw (0.0,1.0,0.0) -- (0.25,0.75,0.0) -- (0.5,0.5,0.0) -- (0.75,0.25,0.0) -- (1.0,0.0,0.0);
	\draw (0.0,0.0,0.3257) -- (0.0,0.0,0.0) -- (0.0,0.0,0.0) -- (0.0,0.0,0.0);
	\draw (0.0,0.1497,0.3257) -- (0.0,0.0,0.0) -- (0.0,0.0,0.0);
	\draw (0.0,0.3869,0.3257) -- (0.0,0.0,0.0);
	\draw (0.0,0.6742,0.3257);
	\draw (0.0,0.0,0.3257) -- (0.0,0.1497,0.3257) -- (0.0,0.3869,0.3257) -- (0.0,0.6742,0.3257);
	\draw (0.0,0.0,0.3257) -- (0.1497,0.0,0.3257) -- (0.0,0.0,0.0) -- (0.0,0.0,0.0);
	\draw (0.0,0.1497,0.3257) -- (0.1934,0.1934,0.3257) -- (0.0,0.0,0.0);
	\draw (0.0,0.3869,0.3257) -- (0.2247,0.4495,0.3257);
	\draw (0.1497,0.0,0.3257) -- (0.1934,0.1934,0.3257) -- (0.2247,0.4495,0.3257);
	\draw (0.0,0.0,0.3257) -- (0.1497,0.0,0.3257) -- (0.3869,0.0,0.3257) -- (0.0,0.0,0.0);
	\draw (0.0,0.1497,0.3257) -- (0.1934,0.1934,0.3257) -- (0.4495,0.2247,0.3257);
	\draw (0.3869,0.0,0.3257) -- (0.4495,0.2247,0.3257);
	\draw (0.0,0.0,0.3257) -- (0.1497,0.0,0.3257) -- (0.3869,0.0,0.3257) -- (0.6742,0.0,0.3257);
	\draw (0.6742,0.0,0.3257);
	\draw (0.0,0.0,0.3257);
	\draw (0.0,0.1497,0.3257) -- (0.1497,0.0,0.3257);
	\draw (0.0,0.3869,0.3257) -- (0.1934,0.1934,0.3257) -- (0.3869,0.0,0.3257);
	\draw (0.0,0.6742,0.3257) -- (0.2247,0.4495,0.3257) -- (0.4495,0.2247,0.3257) -- (0.6742,0.0,0.3257);
	\draw (0.0,0.0,0.6130) -- (0.0,0.0,0.0) -- (0.0,0.0,0.0);
	\draw (0.0,0.1497,0.6130) -- (0.0,0.0,0.0);
	\draw (0.0,0.3869,0.6130);
	\draw (0.0,0.0,0.6130) -- (0.0,0.1497,0.6130) -- (0.0,0.3869,0.6130);
	\draw (0.0,0.0,0.6130) -- (0.1497,0.0,0.6130) -- (0.0,0.0,0.0);
	\draw (0.0,0.1497,0.6130) -- (0.1934,0.1934,0.6130);
	\draw (0.1497,0.0,0.6130) -- (0.1934,0.1934,0.6130);
	\draw (0.0,0.0,0.6130) -- (0.1497,0.0,0.6130) -- (0.3869,0.0,0.6130);
	\draw (0.3869,0.0,0.6130);
	\draw (0.0,0.0,0.6130);
	\draw (0.0,0.1497,0.6130) -- (0.1497,0.0,0.6130);
	\draw (0.0,0.3869,0.6130) -- (0.1934,0.1934,0.6130) -- (0.3869,0.0,0.6130);
	\draw (0.0,0.0,0.8502) -- (0.0,0.0,0.0);
	\draw (0.0,0.1497,0.8502);
	\draw (0.0,0.0,0.8502) -- (0.0,0.1497,0.8502);
	\draw (0.0,0.0,0.8502) -- (0.1497,0.0,0.8502);
	\draw (0.1497,0.0,0.8502);
	\draw (0.0,0.0,0.8502);
	\draw (0.0,0.1497,0.8502) -- (0.1497,0.0,0.8502);
	\draw[orange] (0.0,0.0,0.0) -- (0.0,0.0,0.3257) -- (0.0,0.0,0.6130) -- (0.0,0.0,0.8502) -- (0.0,0.0,1.0);
	\draw (0.0,0.1497,0.0) -- (0.0,0.1497,0.3257) -- (0.0,0.1497,0.6130) -- (0.0,0.1497,0.8502);
	\draw (0.0,0.3869,0.0) -- (0.0,0.3869,0.3257) -- (0.0,0.3869,0.6130);
	\draw (0.0,0.6742,0.0) -- (0.0,0.6742,0.3257);
	\draw (0.0,1.0,0.0) -- (0.0,0.6742,0.3257) -- (0.0,0.3869,0.6130) -- (0.0,0.1497,0.8502) -- (0.0,0.0,1.0);
	\draw (1.0,0.0,0.0) -- (0.6742,0.0,0.3257) -- (0.3869,0.0,0.6130) -- (0.1497,0.0,0.8502) -- (0.0,0.0,1.0);
	\draw (0.1497,0.0,0.0) -- (0.1497,0.0,0.3257) -- (0.1497,0.0,0.6130) -- (0.1497,0.0,0.8502);
	\draw (0.1934,0.1934,0.0) -- (0.1934,0.1934,0.3257) -- (0.1934,0.1934,0.6130);
	\draw (0.2247,0.4495,0.0) -- (0.2247,0.4495,0.3257);
	\draw (0.25,0.75,0.0) -- (0.2247,0.4495,0.3257) -- (0.1934,0.1934,0.6130) -- (0.1497,0.0,0.8502);
	\draw (0.75,0.25,0.0) -- (0.4495,0.2247,0.3257) -- (0.1934,0.1934,0.6130) -- (0.0,0.1497,0.8502);
	\draw (0.3869,0.0,0.0) -- (0.3869,0.0,0.3257) -- (0.3869,0.0,0.6130);
	\draw (0.4495,0.2247,0.0) -- (0.4495,0.2247,0.3257);
	\draw (0.5,0.5,0.0) -- (0.4495,0.2247,0.3257) -- (0.3869,0.0,0.6130);
	\draw (0.5,0.5,0.0) -- (0.2247,0.4495,0.3257) -- (0.0,0.3869,0.6130);
	\draw (0.6742,0.0,0.0) -- (0.6742,0.0,0.3257);
	\draw (0.75,0.25,0.0) -- (0.6742,0.0,0.3257);
	\draw (0.25,0.75,0.0) -- (0.0,0.6742,0.3257);
\end{tikzpicture}}

\def\tetrahedralmacroelementE{
\begin{tikzpicture}[scale=1.8]	\fill[red] (0,0,1) circle (1pt);
	\draw (0.0,0.0,0.0);
	\draw (0.0,0.1102,0.0) -- (0.0,0.0,0.0) -- (0.0,0.0,0.0) -- (0.0,0.0,0.0) -- (0.0,0.0,0.0);
	\draw (0.0,0.2850,0.0) -- (0.0,0.0,0.0) -- (0.0,0.0,0.0) -- (0.0,0.0,0.0);
	\draw (0.0,0.4967,0.0) -- (0.0,0.0,0.0) -- (0.0,0.0,0.0);
	\draw (0.0,0.7366,0.0) -- (0.0,0.0,0.0);
	\draw (0.0,1.0,0.0);
	\draw (0.0,0.0,0.0) -- (0.0,0.1102,0.0) -- (0.0,0.2850,0.0) -- (0.0,0.4967,0.0) -- (0.0,0.7366,0.0) -- (0.0,1.0,0.0);
	\draw (0.0,0.0,0.0) -- (0.1102,0.0,0.0) -- (0.0,0.0,0.0) -- (0.0,0.0,0.0) -- (0.0,0.0,0.0) -- (0.0,0.0,0.0);
	\draw (0.0,0.1102,0.0) -- (0.1425,0.1425,0.0) -- (0.0,0.0,0.0) -- (0.0,0.0,0.0) -- (0.0,0.0,0.0);
	\draw (0.0,0.2850,0.0) -- (0.1655,0.3311,0.0) -- (0.0,0.0,0.0) -- (0.0,0.0,0.0);
	\draw (0.0,0.4967,0.0) -- (0.1841,0.5524,0.0) -- (0.0,0.0,0.0);
	\draw (0.0,0.7366,0.0) -- (0.2,0.8,0.0);
	\draw (0.1102,0.0,0.0) -- (0.1425,0.1425,0.0) -- (0.1655,0.3311,0.0) -- (0.1841,0.5524,0.0) -- (0.2,0.8,0.0);
	\draw (0.0,0.0,0.0) -- (0.1102,0.0,0.0) -- (0.2850,0.0,0.0) -- (0.0,0.0,0.0) -- (0.0,0.0,0.0) -- (0.0,0.0,0.0);
	\draw (0.0,0.1102,0.0) -- (0.1425,0.1425,0.0) -- (0.3311,0.1655,0.0) -- (0.0,0.0,0.0) -- (0.0,0.0,0.0);
	\draw (0.0,0.2850,0.0) -- (0.1655,0.3311,0.0) -- (0.3683,0.3683,0.0) -- (0.0,0.0,0.0);
	\draw (0.0,0.4967,0.0) -- (0.1841,0.5524,0.0) -- (0.4,0.6,0.0);
	\draw (0.2850,0.0,0.0) -- (0.3311,0.1655,0.0) -- (0.3683,0.3683,0.0) -- (0.4,0.6,0.0);
	\draw (0.0,0.0,0.0) -- (0.1102,0.0,0.0) -- (0.2850,0.0,0.0) -- (0.4967,0.0,0.0) -- (0.0,0.0,0.0) -- (0.0,0.0,0.0);
	\draw (0.0,0.1102,0.0) -- (0.1425,0.1425,0.0) -- (0.3311,0.1655,0.0) -- (0.5524,0.1841,0.0) -- (0.0,0.0,0.0);
	\draw (0.0,0.2850,0.0) -- (0.1655,0.3311,0.0) -- (0.3683,0.3683,0.0) -- (0.6,0.4,0.0);
	\draw (0.4967,0.0,0.0) -- (0.5524,0.1841,0.0) -- (0.6,0.4,0.0);
	\draw (0.0,0.0,0.0) -- (0.1102,0.0,0.0) -- (0.2850,0.0,0.0) -- (0.4967,0.0,0.0) -- (0.7366,0.0,0.0) -- (0.0,0.0,0.0);
	\draw (0.0,0.1102,0.0) -- (0.1425,0.1425,0.0) -- (0.3311,0.1655,0.0) -- (0.5524,0.1841,0.0) -- (0.8,0.2,0.0);
	\draw (0.7366,0.0,0.0) -- (0.8,0.2,0.0);
	\draw (0.0,0.0,0.0) -- (0.1102,0.0,0.0) -- (0.2850,0.0,0.0) -- (0.4967,0.0,0.0) -- (0.7366,0.0,0.0) -- (1.0,0.0,0.0);
	\draw (0.0,0.1102,0.0) -- (0.1102,0.0,0.0);
	\draw (0.0,0.2850,0.0) -- (0.1425,0.1425,0.0) -- (0.2850,0.0,0.0);
	\draw (0.0,0.4967,0.0) -- (0.1655,0.3311,0.0) -- (0.3311,0.1655,0.0) -- (0.4967,0.0,0.0);
	\draw (0.0,0.7366,0.0) -- (0.1841,0.5524,0.0) -- (0.3683,0.3683,0.0) -- (0.5524,0.1841,0.0) -- (0.7366,0.0,0.0);
	\draw (0.0,1.0,0.0) -- (0.2,0.8,0.0) -- (0.4,0.6,0.0) -- (0.6,0.4,0.0) -- (0.8,0.2,0.0) -- (1.0,0.0,0.0);
	\draw (0.0,0.0,0.2633) -- (0.0,0.0,0.0) -- (0.0,0.0,0.0) -- (0.0,0.0,0.0) -- (0.0,0.0,0.0);
	\draw (0.0,0.1102,0.2633) -- (0.0,0.0,0.0) -- (0.0,0.0,0.0) -- (0.0,0.0,0.0);
	\draw (0.0,0.2850,0.2633) -- (0.0,0.0,0.0) -- (0.0,0.0,0.0);
	\draw (0.0,0.4967,0.2633) -- (0.0,0.0,0.0);
	\draw (0.0,0.7366,0.2633);
	\draw (0.0,0.0,0.2633) -- (0.0,0.1102,0.2633) -- (0.0,0.2850,0.2633) -- (0.0,0.4967,0.2633) -- (0.0,0.7366,0.2633);
	\draw (0.0,0.0,0.2633) -- (0.1102,0.0,0.2633) -- (0.0,0.0,0.0) -- (0.0,0.0,0.0) -- (0.0,0.0,0.0);
	\draw (0.0,0.1102,0.2633) -- (0.1425,0.1425,0.2633) -- (0.0,0.0,0.0) -- (0.0,0.0,0.0);
	\draw (0.0,0.2850,0.2633) -- (0.1655,0.3311,0.2633) -- (0.0,0.0,0.0);
	\draw (0.0,0.4967,0.2633) -- (0.1841,0.5524,0.2633);
	\draw (0.1102,0.0,0.2633) -- (0.1425,0.1425,0.2633) -- (0.1655,0.3311,0.2633) -- (0.1841,0.5524,0.2633);
	\draw (0.0,0.0,0.2633) -- (0.1102,0.0,0.2633) -- (0.2850,0.0,0.2633) -- (0.0,0.0,0.0) -- (0.0,0.0,0.0);
	\draw (0.0,0.1102,0.2633) -- (0.1425,0.1425,0.2633) -- (0.3311,0.1655,0.2633) -- (0.0,0.0,0.0);
	\draw (0.0,0.2850,0.2633) -- (0.1655,0.3311,0.2633) -- (0.3683,0.3683,0.2633);
	\draw (0.2850,0.0,0.2633) -- (0.3311,0.1655,0.2633) -- (0.3683,0.3683,0.2633);
	\draw (0.0,0.0,0.2633) -- (0.1102,0.0,0.2633) -- (0.2850,0.0,0.2633) -- (0.4967,0.0,0.2633) -- (0.0,0.0,0.0);
	\draw (0.0,0.1102,0.2633) -- (0.1425,0.1425,0.2633) -- (0.3311,0.1655,0.2633) -- (0.5524,0.1841,0.2633);
	\draw (0.4967,0.0,0.2633) -- (0.5524,0.1841,0.2633);
	\draw (0.0,0.0,0.2633) -- (0.1102,0.0,0.2633) -- (0.2850,0.0,0.2633) -- (0.4967,0.0,0.2633) -- (0.7366,0.0,0.2633);
	\draw (0.7366,0.0,0.2633);
	\draw (0.0,0.0,0.2633);
	\draw (0.0,0.1102,0.2633) -- (0.1102,0.0,0.2633);
	\draw (0.0,0.2850,0.2633) -- (0.1425,0.1425,0.2633) -- (0.2850,0.0,0.2633);
	\draw (0.0,0.4967,0.2633) -- (0.1655,0.3311,0.2633) -- (0.3311,0.1655,0.2633) -- (0.4967,0.0,0.2633);
	\draw (0.0,0.7366,0.2633) -- (0.1841,0.5524,0.2633) -- (0.3683,0.3683,0.2633) -- (0.5524,0.1841,0.2633) -- (0.7366,0.0,0.2633);
	\draw (0.0,0.0,0.5032) -- (0.0,0.0,0.0) -- (0.0,0.0,0.0) -- (0.0,0.0,0.0);
	\draw (0.0,0.1102,0.5032) -- (0.0,0.0,0.0) -- (0.0,0.0,0.0);
	\draw (0.0,0.2850,0.5032) -- (0.0,0.0,0.0);
	\draw (0.0,0.4967,0.5032);
	\draw (0.0,0.0,0.5032) -- (0.0,0.1102,0.5032) -- (0.0,0.2850,0.5032) -- (0.0,0.4967,0.5032);
	\draw (0.0,0.0,0.5032) -- (0.1102,0.0,0.5032) -- (0.0,0.0,0.0) -- (0.0,0.0,0.0);
	\draw (0.0,0.1102,0.5032) -- (0.1425,0.1425,0.5032) -- (0.0,0.0,0.0);
	\draw (0.0,0.2850,0.5032) -- (0.1655,0.3311,0.5032);
	\draw (0.1102,0.0,0.5032) -- (0.1425,0.1425,0.5032) -- (0.1655,0.3311,0.5032);
	\draw (0.0,0.0,0.5032) -- (0.1102,0.0,0.5032) -- (0.2850,0.0,0.5032) -- (0.0,0.0,0.0);
	\draw (0.0,0.1102,0.5032) -- (0.1425,0.1425,0.5032) -- (0.3311,0.1655,0.5032);
	\draw (0.2850,0.0,0.5032) -- (0.3311,0.1655,0.5032);
	\draw (0.0,0.0,0.5032) -- (0.1102,0.0,0.5032) -- (0.2850,0.0,0.5032) -- (0.4967,0.0,0.5032);
	\draw (0.4967,0.0,0.5032);
	\draw (0.0,0.0,0.5032);
	\draw (0.0,0.1102,0.5032) -- (0.1102,0.0,0.5032);
	\draw (0.0,0.2850,0.5032) -- (0.1425,0.1425,0.5032) -- (0.2850,0.0,0.5032);
	\draw (0.0,0.4967,0.5032) -- (0.1655,0.3311,0.5032) -- (0.3311,0.1655,0.5032) -- (0.4967,0.0,0.5032);
	\draw (0.0,0.0,0.7149) -- (0.0,0.0,0.0) -- (0.0,0.0,0.0);
	\draw (0.0,0.1102,0.7149) -- (0.0,0.0,0.0);
	\draw (0.0,0.2850,0.7149);
	\draw (0.0,0.0,0.7149) -- (0.0,0.1102,0.7149) -- (0.0,0.2850,0.7149);
	\draw (0.0,0.0,0.7149) -- (0.1102,0.0,0.7149) -- (0.0,0.0,0.0);
	\draw (0.0,0.1102,0.7149) -- (0.1425,0.1425,0.7149);
	\draw (0.1102,0.0,0.7149) -- (0.1425,0.1425,0.7149);
	\draw (0.0,0.0,0.7149) -- (0.1102,0.0,0.7149) -- (0.2850,0.0,0.7149);
	\draw (0.2850,0.0,0.7149);
	\draw (0.0,0.0,0.7149);
	\draw (0.0,0.1102,0.7149) -- (0.1102,0.0,0.7149);
	\draw (0.0,0.2850,0.7149) -- (0.1425,0.1425,0.7149) -- (0.2850,0.0,0.7149);
	\draw (0.0,0.0,0.8897) -- (0.0,0.0,0.0);
	\draw (0.0,0.1102,0.8897);
	\draw (0.0,0.0,0.8897) -- (0.0,0.1102,0.8897);
	\draw (0.0,0.0,0.8897) -- (0.1102,0.0,0.8897);
	\draw (0.1102,0.0,0.8897);
	\draw (0.0,0.0,0.8897);
	\draw (0.0,0.1102,0.8897) -- (0.1102,0.0,0.8897);
	\draw[orange] (0.0,0.0,0.0) -- (0.0,0.0,0.2633) -- (0.0,0.0,0.5032) -- (0.0,0.0,0.7149) -- (0.0,0.0,0.8897) -- (0.0,0.0,1.0);
	\draw (0.0,0.1102,0.0) -- (0.0,0.1102,0.2633) -- (0.0,0.1102,0.5032) -- (0.0,0.1102,0.7149) -- (0.0,0.1102,0.8897);
	\draw (0.0,0.2850,0.0) -- (0.0,0.2850,0.2633) -- (0.0,0.2850,0.5032) -- (0.0,0.2850,0.7149);
	\draw (0.0,0.4967,0.0) -- (0.0,0.4967,0.2633) -- (0.0,0.4967,0.5032);
	\draw (0.0,0.7366,0.0) -- (0.0,0.7366,0.2633);
	\draw (0.0,1.0,0.0) -- (0.0,0.7366,0.2633) -- (0.0,0.4967,0.5032) -- (0.0,0.2850,0.7149) -- (0.0,0.1102,0.8897) -- (0.0,0.0,1.0);
	\draw (1.0,0.0,0.0) -- (0.7366,0.0,0.2633) -- (0.4967,0.0,0.5032) -- (0.2850,0.0,0.7149) -- (0.1102,0.0,0.8897) -- (0.0,0.0,1.0);
	\draw (0.1102,0.0,0.0) -- (0.1102,0.0,0.2633) -- (0.1102,0.0,0.5032) -- (0.1102,0.0,0.7149) -- (0.1102,0.0,0.8897);
	\draw (0.1425,0.1425,0.0) -- (0.1425,0.1425,0.2633) -- (0.1425,0.1425,0.5032) -- (0.1425,0.1425,0.7149);
	\draw (0.1655,0.3311,0.0) -- (0.1655,0.3311,0.2633) -- (0.1655,0.3311,0.5032);
	\draw (0.1841,0.5524,0.0) -- (0.1841,0.5524,0.2633);
	\draw (0.2,0.8,0.0) -- (0.1841,0.5524,0.2633) -- (0.1655,0.3311,0.5032) -- (0.1425,0.1425,0.7149) -- (0.1102,0.0,0.8897);
	\draw (0.8,0.2,0.0) -- (0.5524,0.1841,0.2633) -- (0.3311,0.1655,0.5032) -- (0.1425,0.1425,0.7149) -- (0.0,0.1102,0.8897);
	\draw (0.2850,0.0,0.0) -- (0.2850,0.0,0.2633) -- (0.2850,0.0,0.5032) -- (0.2850,0.0,0.7149);
	\draw (0.3311,0.1655,0.0) -- (0.3311,0.1655,0.2633) -- (0.3311,0.1655,0.5032);
	\draw (0.3683,0.3683,0.0) -- (0.3683,0.3683,0.2633);
	\draw (0.4,0.6,0.0) -- (0.3683,0.3683,0.2633) -- (0.3311,0.1655,0.5032) -- (0.2850,0.0,0.7149);
	\draw (0.6,0.4,0.0) -- (0.3683,0.3683,0.2633) -- (0.1655,0.3311,0.5032) -- (0.0,0.2850,0.7149);
	\draw (0.4967,0.0,0.0) -- (0.4967,0.0,0.2633) -- (0.4967,0.0,0.5032);
	\draw (0.5524,0.1841,0.0) -- (0.5524,0.1841,0.2633);
	\draw (0.6,0.4,0.0) -- (0.5524,0.1841,0.2633) -- (0.4967,0.0,0.5032);
	\draw (0.4,0.6,0.0) -- (0.1841,0.5524,0.2633) -- (0.0,0.4967,0.5032);
	\draw (0.7366,0.0,0.0) -- (0.7366,0.0,0.2633);
	\draw (0.8,0.2,0.0) -- (0.7366,0.0,0.2633);
	\draw (0.2,0.8,0.0) -- (0.0,0.7366,0.2633);
\end{tikzpicture}}

\def\tetrahedralmacroelementF{
\begin{tikzpicture}[scale=1.8]	\fill[red] (0,0,1) circle (1pt);
	\draw (0.0,0.0,0.0) -- (0.0,0.0,0.0) -- (0.0,0.0,0.0) -- (0.0,0.0,0.0) -- (0.0,0.0,0.0) -- (0.0,0.0,0.0) -- (0.0,0.0,0.0);
	\draw (0.0,0.0859,0.0) -- (0.0,0.0,0.0) -- (0.0,0.0,0.0) -- (0.0,0.0,0.0) -- (0.0,0.0,0.0) -- (0.0,0.0,0.0);
	\draw (0.0,0.2220,0.0) -- (0.0,0.0,0.0) -- (0.0,0.0,0.0) -- (0.0,0.0,0.0) -- (0.0,0.0,0.0);
	\draw (0.0,0.3869,0.0) -- (0.0,0.0,0.0) -- (0.0,0.0,0.0) -- (0.0,0.0,0.0);
	\draw (0.0,0.5738,0.0) -- (0.0,0.0,0.0) -- (0.0,0.0,0.0);
	\draw (0.0,0.7789,0.0) -- (0.0,0.0,0.0);
	\draw (0.0,1.0,0.0);
	\draw (0.0,0.0,0.0) -- (0.0,0.0859,0.0) -- (0.0,0.2220,0.0) -- (0.0,0.3869,0.0) -- (0.0,0.5738,0.0) -- (0.0,0.7789,0.0) -- (0.0,1.0,0.0);
	\draw (0.0,0.0,0.0) -- (0.0859,0.0,0.0) -- (0.0,0.0,0.0) -- (0.0,0.0,0.0) -- (0.0,0.0,0.0) -- (0.0,0.0,0.0) -- (0.0,0.0,0.0);
	\draw (0.0,0.0859,0.0) -- (0.1110,0.1110,0.0) -- (0.0,0.0,0.0) -- (0.0,0.0,0.0) -- (0.0,0.0,0.0) -- (0.0,0.0,0.0);
	\draw (0.0,0.2220,0.0) -- (0.1289,0.2579,0.0) -- (0.0,0.0,0.0) -- (0.0,0.0,0.0) -- (0.0,0.0,0.0);
	\draw (0.0,0.3869,0.0) -- (0.1434,0.4303,0.0) -- (0.0,0.0,0.0) -- (0.0,0.0,0.0);
	\draw (0.0,0.5738,0.0) -- (0.1557,0.6231,0.0) -- (0.0,0.0,0.0);
	\draw (0.0,0.7789,0.0) -- (0.1666,0.8333,0.0);
	\draw (0.0859,0.0,0.0) -- (0.1110,0.1110,0.0) -- (0.1289,0.2579,0.0) -- (0.1434,0.4303,0.0) -- (0.1557,0.6231,0.0) -- (0.1666,0.8333,0.0);
	\draw (0.0,0.0,0.0) -- (0.0859,0.0,0.0) -- (0.2220,0.0,0.0) -- (0.0,0.0,0.0) -- (0.0,0.0,0.0) -- (0.0,0.0,0.0) -- (0.0,0.0,0.0);
	\draw (0.0,0.0859,0.0) -- (0.1110,0.1110,0.0) -- (0.2579,0.1289,0.0) -- (0.0,0.0,0.0) -- (0.0,0.0,0.0) -- (0.0,0.0,0.0);
	\draw (0.0,0.2220,0.0) -- (0.1289,0.2579,0.0) -- (0.2869,0.2869,0.0) -- (0.0,0.0,0.0) -- (0.0,0.0,0.0);
	\draw (0.0,0.3869,0.0) -- (0.1434,0.4303,0.0) -- (0.3115,0.4673,0.0) -- (0.0,0.0,0.0);
	\draw (0.0,0.5738,0.0) -- (0.1557,0.6231,0.0) -- (0.3333,0.6666,0.0);
	\draw (0.2220,0.0,0.0) -- (0.2579,0.1289,0.0) -- (0.2869,0.2869,0.0) -- (0.3115,0.4673,0.0) -- (0.3333,0.6666,0.0);
	\draw (0.0,0.0,0.0) -- (0.0859,0.0,0.0) -- (0.2220,0.0,0.0) -- (0.3869,0.0,0.0) -- (0.0,0.0,0.0) -- (0.0,0.0,0.0) -- (0.0,0.0,0.0);
	\draw (0.0,0.0859,0.0) -- (0.1110,0.1110,0.0) -- (0.2579,0.1289,0.0) -- (0.4303,0.1434,0.0) -- (0.0,0.0,0.0) -- (0.0,0.0,0.0);
	\draw (0.0,0.2220,0.0) -- (0.1289,0.2579,0.0) -- (0.2869,0.2869,0.0) -- (0.4673,0.3115,0.0) -- (0.0,0.0,0.0);
	\draw (0.0,0.3869,0.0) -- (0.1434,0.4303,0.0) -- (0.3115,0.4673,0.0) -- (0.5,0.5,0.0);
	\draw (0.3869,0.0,0.0) -- (0.4303,0.1434,0.0) -- (0.4673,0.3115,0.0) -- (0.5,0.5,0.0);
	\draw (0.0,0.0,0.0) -- (0.0859,0.0,0.0) -- (0.2220,0.0,0.0) -- (0.3869,0.0,0.0) -- (0.5738,0.0,0.0) -- (0.0,0.0,0.0) -- (0.0,0.0,0.0);
	\draw (0.0,0.0859,0.0) -- (0.1110,0.1110,0.0) -- (0.2579,0.1289,0.0) -- (0.4303,0.1434,0.0) -- (0.6231,0.1557,0.0) -- (0.0,0.0,0.0);
	\draw (0.0,0.2220,0.0) -- (0.1289,0.2579,0.0) -- (0.2869,0.2869,0.0) -- (0.4673,0.3115,0.0) -- (0.6666,0.3333,0.0);
	\draw (0.5738,0.0,0.0) -- (0.6231,0.1557,0.0) -- (0.6666,0.3333,0.0);
	\draw (0.0,0.0,0.0) -- (0.0859,0.0,0.0) -- (0.2220,0.0,0.0) -- (0.3869,0.0,0.0) -- (0.5738,0.0,0.0) -- (0.7789,0.0,0.0) -- (0.0,0.0,0.0);
	\draw (0.0,0.0859,0.0) -- (0.1110,0.1110,0.0) -- (0.2579,0.1289,0.0) -- (0.4303,0.1434,0.0) -- (0.6231,0.1557,0.0) -- (0.8333,0.1666,0.0);
	\draw (0.7789,0.0,0.0) -- (0.8333,0.1666,0.0);
	\draw (0.0,0.0,0.0) -- (0.0859,0.0,0.0) -- (0.2220,0.0,0.0) -- (0.3869,0.0,0.0) -- (0.5738,0.0,0.0) -- (0.7789,0.0,0.0) -- (1.0,0.0,0.0);
	\draw (1.0,0.0,0.0);
	\draw (0.0,0.0,0.0);
	\draw (0.0,0.0859,0.0) -- (0.0859,0.0,0.0);
	\draw (0.0,0.2220,0.0) -- (0.1110,0.1110,0.0) -- (0.2220,0.0,0.0);
	\draw (0.0,0.3869,0.0) -- (0.1289,0.2579,0.0) -- (0.2579,0.1289,0.0) -- (0.3869,0.0,0.0);
	\draw (0.0,0.5738,0.0) -- (0.1434,0.4303,0.0) -- (0.2869,0.2869,0.0) -- (0.4303,0.1434,0.0) -- (0.5738,0.0,0.0);
	\draw (0.0,0.7789,0.0) -- (0.1557,0.6231,0.0) -- (0.3115,0.4673,0.0) -- (0.4673,0.3115,0.0) -- (0.6231,0.1557,0.0) -- (0.7789,0.0,0.0);
	\draw (0.0,1.0,0.0) -- (0.1666,0.8333,0.0) -- (0.3333,0.6666,0.0) -- (0.5,0.5,0.0) -- (0.6666,0.3333,0.0) -- (0.8333,0.1666,0.0) -- (1.0,0.0,0.0);
	\draw (0.0,0.0,0.2210) -- (0.0,0.0,0.0) -- (0.0,0.0,0.0) -- (0.0,0.0,0.0) -- (0.0,0.0,0.0) -- (0.0,0.0,0.0);
	\draw (0.0,0.0859,0.2210) -- (0.0,0.0,0.0) -- (0.0,0.0,0.0) -- (0.0,0.0,0.0) -- (0.0,0.0,0.0);
	\draw (0.0,0.2220,0.2210) -- (0.0,0.0,0.0) -- (0.0,0.0,0.0) -- (0.0,0.0,0.0);
	\draw (0.0,0.3869,0.2210) -- (0.0,0.0,0.0) -- (0.0,0.0,0.0);
	\draw (0.0,0.5738,0.2210) -- (0.0,0.0,0.0);
	\draw (0.0,0.7789,0.2210);
	\draw (0.0,0.0,0.2210) -- (0.0,0.0859,0.2210) -- (0.0,0.2220,0.2210) -- (0.0,0.3869,0.2210) -- (0.0,0.5738,0.2210) -- (0.0,0.7789,0.2210);
	\draw (0.0,0.0,0.2210) -- (0.0859,0.0,0.2210) -- (0.0,0.0,0.0) -- (0.0,0.0,0.0) -- (0.0,0.0,0.0) -- (0.0,0.0,0.0);
	\draw (0.0,0.0859,0.2210) -- (0.1110,0.1110,0.2210) -- (0.0,0.0,0.0) -- (0.0,0.0,0.0) -- (0.0,0.0,0.0);
	\draw (0.0,0.2220,0.2210) -- (0.1289,0.2579,0.2210) -- (0.0,0.0,0.0) -- (0.0,0.0,0.0);
	\draw (0.0,0.3869,0.2210) -- (0.1434,0.4303,0.2210) -- (0.0,0.0,0.0);
	\draw (0.0,0.5738,0.2210) -- (0.1557,0.6231,0.2210);
	\draw (0.0859,0.0,0.2210) -- (0.1110,0.1110,0.2210) -- (0.1289,0.2579,0.2210) -- (0.1434,0.4303,0.2210) -- (0.1557,0.6231,0.2210);
	\draw (0.0,0.0,0.2210) -- (0.0859,0.0,0.2210) -- (0.2220,0.0,0.2210) -- (0.0,0.0,0.0) -- (0.0,0.0,0.0) -- (0.0,0.0,0.0);
	\draw (0.0,0.0859,0.2210) -- (0.1110,0.1110,0.2210) -- (0.2579,0.1289,0.2210) -- (0.0,0.0,0.0) -- (0.0,0.0,0.0);
	\draw (0.0,0.2220,0.2210) -- (0.1289,0.2579,0.2210) -- (0.2869,0.2869,0.2210) -- (0.0,0.0,0.0);
	\draw (0.0,0.3869,0.2210) -- (0.1434,0.4303,0.2210) -- (0.3115,0.4673,0.2210);
	\draw (0.2220,0.0,0.2210) -- (0.2579,0.1289,0.2210) -- (0.2869,0.2869,0.2210) -- (0.3115,0.4673,0.2210);
	\draw (0.0,0.0,0.2210) -- (0.0859,0.0,0.2210) -- (0.2220,0.0,0.2210) -- (0.3869,0.0,0.2210) -- (0.0,0.0,0.0) -- (0.0,0.0,0.0);
	\draw (0.0,0.0859,0.2210) -- (0.1110,0.1110,0.2210) -- (0.2579,0.1289,0.2210) -- (0.4303,0.1434,0.2210) -- (0.0,0.0,0.0);
	\draw (0.0,0.2220,0.2210) -- (0.1289,0.2579,0.2210) -- (0.2869,0.2869,0.2210) -- (0.4673,0.3115,0.2210);
	\draw (0.3869,0.0,0.2210) -- (0.4303,0.1434,0.2210) -- (0.4673,0.3115,0.2210);
	\draw (0.0,0.0,0.2210) -- (0.0859,0.0,0.2210) -- (0.2220,0.0,0.2210) -- (0.3869,0.0,0.2210) -- (0.5738,0.0,0.2210) -- (0.0,0.0,0.0);
	\draw (0.0,0.0859,0.2210) -- (0.1110,0.1110,0.2210) -- (0.2579,0.1289,0.2210) -- (0.4303,0.1434,0.2210) -- (0.6231,0.1557,0.2210);
	\draw (0.5738,0.0,0.2210) -- (0.6231,0.1557,0.2210);
	\draw (0.0,0.0,0.2210) -- (0.0859,0.0,0.2210) -- (0.2220,0.0,0.2210) -- (0.3869,0.0,0.2210) -- (0.5738,0.0,0.2210) -- (0.7789,0.0,0.2210);
	\draw (0.7789,0.0,0.2210);
	\draw (0.0,0.0,0.2210);
	\draw (0.0,0.0859,0.2210) -- (0.0859,0.0,0.2210);
	\draw (0.0,0.2220,0.2210) -- (0.1110,0.1110,0.2210) -- (0.2220,0.0,0.2210);
	\draw (0.0,0.3869,0.2210) -- (0.1289,0.2579,0.2210) -- (0.2579,0.1289,0.2210) -- (0.3869,0.0,0.2210);
	\draw (0.0,0.5738,0.2210) -- (0.1434,0.4303,0.2210) -- (0.2869,0.2869,0.2210) -- (0.4303,0.1434,0.2210) -- (0.5738,0.0,0.2210);
	\draw (0.0,0.7789,0.2210) -- (0.1557,0.6231,0.2210) -- (0.3115,0.4673,0.2210) -- (0.4673,0.3115,0.2210) -- (0.6231,0.1557,0.2210) -- (0.7789,0.0,0.2210);
	\draw (0.0,0.0,0.4261) -- (0.0,0.0,0.0) -- (0.0,0.0,0.0) -- (0.0,0.0,0.0) -- (0.0,0.0,0.0);
	\draw (0.0,0.0859,0.4261) -- (0.0,0.0,0.0) -- (0.0,0.0,0.0) -- (0.0,0.0,0.0);
	\draw (0.0,0.2220,0.4261) -- (0.0,0.0,0.0) -- (0.0,0.0,0.0);
	\draw (0.0,0.3869,0.4261) -- (0.0,0.0,0.0);
	\draw (0.0,0.5738,0.4261);
	\draw (0.0,0.0,0.4261) -- (0.0,0.0859,0.4261) -- (0.0,0.2220,0.4261) -- (0.0,0.3869,0.4261) -- (0.0,0.5738,0.4261);
	\draw (0.0,0.0,0.4261) -- (0.0859,0.0,0.4261) -- (0.0,0.0,0.0) -- (0.0,0.0,0.0) -- (0.0,0.0,0.0);
	\draw (0.0,0.0859,0.4261) -- (0.1110,0.1110,0.4261) -- (0.0,0.0,0.0) -- (0.0,0.0,0.0);
	\draw (0.0,0.2220,0.4261) -- (0.1289,0.2579,0.4261) -- (0.0,0.0,0.0);
	\draw (0.0,0.3869,0.4261) -- (0.1434,0.4303,0.4261);
	\draw (0.0859,0.0,0.4261) -- (0.1110,0.1110,0.4261) -- (0.1289,0.2579,0.4261) -- (0.1434,0.4303,0.4261);
	\draw (0.0,0.0,0.4261) -- (0.0859,0.0,0.4261) -- (0.2220,0.0,0.4261) -- (0.0,0.0,0.0) -- (0.0,0.0,0.0);
	\draw (0.0,0.0859,0.4261) -- (0.1110,0.1110,0.4261) -- (0.2579,0.1289,0.4261) -- (0.0,0.0,0.0);
	\draw (0.0,0.2220,0.4261) -- (0.1289,0.2579,0.4261) -- (0.2869,0.2869,0.4261);
	\draw (0.2220,0.0,0.4261) -- (0.2579,0.1289,0.4261) -- (0.2869,0.2869,0.4261);
	\draw (0.0,0.0,0.4261) -- (0.0859,0.0,0.4261) -- (0.2220,0.0,0.4261) -- (0.3869,0.0,0.4261) -- (0.0,0.0,0.0);
	\draw (0.0,0.0859,0.4261) -- (0.1110,0.1110,0.4261) -- (0.2579,0.1289,0.4261) -- (0.4303,0.1434,0.4261);
	\draw (0.3869,0.0,0.4261) -- (0.4303,0.1434,0.4261);
	\draw (0.0,0.0,0.4261) -- (0.0859,0.0,0.4261) -- (0.2220,0.0,0.4261) -- (0.3869,0.0,0.4261) -- (0.5738,0.0,0.4261);
	\draw (0.5738,0.0,0.4261);
	\draw (0.0,0.0,0.4261);
	\draw (0.0,0.0859,0.4261) -- (0.0859,0.0,0.4261);
	\draw (0.0,0.2220,0.4261) -- (0.1110,0.1110,0.4261) -- (0.2220,0.0,0.4261);
	\draw (0.0,0.3869,0.4261) -- (0.1289,0.2579,0.4261) -- (0.2579,0.1289,0.4261) -- (0.3869,0.0,0.4261);
	\draw (0.0,0.5738,0.4261) -- (0.1434,0.4303,0.4261) -- (0.2869,0.2869,0.4261) -- (0.4303,0.1434,0.4261) -- (0.5738,0.0,0.4261);
	\draw (0.0,0.0,0.6130) -- (0.0,0.0,0.0) -- (0.0,0.0,0.0) -- (0.0,0.0,0.0);
	\draw (0.0,0.0859,0.6130) -- (0.0,0.0,0.0) -- (0.0,0.0,0.0);
	\draw (0.0,0.2220,0.6130) -- (0.0,0.0,0.0);
	\draw (0.0,0.3869,0.6130);
	\draw (0.0,0.0,0.6130) -- (0.0,0.0859,0.6130) -- (0.0,0.2220,0.6130) -- (0.0,0.3869,0.6130);
	\draw (0.0,0.0,0.6130) -- (0.0859,0.0,0.6130) -- (0.0,0.0,0.0) -- (0.0,0.0,0.0);
	\draw (0.0,0.0859,0.6130) -- (0.1110,0.1110,0.6130) -- (0.0,0.0,0.0);
	\draw (0.0,0.2220,0.6130) -- (0.1289,0.2579,0.6130);
	\draw (0.0859,0.0,0.6130) -- (0.1110,0.1110,0.6130) -- (0.1289,0.2579,0.6130);
	\draw (0.0,0.0,0.6130) -- (0.0859,0.0,0.6130) -- (0.2220,0.0,0.6130) -- (0.0,0.0,0.0);
	\draw (0.0,0.0859,0.6130) -- (0.1110,0.1110,0.6130) -- (0.2579,0.1289,0.6130);
	\draw (0.2220,0.0,0.6130) -- (0.2579,0.1289,0.6130);
	\draw (0.0,0.0,0.6130) -- (0.0859,0.0,0.6130) -- (0.2220,0.0,0.6130) -- (0.3869,0.0,0.6130);
	\draw (0.3869,0.0,0.6130);
	\draw (0.0,0.0,0.6130);
	\draw (0.0,0.0859,0.6130) -- (0.0859,0.0,0.6130);
	\draw (0.0,0.2220,0.6130) -- (0.1110,0.1110,0.6130) -- (0.2220,0.0,0.6130);
	\draw (0.0,0.3869,0.6130) -- (0.1289,0.2579,0.6130) -- (0.2579,0.1289,0.6130) -- (0.3869,0.0,0.6130);
	\draw (0.0,0.0,0.7779) -- (0.0,0.0,0.0) -- (0.0,0.0,0.0);
	\draw (0.0,0.0859,0.7779) -- (0.0,0.0,0.0);
	\draw (0.0,0.2220,0.7779);
	\draw (0.0,0.0,0.7779) -- (0.0,0.0859,0.7779) -- (0.0,0.2220,0.7779);
	\draw (0.0,0.0,0.7779) -- (0.0859,0.0,0.7779) -- (0.0,0.0,0.0);
	\draw (0.0,0.0859,0.7779) -- (0.1110,0.1110,0.7779);
	\draw (0.0859,0.0,0.7779) -- (0.1110,0.1110,0.7779);
	\draw (0.0,0.0,0.7779) -- (0.0859,0.0,0.7779) -- (0.2220,0.0,0.7779);
	\draw (0.2220,0.0,0.7779);
	\draw (0.0,0.0,0.7779);
	\draw (0.0,0.0859,0.7779) -- (0.0859,0.0,0.7779);
	\draw (0.0,0.2220,0.7779) -- (0.1110,0.1110,0.7779) -- (0.2220,0.0,0.7779);
	\draw (0.0,0.0,0.9140) -- (0.0,0.0,0.0);
	\draw (0.0,0.0859,0.9140);
	\draw (0.0,0.0,0.9140) -- (0.0,0.0859,0.9140);
	\draw (0.0,0.0,0.9140) -- (0.0859,0.0,0.9140);
	\draw (0.0859,0.0,0.9140);
	\draw (0.0,0.0,0.9140);
	\draw (0.0,0.0859,0.9140) -- (0.0859,0.0,0.9140);
	\draw[orange] (0.0,0.0,0.0) -- (0.0,0.0,0.2210) -- (0.0,0.0,0.4261) -- (0.0,0.0,0.6130) -- (0.0,0.0,0.7779) -- (0.0,0.0,0.9140) -- (0.0,0.0,1.0);
	\draw (0.0,0.0859,0.0) -- (0.0,0.0859,0.2210) -- (0.0,0.0859,0.4261) -- (0.0,0.0859,0.6130) -- (0.0,0.0859,0.7779) -- (0.0,0.0859,0.9140);
	\draw (0.0,0.2220,0.0) -- (0.0,0.2220,0.2210) -- (0.0,0.2220,0.4261) -- (0.0,0.2220,0.6130) -- (0.0,0.2220,0.7779);
	\draw (0.0,0.3869,0.0) -- (0.0,0.3869,0.2210) -- (0.0,0.3869,0.4261) -- (0.0,0.3869,0.6130);
	\draw (0.0,0.5738,0.0) -- (0.0,0.5738,0.2210) -- (0.0,0.5738,0.4261);
	\draw (0.0,0.7789,0.0) -- (0.0,0.7789,0.2210);
	\draw (0.0,1.0,0.0) -- (0.0,0.7789,0.2210) -- (0.0,0.5738,0.4261) -- (0.0,0.3869,0.6130) -- (0.0,0.2220,0.7779) -- (0.0,0.0859,0.9140) -- (0.0,0.0,1.0);
	\draw (1.0,0.0,0.0) -- (0.7789,0.0,0.2210) -- (0.5738,0.0,0.4261) -- (0.3869,0.0,0.6130) -- (0.2220,0.0,0.7779) -- (0.0859,0.0,0.9140) -- (0.0,0.0,1.0);
	\draw (0.0859,0.0,0.0) -- (0.0859,0.0,0.2210) -- (0.0859,0.0,0.4261) -- (0.0859,0.0,0.6130) -- (0.0859,0.0,0.7779) -- (0.0859,0.0,0.9140);
	\draw (0.1110,0.1110,0.0) -- (0.1110,0.1110,0.2210) -- (0.1110,0.1110,0.4261) -- (0.1110,0.1110,0.6130) -- (0.1110,0.1110,0.7779);
	\draw (0.1289,0.2579,0.0) -- (0.1289,0.2579,0.2210) -- (0.1289,0.2579,0.4261) -- (0.1289,0.2579,0.6130);
	\draw (0.1434,0.4303,0.0) -- (0.1434,0.4303,0.2210) -- (0.1434,0.4303,0.4261);
	\draw (0.1557,0.6231,0.0) -- (0.1557,0.6231,0.2210);
	\draw (0.1666,0.8333,0.0) -- (0.1557,0.6231,0.2210) -- (0.1434,0.4303,0.4261) -- (0.1289,0.2579,0.6130) -- (0.1110,0.1110,0.7779) -- (0.0859,0.0,0.9140);
	\draw (0.8333,0.1666,0.0) -- (0.6231,0.1557,0.2210) -- (0.4303,0.1434,0.4261) -- (0.2579,0.1289,0.6130) -- (0.1110,0.1110,0.7779) -- (0.0,0.0859,0.9140);
	\draw (0.2220,0.0,0.0) -- (0.2220,0.0,0.2210) -- (0.2220,0.0,0.4261) -- (0.2220,0.0,0.6130) -- (0.2220,0.0,0.7779);
	\draw (0.2579,0.1289,0.0) -- (0.2579,0.1289,0.2210) -- (0.2579,0.1289,0.4261) -- (0.2579,0.1289,0.6130);
	\draw (0.2869,0.2869,0.0) -- (0.2869,0.2869,0.2210) -- (0.2869,0.2869,0.4261);
	\draw (0.3115,0.4673,0.0) -- (0.3115,0.4673,0.2210);
	\draw (0.3333,0.6666,0.0) -- (0.3115,0.4673,0.2210) -- (0.2869,0.2869,0.4261) -- (0.2579,0.1289,0.6130) -- (0.2220,0.0,0.7779);
	\draw (0.6666,0.3333,0.0) -- (0.4673,0.3115,0.2210) -- (0.2869,0.2869,0.4261) -- (0.1289,0.2579,0.6130) -- (0.0,0.2220,0.7779);
	\draw (0.3869,0.0,0.0) -- (0.3869,0.0,0.2210) -- (0.3869,0.0,0.4261) -- (0.3869,0.0,0.6130);
	\draw (0.4303,0.1434,0.0) -- (0.4303,0.1434,0.2210) -- (0.4303,0.1434,0.4261);
	\draw (0.4673,0.3115,0.0) -- (0.4673,0.3115,0.2210);
	\draw (0.5,0.5,0.0) -- (0.4673,0.3115,0.2210) -- (0.4303,0.1434,0.4261) -- (0.3869,0.0,0.6130);
	\draw (0.5,0.5,0.0) -- (0.3115,0.4673,0.2210) -- (0.1434,0.4303,0.4261) -- (0.0,0.3869,0.6130);
	\draw (0.5738,0.0,0.0) -- (0.5738,0.0,0.2210) -- (0.5738,0.0,0.4261);
	\draw (0.6231,0.1557,0.0) -- (0.6231,0.1557,0.2210);
	\draw (0.6666,0.3333,0.0) -- (0.6231,0.1557,0.2210) -- (0.5738,0.0,0.4261);
	\draw (0.3333,0.6666,0.0) -- (0.1557,0.6231,0.2210) -- (0.0,0.5738,0.4261);
	\draw (0.7789,0.0,0.0) -- (0.7789,0.0,0.2210);
	\draw (0.8333,0.1666,0.0) -- (0.7789,0.0,0.2210);
	\draw (0.1666,0.8333,0.0) -- (0.0,0.7789,0.2210);
\end{tikzpicture}}

\usetikzlibrary{calc,shapes.geometric,arrows}

% numbering before the word Theorem
\swapnumbers
\setcounter{secnumdepth}{3}
%\numberwithin{equation}{section}

%%%%%%%%%%%%%%%%%%%%%%%%%%%%%%%%
%\usepackage{syntonly}
%\syntaxonly
%%%%%%%%%%%%%%%%%%%%%%%%%%%%%%%%

\DeclareMathAlphabet{\pazocal}{OMS}{zplm}{m}{n}

\def\twoPi{360}
\def\ok{{\color{green}\quad ok}}
\def\xyz{(\hat x_1, \hat x_2, \hat x_3)}
\def\bxi{\boldsymbol{\xi}}
\def\bphi{\boldsymbol{\phi}}
\def\be{\boldsymbol{e}}
\def\bx{\boldsymbol{x}}
\def\by{\boldsymbol{y}}
\def\bh{\boldsymbol{h}}
\def\bq{\boldsymbol{q}}
\def\bp{\boldsymbol{p}}
\def\bv{\boldsymbol{v}}
\def\bw{\boldsymbol{w}}
\def\bu{\boldsymbol{u}}
\def\br{\boldsymbol{r}}
\def\bn{\boldsymbol{n}}
\def\balpha{\boldsymbol{\alpha}}
\def\bbeta{\boldsymbol{\beta}}
\def\btau{\boldsymbol{\tau}}
\def\bz{\boldsymbol{\zeta}}
\def\bnu{\boldsymbol{\nu}}
\def\bgamma{\boldsymbol{\gamma}}
\def\s{\scriptstyle}
\def\sss{\scriptscriptstyle}
\def\diam{\text{diam}}
\def\dv{\mbox{div\,}}
\def\curl{\textbf{curl\,}}
\def\bcurl{\textbf{curl}}
\def\wku{\bw_{\hat{E}}\hat{\bu}}
\def\wkutilde{\bw_{\tilde{E}}\tilde{\bu}}
\def\rku{\br_{\hat{E}}\hat{\bu}}
\def\rZerou{\br_{0}\bu}
\def\rkutilde{\br_{\tilde{E}}\tilde{\bu}}

\newcommand{\Qb}{\pazocal{Q}}
\newcommand{\Qbb}{\pazocal{\bf Q}}
\newcommand{\Tb}{\pazocal{T}}
\newcommand{\Kb}{\pazocal{K}}
\newcommand{\Th}{\pazocal{T}_{\textit{h}}}

{\newcommand{\vertiii}[1]{{\left\vert\kern-0.25ex\left\vert\kern-0.25ex\left\vert #1 
    \right\vert\kern-0.25ex\right\vert\kern-0.25ex\right\vert}}
\newcommand{\forma}[2]{\int_\Omega {\boldsymbol{#1}}\cdot{\boldsymbol{#2}}\,d\bx}
\newcommand{\formb}[2]{\int_\Omega #2\,\dv{\boldsymbol{#1}}\,d\bx}
\newcommand{\hmtn}[2]{\textrm{H}^#1(\textrm{T})^#2}
\newcommand{\RTk}{\textrm{RT}_k}
\newcommand{\RTksomb}{\widehat{\textrm{RT}_k}}
\newcommand{\xSombrero}[1]{\widehat{\textbf{#1}}}
\newcommand{\raviart}[1]{\mathcal{RT}_k(#1)}
\newcommand{\Vh}{\mathbb{V}_h}
\newcommand{\Qh}{\mathbb{Q}_h}



\newcommand{\supp}[1]{\textrm{supp}(#1)}
\newcommand{\p}[2]{\mathcal{P}_{#1}(x,y) \otimes \mathcal{P}_{#2}(z)}
\newcommand{\pp}[2]{\mathcal{P}_{#1}(\hat f_1)\otimes\mathcal P_{#2}(\hat z)}
\newcommand{\wpcurl}[1]{W^p(\textbf{curl}, #1)}
\newcommand{\diag}[3]{
    \left(
    \begin{array}{ccc}
        #1  & 0     & 0\\
        0   & #2    & 0\\
        0   & 0     & #3
    \end{array}
    \right)}
\newcommand{\blockdiag}[5]{
	\left(
	\begin{array}{ccc}
		#1 	& #2 	& 0\\
		#3	& #4 	& 0\\
		0	& 0		& #5
	\end{array}
	\right)}
\newcommand{\dvg}{\text{div}\,}
\newcommand{\lf}[1]{\lambda_{f_{#1}}}
\newcommand{\lhatf}[1]{\lambda_{\hat{f}_{#1}}}
\newcommand{\gancho}{{\scriptstyle\partial}}

\newtheorem{theorem}{Theorem}[section]
\newtheorem{proposition}[theorem]{Proposition}
\newtheorem{corollary}[theorem]{Corollary}
\newtheorem{lemma}[theorem]{Lemma}
\newtheorem{obs}[theorem]{Observaci\'on}
\newtheorem{defi}[theorem]{Definition}
\newtheorem{notation}[theorem]{Notation}
\newtheorem{notacion}[theorem]{Notaci\'on}
\newtheorem{problem}[theorem]{Problem}
\newtheorem{remark}[theorem]{Remark}
\newtheorem{example}[theorem]{Example}

\DeclarePairedDelimiter{\abs}{\lvert}{\rvert}
\DeclarePairedDelimiter{\Abs}{\left|}{\right|}
\DeclarePairedDelimiter{\norm}{\lVert}{\rVert}
\DeclarePairedDelimiter{\Norm}{\left\|}{\right\|} %TODO este delimitador tiene un inconveniente

\DeclareMathOperator{\img}{Im}
\DeclareMathOperator{\Div}{\textrm{div}}


\begin{document}
\thispagestyle{empty}

\begin {center}

\includegraphics[scale=.3]{logofac.jpg}

\medskip
UNIVERSIDAD DE BUENOS AIRES

Facultad de Ciencias Exactas y Naturales

Departamento de Matem\'atica

\vspace{3cm}

\textbf{\large M\'etodos mixtos con mallas h\'ibridas para problemas el\'ipticos
en dominios poliedrales.}

\vspace{2cm}

Tesis presentada para optar al t\'\i tulo de Doctor de la Universidad de Buenos Aires en el \'area Ciencias Matem\'aticas

\vspace{2cm}

\textbf{Lic. Alexis Jawtuschenko}

\end {center}

\vspace{1.5cm}

\noindent Director de tesis: \ Dr. Ariel Lombardi

\noindent Consejero de estudios: \ Dr. Ariel Lombardi

\vspace{1cm}

\noindent Buenos Aires, 2018.
\newpage
\section*{Resumen}
En esta Tesis introducimos un M\'etodo combinado
de Elementos Finitos y Virtuales en dimensi\'on tres 
para
la aproximaci\'on mixta 
de un problema el\'iptico modelo para el operador
de Laplace en un poliedro arbitrario.
El m\'etodo es analizado por completo
cuando las mallas constan de prismas rectos de base triangular, pir\'amdes y tetraedros. Los espacios locales
discretizantes coinciden con los espacios de orden m\'inimo
de Raviart-Thomas 
sobre tetraedros y prismas, y constituyen una extensi\'on
de estos a elementos piramidales. Probamos que el 
esquema discreto es bien planteado y 
probamos estimaciones \'optimas de error
sobre mallas 
que admiten elementos anis\'otropos. En particular, 
 nuestras estimaciones de 
error
de 
interpolaci\'on 
local 
para el espacio discreto son \'optimas y 
anis\'otropas en prismas rectos anis\'otropos. La motivaci\'on
para trabajar con elemento anis\'otropos es que en
distintas situaciones en aproximaciones por elmentos finitos mixtos
es necesario el uso de mallas con elementos elongados.
Este es el caso, por ejemplo, con la ecuaci\'on de Poisson en un
poliedro $\Omega$ con aristas c\'oncavas y v\'ertices entrantes,
que en forma mixta puede escribirse como
\begin{equation*}\label{mf} \left\{\begin{array}{rcl}
\boldsymbol{u}&=&-\nabla p\qquad \mbox{in }\Omega\\
\dv\boldsymbol{u}&=&f\qquad \mbox{in }\Omega\\
p&=&0\qquad\mbox{on }\partial\Omega.\end{array}\right.
\end{equation*}
En esta caso la variable vectorial de la soluci\'on, $\boldsymbol{u}$,
no est\'a en $H^1(\Omega)$ en el caso general debido a las singularidades 
de aristas y v\'ertices. En particular, cerca de las aristas c\'oncavas,
$\boldsymbol{u}$ es m\'as regular en la direcci\'on a lo largo de \'estas
que transversalmente, y consecuentemente las mallas tienen que ser
adecuadamente refinadas para recuperar el orden \'optimo de convergencia
con respecto al n\'umero de grados de libertad. Tales mallas  contienen
elementos arbitrariamente alargados en la direcci\'on de las aristas sigulares.

Asimismo, proponemos un proceso de mallado
para construir una familia de mallas que nos permite
obtener estimaciones de error de aproximacion global
\'optimas cuando la soluci\'on del problema modelo
presenta singularidades de arista o v\'ertice
puesto que las mallas resultan, por construcci\'on,
adecuadamente gra\-duadas y adaptadas 
a las singularidades, como mencionamos en el p\'arrafo anterior.

Adem\'as en la presente Tesis 
obtuvimos cotas de estabilidad y de error de interpolaci\'on
local para Elementos Finitos prism\'aticos anis\'otropos de orden
arbitrario, tanto para la clase de elementos
conformes en $H(\bcurl)$ como para 
la clase de elementos conformes en $H(\mbox{div})$, que 
constituyen resultados adicionales que extienden algunos hechos te\'oricos
que probamos para el problema principal de la tesis.'

Tambi\'en presentamos 
cotas de estabilidad y de error de interpolaci\'on
local para Elementos Finitos piramidales anis\'otropos 
de orden bajo, tanto para la clase de elementos
conformes en $H(\bcurl)$ como para 
la clase de elementos conformes en $H(\mbox{div})$. Este resultado
est\'a incluido para mostrar una variante a nuestro m\'etodo principal,
esto es, un m\'etodo solamente con elementos finitos. Con respecto a esta variante,
como
mostramos expl\'icitamente en la Tesis, las 
funciones de forma, generadoras de 
estos \'ultimos espacios de Elementos Finitos,
son racionales y son singulares, aunque acotadas,
en la pir\'amide de referencia. Esta es una raz\'on por la cual 
consideramos mejor nuestra aproximaci\'on FE--VE combinada, porque estamos
evitando la evaluaci\'on de funciones con dichas propiedades
en implementaciones en computadoras.

\newpage
\section*{Abstract}
In this Thesis we introduce a combined Finite and Virtual Element Method in dimension three
for the mixed approximation of a 
model elliptic problem for the Laplace operator on an 
arbitrary polyhedron. 
The 
method is fully analysed when the meshes are made up of triangularly
right prisms, pyramids and tetrahedra. The local discrete 
spaces coincide with the lowest order Raviart-Thomas 
spaces on tetrahedra and prisms, and 
extend them to pyramidal elements. The discrete scheme 
is well posed and optimal error estimates are proved on meshes which 
allow for anisotropic elements. In particular, local 
interpolation error estimates for the discrete element space are 
optimal and anisotropic on anisotropic right prisms.
The motivation to work with anisotropic elements is that 
in several situations in mixed finite element approximations the use of meshes 
with narrow elements is needed. 
This is the case for instance when dealing with the Poisson equation 
in a polyhedron $\Omega$ with concave edges or vertices, which, in mixed form
can be written as
\begin{equation*}\label{mf} \left\{\begin{array}{rcl}
\boldsymbol{u}&=&-\nabla p\qquad \mbox{in }\Omega\\
\dv\boldsymbol{u}&=&f\qquad \mbox{in }\Omega\\
p&=&0\qquad\mbox{on }\partial\Omega.\end{array}\right.
\end{equation*}
In this case the vectorial variable of the solution, $\boldsymbol{u}$, is in 
general not in $H^1(\Omega)$ due to vertex and edges 
singularities. In particular, close to concave edges, 
$\boldsymbol{u}$ is expected to be more regular in its direction than transversally 
to it, and consequently the mesh has to be accordingly refined in order to 
recover optimal order of convergence with respect to the number of degrees of 
freedom. Those meshes contain elements that are arbitrarily elongated
in the direction of concave edges.

Likewise, we propose a meshing process to construct a mesh that
allows us to  obtain optimal global 
approximation error estimates when the 
solution has edge or vertex singularities as the mesh results, by construction,
suitably graded and adapted to the singularities.

Furthermore, in the present Thesis we obtained
local anisotropic stability
and interpolation error estimates for arbitrary order Prismatic
Finite Elements in both
the $\bcurl$--conforming and div--conforming classes of elements, which are 
additional results that extend some theoretical facts we proved for
the main problem of the Thesis.

Moreover, we present local anisotropic stability
and interpolation error estimates for lowest order Pyramidal
Finite Elements constructed in the literature for both
the $\bcurl$--conforming and div--conforming classes of elements. This result
is included to show a variant to our main method, that is one 
with only Finite Elements. Regarding this variant, as we show
explicitly in the Thesis, the shape functions 
spanning the pyramidal Finite Element spaces are rational functions and are 
singular, yet bounded,
in the reference pyramid. This is a reason why we considered to move
to our combined FE--VE approach better, because we are avoiding the evaluation
of functions with those properties in computer implementations. 

%\frontmatter

\tableofcontents{}
%\mainmatter
\chapter*{Introducci\'on}
\addcontentsline{toc}{chapter}{Introducci\'on} \markboth{INTRODUCCION}{}
Un objeto f\'isico se dice anis\'otropo si exhibe 
propiedades f\'isicas vectoriales con magnitudes
distintas cuando se las mide en direcciones 
distintas (cfr.~\cite{jamet}). En
el caso de un m\'etodo num\'erico diremos que la familia
de mallas usada es anis\'otropa si contiene sucesiones 
de elementos tales que el orden de decrecimiento de
sus tamaños a lo largo de una direcci\'on es superior
al correspondiente a otras direcciones independientes.
Formalizaremos e ilustraremos esto en el 
Cap\'itulo~\ref{auxlabel207}.

En varias situaciones en aproximaciones por elementos 
finitos mixtos es necesario el uso de mallas anis\'otropas.
Este es el caso, por ejemplo, cuando se trata con la 
ecuaci\'on de Poisson en un poliedro 
$\Omega\subseteq\mathbb{R}^3$ con aristas y v\'ertices
c\'oncavos, la cual, introduciendo la variable vectorial
$\bu=\nabla$ puede escribirse en forma mixta como
\begin{equation}\label{mfSpanish} 
\left\{\begin{IEEEeqnarraybox*}[][c]{,r/c/c/c/l,}
	&&&&\\
	\bu     & = & \nabla p   & \qquad & \mbox{in }\Omega\\
	-\dv\bu & = &        f   & \qquad & \mbox{in }\Omega\qquad(f\in L^2(\Omega))\\
	p       & = & 0          & \qquad & \mbox{on }\partial\Omega.\\
	&&&&%
	\end{IEEEeqnarraybox*}\right.
\end{equation}
La formulaci\'on variacional mixta est\'andar es hallar
$\bu\in H(\Div,\Omega)$ and $p\in L^2(\Omega)$ 
tales que
\begin{equation}\label{PSpanish}
	\begin{IEEEeqnarraybox*}[][c]{l/c/r/C/l}
	\forall \bv\in H(\Div,\Omega)  & \qquad & a(\bu,\bv) + b(\bv,p)   & = & 0    \\
	\forall q\in   L^2(\Omega)	   & \qquad &    		   b(\bu,q)   & = & (-f,q)
	\end{IEEEeqnarraybox*}
\end{equation}
en donde
\[
a(\bv,\bw)=\int_\Omega\bv\cdot\bw\,d\bx,\qquad b(\bv,q)=\int_\Omega q\,\mbox{div\,}\bv\,d\bx.
\]
Dado que $\Omega$ no es convexo, la soluci\'on $\bu$ 
de~\eqref{PSpanish} no pertenece a $H^1(\Omega)$ en
el caso general por tener singularidades de arista y 
v\'ertice. En particular, cerca de aristas c\'oncavas,
se espera que la soluci\'on sea m\'as regular a lo largo
de la direcci\'on de esa arista que transversalmente a ella
(ver~\cite{apelNicaise}), y es por eso que las mallas
tienen que ser adecuadamente graduadas y refinadas
para recuperar el orden \'optimo de convergencia de 
la sucesi\'on de soluciones num\'ericas con respecto al 
n\'umero de grados de libertad (ver~\cite{alw,apelNicaise}).
Ese tipo de mallas contiene elementos que son tan
estrechos como se quiera en direcci\'on ortogonal
a las aristas c\'oncavas, con tal de tomarlos lo 
suficientemente cerca de ellas. Si tom\'aramos 
una sucesi\'on de mallas como esta hecha \'unicamente con
tetraedros, incurrir\'iamos en el uso de subfamilias de 
estos que no verifican ciertas condiciones que son 
necesarias para el an\'alisis anis\'otropo, y es aqu\'i donde
nuestro m\'etodo propuesto  con elementos de geometr\'ias
diferentes cobra sentido.

Para aclarar lo \'ultimo y el principal resultado de esta
Tesis (cfr.~\cite{alexisAriel}) ponemos las siguientes definiciones y resultados
previos relacionados.

Primero, $T$ satisface la \emph{propiedad del
v\'ertice regular}
con una 
constante $\bar{c} > 0$ (escrito $T\in \pazocal{RVP}(\bar c)$) si
$T$ tiene un v\'ertice $\bx_T$ tal que,
tomando $M_T$ como la  matriz cuyas columnas
son los vectores unitarios en las direcciones
de las aristas que comparten a $\bx_T$, entonces
$|\det M_T| > \bar{c}$.

Una propiedad geom\'etrica menos restrictiva  es
la siguiente. 
Diremos que un tetraedro $T$ satisface la
\emph{propiedad del \'angulo m\'aximo}
con par\'ametro $\bar\alpha$
(escrito $T\in\pazocal{MAC}(\bar\alpha$))
si los \'angulos de las caras de $T$
y entre caras son menores a $\bar\alpha$. 

Con estas dos nociones en mente, citamos el siguiente
resultado de~\cite{aadl}. Si $T$ es un tetraedro en 
$\pazocal{RVP}(\bar c)$ y $\br_{\sss k,T}$ 
es el interpolador de Raviart-Thomas
(ver~\cite{nedelec2, MR0483555}), entonces existe una
$C(\bar c)>0$ tal que para toda  
$\bu\in H^1(T)^3$
\begin{IEEEeqnarray}{rCl}\label{rvpspanish}
  \|\bu-\br_{\sss k,T}\bu\|_{\sss L^2(T)^3}& \leqslant & C 
    \left\{\sum_{1\leqslant i\leqslant 3} h_i \|{\s\partial_{\xi_i}}\bu\|_{\sss L^2(T)^3}
	  + h_T \|\dv\bu\|_{\sss L^2(T)}\right\}
\end{IEEEeqnarray}
donde escribimos
$\xi_i$ para las coordenadas locales con origen
en el v\'ertice regular
y $h_i$ para las longitudes
de las aristas incidentes a \'el y 
$h_T$ para el di\'ametro de $T$.

Por otro lado, si $T\in\pazocal{MAC}(\bar\alpha)$ entonces
existe una $C(\bar\alpha)>0$
tal que 
para toda $\bu\in H^1(T)^3$
vale
\begin{IEEEeqnarray}{rCl}\label{macspanish}
  \|\bu-\br_{\sss k,T}\bu\|_{\sss L^2(T)^3}& \leqslant & C h_T \sum_{1\leqslant i\leqslant 3}
  \|{\s\partial_{\xi_i}}\bu\|_{\sss L^2(T)^3}
\end{IEEEeqnarray}
($\xi_i$ son coordenadas locales a partir de
un v\'ertice de $T$, pero
no necesariamente tenemos uno regular).

Observamos que la desigualdad~\eqref{macspanish} es estrictamente
m\'as d\'ebil que~\eqref{rvpspanish}, dado que hay elementos
que satisfacen la condici\'on $\pazocal{MAC}(\bar\alpha)$
para $\bar\alpha$ fijo, pero con par\'ametro $\pazocal{RVP}$
arbitrariamente peque\~no, haciendo que se degenere la
cantidad $C(\bar c)$ en~\eqref{rvpspanish}. Ponemos un ejemplo
en la Figura~\ref{tetraedrosSpanish}. Adem\'as, como est\'a
dicho en~\cite{aadl} mediante un contraejemplo, 
la desigualdad~\eqref{rvpspanish} no puede ser probada bajo
condici\'on de \'angulo m\'aximo solamente.

Volviendo al segundo p\'arrafo, lo que dec\'iamos es que 
es posible construir mallas graduadas anis\'otropas para
problemas el\'ipticos en dominios con singularidades que
consistan solamente de tetraedros que satisfacen 
condic\'on  $\pazocal{MAC}(\bar\alpha)$ para $\alpha<\pi$,
pero estos elementos no satisfacen la condici\'on de $\pazocal{RVP}(\bar c)$
para ning\'un par\'ametro uniforme positivo $\bar c$.
En consecuencia, la estimaci\'on 
del error de interpolaci\'on~\eqref{rvpspanish} no puede ser 
usada de manera global y en cambio~\eqref{macspanish} debe ser tomada,
y por esto las propiedades de anisotrop\'ia de las mallas
pueden no traer ninguna ventaja. En otras palabras,
en la segunda desigualdad una \emph{derivada grande} en
la direcci\'on $\xi_i$ no estar\'ia necesariamente compensada
por un $h_i$ peque\~no, as\'i que tendr\'iamos que hacer
peque\~no al di\'ametro $h_T$ y refinar las mallas en
todas las direcciones, que es lo que queremos evitar, y
perder\'iamos orden $\pazocal{O}(\textit{h})$ en la 
convergencia del error num\'erico con la relaci\'on 
asint\'otica $\textit{h}\sim N_{\Th}^{-\nicefrac13}$
 (aqu\'i $\Th$ es una malla y $N_{\Th}$ es su 
cardinal; el significado concreto meaning 
del par\'ametro de mallado $\textit{h}$, que no es
el di\'ametro de sus elementos, como s\'i lo es en el caso
de las mallas uniformes, se har\'a claro en 
la Subsecci\'on~\ref{meshes} cuando propongamos nuestro proceso
de mallado).

\tetsTikzSpanish

Una idea para sobrellevar la mencionada dificultad, para
el caso en que $\Omega$ es un dominio cil\'indrico 
poliedral, fue propuesta en~\cite{MR1866274}. En este caso,
cuando $f$ est\'a en $L^2(\Omega)$, la soluci\'on puede
exhibir solamente singularidades a lo largo de aristas
c\'oncavas. Los autores proponen un m\'etodo mixto de 
Raviart--Thomas en mallas graduadas de prismas triangulares
y probaron estimaciones \'optimas de error por medio de 
resultados de interpolaci\'on con anisotrop\'ia y de este modo
los tetraedros que no satisfacen una propiedad 
$\pazocal{RVP}$ uniforme son evitados. Tambi\'en proponen
un m\'etodo similar con mallas de tetraedros anis\'otropos
 graduados obtenido subdividiendo cada prisma entre tres
 tetraedros. Por supuesto, estas mallas contienen a los 
tetraedros malosm, y para obtener estimaciones
de error \'optimas el precio pagado es el de requerir m\'as
regularidad al dato $f$, m\'as precisamente, debe pertenecer
a un espacio de Sobolev con pesos.

Uno de los resultados presentados en esta Tesis extiende
los resultados de~\cite{MR1866274} puesto que nuestro
m\'etodo fue buscado con el prop\'osito de poder
lidiar con la aproximaci\'on mixta de~\eqref{mfSpanish}
para $f\in L^2(\Omega)$, y tambi\'en para un poliedro
cualquiera $\Omega$. Como estos dominios no necesariamente
admiten una partici\'on en t\'erminos de prismas rectos y como
tambi\'en nosotros quisi\'eramos evitar pedir
m\'as regularidad al dato $f$, proponemos una discretiza\-ci\'on
basada en mallas h\'ibridas consistentes en prismas combinados
con tetraedros, con brechas interelementales rellenadas
con pir\'amides. Obtenemos la estimaci\'on del error de
aproximaci\'on
\begin{equation}\label{auxlabel410}
 \|\bu-\bu_{\textit h}\|_{L^2(\Omega)^3} + \|p-p_{\textit h}\|_{L^2(\Omega)} 
 \leqslant C\,\textit{h}\|f\|_{L^2(\Omega)}{\mbox{,}}
\end{equation}
en donde $\bu_{\textit h}$ y $p_{\textit h}$ son las
aproximaciones de las soluciones $\bu$ y $p$, por medio
de estimaciones con normas de Sobolev con pesos 
(para $\bu$) tratando las singularidades de manera 
localizada. Primero introducimos y analizamos por completo
un nuevo
m\'etodo combinado de Elementos Finitos y Virtuales
en un poliedro cuando las mallas son construidas con
los mencionados poliedros elementales. Las estimaciones
de interpolaci\'on locales obtenidas 
para los espacios discretos son
\'optimas y anis\'otropas en prismas anis\'otropos, lo que 
nos permite obtener estimaciones \'optimas de error de
aproximaci\'on cuando la soluci\'on tiene tanto singularidades
de aristas como de v\'ertices, usando
mallas graduadas adecuadamente que incluyen
elementos anis\'otropos, y que solamente incluyen tetraedros
con un par\'ametro de condici\'on $\pazocal{RVP}$ uniforme positivo.

No solamente probamos el resultado para una familia abstracta
de tales mallas con las mencionadas propiedades,
 sino que tambi\'en mostramos un proceso general de mallado
 que comienza aislando y clasificando las singularidades
del dominio (cfr. Teorema~\ref{thm_regularity}), y
con eso probamos la existencia de una familia de mallas
como la requerimos, mediante su construcci\'on.

Los espacios discretos $V_{\textit h}$ correspondientes
a las mallas propuestas fueron obtenidos satisfaciendo
las siguientes condiciones, adecuadas para una discretiza\-ci\'on
en $\mathbb{R}^3$.
\begin{enumerate}
	\item \label{auxlabel411} Conformidad.
	\item \label{auxlabel412} Propiedades de aproximaci\'on anis\'otropas y \'optimas.
	\item \label{auxlabel413} Generalidad de dominio.
\end{enumerate}
Tal como es sugerido en~\cite{bfm} para el caso $2D$,
presentamos $V_{\textit h}$ como un espacio de elementos 
virtuales que coincide localmente con el espacio original
$3D$ de Raviart--Thomas en tetraedros y prismas, y lo
extiende de manera natural a elementos piramidales. 
En particular, las componentes normales de las
funciones discretas son constantes sobre las caras de
los elementos, coincidiendo de manera conforme a trav\'es 
de caras comunes a elementos de distinta geometr\'ia. De
este modo el requerimiento~\ref{auxlabel411} fue satisfecho.
Una ventaja de esta presentaci\'on es que la definici\'on
de los espacios locales es independiente de la geometr\'ia 
del elemento, ver Secci\'on~\ref{auxlabel290}. Como ya comentamos,
el \'item~\ref{auxlabel413} es verificado mediante la 
incorporaci\'on de pir\'amides de base paralelogramo.
Con respecto al punto~\ref{auxlabel412}, presentamos
estimaciones \'optimas de estabilidad y de interpolaci\'on
anis\'otropas en varias partes aunque no todas ellas
son usadas en la aplicaci\'on final, en algunos casos
porque los espacios funcionales para los cuales son usadas
no se ven involucrados en el problema 
modelo~\eqref{PSpanish} y en otros casos porque
nuestro esquema de mallado funciona sin requerir
que los elementos de todos los tipos presenten 
anisotrop\'ia.

Tambi\'en podr\'ian ser consideradas mallas con elementos
de geometr\'ia m\'as general. De hecho esta es una 
de las principales propiedades de los m\'etodos
de elementos virtuales. Nosotros decidimos
restringirnos a unas pocas geometr\'ias elementales
pues nuestro principal objetivo fue admitir 
mallas anis\'otropas, y con estimaciones anis\'otropas
uniformemente v\'alidas. Una dificultad que aparece
cuando se consideran otras geometr\'ias elementales 
(como prismas obl\'icuos) es que los espacios virtuales
locales no necesariamente se ve preservado 
por transformaciones \emph{push--forward} basadas 
en aplicaciones afines (nosotros 
usamos una propiedad de $\bcurl$ nulo que no
es invariante en esta situaci\'on), con
la consecuencia de que los argumentos de
reescalado est\'andar son dif\'iciles de usar
(ver Secci\'on~\ref{auxlabel290}). Sin embargo, esto no deviene
en una limitaci\'on en los dominios poliedrales que
podemos tratar con nuestro m\'etodo, pues podemos
restringirnos a usar prismas rectos duplicando, en el
peor de los casos, el n\'umero de elementos de la
primera malla, y as\'i aumentando el n\'umero de elementos
en una malla cualquiera posterior en un factor constante.

Nuestro m\'etodo presenta distinta cantidad de grados
de libertad a trav\'es de la malla, pues prismas, tetraedros
y pir\'amides tienen distinta cantidad de caras, y tambi\'en
admite distintas geometr\'ias para los grados de libertad pues
estamos trabajando con caras triangulares y cuadril\'ateras
al mismo tiempo. Como consecuencia de esto, nuestro m\'etodo
es en cierto sentido una generalizaci\'on de los
m\'etodos de elementos virtuales cl\'asicos, en los cuales todos
los elementos tienen la misma geometr\'ia arbitraria, y de 
cada elemento de toma el mismo n\'umero de grados de libertad, todos
del mismo tipo (por ejemplo, evaluaci\'on en los mismos puntos en
el borde de cada elemento).

Las mallas h\'ibridas que incluyen tetraedros y prismas (y hasta
hexaedros) pueden ser necesarias para satisfaces las demandas
de la geometr\'ia espec\'ifica de un problema (regiones no triviales)
o para alcanzar c\'alculos eficientes. Si se requiere que estas mallas
eviten nodos colgantes entonces a menudo incluir\'an pir\'amides; ver
por ejemplo~\cite{owenSaigal}. Varios art\'iculos contienen
la construcci\'on de elementos finitos conformes en pir\'amides,
algunos de los cuales son~\cite{bergot} 
para elementos $H^1$ y~\cite{gh99, Nigam-2012}
para elementos $H(\mbox{div})$ y $H({\bf curl})$,
el primero para orden bajo y el segundo para orden arbitrario.
En~\cite{Nigam-2012} los autores prueban que no es posible
construir elementos finitos $H^1$ \'utiles en pir\'amides
usando solamente funciones polinomiales y muestran que, 
en el caso $H(\mbox{div})$,
todos los espacios construidos en la literatura contienen
funciones no polinomiales.

Queremos finalizar esta introducci\'on mencionando otros
resultados que obtuvimos, presentados como resultados
adicionales porque no fueron usados para el problema principal 
de la Tesis, y que pueden ser vistos como extensiones 
de ciertos resultados te\'oricos.

Probamos estimaciones locales anis\'otropas 
de estabilidad y de error 
de interpolaci\'on para elementos finitos prism\'aticos
tanto para las clases de elementos 
$\bcurl$--conformes como para las div--conformes.
Nuestro m\'etodo usa solo el caso de orden bajo de las 
estimaciones en $H(\Div)$ y dejamos una versi\'on 
de orden alto del m\'etodo para investigaciones futuras.

Finalmente, obtuvimos y presentamos estimaciones locales anis\'otropas 
de estabilidad y de error 
de interpolaci\'on para los elementos finitos piramidales
introducidos en~\cite{gh99, Nigam-2012} 
tanto para las clases de elementos 
$\bcurl$--conformes como para las div--conformes.
Como mostramos en el Cap\'itulo~\ref{auxlabel202}, las funciones
de forma que generan estos espacios son racionales y singulares,
aunque acotadas, en la pir\'amide de referencia. Por esta 
raz\'on consideramos que nuestro m\'etodo combinado FE--VE
presenta una ventaja, la de evitar las evaluaciones de 
funciones con esas propiedades en implementaciones en
computadoras.
% section intro (end)
\chapter{Introduction}
\section{Motivation of the Problems and Previous Results} % (fold)
Input of using prisms combined with tetrahedra, with gaps filled with pyramids.
We say that a tetrahedron $T$ satisfies the ``regular vertex property'' with a
constant $c > 0$ (written RVP($c$)) if $T$ has a vertex $\bx_T$ such that,
if $M_T$ is the matrix made up of the unitary vectors in the directions
of the edges sharing $\bx_T$ as columns, then $|\det M_T| > c$.

If we mesh using only tetrahedra we will incur in using sequences of tetrahedra
wich do not fullfil uniformly the RVP($c$) for any positive constant $c$.

{\color{Orange}\#\#\#\#\#\#\#\# seguir aca.}


In other words, deficiency in using $\pazocal{F}_2$ class tetrahedra: we don't
have $\textit{h}$ order with the asymptotic relation 
$\textit{h}\sim N_{\Th}^{-\nicefrac13}$ preserved because we use strongly
that
\begin{IEEEeqnarray*}{rCl}
  \|\bu-\br_{\sss k,T}\bu\|_{\sss L^2(T)}& \leqslant & \sum_{1\leqslant i\leqslant 3} h_i \|{\s\partial_{\xi_i}}\bu\|_{\sss L^2(T)}
  	+ h_T\|\dv \bu\|_{\sss L^2(T)}
\end{IEEEeqnarray*}
for the family $\pazocal{F}_1$ while, for the family $\pazocal{F}_2$, the sharp
estimation is
\begin{IEEEeqnarray*}{rCl}
  \|\bu-\br_{\sss k,T}\bu\|_{\sss L^2(T)}& \leqslant & h_T \sum_{1\leqslant i\leqslant 3}
  \|{\s\partial_{\xi_i}}\bu\|_{\sss L^2(T)}.
\end{IEEEeqnarray*}
In the latter case, a big derivative in the direction $\xi_i$ will not be 
compensated with a small $h_i$, so we necessarily have to make the diameter
$h_T$ small and refine the meshes in all the directions.



{\color{Orange}\#\#\#\#\#\#\#\# intro vems and Motivation}
% section intro (end)
\chapter{Preliminaries}\label{auxlabel207}
\section*{Introducci\'on al cap\'itulo}
En este cap\'itulo recolectamos definiciones y propiedades
dentro de la teor\'ia que desarrollamos y fijamos nuestra 
elecci\'on de notaci\'on, est\'andar en la mayor\'ia de los casos 
en la literatura.
\section*{Introduction to the chapter}
In this chapter we gather definitions and properties
within the theory we develop and
fix our chosen notation, which is standard in most cases in
the literature.
\label{chap_prelim}
\section{Preliminaries} % (fold)
\label{sec:preliminaries}
\subsection{Definitions and Notations} % (fold)
\label{sub:definitions_notations}
For the first definition we refer the reader to~\cite{ciarlet}. 
\begin{defi}
A Finite Element in $\mathbb{R}^n$ is defined by the following triple $(E, P_E, \Sigma) $:
\begin{enumerate}
  \item 
$E$ is an open non empty polytope of $\mathbb{R}^n$ with Lipschitz--continuous 
boundary.
  \item
$P_E$ is a finite--dimensional space of real--valued scalar or vectorial 
functions defined over $E$.
  \item
$\Sigma$ is a finite set of linearly independent linear functionals acting 
on a functional space containing $P_E$. The elements of $\Sigma$ are often
referred to as  \emph{degrees of freedom} or \emph{moments} of the Finite
Element.
\end{enumerate}
\end{defi}
\begin{defi} 
Given  a Finite Element $E$ set
\begin{IEEEeqnarray*}{rCl}
  h_E & = & \mbox{ diameter of the element } E.\\
  \rho_E & = & \mbox{ supremum of the diameters of the spheres contained in } E.
\end{IEEEeqnarray*}
Consider a family of meshes $\{\pazocal{T}_n\}_{n\in\mathbb{N}}$ such that 
$\max_{E\in\pazocal{T}_{n}} h_E$
tends to zero as $n$ tends to infinity.
\begin{itemize}
	\item [i)] The family of meshes is isotropic if 
	there is a positive constant $\sigma$ such that
	for each $n$ and each $E\in\pazocal{T}_n$ 
	\[
		h_E \leqslant \sigma\rho_E.
	\]
	\item [ii)] The family of meshes is anisotropic if it is not
	isotropic.
\end{itemize}
\end{defi}
\begin{defi}
  If $V(E)$ denotes a functional space over an element $E$ from a 
  mesh $\pazocal{T}_n$ for a domain $\Omega$, the symbol $V(\pazocal{T}_n)$
  denotes the space of functions in $\Omega$ whose restrictions to each 
  $E\in\pazocal{T}_n$ belongs to $V(E)$.
\end{defi}
\subsection{Polynomials} % (fold)
\label{sub:polynomials}
The finite elements involved in this Thesis are built with piecewise
polynomial functions or rational functions (that is, quotients of polynomials).
As we will see, the virtual elements also involve functions of these
two kinds amongst others.
For that reason we state some handy notations here.
\begin{defi}
\begin{IEEEeqnarray*}{rCl}
          P_k & = & \{\,\mbox{polynomials of degree less than or equal to k}\,\}.\\[5pt]
  \tilde{P}_k & = & \{\,\mbox{homogeneous polynomials of degree k}\,\}.\\[5pt]
\end{IEEEeqnarray*}
In the case of a line segment $\be$ or a subdomain $f$ of a plane (typically
an edge or a face of a polyhedron) we will write interchangeably
\begin{IEEEeqnarray*}{rCcCl}
	P_k (\be) & = & P_k(t) & = & \{\,\mbox{polynomials of maximum degree k in arc length on $\be$}\,\}\mbox{,}\\[5pt]
	P_k (f) & = & P_k(t_1,t_2) & = & \{\,\mbox{polynomials of maximum degree k in ${\xi_1}$, ${\xi_2}$ on f}\,\}\mbox{,}
\end{IEEEeqnarray*}
where we have used an orthogonal coordinate system $(\xi_1,\xi_2)$ in the plane
containing $f$ and the variable $\xi$ along the edge $\be$.
Moreover, we will make the identifications
\begin{IEEEeqnarray*}{rClCrCl}
	P_k(\be)  & = & \{p|_{\be}\,:\,p\in P_k\} &\quad\mbox{ and }\quad&P_k(f)
			& = & \{p|_f\,:\,p\in P_k\}.
\end{IEEEeqnarray*}
\end{defi}
\noindent The tensor product of polynomials. % (fold)
\begin{defi} \label{tensor_product} Given two polynomials 
\begin{IEEEeqnarray*}{rClcrCl}
	p(x)&=&\sum_i a_ix^i&\quad\mbox{ and }\quad&q(y)&=&\sum_j b_jy^j
\end{IEEEeqnarray*}
the tensor product of $p$ and $q$ is the following
polynomial of \emph{separate variables}
\begin{IEEEeqnarray}{rCl}
	p\otimes q (x,y)&=&\sum_{i,j}a_ib_jx^iy^j.
\end{IEEEeqnarray}
If $\mathbb{Q}_1$ and $\mathbb{Q}_2$ are two polynomial spaces
\begin{IEEEeqnarray}{rCl}
	\mathbb{Q}_1\otimes \mathbb{Q}_2 & = &\{p\otimes q\,|\,p\in\mathbb{Q}_1, q\in\mathbb{Q}_2\}.
\end{IEEEeqnarray}
\end{defi}
\begin{remark}\label{tensor_prod_dim} An observation that we will use several times is that,
from the definition, $\dim \pazocal{P}\otimes \pazocal{Q} = \dim \pazocal{P} \dim \pazocal{Q}$.
\end{remark}
% subsubsection tensor_product (end)
% subsection polynomials (end)
\subsection{Functional Spaces and Trace Spaces} % (fold)
\label{sub:functional_spaces_trace_spaces}
Let $\bx$, $\by$, \dots\, denote points in $\mathbb{R}^n$
and let $d\bx$, $d\by$, \dots denote Lebesgue measure. If $\Omega$
is a measurable set, $1\leqslant p\leqslant\infty$ and $f$ is a 
real or complex valued measurable function, we say $f\in L^p(\Omega)$
if
\begin{IEEEeqnarray*}{rCcCl}
    \|f\|_{L^p(\Omega)}\,=\,\|f\|_{0,p,\Omega}& := &\left\{\int_\Omega |f(\bx)|^p\,d\bx\right\}^{\nicefrac1p} 
    & < & \infty
\end{IEEEeqnarray*}
with the usual modification when $p=\infty$.

Let $\mathbb{Z}_{\geqslant 0}$ denote the set of nonnegative integers. A
\emph{multi--index} $\balpha$ is an $n$--tuple of nonnegative integers:
$\balpha = (\alpha_1,\ldots,\alpha_n)$, $\alpha_i\in\mathbb{Z}_{\geqslant 0}$,
${\s1\leqslant i\leqslant n}$. With multi--indices we will establish the following
notations:
\begin{IEEEeqnarray*}{rCl}
  |\balpha|&=&\sum_i\alpha_i\mbox{,}\\[5pt]
  \balpha&\leqslant&\boldsymbol{\beta}\mbox{\quad iff\quad} \alpha_i\leqslant\beta_i
  \mbox{,\quad}
  {\s 1\leqslant i\leqslant n}\mbox{,}\\[5pt]
  \balpha+\boldsymbol{\beta}&=&(\alpha_1+\beta_1,\ldots,\alpha_n+\beta_n)\mbox{,}\\[5pt]
  \balpha-\boldsymbol{\beta}&=&(\max\{\alpha_1-\beta_1,0\},\ldots,\max\{\alpha_n-\beta_n,0\})\mbox{,}\\[5pt]
  \balpha!&=&\Pi_{i=1}^n \alpha_i!\mbox{,}\\[5pt]
  \bx^{\balpha}&=&\Pi_{i=1}^n x_i^{\alpha_i}\mbox{\quad and}\\[5pt]
  {\s\partial}^{\balpha}\,=\,\left(\frac{{\s\partial}}{{\s\partial}\bx}\right)^{\balpha}&=&\frac{{\s\partial}^{|\balpha|}}{{\s\partial} x_1^{\alpha_1}
  \ldots{\s\partial} x_n^{\alpha_n}}\\[5pt]
  &=&\Pi_{i=1}^n \left(\frac{{\s\partial}}{{\s\partial} x_i}\right)^{\alpha_i}.
\end{IEEEeqnarray*}
If $|\alpha| = 0$, $\partial^{\alpha}f=f$.\\

Let $\Omega$ be an open set with Lipschitz boundary, let $\pazocal{C}^\infty(\Omega)$ denote
the space of infinitely differentiable functions in $\Omega$.                            %For $f\in\pazocal{C}^\infty(\Omega)$
Let $\mathcal{D}(\Omega)$ denote the vectorial subspace of $\pazocal{C}^\infty(\Omega)$
functions that have compact support in $\Omega$, called the \emph{space of test functions}
with the usual topology (cfr.~\cite{rudin}, page 136) 
for which we say a sequence $\phi_n\to 0$ in the topology of $\mathcal{D}(\Omega)$
if and only if there is a compact $K\subseteq\Omega$ which contains the support
of every $\phi_n$, and $\partial^{\alpha}\phi_n\to 0$ uniformly on $K$, as $n\to\infty$,
for every multi--index $\balpha$. The dual $\mathcal{D}'(\Omega)$ of $\mathcal{D}(\Omega)$
is called the space of \emph{distributions} on $\Omega$. If $\phi\in\mathcal{D}'(\Omega)$
and $\alpha$ is a multi--index, $\partial^{\alpha}\phi$ is called a distributional
or weak derivative of $\phi$, where $\partial^{\alpha}\phi$ is defined by
\begin{IEEEeqnarray*}{rCl}
  (\partial^{\alpha}\phi)(f) & = & (-1)^{|\alpha|}\phi(\partial^{\alpha}f)\mbox{,\qquad}
    f\in\mathcal{D}(\Omega).
\end{IEEEeqnarray*}
A distribution $\phi\in\mathcal{D}'(\Omega)$ will be identified with a function
$\psi$ defined on $\Omega$ if for each $f\in \mathcal{D}(\Omega)$, $\psi f\in L^1(\Omega)$
and $\phi(f) = \int_\Omega \psi f\,d\bx$. In this case we shall let
$\phi$ denote the identified function, $\psi$, as well. 

If $m\in\mathbb{Z}_{\geqslant 0}$ and if for each multi--index $\alpha$
with $|\alpha|\leqslant m$, $\partial^{\alpha}\phi$ is  given by a function such that
\begin{IEEEeqnarray*}{rCcccCl}
  \|\phi\|_{W^{m,p}(\Omega)} & = & 
  \|\phi\|_{m,p,\Omega} & := & 
  \left\{\sum_{|\alpha|\leqslant m}\|{\s\partial}^\alpha\phi\|^p_{L^{p}(\Omega)}\right\}^{\nicefrac1p} 
  & < & \infty\mbox{,}
\end{IEEEeqnarray*}
then we will say $\phi\in W^{m,p}(\Omega)$. For $\phi\in W^{m,p}(\Omega)$ let
\begin{IEEEeqnarray*}{rCcCl}
  |\phi|_{W^{m,p}(\Omega)} & = & |\phi|_{m,p,\Omega} 
    & := & \left\{\sum_{|\alpha|=m}\|\partial\phi\|_{L^{p}(\Omega)}\right\}^{\nicefrac1p}.
\end{IEEEeqnarray*}
In the previous notation we will skip the number $2$ for the special case $p=2$ and 
write $\|\phi\|_{W^{m,2}(\Omega)}=\|\phi\|_{m,\Omega}$,
$|\phi|_{W^{m,2}(\Omega)}=|\phi|_{m,\Omega}$.
Sometimes we will write, in the vectorial case, $|\bu|^p = |u_1|^p + |u_2|^p + |u_3|^p$.\\

In the definitions of the following two functional spaces, the differential
operators must be understood in the distributional sense.
\begin{defi} As in page 26 of~\cite{giraultRaviart} we make the present definition. Let 
$\Omega\subseteq\mathbb{R}^3$ be a lipschitz domain.
  \begin{IEEEeqnarray*}{rCl}
    H(\Div, \Omega) & := & \{\bu\in L^2(\Omega)^3\,:\,\dv \bu \in L^2(\Omega)\}\mbox{,}\\[5pt]
    \|\bu\|_{H(\Div, \Omega)} & := & \left(\|\bu\|^2_{L^2(\Omega)^3}+
      \|\dv \bu\|^2_{L^2(\Omega)}\right)^{\nicefrac12}.
  \end{IEEEeqnarray*}
\end{defi}
\begin{defi} As in page 32 of~\cite{giraultRaviart} we make the present definition. Let 
$\Omega\subseteq\mathbb{R}^3$ be a lipschitz domain.
  \begin{IEEEeqnarray*}{rCl}
    H(\bcurl,\Omega) & := & \{\bu\in L^2(\Omega)^3\,:\,\curl \bu \in L^2(\Omega)^3\}\mbox{,} \\[5pt]
    \|\bu\|_{H(\bcurl,\Omega)} & := & \left(\|\bu\|^2_{L^2(\Omega)^3}+
      \|\curl\bu\|^2_{L^2(\Omega)^3}\right)^{\nicefrac12}.
  \end{IEEEeqnarray*}
\end{defi}
\begin{defi} As in page 4 of~\cite{ariel} we make the present definition.
\begin{IEEEeqnarray*}{rCl}
  W^p(\bcurl, \Omega) & = & \{\bu\in W^{1,p}(\Omega)^3\,:\,
  \curl\bu\in W^{1,1}(\Omega)^3\}\mbox{,}\\
  \label{normaWpcurl}\yesnumber \|\bu\|_{_{W^p(\bcurl, \Omega)}} & = & 
  \|\bu\|_{_{W^{1,p}(\Omega)}} +
  \| \curl\bu \|_{_{W^{1,1}(\Omega)}}. 
\end{IEEEeqnarray*}
\end{defi}
The concept of a component that is normal to the boundary can be extended
to $H(\Div,\Omega)$ using a density argument. This is what we call \textsl{the
normal trace}. Let $\bnu$ be, in the following, the unit outward normal
to $\partial\Omega$. For a function $\bv\in\pazocal{C}^\infty(\overline{\Omega})^3$
the normal trace $\gamma_{\boldsymbol{\nu}}$ is defined simply by
\begin{IEEEeqnarray}{rCl}\label{normal_trace}
  \gamma_{\boldsymbol{\nu}}(\bv) & := & \bv|_{\partial\Omega}\cdot\boldsymbol{\nu}.
\end{IEEEeqnarray}
The proof of next Theorem can be found in~\cite{monk}, page $53$ and uses merely
the density of $\pazocal{C}^\infty(\overline{\Omega})^3$ in $H(\Div,\Omega)$.
\begin{theorem} The operator $\gamma_{\boldsymbol{\nu}}$ defined in~(\ref{normal_trace})
can be extended by continuity to a continuous linear map $\gamma_{\boldsymbol{\nu}}$ from
$H(\Div,\Omega)$ onto $(H^{\nicefrac12}(\partial\Omega))'.$
\end{theorem}
With regard to trace properties in $H(\bcurl)$, we must verify that
functions in this space have a well--defined tangential trace (cfr.
page 34 of~\cite{giraultRaviart} and page 59 of~\cite{monk}). The next Definition
and proof of the Theorem after it can be found in~\cite{chenDuZou} and gives the
form that the surface degrees of freedom in $H(\bcurl)$--Conforming
Elements will take.

For a smooth vector function $\bv\in\pazocal{C}^\infty(\overline{\Omega})^3$ we define
the two traces
\begin{IEEEeqnarray}{rCl}
\label{aux_label80}\gamma_t(\bv)&=&\bnu\times\bv|_{\partial\Omega}\\
\gamma_T(\bv)&=&(\bnu\times\bv|_{\partial\Omega})\times\bnu
\end{IEEEeqnarray}
Theorem $3.29$ in~\cite{monk} states that~(\ref{aux_label80}) can be extended
by continuity to a continuous linear map from $H(\bcurl,\Omega)$ into 
$(H^{\nicefrac12}(\partial\Omega)')^3$
\begin{defi}
  \begin{IEEEeqnarray}{rCl}
  \nonumber
    Y(\partial\Omega) &=& \left\{ \boldsymbol{f}\in 
    (H^{\nicefrac12}(\partial\Omega)')^3\,:\,\mbox{ there exists }\bu\in 
    H(\bcurl,\Omega) \right.\\
    &&\quad\left.\mbox{with } \gamma_t(\bu) = \boldsymbol{f}\right\}\mbox{,}
  \end{IEEEeqnarray}
  \begin{IEEEeqnarray*}{rCl}
    \|\boldsymbol{f}\|_{Y(\partial\Omega)} &=& 
    \inf_{\bu\in H(\bcurl,\Omega), \gamma_t(\bu) = \boldsymbol{f}}
    \|\bu\|_{H(\bcurl,\Omega)}.
  \end{IEEEeqnarray*}
\end{defi}
So we can consider the epimorphism $\gamma_t\,:\,H(\bcurl,\Omega)\rightarrow
Y(\partial\Omega)$ with which we have the following Theorem (cfr. Theorem 3.31
in~\cite{monk}).
\begin{theorem} The map 
$\gamma_T\,:\,H(\bcurl,\Omega)\rightarrow
Y(\partial\Omega)'$ is well--defined. For any $\bv$, $\bphi$ in $H(\bcurl,\Omega)$
\begin{IEEEeqnarray}{rCl}\label{aux_label5}
   \int_\Omega \curl\bv\cdot\bphi\,d\bx 
    - \int_\Omega \bv\cdot\curl\bphi\,d\bx & = & 
    \langle\gamma_t(\bv), \gamma_T(\bphi)\rangle_{\partial\Omega}
 \end{IEEEeqnarray}
($\langle\cdot,\cdot\rangle$ denotes a duality pairing).
\end{theorem}
%%=============================================================================
% {\color{blue}\#\#\#\#\#\#\#\# es el de pag. 51 Adams?.}
% {\color{blue}\#\#\#\#\#\#\#\# tal vez mejor el Teorema 3.26 de monk, pag.55?.}
% {\color{blue}\#\#\#\#\#\#\#\# poner un esbozo de la prueba, mirar chap 3 adams
%  si queda tiempo.}
%%=============================================================================
\begin{defi}
  We shall say that a domain $\Omega$ has the \textsl{segment property}
  if for every $x$ in the boundary of $\Omega$ there exists an open set
  $U_x$ and a nonzero vector $y_x$ such that $x\in U_x$ and if 
  $z\in\overline{\Omega}\cap U_x$, then $z+ty_x \in \Omega$ for $0<t<1$.
\end{defi}
A domain having this property must have $(n-1)$--dimensional boundary
and cannot simultaneously lie on both sides of any given part of its
boundary.  Since the polyhedra we will consider to build finite elements are
  convex and obviously satisfy the segment property,
  with techniques that can be learned from Chapter $3$
  in the book~\cite{adams} we can prove the following Proposition for our 
  elements.
\begin{proposition}\label{density_wpcurl}
  The space $\pazocal{C}^\infty(\bar{E})^3$ is dense in
  $W^p(\bcurl,{E})$ with its norm defined in~(\ref{normaWpcurl}).
\end{proposition}

With elementary applications of integration by parts for distributional
derivatives we get the following two Lemata.
\begin{lemma} Let two disjoint Lipschitz domains $\Omega_1$ and $\Omega_2$
be given  in $\mathbb{R}^3$, such that $\overline{\Omega_1}\cap\overline{\Omega_2}$ is an
$2$--dimensional surface $f$ of positive measure. Take
$\Omega = \Omega_1\cup \Omega_2\cup f$. Let $\bu_1 \in H(\Div,\Omega_1)$ 
and $\bu_2 \in H(\Div,\Omega_2)$. Consider 
\begin{IEEEeqnarray*}{rCl}
  \bu & = &
    \begin{cases}
      \bu_1, &\text{ on }\Omega_1\\
      \bu_2, &\text{ on }\Omega_2     
    \end{cases}.
\end{IEEEeqnarray*}
Then $\bu$ is in $H(\Div,\Omega)$ if and only if
the normal traces of $\boldsymbol{u}_1$ and $\boldsymbol{u}_2$ coincide on $f$.
\end{lemma}
\begin{lemma} Let two disjoint Lipschitz domains $\Omega_1$ and $\Omega_2$
be given  in $\mathbb{R}^3$, such that $\overline{\Omega_1}\cap\overline{\Omega_2}$ is an
$2$--dimensional surface $f$ of positive measure. Take
$\Omega = \Omega_1\cup \Omega_2\cup f$. Let $\bu_1 \in H(\bcurl,\Omega_1)$ 
and $\bu_2 \in H(\bcurl,\Omega_2)$. Consider 
\begin{IEEEeqnarray*}{rCl}
	\bu & = &
	  \begin{cases}
	  	\bu_1, &\text{ on }\Omega_1\\
	  	\bu_2, &\text{ on }\Omega_2	  	
	  \end{cases}.
\end{IEEEeqnarray*}
Then $\bu$ is in $H(\bcurl,\Omega)$ if and only if
the tangential traces $\gamma_T(\cdot)$
of $\bu_1$ and $\bu_2$, as defined in~(\ref{aux_label5}), coincide on $f$.
\end{lemma}
%=================
%\begin{proof}
%	{\color{red} TODO}
%\end{proof}
%=================
% subsection functional_spaces_trace_spaces (end)

\begin{defi}\label{defi_of_ref_prism}
  Let $\hat T$ be the triangle $\{ 0 < x + y < 1 \}$. 
  The reference prism is the interior of 
  $\hat T\times\{ 0 < z < 1 \}$ (cfr. Figure~\ref{reference_prism}).
\end{defi}
\begin{defi}\label{defi_of_ref_pyr}
The reference pyramid $\hat E$ is the polyhedron 
$\{\bx\in\mathbb{R}^3\,:\,0< x_3< 1,
0<  x_1<  1-x_3, 0<  x_2<  1-x_3\}$
with vertices at $(0,0,0)'$,
$(1,0,0)'$, $(0,1,0)'$, $(1,1,0)'$ and $(0,0,1)'$ (cfr. Figure~\ref{refpyr}).
\end{defi}
\begin{defi}\label{def_of_ref_elems}
The reference tetrahedron is (cfr. Figure~\ref{reftetr})
$\{\bx\in\mathbb{R}^3\,:\,x_1 > 0, x_2 > 0, x_3 > 0, x_1+x_2+x_3 < 1\}$.
\end{defi}

\begin{figure}
	\centering
  \subfloat[Pyramid]
  {
    \label{refpyr}
    \referencePyramidTikz{1.5}
  }
  \hspace{1cm}
  \subfloat[Prism]
  {
    \label{reference_prism}
    \referencePrismTikz{1.5}
  }\\
  \subfloat[Tetrahedron]
  {
    \label{reftetr}
    \referenceTetrahedronTikz{1.5}
  }
  \hspace{1cm}
  \subfloat[Triangle]
  {
    \label{reference_triangle}
    \referenceTriangleTikz{1.3}
  }
	\caption{Reference Elements}
\end{figure}

\section{Regularity of the solution for a model Elliptic Problem}
\label{sec:regularity}
\macroRegularity
%\macroPrismRegularity
\noindent We are concerned in solving 
the following problem.
\begin{problem}\label{mixedContinuous}
Suppose we have a simply connected Lipschitz polyhedron
$\Omega\subseteq\mathbb{R}^3$ which is not convex. Given $f\in L^2(\Omega)$
we look for a $\bu\in H(\dv,\Omega)$ such that 
\begin{IEEEeqnarray*}{rCl}
  \boldsymbol{u}              & = & \nabla p \\
  -\text{div\,}\boldsymbol{u} & = & f \\
   p|_{\partial\Omega}
  & = & 0.
\end{IEEEeqnarray*}
\mbox{\color{Orange}la cond de borde VA?\quad COMO LO PONGO? \quad (ver apunte de FEM2012 talvez)}
\end{problem}
\begin{problem}[Weak formulation.]\label{weakMixedContinuous}
Suppose we have a simply connected Lipschitz polyhedron
$\Omega\subseteq\mathbb{R}^3$ which is not convex. Given $f\in L^2(\Omega)$,
find       $(\bu, p)  \in  H(\dv, \Omega) \times L^2(\Omega)$ 
    such that for all   $(\bv, q)  \in  H(\dv, \Omega) \times L^2(\Omega)$
  \begin{IEEEeqnarray*}{rCl}
    \int_{\Omega} \bu\cdot\bv\,d\bx + 
    \int_{\Omega} p\,\dv\bv\,d\bx                     & = & 0\\
     -\int_{\Omega} q\dv\bu\,d\bx     & = & 
    \int_{\Omega} f\,q\,d\bx.    
  \end{IEEEeqnarray*}
\end{problem}
Due to the polyhedron not being convex there will be ingoing
vertices and edges whose interior angles are obtuse. In the following
we are going to recall how to formalize the singularities in a polyhedron
and how to classify the regularity of the solution of problem~(\ref{mixedContinuous}).\\

Recall the description in page 522 of~\cite{apelNicaise} for the following 
definition. 
\begin{defi}
Suppose that a vertex
$\bv\in\partial\Omega$ is the origin of our Cartesian system of coordinates. Let $C_{\bv}$ be the infinite
polyhedral cone of $\mathbb{R}^3$ which coincides with $\Omega$ in a neighbourhood of 
$\bv$. Set $G_{\bv} = C_{\bv}\cap S^2(\bv)$, the intersection of $C_{\bv}$ with
the unit sphere centered at $\bv$. The vertex singular exponent related to $\bv$
is defined by $\lambda_{\bv} := -\nicefrac12 + \sqrt{\lambda_{\bv,1} + \nicefrac14}$
where 
$\lambda_{\bv,n} > 0$, $n\in\mathbb{N}$, are the eigenvalues, in increasing order, of a positive
Laplace--Beltrami operator $\Delta'$ on $G_{\bv}$ with Dirichlet boundary
conditions. For any edge of $\partial\Omega$ the edge
singular exponent is 
$\pi/\omega_{\be}$ where $\omega_{\be}$ is the interior angle between
the faces sharing  $\be$.
We say that
  \begin{enumerate}
    \item $\be$ is singular if $\lambda_{\be} < 1.$ 
    \item $\bv$ is singular if $\lambda_{\bv} < \dfrac12$
  \end{enumerate}
\end{defi}
The motivation of calling \textsl{singular} an edge or a vertex is that
the solutions of elliptic problems in non--convex domains have singularities
at those vertices or approaching those edges.

To talk about the regularity of a solution in a \emph{non--convex} domain
we will make use of the following weighted Sobolev spaces.
\begin{defi} Suppose $\Omega$ is a non--convex polihedron where $\Lambda \subseteq \Omega$ is
a subdomain such that 
$\bar{\Lambda}$ contains at most one singular vertex $\bv$ or at most one singular
edge $\be$ of $\Omega$.
In case it contained both, the edge is incident to the vertex. Fix a local system
of coordinates in $\Lambda$ with origin at $\bv$ (in case we had a singular
edge and no singular vertex, the origin is at one end of the edge).
$R(\bx)$ will be the distance from $\bx$ to $\bv$,
$r(\bx)$ will be the distance from $\bx$ to $\be$ and $\theta(\bx)$ will be
the \textsl{angular distance} $\theta(\bx)=\frac{r(\bx)}{R(\bx)}$. With these weights,
given two 
positive parameters $\beta$ and $\delta$ we define the norm
\begin{IEEEeqnarray}{rCl}\label{weighted_norm}
  \|v\|^{1,2}_{\beta,\delta} & := & \left\{\sum_{|\alpha|\leqslant 1}
  \|R^{\beta-1+|\alpha|}\theta^{\delta-1+|\alpha|}\,\partial^\alpha v\|_{L^2(\Lambda)}^2\right\}^{1/2}.
\end{IEEEeqnarray}
%\left\{v\in \mathcal D'(\Lambda):
%      R^{\beta-1+|\alpha|}\theta^{\delta-1+|\alpha|}\partial^\alpha v\in L^2(\Lambda),
%      \alpha\in \mathbb N_0^3, |\alpha|\leqslant 1
%\right\}
And the symbol $V_{\beta,\delta}^{1,2}(\Lambda)$ will
denote the space
\begin{IEEEeqnarray}{rCl}\label{weighted_sobolev}
  V_{\beta,\delta}^{1,2}(\Lambda) & = &
  \left\{ v \in \mathcal D'(\Lambda): \|v\|^{1,2}_{\beta,\delta} < \infty\right\}.  
\end{IEEEeqnarray}
\end{defi}
\begin{remark}
If in the last definition we had $\beta = \delta$, which will hold true in the
case of just an edge singularity, then we would write 
$V_{\delta, \delta}^{1,2}(\Lambda)  = V_{\delta}^{1,2}(\Lambda)$
and, consistently, 
\begin{IEEEeqnarray*}{rCl}
\|v\|_{\delta}^{1,2} & := & \left\{\sum_{|\alpha|\leqslant 1}
\|r^{\delta-1+|\alpha|}\partial^\alpha v\|_{L^2(\Lambda)}^2\right\}^{1/2}.
\end{IEEEeqnarray*}
\end{remark}
%========
%First we introduce the space $V^{1,2}_{\beta,\delta}(\Lambda)$ for a macroelement $\Lambda$ as
%where $R(\bx)$ is the distance of $\bx$ to the vertices of $\Lambda$, $r(\bx)$ is the distance from $\bx$ to the edges of $\Lambda$ and finally $\theta(\bx)$ is the angular distance $\theta(\bx)=\frac{r(\bx)}{R(\bx)}$.
%========
Let $\Omega=\cup_{\ell=1}^N \Lambda_\ell$ be 
a decomposition of $\Omega$ in
{\bf prismatic} or {\bf tetrahedral} macro--elements having, each one of them,
at most a singular edge $\lambda_{\be}^{(\ell)}$ and a singular vertex
$\lambda_{\bv}^{(\ell)}$. This corresponds to the fact
that the sequence of meshes we will introduce 
at the end of this thesis has a first coarse term consisting only of prisms
and tetrahedra.
We have the following regularity result, for which we refer again
to~\cite{apelNicaise}.
\begin{theorem}\label{thm_regularity}
The solutions $\bu$ and $p$ of problem \eqref{weakMixedContinuous} satisfy
\[
  p\in H^1(\Omega)
\] 
and for each $\ell$
\[
  \bu=\bu_r + \bu_s
\]
with $\bu_r\in H^1(\Omega)$ and
\[
  \bu_s\cdot \xi_i\in V^{1,2}_{\beta_{\ell},\delta_{\ell}}(\Lambda_\ell), \quad i=1,2, \qquad
  \bu_s\cdot\xi_3\in V^{1,2}_{\beta_{\ell},0}(\Lambda_\ell)
\]
where $\xi_i$, $i=1,2,3$, are the directions of three edges of $\Lambda_\ell$ 
concurrent to $\bv$ with $\xi_3$ being the direction of the
singular edge, and $\beta_{\ell},\delta_{\ell}\geqslant 0$
satisfying $\beta_{\ell}>\frac12-\lambda_{\bv}^{(\ell)}$ and
$\delta_{\ell}>1-\lambda_{\be}^{(\ell)}$.
Furthermore, the following estimates hold:
%provided
%\begin{IEEEeqnarray*}{rCl}
%  {\color{violet} \delta_{\ell}} & > & 1 - \frac{\pi}{\omega_{\textbf{e}}}\text{,}\\
%  {\color{violet} \beta_{\ell}} & > & \frac{1}{2} - \lambda_{\textbf{v}}.
%\end{IEEEeqnarray*}
\begin{IEEEeqnarray}{rCl}
  \label{aux_label11}
  \| \bu_r \|_{H^1(\Omega)} & \leqslant & c\,\|f\|_{L^2(\Omega)}\\[5pt]
  \| \bu_s\cdot\xi_i \|_{V_{\beta_{\ell},\delta_{\ell}}^{1,2}(\Lambda_\ell)} & \leqslant & c\,\| f \|_{L^2(\Omega)}\\[5pt]
  \| \bu_s\cdot\xi_3 \|_{V_{\beta_{\ell},0}^{1,2}(\Lambda_\ell)}      & \leqslant & c\,\| f \|_{L^2(\Omega)}
\end{IEEEeqnarray}
\end{theorem}
\begin{remark}\label{sobreBetaYDelta}
Note that it is always possible to take $0<\beta_{\ell}=\delta_{\ell}<1$ in the previous Theorem
and note the dependence of $\beta$ and $\delta$ on the macro--element, which
we made explicit by putting the subindex $\ell$, meaning the result is ahout
local regularity.
\end{remark}
\section{Polynomial Approximation} % (fold)
\label{sec:polynomial_approximation}

\begin{defi}
  Given $f\in\pazocal{C}^{k}(\Omega)$ the $k$--th degree
  Taylor polynomial of $f$ centered at ${\by}\in\Omega$, $T_{\by}^kf$, is defined by
  \begin{IEEEeqnarray}{rCl}\label{taylor}
      (T_{\by}^kf)({\bx})&=&\sum_{|{\balpha}|\leqslant k} 
      \frac{f^{({\balpha})}({\by})}{{\balpha}!}(\bx-{\by})^{{\balpha}}.
  \end{IEEEeqnarray}
\end{defi}

\begin{defi} $\Omega$ is star--shaped with respect to $B$ if, for all
$x\in\Omega$, the closed convex hull of $\{x\}\cup B$ is 
a subset of $\Omega$.
\end{defi}

\begin{defi}
Assume that $\Omega$ is star--shaped with respect to a set $B\subseteq\Omega$
of positive measure. Given an integer $k\geqslant 0$ and 
$f\in W^{k+1,p}(\Omega)$ we introduce the averaged Taylor polynomial
of $f$, $\Qb_{k,B}f\in P_k$ defined by
\begin{IEEEeqnarray}{rCl}\label{averagedTaylor}
  (\Qb_{k,B}f)({\bx}) & = & \frac{1}{|B|} \int_B (T_{\by}^kf)({\bx})\,dy 
\end{IEEEeqnarray}
with $T_{\by}^k$ as in~(\ref{taylor}) but with distributional
derivatives.

Given a field $\bu = (u_1, u_2, u_3)' \in W^{m,p} (\Omega)^3$,
the vectorial averaged Taylor polynomial of $\bu$ is defined
componentwise as
\begin{IEEEeqnarray*}{rCl}
  \Qbb_{m,B}\,\bu  & = &  
  (\Qb_{m,B}\,u_1 , \Qb_{m,B}\,u_2 , \Qb_{m,B}\,u_3 ).
\end{IEEEeqnarray*}
\end{defi}
\begin{lemma}\label{avg_taylor_commutes}
Let ${\bbeta}$ be a multi--index such that  $|{\bbeta}| \leqslant m$,
then 
\begin{IEEEeqnarray}{rCl}
  \partial^{\bbeta} \Qb_{m,B} f & = & \Qb_{m-|{\bbeta}|,B} \partial^{\bbeta} f.
\end{IEEEeqnarray}
\end{lemma}
\begin{lemma} \label{aux_label40}
  Let $\Omega\subseteq\mathbb{R}^n$ be an open connected set
  with diameter $d$ which is star--shaped with respect to a 
  set $B\subseteq\Omega$ of positive measure. Given $p\geqslant 1$
  and an integer
  $k\geqslant 0$ and $f\in W^{k+1,p}(\Omega)$ there exists a 
  positive $C=C(k,n)$ such that, for $|{\bbeta}|\geqslant k+1$,
  \begin{IEEEeqnarray*}{rCl}
      \|\partial^{{\bbeta}}(f-\Qb_{k,B}f)\|_{L^p(\Omega)}
        &\leqslant&C\frac{|\Omega|^{\nicefrac1p}}{|B|^{\nicefrac1p}}
          d^{k-|{\bbeta}|+1}|f|_{k+1,p,\Omega}.
  \end{IEEEeqnarray*}
  In particular, if $\Omega$ is convex,
  \begin{IEEEeqnarray*}{rCl}
    \|\partial^{{\bbeta}}(f-\Qb_{k,\Omega}f)\|_{L^p(\Omega)}
        &\leqslant&Cd^{k-|{\bbeta}|+1}|f|_{k+1,p,\Omega}.
  \end{IEEEeqnarray*}
\end{lemma} 
\begin{proof} In view of Lemma~\ref{avg_taylor_commutes} we may only prove
the estimate for the case $|{\bbeta}| = 0$ and, for $|{\bbeta}|>0$, apply it to
$\partial^{{\bbeta}}f-\Qb_{k-|{\bbeta}|,B}\partial^{{\bbeta}}f$.

Consider $|{\bbeta}| = 0$ and take $q$ as the H\"older conjugate of $p\geqslant 1$.
By density we may assume $f\in \pazocal{C}^\infty(\Omega)$.
Firt use Taylor's Theorem
\begin{IEEEeqnarray*}{rCl}
  f(\bx)-(T_{\by}^kf)(\bx) & = & {(k+1)}
    \sum_{|{\balpha}|=k+1} \frac{(\bx-\by)^{\balpha}}{{\balpha}!}
    \int_0^1 \partial^{\balpha} f(t\by+(1-t)\bx)\,t^k\,dt
\end{IEEEeqnarray*}
which implies
\begin{IEEEeqnarray*}{rCl}
  f(\bx)-\Qb_{k,B}f(\bx) & = & \frac{k+1}{B}
    \sum_{|{\balpha}|=k+1} \int_B\int_0^1 \frac{(\bx-\by)^{\balpha}}{{\balpha}!}
      \partial^{\balpha} f(t\by+(1-t)\bx)\,t^k\,dt\,d\by.
\end{IEEEeqnarray*}
By H\"older's inequality twice (once in a finite dimensional version) we have
\begin{IEEEeqnarray*}{rCl}
  \IEEEeqnarraymulticol{3}{l}{\int_{\Omega}|f(\bx)-\Qb_{k,B}f(\bx)|^p\,d\bx\,\leqslant\,} \\
  \IEEEeqnarraymulticol{3}{r}{
\begin{IEEEeqnarraybox}{rCl}
  &\leqslant&
    C\dfrac{d^{\,p(k+1)}}{|B|^p}
      \sum_{|{\balpha}|=k+1}\int_{\Omega}
        \left(\int_B\int_0^1|\partial^{\balpha} f(t\by+(1-t)\bx)|^p\,dt\,d\by\right)
        \left(\frac{|B|}{qk+1}\right)^{\nicefrac{p}{q}}
        \\[5pt]
  &=&\frac{C}{(qk+1)^{\nicefrac{p}{q}}}\frac{d^{\,p(k+1)}}{|B|}
    \sum_{|{\balpha}|=k+1} \int_{\Omega}\int_B\int_0^1
      |\partial^{\balpha} f(t\by+(1-t)\bx)|^p\,dt\,d\by d\bx.
\end{IEEEeqnarraybox}}
\end{IEEEeqnarray*}
For any multi--index ${\balpha}$ we split
\begin{IEEEeqnarray}{rCl}
\IEEEeqnarraymulticol{3}{l}{\nonumber
\int_{\Omega}\int_B\int_0^1
  |\partial^{\balpha} f(t\by+(1-t)\bx)|^p\,dt\,d\by d\bx\, = \,}\\[5pt]
\IEEEeqnarraymulticol{3}{r}{
\begin{IEEEeqnarraybox}{rCl}
\hspace{2.7cm}& = & \int_{\Omega}\int_B\int_0^{\nicefrac12}
      |\partial^{\balpha} f(t\by+(1-t)\bx)|^p\,dt\,d\by d\bx\\[5pt]
\label{first_auxiliary}
   &  &\, +  
\int_{\Omega}\int_B\int_{\nicefrac12}^1
      |\partial^{\balpha} f(t\by+(1-t)\bx)|^p\,dt\,d\by d\bx.\hspace{.3cm}
\end{IEEEeqnarraybox}} 
\end{IEEEeqnarray}
Let $\phi_{\balpha}$ be the extension by zero of $\partial^{\balpha} f$ to $\mathbb{R}^n$.
By Fubini's Theorem and change of variables, the first term in~(\ref{first_auxiliary})
is less than or equal to
\begin{IEEEeqnarray*}{rCl}
  \int_{B}\int_0^{\nicefrac12}\int_{\mathbb{R}^n}
      |\phi_{\balpha}(\bz)|^p(1-t)^{-n}\,d\bz\,dt\, d\by
      &\leqslant& 2^{n-1} |B|\|\partial^{\balpha} f\|^p_{p,\Omega}.
\end{IEEEeqnarray*}
The second term in~(\ref{first_auxiliary}) is less than or equal to
\begin{IEEEeqnarray*}{rCl}
  \int_{\Omega}\int_{\nicefrac12}^1\int_{\mathbb{R}^n}
      |\phi_{\balpha}(t\by)|^p\,d\by\,dt\, d\bx
  &=& \int_{\Omega}\int_{\nicefrac12}^1\int_{\mathbb{R}^n}
      |\phi_{\balpha}(\bz)|^pt^{-n}\,d\bz\,dt\, d\bx\\
      &\leqslant& 2^{n-1} |\Omega|\|\partial^{\balpha} f\|^p_{p,\Omega}.
\end{IEEEeqnarray*}
Summing theese up for any ${\balpha}$ of order $k+1$ we obtain
\begin{IEEEeqnarray*}{rCl}
  \|f-\Qb_{k,B}f\|_p^p & \leqslant & 
  \frac{C}{(qk+1)^{\nicefrac{p}{q}}}d^{\,p(k+1)}\frac{|\Omega|}{|B|} |f|^p_{p,k+1,\Omega}.
\end{IEEEeqnarray*}
\end{proof}
For the following paragraphs we refer to the exposition in
Theorem 3.2 of~\cite{dupontScott}.
\begin{theorem}
  \label{aux_label21}
Let $m\geqslant 0$ and $p$, $\bar{p}\in [1,\infty]$. Suppose
\begin{IEEEeqnarray*}{rCl}
  \frac{1}{\bar{p}} - \frac{1}{p} + \frac{m+1}{3} & \geqslant & 0
\end{IEEEeqnarray*}
and that there exists $\sigma$ with 
\begin{IEEEeqnarray*}{rCcCl}
  0 & < & \sigma & \leqslant & 
  \max\left\{
    \left\lfloor \frac{m+1}{3} \right\rfloor,
    \frac{1}{\bar{p}} - \frac{1}{p} + \frac{m+1}{3},
    \min\left\{1-\frac{1}{\bar{p}},\frac{1}{\bar{p}}\right\}
  \right\}\mbox{,}
\end{IEEEeqnarray*}
then there is a positive $C$ depending only on $m$,$\sigma$ and $\Omega$ such
that, for all $g\in W^{m+1,p}(\Omega)$
\begin{IEEEeqnarray}{rCl} \label{aux_label19}
  \|\partial^{{\bbeta}}(g-\Qb_m g)\|_{L^{\bar{p}}(\Omega)} & \leqslant & C|g|_{W^{m+1,p}(\Omega)}
\end{IEEEeqnarray}
whenever $0 \geqslant |{\bbeta}| \geqslant m+1$.
\end{theorem}
% section polynomial_approximation (end)


%==========================================================================
%\begin{defi} We say that the unisolvent Finite Element $(E, P_E, \Sigma)$ is
%$H(\bcurl)$--conforming if every time we take two
%\emph{push-forward} elements $(E_1, P_{E_1}, \Sigma_1)$
%and $(E_2, P_{E_2}, \Sigma_2)$ according
%to~(\ref{sub:transformations}), and denote $\pi_1$, $\pi_2$
%the interpolation operators determined by the degrees
%of freedom $\Sigma_1$ and $\Sigma_2$, the field defined as
%\begin{IEEEeqnarray*}{rCl}
% \bw & = &
%   \begin{cases}
%     \pi_1(\bu|_{\Omega_1}), &\text{ on }\Omega_1\\
%     \pi_2(\bu|_{\Omega_2}), &\text{ on }\Omega_2      
%   \end{cases}
%\end{IEEEeqnarray*}
%results in $H(\bcurl,\Omega_1\cup\Omega_2)$.
%\end{defi}
% section preliminares (end)
%==========================================================================
\chapter{Vectorial Finite Elements}\label{aux_label43}
$H(\bcurl)$--conforming elements were
defined to determine a natural interpolation operator for fields with continuous tangential components.
Similarly, the other elements considered are $H(\Div)$--conforming, and we use them 
to interpolate fields with continuous normal components. These two are the cases of the
solutions of Stokes' equations, the solutions of time harmonic Maxwell's equations, and the
solution of the mixed formulation of the Poisson problem.

We will adopt the convention that objects like functions, variables, normals 
in the reference
prism $\hat{E}$ will have a hat as in $\hat{\bu}, \hat{\bx}, \hat{\bn},\ldots$
and then also the hat will denote a pullback transformation corresponding
to a mapping from $\hat{E}$ onto  a physical element $E$.

As we are going to deal with vectorial Finite Elements and
vectorial Virtual Elements, we are going to need to use the following notational convention.
\begin{notation} If\hspace{5pt}$V$ denotes a functional space of vectorial
fields in $\mathbb{R}^n$, whether finite dimensional
or not, $(V)_i$ denotes the space of scalar functions in the $i-$th
coordinate projection
of $V$, for $1\leqslant i\leqslant n$.
\end{notation}
For tensor product polynomial spaces involved in the construction
of $H(\bcurl)$ or $H(\Div)$ elements we will need the following notation.
\begin{notation}
  \begin{IEEEeqnarray*}{rCl}
    P_{k,l} 		&=& P_{k}(\hat x_1,\hat x_2) 	 \otimes P_{l}(\hat x_3)\mbox{,} \\
    Q_{k,l,m} 	&=& P_k(\hat{x}_1) \otimes P_l(\hat{x}_2)\otimes P_m(\hat{x}_3).
  \end{IEEEeqnarray*}
\end{notation}
%=============================================================
%\noindent---------------------------------------------------------
%% h div low order reference prism
\paragraph{$k=1$} (lowest order basis elements)\\[5pt] % (fold)
\label{par:_k_0_}
\begin{IEEEeqnarraybox*}{rCl}
	v_1&=&\left(
		\begin{array}{c}
			-1+x_1\\
			x_2\\[3pt]
			0\\
		\end{array}
	\right)
\end{IEEEeqnarraybox*}
\begin{IEEEeqnarraybox*}{rCl}
	v_2&=&\left(
		\begin{array}{c}
			x_1\\
			-1+x_2\\[3pt]
			0\\
		\end{array}
	\right)
\end{IEEEeqnarraybox*}
% paragraph _k_0_ (end)\\
%---------------------------------------------------------
%=============================================================
\section{Prismatic Finite Elements}
%>this intends to be an extension of the treatment
%in~\cite{giraultRaviart} to prismatic elements?
\subsection{$H(\bcurl)$--Conforming Element on Prisms} % (fold)
\label{sub:defEdgeElement}
First we introduce two more polynomial spaces on the reference prism of
Figure~\ref{reference_prism}.
$\hat{T}$ denotes the triangle in Definition~\ref{defi_of_ref_prism} and 
$\hat I = [0,1]$.
\begin{defi} For an integer $k\geqslant 1$, let $R_k(\hat{T})$ denote the space of polynomials, defined over the
triangle $\hat{T}$, given by
\begin{IEEEeqnarray}{rCl}
    \label{defRk}
    R_k(\hat{T}) & := & P_{k-1}(\hat{T})^2 \oplus S_k(\hat{T})
\end{IEEEeqnarray}
where
\begin{IEEEeqnarray}{rClCrCl}
    \label{defSk}
    S_k(\hat{T}) & := & \{ \bp\in \widetilde{P}_k^2 \,:\;\bp\cdot\hat\bx =
    0\}\mbox{,}\quad\hat\bx & = & (\hat x_1, \hat x_2)'.
\end{IEEEeqnarray}
\end{defi}
\facesOfPrism
\edgesOfPrism
\begin{figure}[!h]
  \centering
  \subfloat
  {
    \label{unitTanPrism}
    \unitTangentsPrism
  }
  \caption{Directions of positive unit tangents (cfr. Table~\ref{prismNotationTableEdges}).}
\end{figure}
\begin{defi}\label{edgeelement} Given a natural $k$, the $H(\bcurl)$--Conforming 
Finite Element of degree $k$ is defined by the following triple.
\begin{enumerate}
  \item $\hat{E}$ is the reference prism in Definition~\ref{defi_of_ref_prism}.
  \item The polynomial space $P_{\hat{E}}$ is
        \begin{IEEEeqnarray}{rCl} \label{spaceFEprismHcurl}
            P_{\hat{E}} & = & R_k(\hat{T}) \otimes P_k(\hat{I}) \times 
            P_k(\hat{T}) \otimes P_{k-1}(\hat{I}).
         \end{IEEEeqnarray} 
  \item The degrees of freedom are (cfr. Tables~\ref{prismNotationTableFaces} 
and~\ref{prismNotationTableEdges}):
  {\color{Orange} ver si en~\ref{momentos2hcurl} y los otros de caras
    hay que cambiar orden de componentes o signos por lo que acomod'e para definir el elemento
    af'in; ver~(\ref{momentos2hcurlPhys})}.
\begin{IEEEeqnarray}{ll}
    \label{momentos1hcurl}  
    \hat\varphi_{\hat{\be},\hat{q}}\,(\hat\bu) = 
    \int_{\hat\be} \hat q\,\hat\bu\cdot d\hat\balpha\mbox{,}  
      & \hat q\in P_{k-1}(\hat\be)\mbox{, for each edge $\hat\be$;}\\[8pt]%with unit tangent } \boldsymbol{\tau} \mbox{;}}
    \nonumber\hat\varphi_{\hat f,\hat\bq}\,(\hat\bu) =  
    \int_{\hat f} \hat\bu \times \hat\bn \cdot \hat\bq\,
    d\hat S\mbox{, }\quad&\hat\bq = (\hat q_1,\hat q_2,0) \in P_{k-2}^2 \times \{ 0 \},\\[4pt] 
    \label{momentos2hcurl} 
      &\mbox{ for each face $f=\hat f_3$ or$\hat f_4$;}\\[8pt]
    \nonumber\hat\varphi_{\hat f_1,\hat\bq}\,(\hat\bu) =  
    \int_{\hat f_1} \hat\bu \times \hat\bn_1 \cdot \hat\bq\,
    d\hat S\mbox{, }\quad&
      \hat\bq = (0,\hat q_3,\hat q_2)\mbox{, }\hat q_3\in Q_{k-2,k-1}\mbox{,}\\[4pt]
    \label{momentos3hcurl}
      &\hat q_2 \in Q_{k-1,k-2}\mbox{;}\\[8pt]   %\mbox{, for the face } \hat f_1
    \nonumber\hat\varphi_{\hat f_2,\hat\bq}\,(\hat\bu) =  
    \int_{\hat f_2} \hat\bu \times \hat\bn_2 \cdot \hat\bq\,
    d\hat S\mbox{, }\quad& 
      \hat\bq = (\hat q_3,0,\hat q_1)\mbox{, }\hat q_3 \in Q_{k-2,k-1}\mbox{,}\\[4pt]
    \label{momentos4hcurl}
      &\hat q_1 \in Q_{k-1,k-2}\mbox{;}\\[8pt]   %\mbox{, for the face } \hat f_2
    \nonumber\hat\varphi_{\hat f_5,\hat\bq}\,(\hat\bu) =  
    \int_{\hat f_5} \hat\bu \times \hat\bn_5 \cdot \hat\bq\,
    d\hat S\mbox{, }\quad&
      \hat\bq = (0,\hat q_3,\hat q_1)\mbox{, }\hat q_3 \in Q_{k-2,k-1}\mbox{,}\\[4pt]
    \label{momentos5hcurl}
      &\hat q_1 \in Q_{k-1,k-2}\mbox{;}\\[8pt]   %\mbox{, for the face } \hat f_5
    \nonumber\hat\varphi_{\hat\br}\,(\hat\bu) = 
    \int_{\hat{E}} \hat\bu \cdot \hat\br \, d\hat\bx\mbox{, }&\hat r_1\mbox{, } 
    \hat r_2 \in P_{k-2,k-2}\mbox{, }\\[4pt]
    \label{momentos6hcurl}
      &\hat r_3 \in P_{k-3,k-1}.
\end{IEEEeqnarray}
\end{enumerate}
\end{defi}

%%==============================================================================
%% \noindent{\color{blue}\#\#\#\#\#\#\# poner mas sinteticos los dofs de superficie
%% aca arriba y aca abajo aclararlos para mostrar como se computan}
%% In order to clarify how to compute the degrees of
%% freedom~(\ref{momentos2hcurl})--(\ref{momentos5hcurl}) for an implementation
%% we write their test spaces more explicitly here.
%% \begin{IEEEeqnarray}{ll}
%%     (\ref{momentos2hcurl}) \int\limits_{f} \textbf{u} \times \boldsymbol{\nu} \cdot \hat\bq\,
%%     d\gamma\mbox{, } &\bq = (q_1,q_2,0) \in P_{k-2}^2 \times \{ 0 \},\\ 
%%     \IEEEeqnarraymulticol{2}{l}{\nonumber\mbox{ for each horizontal face $f$ with normal } \boldsymbol{\nu} = (0,0,\pm1) \mbox{;}}\\[8pt]
%%     (\ref{momentos3hcurl}) \int\limits_{f} \textbf{u} \times \boldsymbol{\nu} \cdot \bq\,
%%     d\gamma\mbox{, } &\bq = (0,q_3,q_2) \in \{ 0 \} \times Q_{k-2,k-1} \times 
%%     Q_{k-1,k-2}\mbox{, } \\
%%     \IEEEeqnarraymulticol{2}{l}{\nonumber\mbox{ for the face } f \subseteq \{ x=0 \} \mbox{ with normal }\boldsymbol{\nu} = (-1,0,0) \mbox{;}}\\[8pt]
%%     (\ref{momentos4hcurl}) \int\limits_{f} \textbf{u} \times \hat\bn \cdot \bq\,
%%     d\gamma\mbox{, } & \bq = (q_3,0,q_1) \in Q_{k-2,k-1} \times \{ 0 \} \times
%%     Q_{k-1,k-2},\\
%%     \IEEEeqnarraymulticol{2}{l}{\nonumber\mbox{ for the face } f \subseteq \{ y=0 \} \mbox{ with normal }\boldsymbol{n} = (0,-1,0) \mbox{;}}\\[8pt]
%%     (\ref{momentos5hcurl}) \int\limits_{f} \textbf{u} \times \boldsymbol{n} \cdot \bq\,
%%     d\gamma\mbox{, } & \bq = (0,q_3,q_1) \in \{ 0 \} \times Q_{k-2,k-1} \times
%%     Q_{k-1,k-2}\mbox{, }\\
%%     \IEEEeqnarraymulticol{2}{l}{\nonumber\mbox{ for the face }f \subseteq \{x+y=1\} \mbox{ with normal }\boldsymbol{n} = (1,1,0) \mbox{;}}
%% \end{IEEEeqnarray}
%% \noindent{\color{blue}\#\#\#\#\#\#\# }
%%==============================================================================
In the following Remark we make an explicitation of the elements
of $P_{\hat E}$.
\begin{remark} \label{aux_label6}
Take $\hat{\textbf{s}}=(\hat s_1,\hat s_2)\in {S}_k$ as defined in~(\ref{defSk}).
Set $\hat s_1 = \sum_{i+j=k} a_{i, j}\,\hat x_1^i\hat x_2^j$, 
$\hat s_2 = \sum_{i+j=k} b_{i, j}\,\hat x_1^i\hat x_2^j$.
By definition is 
\begin{IEEEeqnarray*}{rCl}
    0&=&\hat x_1\hat s_1 + \hat x_2\hat s_2\\[4pt]
     &=&a_{k,0}\,\hat x_1^{k+1} + b_{0,k}\,\hat x_2^{k+1}+\sum_{i+j=k}(a_{i-1,j+1} + b_{i, j})\,
     \hat x_1^i\hat x_2^{j+1}
\end{IEEEeqnarray*}
so $a_{k,0} = b_{0,k} = 0$ and for all pair $(i,j)$ with $i+j=k$ and
$i\geqslant 1$ it holds the relation $a_{i-1, j+1} = -b_{i, j}$.
Then
\begin{IEEEeqnarray*}{rCcCl}
    \hat s_1 & = & \sum_{i+j = k,\,j\geqslant 1} a_{i, j}\,\hat x_1^i\hat x_2^j
        & = & \hat x_2\sum_{i+j = k,\,j\geqslant 1} a_{i, j}\,\hat x_1^i\hat x_2^{j-1} \\[5pt]
    \hat s_2 & = & -\sum_{i+j = k,\,j \geqslant 1} a_{i, j}\,\hat x_1^{i+1}\hat x_2^{j-1}
        & = & -\hat x_1\sum_{i+j = k,\,j \geqslant 1} a_{i, j}\,\hat x_1^{i}\hat x_2^{j-1}.
\end{IEEEeqnarray*}
So any $\hat{\bp} \in P_{\hat E}$ may be written as
\begin{IEEEeqnarray*}{rCl}
  \hat{\bp} & =  & (\hat p_1, \hat p_2, \hat p_3)'\\
  \yesnumber\label{elemento_P_k} 
            & =  & (\hat \xi_1 + \hat x_2\,\hat h, \hat\xi_2 - \hat x_1\,\hat h, \hat \xi_3), \\[6pt]
  \hat\xi_1, \hat\xi_2   & \in & P_{k-1,k}\mbox{,}\\
             \hat\xi_3   & \in & P_{k,k-1}\mbox{,}\\
                \hat h   & \in & \widetilde{P}_{k-1}(\hat x_1,\hat x_2) \otimes P_k(\hat x_3).\\
\end{IEEEeqnarray*}
\end{remark}
An illustrative example.
\begin{example}[edge elements of degree 1]
\begin{IEEEeqnarray*}{rCl}
\hat{\bu}\,\xyz &=& 
\left(
    \begin{array}{c}
        a_1 + a_3\hat{x}_2 + a_4\hat{x}_3 + a_6\hat{x}_2\hat{x}_3 \\[8pt]
        a_2 - a_3\hat{x}_1 + a_5\hat{x}_3 - a_6\hat{x}_1\hat{x}_3 \\[8pt]
        a_7 + a_8\hat{x}_1 + a_9\hat{x}_2
    \end{array}
\right)\mbox{,}\\[10pt]
(\hat{\bu}\cdot\hat{\btau})|_{\hat{\be}}
    &\in& {P}_0(\hat{\be}).
\end{IEEEeqnarray*}
\end{example}
The following can be proved exactly as in~\cite{monk}, Lemma 5.38.
\begin{remark}
Given $p>2$, the degrees of freedom~(\ref{momentos1hcurl}),~(\ref{momentos2hcurl}),~(\ref{momentos3hcurl}),~(\ref{momentos4hcurl}),~(\ref{momentos5hcurl}) and~(\ref{momentos6hcurl}) 
are well defined and bounded as linear functionals from
$W^{1,p}(\hat{E})$ to $\mathbb{R}$.
\end{remark}
\begin{remark}[dimension of the space~(\ref{spaceFEprismHcurl})]
%%=============
%% $\dim R_k(\hat{T}) \otimes P_k(\hat{I})$.
%% \begin{IEEEeqnarray*}{rCl}
%%     \dim\left(R_k(\hat{T}) \otimes P_k(\hat{I})\right) 
%%     & = & \dim\left(R_k(\hat{T})\right) \dim\left(P_k(\hat{I})\right) \\
%%     & = & \dim\left(R_k(\hat{T})\right) (k+1).
%% \end{IEEEeqnarray*}
%% \begin{IEEEeqnarray*}{rCl}
%%     \dim\left(R_k(\hat{T})\right) 
%%     & = & \dim\left(P_{k-1}(\hat{T})^2 \oplus S_k(\hat{T}) \right)\\
%%     & = & 2\dim\left(P_{k-1}(\hat{T})\right) + \dim\left(S_k(\hat{T}) \right)\\
%%     & = & k(k+1) + \dim\left(S_k(\hat{T}) \right).
%% \end{IEEEeqnarray*}
%%=============
Recall the space $S_k$ as 
\begin{IEEEeqnarray*}{rCl}
S_k(\hat{T}) = \{ \bp \in \widetilde{P}_k^2 \, : \, \bp\cdot\bx = 0 \}.
\end{IEEEeqnarray*}
To know its dimension consider
\begin{IEEEeqnarray*}{lll}
    \phi\,:\,\widetilde{P}_k^2 & \longrightarrow & \widetilde{P}_{k+1}\\
    \phi(\bp)    & := & \bp\cdot\bx\\
                        & := & \hat x_1\hat p_1 + \hat x_2\hat p_2.
\end{IEEEeqnarray*}
It results
\begin{IEEEeqnarray*}{rCl}
    S_k(\hat{T})        & = & \ker(\phi)\\
    \dim(S_k(\hat{T}))  & = & \dim(\widetilde{P}_k^2) - \dim(\img(\phi)).
\end{IEEEeqnarray*}
Now, any $\hat p \in \widetilde{P}_{k+1}$ is
\begin{IEEEeqnarray*}{rCl}
    \hat x_1(a_{k+1,0} \hat x_1^k + a_{k,1} \hat x_1^{k-1}\hat x_2 + \ldots + a_{1,k} \hat x_2^k) + \hat x_2(a_{0,k+1} \hat x_2^k)
        & = & \hat x_1\hat p_1 + \hat x_2\hat p_2
\end{IEEEeqnarray*}
where precisely $\hat p_1$ y $\hat p_2$ belong to $\widetilde{P}_k$, so $\phi$ 
is onto. So it holds
\begin{IEEEeqnarray*}{rCl}
    \dim S_k(\hat{T})  & = & \dim(\widetilde{P}_k^2) - \dim(\widetilde{P}_{k+1})\\
                        & = & 2 \dim(\widetilde{P}_k) - \dim(\widetilde{P}_{k+1})\\
                        & = & 2 (k+1) - (k+2)\\
                        & = & k,
\end{IEEEeqnarray*}
so
\begin{IEEEeqnarray*}{rCl}
    \dim R_k(\hat{T})   & = & k(k+1) + k\\
                                        & = & k(k+2)
\end{IEEEeqnarray*}
and finally
\begin{IEEEeqnarray*}{rCl}
    \dim R_k(\hat{T}) \otimes P_k(\hat{I}) 
        & = & k(k+1)(k+2).
\end{IEEEeqnarray*}
Immediately we arrive at $\dim P_{\hat E} = 3\frac{k(k+1)(k+2)}{2}$.
\end{remark}
The following result can be found in~\cite[page 75]{nedelec2} only for the reference
element.
\begin{lemma}
  The Finite Element in Definition~\ref{edgeelement} is unisolvent in $\hat E$.
\end{lemma}
Later we will establish the unisolvence of a general finite element of
this family on an arbitrary physical prism. The most important consequence of
unisolvence is the existence of an interpolation operator defined in terms of
the set of degrees of freedom.
\begin{defi}\label{aux_label90}
Given $p>2$, we define the $H(\bcurl)$--conforming interpolation operator on 
the reference
prism $\bw_{\hat{E}}\,:\,W^{1,p}(\hat{E})\to P_{\hat E}$ on each $\hat\bu$
as the unique element $\wku$ such that 
  \begin{IEEEeqnarray}{lCll}
    \label{aux_label45}
    \hat\varphi_{\hat\be,\hat p}\,(\hat{\bu} - \wku) & = & 0 
      &\quad\quad\mbox{for all $\hat\varphi_{\hat\be,\hat p}$ as in ~(\ref{momentos1hcurl}).}  \\
    \hat\varphi_{\hat f,\hat \bq}\,(\hat{\bu} - \wku) & = & 0 
      &\quad\quad\mbox{for all $\hat\varphi_{\hat f,\hat \bq}$ as in~(\ref{momentos2hcurl})--(\ref{momentos5hcurl})}  \\
    \hat\varphi_{\br}\,(\hat{\bu} - \wku) & = & 0 
      &\quad\quad\mbox{for all $\hat\varphi_{\br}$ 
    as in~(\ref{momentos6hcurl})}.
  \end{IEEEeqnarray}
\end{defi}
\begin{remark}
The interpolator in Definition~\ref{aux_label90} is well defined and bounded. 
Because of the presence of degrees of freedom~(\ref{momentos1hcurl})
we can't define the operator for an arbitrary field in $H(\bcurl, \hat E)$, 
which is why we put the hypothesis $p>2$ (cfr.~\cite{monk}, page 134, 
and~\cite{adams}, Theorem 5.4).
\end{remark}
\begin{remark} The interpolation operator
can be written 
\begin{IEEEeqnarray}{rCl}\label{edge_interp_explicit}  
  \wku & = & 
  \sum_{\hat\be,\hat p} \hat\varphi_{\hat\be,\hat p}(\hat{\bu})\,\hat{\bv}_{\hat\be,\hat p} +
  \sum_{\hat f,\hat\bq} \hat\varphi_{\hat f,\hat\bq}(\hat{\bu})\,\hat{\bv}_{\hat f,\hat\bq} +
  \sum_{\hat\br}        \hat\varphi_{\hat\br}       (\hat{\bu})\,\hat{\bv}_{\hat\br}\mbox{,}
\end{IEEEeqnarray}
where we chose suitable dual bases to the bases of degrees of freedom.
\end{remark}
\subsection{$H(\Div)$--Conforming Element on Prisms} % (fold)
\label{sub:definition_of_the_h_div_element_on_prisms}
With the notation introduced in Subsection~\ref{sub:polynomials} we consider
the polynomial space
\begin{IEEEeqnarray*}{rCl}
    \yesnumber\label{dk}
    D_k & = & P_{k-1}^2(\hat x_1,\hat x_2) \oplus \widetilde{P}_{k-1}(\hat x_1,\hat x_2) \hat\bx,\\
    \hat\bx & = & (\hat x_1,\hat x_2).
\end{IEEEeqnarray*}
\begin{defi}\label{defi_h_div_conforme} Given a non--negative integer $k$, the 
$H(\Div)$--Conforming 
Finite Element of degree $k$  on a prism is defined by the following triple.
\begin{enumerate}
  \item $\hat{E}$ is the reference prism in Figure~\ref{reference_prism}.
  \item The polynomial space $P_{\hat{E}}$ is
    \begin{IEEEeqnarray*}{rCl}
      P_{\hat{E}} & = & \{ \hat\bv = (\hat v_1,\hat v_2,\hat v_3)':\,
      (\hat v_1,\hat v_2)'\in D_k\otimes P_{k-1}(\hat x_3),\\ 
      \yesnumber\label{prismaticSpace}&   &\,\hat v_3\in P_{k-1, k} \}.
    \end{IEEEeqnarray*} 
  \item The degrees of freedom are of two types, surface and volumen integrals.
\begin{IEEEeqnarray}{cCccl}
    \label{momentos1hdiv} 
    \hat\rho_{\hat f,\hat q}(\hat\bv) & = & \int_{\hat f} (\hat\bv\cdot\hat{\bn})\hat{q}\,d\hat{S} 
        &\quad & \mbox{for } \hat{q} \in P_{k-1}(\hat{f})\mbox{,}\\
    \nonumber&&&\quad&\mbox{if $\hat f = \hat f_3$ or $\hat f_4$;}\\[5pt]
    \label{momentos2hdiv}
    \hat\rho_{\hat f,\hat q}(\hat\bv) & = & \int_{\hat f} (\hat\bv\cdot\hat{\bn})\hat{q}\,d\hat{S} 
        &\quad & \mbox{for } \hat{q} \in Q_{k-1, k-1}(\hat f)\mbox{,}\\
    \nonumber&&&\quad&\mbox{ if $\hat f = \hat f_1$, $\hat f_2$ or $\hat f_5$;}\\[5pt]
    \hat\rho_{\hat \br}(\hat\bv) & = & \int_{\hat{E}} (\hat v_1\,\hat r_1 + \hat v_2\,\hat r_2)\,d\hat\bx 
        &\quad& \mbox{for }\hat r_1\mbox{, }\hat r_2\in P_{k-2,k-1};
    \label{momentos3hdiv}\\
    \label{momentos4hdiv}
    \hat\rho_{\hat \br}(\hat\bv) & = & \int_{\hat{E}} \hat v_3\,\hat r_3\,d\hat\bx 
        &\quad& \mbox{for }\hat r_3\in P_{k-1,k-2}. 
\end{IEEEeqnarray}
\end{enumerate}
%%========
% {\color{red}TODO: tal vez en los dofs (\ref{momentos2hdiv}) haya que separar los espacios test como hice
% en pyramids}
%%========
%%=======================================================
%% original version of dofs
%\begin{IEEEeqnarray}{lcl}
%   \label{momentos1hdiv} \int\limits_{f} (\bv\cdot\boldsymbol{\nu})q\,d\gamma 
%       && \mbox{for } q \in P_{k-1}(x,y)\mbox{,}\\
%   \nonumber&& \mbox{ if $f\subseteq\{\hat{z}=0\}$ or $f\subseteq\{\hat{z}=1\}$; }\\
%   \label{momentos2hdiv} \int\limits_{f} (\bv\cdot\boldsymbol{\nu})q\,d\gamma 
%       && \mbox{for } q \in Q_{k-1, k-1, k-1}\mbox{,}\\
%   \nonumber&& \mbox{ if $f\subseteq\{\hat{x}=0\}$ or $f\subseteq\{\hat{x}_2=0\}$
%    or $f\subseteq\{\hat{x} + \hat{y} = 1\}$; } \\
%   \label{momentos3hdiv} \int\limits_{\hat{E}} (v_1q_1 + v_2q_2)\,d\textbf{x} 
%       &\quad& {q_1\mbox{, } q_2 \in P_{k-2}(x,y) \otimes P_{k-1}(z);}\\
%   \label{momentos4hdiv} \int\limits_{\hat{E}} v_3q_3\,d\textbf{x} 
%       &\quad& { q_3\in P_{k-1}(x,y) \otimes P_{k-2}(z).} 
%\end{IEEEeqnarray}
%%=======================================================
\end{defi}
The following result can be found in page 66 of~\cite{nedelec2}.
\begin{lemma} The Finite Element in Definition~\ref{defi_h_div_conforme} is
  unisolvent in $\hat E$.
\end{lemma}
\begin{defi}\label{defi_face_element} Let $P_{\hat E}$ be as in~(\ref{prismaticSpace}).
The interpolator $\boldsymbol{r}_{\hat{E}}\,:\,W^{1,1}(\hat{E})\to P_{\hat E}$
is defined as the operator such that, 
for each $\hat\bu\in W^{1,1}(\hat E)$, $\rku$ is
defined as the unique element in $P_{\hat E}$ satisfying
  \begin{IEEEeqnarray}{lClc}
    \hat\rho_{\hat f,\hat\bq}\,(\hat{\bu} - \rku) & = & 0 &
    \quad\mbox{for $\hat\rho_{\hat f, \hat q}$ as in~(\ref{momentos1hdiv})
      and~(\ref{momentos2hdiv})}\\
    \hat\rho_{\hat\br}\,(\hat\bu - \rku) & = & 0 &
    \quad\mbox{for $\hat\rho_{\hat\br}$ as in~(\ref{momentos3hdiv})
      and~(\ref{momentos4hdiv})}.
  \end{IEEEeqnarray}
\end{defi}
Because of the degrees of freedom~(\ref{momentos1hdiv}) and~(\ref{momentos2hdiv})
we had to restrict the condition $\hat\bu\in H(\Div, \hat E)$
to the one of $\hat\bu\in W^{1,1}(\hat E)$ in order to define the 
interpolation operator. The proof of the following Lemma follows
from Trace Theorems in Sobolev spaces.
\begin{lemma}
  The operator of Definition~\ref{defi_face_element} is well defined and
  bounded.
\end{lemma}
\begin{proposition} On the Finite Element~(\ref{defi_h_div_conforme}). 
$\dim(P_{\hat{E}}) = k^2\,(k+2) + k\,(k+1)^2/2$
which is, at the same time, equal to the number of its independent degrees of freedom.
\end{proposition}
\begin{proof}
  From~(\ref{tensor_prod_dim}) and~(\ref{dk}) it is really straightforward to do
  \begin{IEEEeqnarray*}{rCl}
    \dim (P_{k-1}^2(\hat x_1,\hat x_2) \oplus \widetilde{P}_{k-1}(\hat x_1,\hat x_2))
    \otimes P_{k-1}(\hat x_3) + \dim P_{k-1,k} & = &\\[5pt]
    \IEEEeqnarraymulticol{3}{r}{=\,(k(k+1)+k)k+\dfrac{k(k+1)}{2}(k+1)} 
  \end{IEEEeqnarray*}
  and the same quantity is obtained by summing up the dimensions of all the
  polynomial spaces on the right--hand sides of~(\ref{momentos1hdiv})--(\ref{momentos4hdiv}).
\end{proof}

\begin{remark} The interpolation operator
can be written
%% \begin{IEEEeqnarray*}{rCl}
%%   \int\limits_{\hat f}(\bv_{f,\hat{p}}\cdot\boldsymbol{\nu})\hat{q}\,d\gamma  & = & \delta_{\hat{p},\hat{q}}
%%   \,\,etc\,\,etc
%% \end{IEEEeqnarray*}
\begin{IEEEeqnarray}{rCl}\label{face_interp_explicit}  
  \hat{\br}_k\hat{\bu} & = & \sum_{\hat f,\hat\bq} \hat\rho_{\hat f,\hat\bq}(\hat{\bu})\,\hat{\bv}_{\hat f,\hat\bq} +
    \sum_{\hat r} \hat\rho_{\hat r} (\hat{\bu})\,\hat{\bv}_{\hat r}
\end{IEEEeqnarray}
where we chose suitable dual bases to the bases of degrees of freedom.
\end{remark}

\section{Pyramidal Finite Elements}
The Finite Elements defined here are the ones found in~\cite{gh99}. There the authors
perform a
construction of discrete differential
forms by solving a local interpolation problem on the reference pyramid. The
finite element in an arbitrary mesh pyramid is obtained by pushing forward
the vector proxies of those discrete forms.
\begin{table}[!h]
    \centering  
    \caption{Notation for the faces and positive normals of the
    reference pyramid.}
    \label{pyramidNotationTableFaces}
    \begin{IEEEeqnarraybox*}
      [\IEEEeqnarraystrutmode
      \IEEEeqnarraystrutsizeadd{2pt}{6pt}]{v/c/x/c/x/c/x/v/x/c/x/c/x/c/v/}
        \IEEEeqnarrayrulerow\\
        \IEEEeqnarrayseprow[5pt]\\
          & \hat f_1 && \subseteq &&  \{\hat x_2 = 0 \}            && && \hat{\bn}_1 && = && (0,-1,0)' & \\
        \IEEEeqnarrayrulerow\\
        \IEEEeqnarrayseprow[5pt]\\
          & \hat f_2 && \subseteq &&  \{\hat x_1 = 0 \}            && && \hat{\bn}_2 && = && (-1,0,0)' &\\
        \IEEEeqnarrayrulerow\\
        \IEEEeqnarrayseprow[5pt]\\
          & \hat f_3 && \subseteq &&  \{\hat x_1 + \hat x_3 = 1 \} && && \hat{\bn}_3 && = && 2^{-\nicefrac{1}{2}}(1,0,1)' &\\
        \IEEEeqnarrayrulerow\\
        \IEEEeqnarrayseprow[5pt]\\
          & \hat f_4 && \subseteq &&  \{\hat x_2 + \hat x_3 = 1 \} && && \hat{\bn}_4 && = && 2^{-\nicefrac{1}{2}}(0,1,1)' &\\
        \IEEEeqnarrayrulerow\\
        \IEEEeqnarrayseprow[5pt]\\
          & \hat f_5 && \subseteq &&  \{\hat x_3 = 0\}             && && \hat{\bn}_5 && = && (0,0,-1)' &\\
        \IEEEeqnarrayrulerow
    \end{IEEEeqnarraybox*}
\end{table}
\begin{table}[!h]
    \centering  
    \caption{Notation for the edges and positive tangents of the
    reference pyramid.}
    \label{pyramidNotationTableEdges}
    \begin{IEEEeqnarraybox*}
      [\IEEEeqnarraystrutmode
      \IEEEeqnarraystrutsizeadd{2pt}{6pt}]{v/c/x/c/x/c/x/v/x/c/x/c/x/c/v/}
        \IEEEeqnarrayrulerow\\
        \IEEEeqnarrayseprow[5pt]\\
   & \hat \be_1 && = && \{(\hat x_1,0,0)^t\,:\,0\leqslant\hat x_1\leqslant 1\} && && \hat \btau_1 && = && (1,0,0)' & \\
        \IEEEeqnarrayrulerow\\
        \IEEEeqnarrayseprow[5pt]\\
   & \hat \be_2 && = && \{(1,\hat x_2,0)^t\,:\,0\leqslant\hat x_2\leqslant 1\} && && \hat \btau_2 && = && (0,1,0)' & \\
        \IEEEeqnarrayrulerow\\
        \IEEEeqnarrayseprow[5pt]\\
   & \hat \be_3 && = && \{(\hat x_1,1,0)^t\,:\,0\leqslant\hat x_1\leqslant 1\} && && \hat \btau_3 && = && (-1,0,0)' & \\
        \IEEEeqnarrayrulerow\\
        \IEEEeqnarrayseprow[5pt]\\
   & \hat \be_4 && = && \{(0,\hat x_2,0)^t\,:\,0\leqslant\hat x_2\leqslant 1\} && && \hat \btau_4 && = && (0,1,0)' & \\
        \IEEEeqnarrayrulerow\\
        \IEEEeqnarrayseprow[5pt]\\
   & \hat \be_5 && = && \{(0,0,\hat x_3)^t\,:\,0\leqslant\hat x_3\leqslant 1\} && && \hat \btau_5 && = && (0,0,1)' & \\
        \IEEEeqnarrayrulerow\\
        \IEEEeqnarrayseprow[5pt]\\
   & \hat \be_6 && = && \{(1-\hat x_3,0,\hat x_3)^t\,:\,0\leqslant\hat x_3\leqslant 1\} && && \hat \btau_6 && = && 2^{-\nicefrac{1}{2}}(-1,0,1)' & \\
        \IEEEeqnarrayrulerow\\
        \IEEEeqnarrayseprow[5pt]\\
   & \hat \be_7 && = && \{(0,1-\hat x_3,\hat x_3)^t\,:\,0\leqslant\hat x_3\leqslant 1\} && && \hat \btau_7 && = && 2^{-\nicefrac{1}{2}}(0,-1,1)' & \\
        \IEEEeqnarrayrulerow\\
        \IEEEeqnarrayseprow[5pt]\\
   & \hat \be_8 && = && \{(1-\hat x_3,1-\hat x_3,\hat x_3)^t\,:\,0\leqslant\hat x_3\leqslant 1\} && && \hat \btau_8 && = && 3^{-\nicefrac{1}{2}}(-1,-1,1) & \\
        \IEEEeqnarrayrulerow
    \end{IEEEeqnarraybox*}
\end{table}
\begin{figure}[!h]
\centering
  \unitTangentsPyramid
  \caption{Directions of the positive unit tangents (cfr. Table~\ref{pyramidNotationTableEdges}).}
  \label{reference_pyramid}
\end{figure}

\subsection{$H(\bcurl)$--Conforming Element on Pyramids} % (fold)
\label{sub:edge}
\begin{defi}\label{aux_label50}
  The following items define a least order $\bcurl$--conforming finite element
  on the reference Pyramid.
  \begin{enumerate}
    \item $\hat E$ is the reference Pyramid in Definition~\ref{defi_of_ref_pyr}. 
    \item The rational space $P_{\hat E}$ is the span of
    $\{\hat{\bgamma}_1,\,\ldots,\,\hat{\bgamma}_8\}$ with $\hat{\bgamma}_i$
    as in Table~\ref{shape_edge_table}.
    \item The degrees of freedom are the line integrals
      \begin{IEEEeqnarray*}{c}
        \int_{\hat\be_j}\hat\bu\cdot\,d\hat\balpha
      \end{IEEEeqnarray*}
      for every edge $\hat\be_j$ of $\hat E$, $1\leqslant j\leqslant 8$.
  \end{enumerate}
\end{defi}
\edgeShapeTable
A direct computation yields the following Lemma.
\begin{lemma}
  For $1\leqslant i,j\leqslant 8$,
  $\int_{\hat\be_j}\hat\bgamma_i\cdot d\hat\balpha = \delta_{ij}$ which
  implies immediately that the finite element in Definition~\ref{aux_label50}
  is unisolvent in $\hat E$. %$H(\bcurl)$--conforming and 
\end{lemma}
% subsection edge (end)
\subsection{$H(\Div)$--Conforming Element on Pyramids} % (fold)
\label{sub:face}
\begin{defi}\label{aux_label71}
The following items define a least order $\Div$--conforming finite element
  on the reference Pyramid.
\begin{enumerate}
  \item $\hat{E}$ is the reference pyramid of Figure~\ref{reference_pyramid}.
  \item The space $P_{\hat{E}}$ is the span of 
  $\{\hat{\bz}_1,\,\ldots,\,\hat{\bz}_5\}$ with $\hat{\bz}_i$
    as in Table~\ref{shape_face_table}.
  \item The degrees of freedom are the surface integrals
  \begin{IEEEeqnarray*}{c}
    \label{dofsdivpyramid} \iint_{\hat{f}_j} \hat\bv\cdot\hat\bn\,d\hat S
  \end{IEEEeqnarray*}
  for every face $\hat{f}_j$ of $\hat E$, $1\leqslant j\leqslant 5$.
\end{enumerate}
\end{defi}
\faceShapeTable
A direct computation yields the following Lemma.
\begin{lemma}
  For $1\leqslant i,j\leqslant 5$,
  $\iint_{f_j}\hat\bz_i\cdot\hat\bn\,d\hat S = \delta_{ij}$ which
  implies immediately that the finite element in Definition~\ref{aux_label71}
  is unisolvent in $\hat E$. %$H(\bcurl)$--conforming and 
\end{lemma}
% subsection face (end)
\section{Tetrahedral Finite Elements}\label{sec:tetrahedralFEs}
\subsection{$H(\Div)$--Conforming Element on Tetrahedra} % (fold)
\label{sub:definition_of_the_h_div_element_on_tetrahedra}
For $k > 0$, let
\begin{IEEEeqnarray}{rCl}\label{tetrahedralSpace}
  P_{\hat{E}} & = & (P_{k-1})^3 + P_{k-1}\,\hat\bx\\
  &=& (P_{k-1})^3 \oplus \widetilde{P}_{k-1}\,\hat\bx.
\end{IEEEeqnarray}
\begin{lemma}
  The dimension of $P_{\hat{E}}$ is $\nicefrac{1}{2}(k+3)(k+1)k$.
\end{lemma}
\begin{lemma}\label{lema_div} $\dv P_{\hat E} = P_{k-1}$.%\noindent{\color{BrickRed}\#\#\#\#\#\#\# esto ya no lo tenemos en%virtuales :)}
\end{lemma}
\begin{defi}
Given a non--negative integer $k$, the 
$H(\Div)$--Conforming 
Finite Element of degree $k$ is defined by the following triple.
\label{defi_face_element_tetra}
\begin{itemize}
  \item $\hat{E}$ is the reference tetrahedron of Definition~\ref{def_of_ref_elems}. 
  \item $P_{\hat{E}}$ is the polynomial space in~(\ref{tetrahedralSpace}).
    \item The degrees of freedom are
    \begin{IEEEeqnarray*}{lll}
      \iint_{\hat f}\hat{q}\hat\bu\cdot\hat\bn\,d\hat{S}
      \quad  &\mbox{for all $\hat q\in P_{k-1}(\hat f)$}&\mbox{for all face $\hat f$ of $\hat E$}\\
      \int_{\hat E} \hat\bu\cdot\hat{\bq}\,d\hat\bx
      \quad  &\mbox{for all $\hat\bq\in (P_{k-2}(\hat E))^3$}.&
    \end{IEEEeqnarray*} 
\end{itemize}
\end{defi}
% subsection definition_of_the_h_div_element_on_tetrahedra

% subsection defEdgeElement (end)
% \begin{proposition} On the Finite Element~(\ref{defi_h_div_conforme}). 
% $\dim(P_{\hat{E}}) = k^2\,(k+2) + k\,(k+1)^2/2$
% which is, at the same time, equal to the number of its independent degrees of freedom.
% \end{proposition}
% \begin{proof}
%     From~(\ref{tensor_prod_dim}) and~(\ref{dk}) it is really straightforward to do
%     \begin{IEEEeqnarray*}{rCl}
%         \dim (P_{k-1}^2(x,y) \oplus \tilde{P}_{k-1}(x,y)) \otimes P_{k-1}(z) + \dim P_{k-1}(x,y)\otimes P_k(z) & = &\\[5pt]
%         \IEEEeqnarraymulticol{3}{r}{=\,(k(k+1)+k)k+\dfrac{k(k+1)}{2}(k+1)} 
%     \end{IEEEeqnarray*}
%     and the same quantity is obtained by summing up the dimensions of all the
%     \emph{test} spaces on the right of~(\ref{momentos1hdiv})--(\ref{momentos4hdiv}).
% \end{proof}
\section{Differentials, coordinate transformations, 
unisolvence and conformity of Finite Elements}

For \noindent{\color{BrickRed} the motivation of (SOME of)}  the following exposition we
refer to~\cite{ciarlet},
 Section 3.9 of~\cite{monk} and~\cite{gh99}.

We want this same setting to build Finite Elements on 
arbitrary contiguous prisms
belonging to a fixed mesh, all of which being an affine 
image of the reference prism.
In order to do so we transform the set $\hat{E}$ and show how to
pull--back and forward the scalars and fields in the 
discrete local spaces and the corresponding 
degrees of freedom.

%===============
%exactly as we are told by the 
%push-forward
%transformations~(\ref{push-forward}). 
%===============

Consider the application $\hat{\bx}\longmapsto{\bx} = 
F_E(\hat{\bx})$, where 
\begin{IEEEeqnarray}{rCl} \label{aux_label8}       
  F_E &= &M_E\hat{\bx} + \bx_0\mbox{, } 
\end{IEEEeqnarray}
that transforms
$\hat{E}$ in an element $E$ of the mesh and $M_E$ is invertible.

A scalar function $\hat{p} \in H^1(\hat{E})$ is transformed into a 
scalar function $p$ on $E$ with
\begin{IEEEeqnarray}{rCl}
    \label{transfEscalar} p\circ F_E & = & \hat{p}.
\end{IEEEeqnarray}
As it holds~(\cite{ciarlet})
\begin{IEEEeqnarray}{rCl} \label{aux_label4}
  \nabla p = M^{-t}\hat{\nabla} \hat{p} \circ F_K^{-1}\mbox{,}
\end{IEEEeqnarray}
then it results 
$p \in H^1(K)$. Gradients are taken with respect to the local coordinates
of $E$ and $\hat{E}$ in each case. The same will apply for every finite element
and for $\dv$ and $\curl$.

Now let $\hat{\bu} \in H(\bcurl, \hat{E})$. We want to assign to it a function
$\bu$ on $E$. As for $\hat{p}$ and $p$ as before it holds
$\nabla p \in H(\bcurl, E)$ and $\hat{\nabla} \hat{p} \in H(\bcurl, \hat{E})$,
equality~(\ref{aux_label4}) suggests the following transformation.
\begin{IEEEeqnarray}{rCl}
    \label{transfHcurl} \bu\circ F_E & = & M_E^{-t}\hat{\bu}.
\end{IEEEeqnarray} 
With this definition we have $\bu\in H(\bcurl, E)$ and also
\begin{IEEEeqnarray}{rCl}
    \label{transfCurl} (\textbf{curl}\,\bu)\circ F_E & = & 
    \frac{1}{\det M_E} M_E (\curl\hat{\bu})\mbox{,}
\end{IEEEeqnarray}
(cfr. Lemma 3.57, page 77 and Corollary 3.58 in~\cite{monk}). Additionally
\begin{IEEEeqnarray}{rCl}\label{aux_label29}
  \bh^\alpha(\partial^\alpha \bu)\circ F_E & = & 
    M_E^{-t}\hat{\partial}^\alpha \hat\bu 
\end{IEEEeqnarray}
that is,
\begin{IEEEeqnarray}{rCl}
  \bh^\alpha(\partial^\alpha \bu)\hat{} & = & 
    \hat{\partial}^\alpha \hat\bu. 
\end{IEEEeqnarray}

Now we proceed in a similar way for $H(\dv)$. The relation 
$\hat\bu\in H(\curl,\hat E)$ implies $\curl\hat\bu\in H(\dv, \hat E)$
so~(\ref{transfCurl}) shows that to transform $\hat\bu\in H(\dv,\hat E)$
into $\bu\in H(\dv,E)$ we must do it via
\begin{IEEEeqnarray}{rCl}\label{transfDiv}
	\bu\circ F_E & = & \frac{1}{\det M_E}M_E\hat\bu.
\end{IEEEeqnarray}
If $\bu$ and $\hat\bu$ are related
by~(\ref{transfDiv}), then with a $\mathcal{D}(E)$ density argument we get 
\begin{IEEEeqnarray}{rCl} %% By Lemma 3.59 in~\cite{monk}
  \label{derivadaPiola} (\dv\,\bu)\circ F_E & = & (\det M_E)^{-1}\dv\,\hat\bu.
\end{IEEEeqnarray}
and hence $\bu\in H(\dv,E)$ if and only if $\hat\bu\in H(\dv,\hat E)$.

Normals and tangents are transformed as follows (cfr. page 265 of~\cite{giraultRaviart}).
Let $\hat{\bnu}$ be the unit outward normal to $\hat E$
If $\hat\bx\in\partial \hat{E}$ and $\bnu$ is defined by
\begin{IEEEeqnarray}{rCl} \label{aux_label9}
  \bnu(F_E\hat\bx)&=&
    \frac{M_E^{-t}\hat{\bnu}(\hat\bx)}{\|M_E^{-t}\hat{\bnu}(\hat\bx)\|}\mbox{,}
\end{IEEEeqnarray} 
then $\bnu$ is a unit normal to $E$. Second, let $\hat\btau$ be any
unit vector tangent to $\partial{\hat{E}}$ at $\hat\bx$. If $\btau$ is
given by
\begin{IEEEeqnarray}{rCl} \label{aux_label10}
  \btau(F_E\hat\bx)&=&
    \frac{M_E\hat{\btau}(\hat\bx)}{\|M_E\hat{\btau}(\hat\bx)\|}\mbox{,}
\end{IEEEeqnarray}
then $\btau$ is a unit vector tangent to $\partial E$ at $F_E\hat\bx$.
Surface differentials are changed in the following way.
\begin{IEEEeqnarray}{rCl} \label{surface_diffs}
    dS & = & \|M_E^{-t}\hat\bnu\|\,|\det B|\,d\hat{S}
\end{IEEEeqnarray}
%======================================================================
% With~(\ref{transfDiv})
% \begin{IEEEeqnarray*}{rCl}
%    \|\hat{\textbf{v}}\|^2_{L^2(\hat{E})}
%    &=& \sum_{1\leqslant i\leqslant 3}\|\hat{v}_i\|^2_{0,\hat{E}}\\[7pt]
%    &=& \sum_{1\leqslant i\leqslant 3}\frac{h_jh_k}{h_i}\,\|v_i\|^2_{0,K};\\[7pt]
%    \|\hat{v}_i\|^2_{0,\hat{E}}&=&\frac{h_jh_k}{h_i}\,\|v_i\|^2_{0,K}
% \end{IEEEeqnarray*}
% where $\{i,j,k\} = \{1,2,3\}$.
%======================================================================
%=================================================================
%\begin{lemma} For all $\tilde{\boldsymbol{\sigma}} \in H(\dvg, \tilde{K})$,
%$\boldsymbol{\sigma}$ results in
%$H(\dvg, K)$ and in fact
%\[
%    {div}\,\boldsymbol{\sigma}({\bx}) =
%        \frac{1}{\det DF}\,\tilde{{div}}\,\tilde{\boldsymbol{\sigma}}(\tilde{{\bx}}).
%\]
%\end{lemma}
%\begin{proof}
%Observe that
%\begin{IEEEeqnarray*}{rCl}
%    trace(A\cdot B\cdot A^{-1}) &=& trace(B)\\
%    \label{Piola}\yesnumber\boldsymbol{\sigma} \circ F & = & \frac{1}{\det(A)} A\,\tilde{\boldsymbol{\sigma}}\\
%    \label{derivadaPiola}\yesnumber D\tilde{\boldsymbol{\sigma}}(\tilde{\bx}) & = &
%        \det(A)\,A^{-1}\,D\boldsymbol{\sigma} (F(\tilde{\bx})) \,A.
%\end{IEEEeqnarray*}
%Then
%\begin{IEEEeqnarray*}{rCl}
%    \text{div}\,\boldsymbol{\sigma}(\bx) & = & trace(D\boldsymbol{\sigma} (\bx))\\
%                                        & = & \frac{1}{\det(A)}\,trace(A\,\tilde{D}\tilde{\boldsymbol{\sigma}} (F^{-1}(\bx))\,A^{-1})\\
%                                        & = & \frac{1}{\det(A)}\,\tilde{\text{div}}\,\tilde{\boldsymbol{\sigma}}(\tilde{\bx}).   
%\end{IEEEeqnarray*}
%\end{proof}
%=================================================================

First a key result that establishes a relation between the interpolation
operators. It will used as an important step in the proof of the stability of the
edge element as well as in the proof of the stability of the face elements.
\begin{remark} Lemma 5.40 in page 135 and the first Paragraph of Section 5.7 
in page 149 of~\cite{monk} state the following facts.
  
Take $\bw_E$ es el operador de interpolaci\'on determinado por el elemento en
la Definici\'on~\ref{edgeelement}, $\br_E$ es el operador de interpolaci\'on determinado por el elemento en la
Definici\'on~(\ref{defi_h_div_conforme}) and $\pi^{\perp}_E$ is
the $L^2$--orthogonal projection onto $P_k(E)$, then 
\begin{enumerate}
  \item 
For all sufficiently smooth $\bu$ such that both the interpolants
$\bw_E\bu$ and $\br_E\curl\bu$ are defined, then
\begin{IEEEeqnarray}{rCl}
\label{curl_commutativity}
  \curl\bw_E\bu &=& \br_E\curl\bu.
\end{IEEEeqnarray}
  \item 
For all sufficiently smooth $\bu$ such that both the interpolants
$\br_E\bu$ and $\pi^{\perp}_E\dv\bu$ are defined, then
\begin{IEEEeqnarray}{rCl}
\label{div_commutativity}
  \dv\,\br_E\bu & = & \pi^{\perp}_E\dv\,\bu.
\end{IEEEeqnarray}
\end{enumerate}
\end{remark}
%=============================================================
% \begin{proof}
% Vamos a usar la siguiente versión superficial del Teorema de Stokes. Sea dado un dominio Lipschitz acotado 
% $S\subseteq\mathbb{R}^2$ con tangente unitaria $\boldsymbol{\tau}$ al borde $\partial S$. Para 
% $\bu \in \mathcal{C}^1(\bar{S})^2 $ y $\phi \in \mathcal{C}^1(\bar{S})$ tenemos
% \begin{IEEEeqnarray}{rCl}
%     \int\limits_S \bu d\gamma & = &   %% HACER SEGUIR ACA
% \end{IEEEeqnarray}
% \end{proof}
%=============================================================
\noindent The last result can be expressed saying that the following diagram commutes:
\begin{center}
        \begin{tikzpicture}[point/.style={circle, inner sep=0pt, minimum size=2pt,fill=red}]
            \matrix[column sep = 1.82mm, row sep = 1.1mm, ampersand replacement = \&] {
             \node {$\text{H}(\textbf{curl},E)$};  
              \& \node (n0) {};
              \& \node      {};
              \& \node (n1) {};
              \& \node (n2) {};
              \& \node {$\text{H}(\Div, E)$}; 
              \& \node (r1c7) {};
              \& \node {};
              \& \node {};
              \& \node (r1c10) {};
              \& \node {$L^2(E)$};\\
             \node (n3) {}; \&\&\&\&\& \node (n5)   {};
              \& \node (r2c7) {};
              \& \node {};
              \& \node {};
              \& \node {};
              \& \node (r2c11) {};\\
             \node (n4) {}; \&\&\&\&\& \node (n6)   {};
              \& \node (r3c7) {};
              \& \node {};
              \& \node {};
              \& \node {};
              \& \node (r3c11) {};\\
             \node (v)  {$V$}; \&\node(fromV){};\&\&\&\node(toW){};\& \node (w) {$W$};
              \& \node (r4c7) {};
              \& \node {};
              \& \node {};
              \& \node (r4c10) {};
              \& \node {$Q$};\\
             \node (n7) {}; \&\&\&\&\& \node (n8)   {};
              \& \node {};
              \& \node {};
              \& \node {};
              \& \node {};
              \& \node (r5c11) {};\\
             \node      {}; \&\&\&\&\& \node        {}; 
              \& \node {};
              \& \node {};
              \& \node {};
              \& \node {};
              \& \node {};\\
             \node      {}; \&\&\&\&\& \node        {}; 
              \& \node {};
              \& \node {};
              \& \node {};
              \& \node {};
              \& \node {};\\
             \node (n11)    {}; \&\&\&\&\& \node (n12) {};
              \& \node {};
              \& \node {};
              \& \node {};
              \& \node {};
              \& \node (r8c11) {};\\
             \node {$V_{\textit{h}} $};                                 
              \& \node (n13) {};
              \& \node       {};
              \& \node (n14) {};
              \& \node (n15) {};
              \& \node {$W_{\textit{h}} $}; 
              \& \node (r9c7) {};
              \& \node {};
              \& \node {};
              \& \node (r9c10) {};
              \& \node {$Q_h$};\\
             };
            \draw[->] (n0) to node[above] {$\textbf{curl}$} (n2); 
            \draw[->] (fromV) to node[above] {$\textbf{curl}$} (toW); 
            \draw[white] (n3) to node {{\color{black}$\cup$}} (n4);
            \draw[white] (n5) to node {{\color{black}$\cup$}} (n6);
            \draw[->] (n7) to node[left] {$\bw_E$} (n11); 
            \draw[->] (n8) to node[left] {$\boldsymbol{r}_k$} (n12); 
            \draw[->] (n13) to node[above] {$\textbf{curl}$} (n15); 
            \draw[->] (r1c7) to node[above] {$\text{div}$} (r1c10);
            \draw[->] (r4c7) to node[above] {$\text{div}$} (r4c10);
            \draw[->] (r9c7) to node[above] {$\text{div}$} (r9c10);
            %\draw[white] (r2c7) to node {{\color{black}$\cup$}} (r3c7);
            \draw[white] (r2c11) to node {{\color{black}$\cup$}} (r3c11);
            \draw[->] (r5c11) to node[left] {$\pi^{\perp}$} (r8c11);
        \end{tikzpicture}
    \end{center}
\begin{equation}\label{push-forward}
  \mbox{\color{red} \ldots \mbox{push-forward} de los interpoladores\\
    y que los pull--backs conmutan con los interpoladores}
\end{equation}
{\bf hcurl}
  
\begin{lemma} \label{aux_label7}
Let $\hat E$ be the reference prism~(\ref{defi_of_ref_prism}) and
let $P_{\hat E}$ be the space in~(\ref{spaceFEprismHcurl}), that is,
\begin{IEEEeqnarray*}{rCl}
  P_{\hat{E}} & = & R_k(\hat{x}_1,\hat{x}_2) \otimes P_k(\hat{x}_3) \times 
            P_k(\hat{x}_1,\hat{x}_2) \otimes P_{k-1}(\hat{x}_3).
\end{IEEEeqnarray*}
Provided the matrix $M_E$ is block--diagonal in this manner
\begin{IEEEeqnarray*}{rCl}
  M_E & = & \blockdiag{m_{1,1}}{m_{1,2}}{m_{2,1}}{m_{2,2}}{m_{3,3}}\mbox{,}
\end{IEEEeqnarray*}
then $P_{\hat{E}}$ is invariant by transformation~(\ref{transfHcurl}).
\end{lemma}
\begin{proof}
  Starting with transformation~(\ref{transfHcurl}) take 
  $\hat{\bu}=(\hat u_1,\hat u_2,\hat u_3)'\in P_{\hat E}$
  and $\bu := (M_E^{-t}\hat{\bu})\circ F_E^{-1}$.
  Let 
  \[
    M_2 = \left(
    \begin{array}{cc}
      m_{1,1}  & m_{1,2}\\
      m_{2,1}  & m_{2,2} 
    \end{array}
    \right)
  \]
  and let $m_{i,j}^{(-1)}$ denote the coefficients of the inverse.
  By definition, 
  $(\hat u_1,\hat u_2)' = \hat{q}(\hat\bp + \hat{\boldsymbol{s}})$ for some
  $\hat\bp$, $\hat{\boldsymbol{s}}$ as in~(\ref{defSk}) and~(\ref{defRk}) and
  $\hat{q}\in P_{k}(\hat x_3).$ Using the blocks of $M_E$ and the tensor nature of
  $P_{\hat E}$,
  \begin{IEEEeqnarray*}{rCl}
    (\hat u_1(\bx),\hat u_2(\bx))' & = & (\hat{q}M_2^{-t}(\hat\bp + 
      \hat{\boldsymbol{s}}))(M_E^{-1}\bx - M_E^{-1}\bx_E)\\
    & = & \hat{q}(m_{3,3}^{(-1)}(x_3-x_{E,3}))(M_2^{-t}(\hat\bp + 
      \hat{\boldsymbol{s}}))(M_2^{-1}\bx - M_2^{-1}\bx_E).
  \end{IEEEeqnarray*}
  Now, $\hat{q}(m_{3,3}^{(-1)}(x_3-x_{E,3}))$ is a polynomial $q(x_3)$ with the
  same 
  degree as $\hat{q}$ is qith respect to $\hat x_3$. Furthermore,
  as by Remark~(\ref{aux_label6}) $\hat{\boldsymbol{s}}$ has degree $\leqslant k-1$,
  $(M_2^{-t}(\hat\bp + \hat{\boldsymbol{s}}))(M_2^{-1}\bx - M_2^{-1}\bx_E)$
  can be seen as
  $M_2^{-t}(\bp(x_1,x_2) + \hat{\boldsymbol{s}}(M_2^{-1}\bx))$ with $\bp$
  having the expected degrees. And now
  \begin{IEEEeqnarray*}{rCcCl}
    M_2^{-t}\hat{\boldsymbol{s}}(M_2^{-1}\bx)\cdot(x_1,x_2)'
        &=& 
    \hat{\boldsymbol{s}}(M_2^{-1}(x_1,x_2)')\cdot M_2^{-1}(x_1,x_2)' &=& 0.
  \end{IEEEeqnarray*}
  The invariance of the third component of elements in $P_{\hat E}$ is
  an immediate consequence of the affine nature of the coordinate change.
\end{proof}
%=======================================================================
%\begin{remark} this is not a restriccion...
%  \begin{figure}
%    \centering
%    *** ***
%    \caption{hybrid mesh}\label{obliquePrism}
%  \end{figure}
%  dibujo prisma oblicuo subdividido: tetra + prism + pyramid. Figure~\ref{obliquePrism}
%\end{remark}
%=======================================================================
\begin{lemma}\label{aux_label14}
Let a physical prism $E = F_E(\hat{E})$ for $F_E$ as in~(\ref{aux_label8}).
Given $\hat\bu$ defined in $\hat{E}$ let $\bu$ in $E$ dtermined by~(\ref{transfHcurl}), and
let $\hat\btau$, $\btau$, $\hat\bnu$ and $\bnu$ be tangents and normals in
$\hat E$ and $E$ related by~(\ref{aux_label10}) and~(\ref{aux_label9}). Suppose
we define degrees of freedom of $\bu$ on $E$ as
\begin{IEEEeqnarray}{ll}
    \label{momentos1hcurlPhys}  
    \varphi_{\be_i,p}\,(\bu) = 
    \int_{\be} \bu \cdot \boldsymbol{\tau} \,q\, ds  
        & q\in P_{k-1}\mbox{,} \\
    \IEEEeqnarraymulticol{2}{l}{\nonumber\mbox{ for each edge $\be$ with unit tangent } \boldsymbol{\tau} \mbox{;}}\\[8pt]
    \label{momentos2hcurlPhys} 
    \varphi_{f,\bq}\,(\bu) =  
    \int_{f} \bu \cdot \bq_0\,
    dS\mbox{, } &\bq_0 = (\det M_E\|M_E^{-t}\hat\bnu\|)^{-1} M_E\hat{\bq}_T\\
    &\hat{\bq}_T := (\hat\bnu\times\hat\bq)\times\hat\bnu\\
    &\mbox{$\hat\bq$ as in~(\ref{momentos2hcurl})--(\ref{momentos5hcurl})}.\\[8pt]
    \label{momentos3hcurlPhys}
    \varphi_{\br}(\bu) = 
    \int_{E} \bu \cdot \br \, d\bx\mbox{, }&\br = (\det M_E)^{-1} M_E\hat\br \circ F_E^{-1}\\
    &{\nonumber\hat\br \in P_{k-2,k-2} \times P_{k-3,k-1}.}
\end{IEEEeqnarray}
Provided $\det > 0$, the degrees of freedom are identical
\end{lemma}
\begin{proof}
  Edge dofs\\
  Take a Parametrization $\hat{\boldsymbol{\alpha}}(t):I=[0,1]\to \hat\be$ such that
  $\hat\btau = \stackrel{.}{\hat{\boldsymbol{\alpha}}}(t)/\|\stackrel{.}{\hat{\boldsymbol{\alpha}}}(t)\|$.
  As a jacobian maps tangents into tangents,
  $\boldsymbol{\alpha}(t) = M_E\hat{\boldsymbol{\alpha}}(t) + \bx_E$ parameterizes
  $\be$. Then
  \begin{IEEEeqnarray*}{rCl}
    \int_{\be}(q\bu)\cdot d\boldsymbol{\alpha}
    &=&\int_{I}
    q\bu(F\hat{\boldsymbol{\alpha}}(t)))\cdot M_E
    \stackrel{.}{\hat{\boldsymbol{\alpha}}}(t)dt  \\[5pt]
    &=&\int_{I}
    \hat q(\hat{\boldsymbol{\alpha}}(t)) M_E^{-t}
    \hat\bu(\hat{\boldsymbol{\alpha}}(t))\cdot M_E\stackrel{.}{\hat{\boldsymbol{\alpha}}}(t)dt\\[5pt]
    &=&\int_{\hat\be}(\hat q\hat\bu)\cdot d\boldsymbol{\hat\alpha}.
  \end{IEEEeqnarray*}
  Face dofs: 
  \begin{IEEEeqnarray*}{rCl}
    \int_{\hat f} \hat\bv\times\hat\bnu\cdot\hat\bq\times\hat\bnu\,d\hat S 
    & = & \int_{\hat f} (\hat\bnu\times\hat\bq)\times\hat\bnu\cdot\hat\bu\,d\hat S \\
    & = & \int_{\hat f} \det M_E\|M_E^{-t}\hat\bnu\| \bq_0 (F\hat\bx) \cdot \bu(F\hat \bx) d\hat S \\
    & = & \int_{ f} \bq_0 (\bx) \cdot \bu( \bx) d S
  \end{IEEEeqnarray*}
  Vol dofs:
  \begin{IEEEeqnarray}{rCl}
    \int_{E}\bv\cdot\br\,d\bx
    & = & 
    \int_{\hat E}M_E^{-t}\hat\bv(F_E^{-1}\bx)\cdot
      (\det M_E)^{-1}M_E\hat\br(F_E^{-1}\bx)\,d\hat\bx.\\
    & = & 
    \int_{\hat E}\hat\bv(\hat\bx)\cdot\hat\br(\hat\bx)\,d\hat\bx.
  \end{IEEEeqnarray} 
\end{proof}

\begin{corollary} The finite element in Definition~\ref{edgeelement}  is conforming
in $H\curl$ and unisolvent.
  theorem 8 page 75 in~\cite{nedelec2} is incomplete, because it is only on
  the reference element.
\end{corollary}
\begin{proof}
  Transform the dofs to $\hat E$ and use proof of Theorem 8, page 75 of
  ~\cite{nedelec2}. 
\end{proof}

\begin{corollary}
 \label{aux_label26}\noindent{\color{BrickRed}\#\#\#\#\#\#\# ojo tipeo ac'a\\} 
  For any $\bv\in W^{1,p}(E)$ expressions 
  ~(\ref{momentos1hcurlPhys})--(\ref{momentos3hcurlPhys}) 
  determine a well defined local interpolate
  $\bw_E\,\bu$ defined as the unique finite element function in $P_E$ such that
  \begin{IEEEeqnarray}{lClc}
    \varphi_{\be,q}\,(\bv - \bw_E\,\bu) & = & 0 &
    \quad\mbox{for $\varphi_{\be,q}$ as in~(\ref{momentos1hcurlPhys})}\\
    \varphi_{f,\bq}\,(\bv - \bw_E\,\bu) & = & 0 &
    \quad\mbox{for $\varphi_{f,\bq}$ as in~(\ref{momentos2hcurlPhys})}\\
    \varphi_{\br}\,(\bv - \bw_E\,\bu) & = & 0 &
    \quad\mbox{for $\varphi_{\br}$ as in~(\ref{momentos3hcurlPhys})}.
  \end{IEEEeqnarray}
\end{corollary}


This will be used in the proof of local interpolation estimates.
\begin{lemma} Provided $\det M^{t} > 0$, the edge element interpolators satisfy
\begin{IEEEeqnarray}{rCl}\label{piTransformado}
    \wku(\hat{\bx}) & = & M^{t} \bw_E\bu(F_E(\hat{\bx}))
\end{IEEEeqnarray}
That is, the interpolator commutes with the coordinate change~(\ref{transfHcurl}).
\end{lemma}
\begin{proof} 
  By definition, for all dof $\varphi$ in (poner refs)
  $\varphi(\bu - \bw_E\bu) = 0$. By dofs identity in Lemma~\ref{aux_label14}
  \[
  \hat{\varphi}\left((\bu - \bw_E\bu)^{\hat{}\,}\right) = \hat{\varphi}\left(\hat{\bu} - (\bw_E\bu)^{\displaystyle\hat{}\,}\right) = 0
  \]
  which implies $\wku = \hat{\bw}_E(\bw_E\bu)^{\hat{}}$.
  The space $P_{\hat E}$ was proven to be invariant under~(\ref{transfCurl}), so
  $\wku = (\bw_E\bu)^{\hat{}}$.
\end{proof}
{\bf hcurl end}

{\bf hdiv begin}

\begin{lemma}\label{aux_label13} Let $\hat E$ be the reference prism~(\ref{defi_of_ref_prism}) and
let $P_{\hat E}$ be the space in~(\ref{prismaticSpace}).
Provided the matrix $M_E$ is block--diagonal as in Lemma~\ref{aux_label7},
then $P_{\hat{E}}$ is invariant by transformation~(\ref{transfDiv}).
\end{lemma}
\begin{proof} The proof uses the same straightforward approach as the proof of 
Lemma~\ref{aux_label7} and in a much simpler case. \noindent{\color{BrickRed}pensar esto}   
\end{proof}

\begin{lemma} \label{aux_label12}
The degrees of freedom are identical, surface and volumen integrals:
in which the faces are related by $F\hat{f} = f$.
\begin{IEEEeqnarray}{cCccl}
    \label{momentos1hdivPhys} 
    \rho_{ f,q}(\bv) & = & \int_{f} (\bv\cdot\bnu)q\,dS 
        &\quad & \mbox{for } q = \hat{q}\circ F_E^-1, \hat{q} \in P_{k-1}(\hat{f})\mbox{,}\\
    \nonumber&&&\quad&\mbox{if $ \hat{f} =  \hat{f}_3$ or $ \hat{f}_4$;}\\[5pt]
    \label{momentos2hdivPhys}
    \rho_{ f,q}(\bv) & = & \int_{f} (\bv\cdot\bnu)q\,dS 
        &\quad & \mbox{for } q = \hat{q}\circ F_E^-1, \hat{q} \in Q_{k-1, k-1}(\hat f)\mbox{,}\\
    \nonumber&&&\quad&\mbox{ if $ \hat{f} =  \hat{f}_1$, $ \hat{f}_2$ or $ \hat{f}_5$;}\\[5pt]
    \label{momentos3hdivPhys}
    \rho_{ \br}(\bv) & = & \int_{{E}} \bv\cdot\br\,d\bx 
        &\quad& \mbox{for }\br = M_E^{-t}\hat\br\circ F_E^{-1}, \hat\br\in (P_{k-2,k-1})^2 \times P_{k-1,k-2};
\end{IEEEeqnarray}
\end{lemma}
\begin{proof}
  Face dofs, by~(\ref{transfDiv}) and~(\ref{surface_diffs}) 
  \begin{IEEEeqnarray*}{rCl}
    \int_{f} \bv\cdot\bnu\bq\,d S & = & 
    \int_{f} (\det M_E)^{-1} \hat\bv(F_E^{-1}\bx)\cdot\|M_E^{-t}\hat\bnu\|^{-1}\hat\bnu\hat\bq(F_E^{-1}\bx)\,d S \\
    & = &\int_{\hat f} \hat\bv(\hat\bx)\cdot\hat\bnu\hat\bq(\hat\bx)\,d\hat S 
  \end{IEEEeqnarray*}    
Vol dofs
  \begin{IEEEeqnarray}{rCcCl}
    \int_{E}\bv\cdot\br\,d\bx
    & = & 
    \int_{\hat E}M_E\hat\bv\cdot M_E^{-t}\hat\br\,d\hat\bx.
    & = & 
    \int_{\hat E}\hat\bv\cdot\hat\br\,d\hat\bx.
  \end{IEEEeqnarray}
\end{proof}

\begin{lemma}
  If $\bu\in P_{E}$ is such that all the
  dofs~(\ref{momentos1hdivPhys}) or~(\ref{momentos2hdivPhys}) vanish
  on the respective face $f$, then the normal component of $\bu$ 
  vanishes identically on $f$.
\end{lemma}
\begin{proof}
  This fact is stated in the proof of Theorem 4, page $66$ of~\cite{nedelec2} 
  only for the reference prism. By transforming the degrees of freedom to 
  the faces of $\hat E$ and back to the faces of $E$ using Lemma~\ref{aux_label12}
  we get the result.
\end{proof}
\begin{lemma}
  If $\bu\in P_{E}$ is such that all the
  dofs~(\ref{momentos1hdivPhys})--(\ref{momentos3hdivPhys}) vanish, 
  then $\bu$ vanishes identically on $E$.
\end{lemma}
\begin{proof}
  In the case of the reference prism $\hat E$, we refer to the proof of 
  Theorem 4 in page $66$ of~\cite{nedelec2}.
  by the invariance of the finite element space under
  the change~(\ref{transfDiv}) (Lemma~\ref{aux_label13}) we have our result.
\end{proof}
\begin{corollary}
  Finite Element on $E$ is $H(\dv)$--conforming and unisolvent.
\end{corollary}
\begin{corollary}
  For any $\bv\in W^{1,1}(E)$ expressions 
  ~(\ref{momentos1hdivPhys})--(\ref{momentos3hdivPhys}) 
  determine a well defined local interpolate
  $\br_E\,\bu$ defined as the unique finite element function in $P_E$ such that
  \begin{IEEEeqnarray}{lClc}
    \rho_{f,\bq}\,(\bv - \br_E\,\bu) & = & 0 &
    \quad\mbox{for $\rho_{f,\bq}$ as in~(\ref{momentos1hdivPhys})
      and~(\ref{momentos2hdivPhys})}\\
    \rho_{\br}\,(\bv - \br_E\,\bu) & = & 0 &
    \quad\mbox{for $\rho_{\br}$ as in~(\ref{momentos3hdivPhys})}.
  \end{IEEEeqnarray}
\end{corollary}
And the most important, whose proof is like the one of~(\ref{piTransformado}).
\begin{corollary}\label{aux_label17}
  Given $\bu \in W^{1,1}(E)$, then
  \begin{IEEEeqnarray}{rCl} \label{div_interp_commutes}
    \rku & = & \det M_E\,(M_E^{-1}\,\br_E\,\bu)\circ F_E\mbox{,}
  \end{IEEEeqnarray}
  that is, the $H(\dv)$ interpolator commutes with the coordinate
  change~(\ref{transfDiv}).
\end{corollary}

{\bf hdiv end}


\begin{remark}
  Unisolvence and conformity of the finite elements on tetrahedra in
  Section~\ref{sec:tetrahedralFEs}
  is proved in~\cite{monk}, Sections 5.4 and 5.5.
\end{remark}
%===========================
% See Lemmas 5.32 and 5.34 in pages 130 and 131 of~\cite{monk}, where
% it is proved for tetrahedra. Exacly the same considerations about the degrees
% of freedom therein apply to prove the Lemma for prisms or pyramids.  
% Ver (3.79) (3.80) en monk: transf. de normales y tangentes.\\\\
%===========================
\begin{chapter}[Virtual Elements]{Virtual Elements}
\section*{Introducci\'on al cap\'itulo}
Los Elementos Virtuales son definidos en t\'erminos de un dado problema. 
Es nestra intenci\'on
proponer un esquema para un m\'etodo combinado 
de elementos Finitos/Virtuales
como una generalizaci\'on de los Elementos Finitos $H(\Div)$--conformes
en mallas consistentes en
poliedros de geometr\'ia arbitraria. En este
Cap\'itulo presentamos un desarrollo de los espacios, las formas
bilineales y las propiedades concernientes a los  
elementos virtuales.
Trabajamos con tetraedros, prismas triangulares y pir\'amides y,
en presencia de estas \'ultimas,
nuestro esquema FEM/VEM se ubica en
el marco de los Elementos Finitos no--polinomiales.

El m\'etodo de elementos virtuales (VEM) 
fue introducido recientemente~\cite{MR2997471}
como una generalizaci\'on de elementos $H^1$--conformes 
para elementos de geometr\'ia arbitraria y como una
generalizaci\'on de las Diferencias Finitas Mim\'eticas
a grado arbitrario de precisi\'on. Una extensi\'on de 
la discretizaci\'on de campos vectoriales $H(\Div)$--conformes
y aproximaciones por elementos finitos mixtos
fue propuesta en~\cite{bfm} en dimensi\'on 2. 
Adem\'as, in~\cite{MR3507271} un VEM mixto
fue analizado
para la aproximaci\'on de 
problemas lineales el\'ipticos con coeficientes variables.

Un espacio de elementos virtuales puede contener funciones
no \emph{polinomiales a trozos}
y principalmente contiene funciones
que son desconocidas y no pueden ser evaluadas en ning\'un punto.
En el VEM los espacios y los grados de libertad son tomados de tal manera
que la matriz de rigidez elemental puede ser computada
sin conocer estas funciones no polinomiales, en nuestro caso gracias 
al Teorema de la Divergencia.

\section*{Introduction to the chapter}
The Virtual Elements are customarly defined in terms of a given problem. Our
intention 
is to propose a combined Finite/Virtual Elements Method scheme
as a generalization of $H(\Div)$--conforming Finite Elements
in meshes consisting of polyhedra of arbitrary kind. In this Chapter we develop
the spaces, bilinear forms and properties concerning 
virtual elements.
We deal with tetrahedra, triangular prisms and pyramids and, in the presence of
the latter, our FEM/VEM scheme is put in the framework of 
non--polynomial Finite Elements.

The virtual element method (VEM) has been recently introduced \cite{MR2997471} 
as a generalization of $H^1$-conforming finite elements to arbitrary 
element--geometry 
and as a generalization of Mimetic Finite Differences to arbitrary degree 
of
accuracy and arbitrary continuity properties. An extension to 
the discretization of $H(\Div)$--conforming vector fields and mixed finite 
element 
approximations has been proposed in~\cite{bfm} 
in the two dimensional case. Furthermore, in \cite{MR3507271} a mixed VEM 
has been
analysed for the approximation of general linear elliptic problems with variable
coefficients.  
The virtual element space can contain non piecewise polynomial
functions and mainly functions which are a priori unknown, 
in the sense that they
can't be evaluated in any point. In the VEM approach, the space and the degrees
of freedom are taken in such a way that the elementary stiffness matrix can be 
computed without actually computing these non--polynomial functions, but just using 
the degrees of freedom. In this respect, a key
point in this approach and particularly in our case is that, 
given an element $E$, if $\bu=\nabla q_2$ for a known 
polynomial $q_2$, then for a field $\bv$ the quantity 
\[
\int_E\bu\cdot\bv\,d\bx
\]
can be computed if $\mbox{div\,}\bv$ on $E$ and the outer normal 
component $\bv\cdot{\bf n}$ of $\bv$ on 
$\partial E$ are known polynomials, by using the divergence Theorem.
%In Section~\ref{auxlabel290} the discrete spaces for the discrete mixed 
%formulation are defined together with the 
%variational discrete forms. The discrete form, $a_h$, requires of projections 
%onto subspaces of the local discrete 
%fields. Those projections become the identity in the case of tetrahedral or 
%prismatic elements, 
%but for pyramids, they require some additional analysis. 
%In Section \ref{liee} the interpolation on the virtual element spaces is 
%considered and 
%interpolation error estimates are proved under different shape assumptions. Also a 
%discrete $\inf--\sup$ property is proved, which is used in Section \ref{appr err} to 
%give an abstract approximation error result. 
\section{Definition and Construction}
\label{auxlabel290}
First we recall Problem~\ref{weakMixedContinuous} and add some notation for the 
bilinear forms.
Let  $V:=H(\Div, \Omega)$, $Q:=L^2(\Omega)$ and
\begin{IEEEeqnarray*}{rClCrCl}
  a(\bv,\bw) & = & \forma{v}{w} &\quad\mbox{and}\quad& b(\bv,q) & = & \formb{v}{q}.
\end{IEEEeqnarray*}
Then Problem~\ref{weakMixedContinuous} reads to
find $\bu\in V$ and $p\in Q$ such that for every $\bv\in V$ and every $q\in Q$
\begin{IEEEeqnarray*}{rCcCl}                          % Let be given an open \emph{non--convex} domain $\Omega\subseteq\mathbb{R}^3$ with
  a(\bu,\bv) & + &b(\bv,p) & = & 0\\[5pt]                % Lipschitz--continuous boundary
	       	   &   &b(\bu,q) & = & (-f,q).                     % consisting of planar faces and define 
\end{IEEEeqnarray*}                                   % Let us consider the following continuous problem.\\[5pt]
We start supposing we are given with a conforming polihedral triangulation $\Th$ of $\Omega$ to define the 
virtual spaces $V_h$ and $Q_h$ as discretizations of $V$ and $Q$ respectively. For now 
we will suppose the aspect ratios of tetrahedra and pyramids are bounded
in terms of a constant independent of $\textit{h}$. However, in Chapter~\ref{auxLabel100}
we will construct a mesh using the three types of polihedra mentioned before, and
we will see that the last assumtion is not a restriction.
We will assume a strictly decreasing parameter $\textit{h}$ tending to zero 
to mitigate the abuse of notation present in expressions like 
``$\bu_{\textit{h}}$'' and ``$\Th$''.
Then we are able to write, for example, $\bu_{\textit{h}}$ or 
$\bu_{\Th}$ indistinctly.

For $E\in\Th$ the local space of virtual vector fields will be
\begin{IEEEeqnarray*}{rCl}
  V_h(E)& = &\Big\{\bv\in H(\Div,E)\cap H(\bcurl,E)\,:\,\\
  \yesnumber\label{vhE}
   & & \qquad \bv\cdot\bn|_f\in P_0(f) \,\,\mbox{for all face $f$ of }E, \\
   & & \qquad \dv\bv  \in P_0(E) \mbox{ and } \curl\bv = 0 \Big\}
\end{IEEEeqnarray*}
and the  global space $V_h$ will consist of functions defined piecewise with the former
local spaces:
\begin{IEEEeqnarray*}{ccrCl}
  V_h & = & V_h(\Th) & := & \Big\{\bv\in H(\Div,\Omega): \bv|_E\in V_h(E),
  \mbox{for all element } E\in\Th\Big\}.
\end{IEEEeqnarray*}
The  condition on the $\curl$ is put because we are
considering gradients, as we will see later.
The scalar discrete space we will consider is
\begin{IEEEeqnarray}{rCl}
  Q_h & = & {P}_0(\Th)
\end{IEEEeqnarray}
meaning the functions that are constant on each element of $\Th$. As expected,
with $V_h$ we consider the $H(\Div,\Omega)$ norm, 
$\|\bv\|^2_{V_h} = \|\bv\|^2_{L^2(\Omega)} + \|\dv\bv\|^2_{L^2(\Omega)}$,
and with $Q_h$ we consider the $L^2(\Omega)$ norm.
As Problem~\ref{weakMixedContinuous}  has solution in $H(\Div,\Omega)$
it suggests to use a face--like  operator as virtual interpolator. For that reason,
let the degrees of freedom be
\begin{IEEEeqnarray}{rCl}\label{dofs}
  \iint_f \bv\cdot\bn\,dS & \qquad\mbox{ for all face $f$ of } & \Th.
\end{IEEEeqnarray}
With \emph{faces of $\Th$} we mean the family of all faces forming the boundary
of the elements, with $\bn$ being a unit normal vector, chosen for each face among the
two possibilities in the case of
neighboring elements. 

Something that has been already established for Finite Elements in
Chapter~\ref{aux_label43} must be proved for our Virtual Elements.
\begin{lemma}\label{unisolvence} Given a polyhedron $E\in\Th$, the degrees
  of freedom~(\ref{dofs}) corresponding to the faces of $E$ are unisolvent
  in $V_h(E)$.
\end{lemma}
\begin{proof} \emph{Existence.} Let $n_{f,E}$ be the number of faces of $E$ and
take real numbers $\{\alpha_i\}_{i=1}^{n_{f,E}}$. Let $g$ be the  piecewise constant
function on $\partial E$ satisfying, for all $1\leqslant i\leqslant n_{f,E}$, %face $f$ of $E$
\[
  \iint_{f_i} g\,dS = \alpha_i
\]
and let $d$ be the constant function in $E$ such that
\[
 \int_E d\,d\bx = \sum_i \alpha_i.
\]
As this proof is performed locally on a fixed element $E$, we can fix the degrees
of freedom in~\eqref{dofs} so that the normal vector points outward.
Then we consider the auxiliary problem of seeking a solution of
\begin{IEEEeqnarray}{rClrCl}
  \label{aux_prob}
  \Delta \phi & = & d \quad \mbox{in $E$,} \qquad & 
  \frac{\partial \phi}{\partial \bn}& = &g \quad \mbox{on }\partial E.
\end{IEEEeqnarray}
By definition we obtain the compatibility condition
\begin{IEEEeqnarray*}{rCl}
   \int_E d\,d\bx& = & \iint_{\partial E} g\,dS
\end{IEEEeqnarray*}
so the solution $\phi$ to the problem~(\ref{aux_prob}) exists. Now
we take $\bu:=\nabla \phi$ and  it holds immediately that $\dv\bu$ is constant in $E$,
$\bu\cdot\bn$ is constant on each face of $\partial E$ and $\curl\bu = 0$. So
$\bu$ lies in $V_h(E)$ and also for all $1\leqslant i\leqslant n_{f,E}$ 
$\iint_{f_i} \bu\cdot\bn\,dS = \alpha_i$.\\[4pt]
\emph{Uniqueness.} Suppose that $\bv\in V_h(E)$ has vanishing
degrees of freedom. Condition $\curl\bv=0$ implies
$\bv=\nabla\phi$ for certain $\phi$. Now, since $\dv\bv$ is constant on $E$, the
relation
\begin{IEEEeqnarray*}{rCl}
   0 & = & \int_{\partial P}\bv\cdot\bn\,dS
\end{IEEEeqnarray*} %& = & \int\limit s_P\dv\bv\,d\bx
implies $\dv\bv=0$ by Green Theorem. Then, the potential $\phi$ satisfies
\begin{IEEEeqnarray*}{rClrCl}
  \Delta \phi & = & 0 \quad \mbox{in $E$,} \qquad & 
  \frac{\partial \phi}{\partial \bn}& = &0 \quad \mbox{on }\partial E
\end{IEEEeqnarray*}
which means it is a constant, and it follows $\bv=0$.
\end{proof}
With this Lemma already demonstrated we are able to consider
an $H(\Div)$--like local interpolation operator well defined.
\begin{corollary} \label{interpolant}
  For every $\bv\in H^1(E)^3$ there exists a $V_h(E)$--interpolant $I\bv$
  defined as the unique element in $V_h(E)$ such that for every face $f$ of $E$
    \begin{IEEEeqnarray*}{rCl}
      \iint_f I\bv\cdot\bn\,dS & = & \iint_f \bv\cdot\bn\,dS.       
    \end{IEEEeqnarray*}
\end{corollary}
\begin{lemma} \label{p0_projection} Consider the projection $P_0$ onto the constants on $E$. It holds
\begin{IEEEeqnarray*}{rCl}
  \dv I\bv & = & P_0\,\dv\bv.
\end{IEEEeqnarray*}
\begin{proof}
  \begin{IEEEeqnarray*}{rCl}
    \int_E(\dv I\bv - \dv\bv)\,d\bx& = & \int_E\dv (I\bv - \bv)\,d\bx\\
    & = & \iint_{\partial E}(I\bv - \bv)\cdot\bn\,dS\\
    & = & \sum_{f\subseteq \partial E} \iint_{f}(I\bv - \bv)\cdot\bn\,dS = 0
  \end{IEEEeqnarray*}
\end{proof}
\end{lemma}
\begin{proposition}\label{vem_equal_fem}
Recall the space $D_k$ introduced in~(\ref{dk}). Then
\begin{enumerate}
  \item 
if $E$ is a right prism, we have
\begin{IEEEeqnarray}{rCl}\label{d1p1}
  V_{\textit{h}}(E) & = & D_1(x,y)\otimes P_1(z)\mbox{,}
\end{IEEEeqnarray}
with $x,y,z$ being the variables in a cartesian system of coordinates in which
the $z$--axis is orthogonal to the planes containing the triangular faces of $E$.
  \item 
If $E$ is a tetrahedron, we have
\begin{IEEEeqnarray}{rCl}\label{p03}
  V_{\textit{h}}(E) & = & P_0^3(\bx) + P_0\bx\mbox{,}
\end{IEEEeqnarray}
with $\bx = (x,y,z)'$ being the vector of variables in a cartesian system of
coordinates.
\end{enumerate}
\end{proposition}
\begin{proof}
  In the case of a prism $E$, the space $D_1(x,y)\otimes P_1(z)$ can be written as
  \[
    \{\bv \in P_1(E)\,:\, \bv=(a+\gamma x,b+\gamma y,c+d z),
        a,b,c,d,\gamma\in\mathbb{R})\}
  \]
  and then we see immediately that it has the same dimension 
  as $V_{\textit{h}}(E)$ which has dimension five. Finally, recalling that 
  the $\dv$ and $\curl$ operators
  can be computed in the chosen local variables, given $\bv\in D_1(x,y)\otimes P_1(z)$
  we can compute $\bv|_{f}\cdot\bn_f$ (on each face $f$ of $E$), $\dv\bv$ and $\curl\bv$
  to verify explicitly that the space on the right side of~(\ref{d1p1}) fulfills the 
  definition of $V_h(E)$ given in~(\ref{vhE}).
  Exactlty the same argument works for the case of a tetrahedron $E$, in which we
  deal with two vectorial spaces of dimension four that are such that 
  every element in
  \[
    P_0^3(\bx) + P_0\bx=\{\bv \in P_1(E)\,:\,
    \bv=(a+\gamma x,b+\gamma y,c+\gamma z)',
    a,b,c,\gamma\in\mathbb{R}\}
  \]
  fulfills the conditions defining $V_{\textit{h}}(E)$.
\end{proof}
\begin{remark}
  By Proposition~\ref{vem_equal_fem} the spaces $V_h(E)$ coincide with the
  lowest order $H(\Div)$--conforming local Finite Element spaces
  introduced in~(\ref{prismaticSpace})
  for the prismatic case and in~(\ref{tetrahedralSpace}) for the tetrahedral case.
\end{remark}
\begin{lemma}
  The definition of the lowest order $H(\Div)$ Finite Elements on
  right prisms or tetrahedra is independent of
the choice of the cartesian axes (as long as the $z$-axis is
perpendicular to the trianguar faces in case of prisms).
\end{lemma}
\begin{proof}
  This can be proved by hand. Let us prove the case of a prism $E$. Let 
  $(x,y,z)'$ and $(x',y',z')'$ be two cartesian coordinate systems satisfying the
  required properties. For them there is a change of coordinates such as
  \begin{IEEEeqnarray*}{rCl}
    x & = & p+\alpha x'-\beta  y' \\
    y & = & q+\beta x' -\alpha y' \\
    z & = & r+z'
  \end{IEEEeqnarray*}
for $\alpha$ and $\beta$ satisfying $\alpha^2+\beta^2 = 1$.
Let 
$\bv$ be an element in $V_{\textit{h}}(E)$. So
\begin{IEEEeqnarray*}{rCl}
\bv(x,y,z) & = & (a+\gamma x,b+\gamma y,c+d z)'\\
&=&((a+\gamma p) + \gamma(\alpha x' - \beta y'),\\
&& \,\,        (b+\gamma q) + \gamma(\beta x'+\alpha y'),\\
 && \,\,       (c+dr)+dz')'.
\end{IEEEeqnarray*}
Then, the components of $\bv$ in the new coordinate versors are
\begin{IEEEeqnarray*}{c}
 \begin{cases}
  \begin{IEEEeqnarraybox*}{lClCl}
    \bv\cdot(\alpha,\beta,0)' & = &
     (\alpha a + \beta b + \gamma(\alpha p + \beta q )) +\gamma x' & =: & a' + \gamma x' \\
    \bv\cdot(-\beta,\alpha,0)' & = &
     (-\beta a + \alpha b + \gamma(-\beta p + \alpha q )) +\gamma y' & =: & b' + \gamma y' \\
    \bv\cdot(0,0,1)' & = &
     (c+dr)+d z' & =: & c' + dz'.
  \end{IEEEeqnarraybox*}
 \end{cases}
\end{IEEEeqnarray*}
It follows that, in the $x'y'z'$ system,
\[
  \bv(x',y',z') = (a' + \gamma x', b' + \gamma y', c' + dz')'
  \in D_1(x',y')\otimes P_1(z').
\]
\end{proof}
%\paragraph{discretized forms} % (fold)
%\label{par:discretized_forms}
%\[
%\noindent{\color{blue}\#\#\#\#\#\#\# \mbox{¿hace falta?}} 
%  a(\bv,\bw)=\sum_{E\in\Th}a^E(\bv,\bw), \qquad
%  b(\bv,q)=\sum_{E\in\Th}b^E(\bv,q).
%\]
Next, together with the finite dimensional spaces already defined,
we present discretized version of the bilinear forms. The evaluation of
the form $b(\cdot,\cdot)$ at a pair $(\bv,q)\in V_h\times Q_h$ can
be computed using the degrees of freedom~(\ref{dofs}) applied to $\bv$. For if $q\in Q_h$, then
\[
  b(\bv,q)=\int_\Omega q\,\dv\bv\,d\bx = \sum_{E\in\Th}
  \int_E q\,\dv\bv\,d\bx = \sum_{E\in\Th}\iint_{\partial E}q \bv\cdot\bn\,dS.
\]
So in the case of the form $b_h(\cdot,\cdot)$ we put simply
\begin{IEEEeqnarray}{rCcCl}\label{aux_label44}
  b_h(\bv,q) & = & b(\bv,q) & = &
  \sum_{E\in\Th}b^E(\bv,q)\mbox{,}
\end{IEEEeqnarray}
where $b^E(\bv,q) = \int_E q\,\dv\bv\,d\bx$.
For the bilinear form $a(\cdot,\cdot)$, we can decompose it as
\[
  a(\bv,\bw)=\sum_{E\in\Th}\int_E\bv\cdot\bw\,d\bx
\]
but this is not computable in terms of the 
degrees of freedom. The following construction
is done in order to get a calculable discrete form and
also we will introduce a term for the
stability.\\[4pt]
For each element $E\in\Th$, let the space $W(E)$
be defined by
\begin{IEEEeqnarray*}{rCl}
  W(E) & = & \left\{ \bw\in V_h(E):  \bw = \nabla  q_2,
\mbox{ for some }  q_2\in P_2(E)\right\}\\[5pt]
       & = & V_h(E)\cap\nabla {P}_2(E)\mbox{,}
\end{IEEEeqnarray*}
and with these we consider $W(\Th) = \{{\bw} : {\bw}|_{E} \in W(E)
\mbox{ for each }E\in\Th\}$. Inspecting arbitrary elements
in~(\ref{d1p1}) and~(\ref{p03}) yields the following result.
\begin{lemma} When $E\in\Th$ is a tetrahedron or 
a prism, then $W(E) = V_{\textit{h}}(E)$.  
\end{lemma}
Again, observe that if $\bv\in V_h(E)$ and $\bw\in W(E)$, $a^E(\bw,\bv)$ can be 
computed using the degrees of freedom of $\bv$, because it holds
\begin{IEEEeqnarray*}{rCCCl}
a^E(\bw,\bv) &:=& \int_E\bw\cdot\bv\,d\bx &=& \int_E\nabla q_2\cdot\bv\,d\bx\\
                 &=&-\int_E q_2 \dv\bv\,d\bx &+& \iint_{\partial E} q_2\bv\cdot\bn\,dS.
\end{IEEEeqnarray*}
This means we can consider an auxiliary projection  operator $\pi^E$
from $H(\Div,E)$ onto $W(E)$ defined by
\begin{IEEEeqnarray}{rCl}\label{projection}
  a^E(\bv-\pi^E\bv,\bw) = 0 &\qquad& \mbox{for all }\bw\in W(E)
\end{IEEEeqnarray}
fully computable in terms of the degrees of freedom of $\bv$.

We need the following last object to complete the discretization of $a(\cdot,\cdot)$.
Thanks to Lemma~\ref{unisolvence} we can consider a basis $B=\{\bv_{i,E}\}$
of $V_h(E)$ dual to the functionals~(\ref{dofs}), that is, for every
$1\leqslant i,j\leqslant n_{f,E}$
\begin{IEEEeqnarray}{rCl}
  \iint_{f_j} \bv_{i,E}\cdot\bn_j\,dS & = & \delta_{i,j}.
\end{IEEEeqnarray}
What we are considering is the inner product
$\langle(\bv)_B,(\bw)_B\rangle$ of the coordinates
of the fields $\bv,\bw\in V_h(E)$ with respect 
to this local dual basis $B$.

Finally, the local discrete form $a_h^E$ is stated in the following definition.
\begin{defi} Given an element $E\in\Th$, for $\bv$ and $\bw$ in $V_{\textit{h}}(E)$
\begin{IEEEeqnarray}{rCl}\label{discreteLocal_a}
  a^E_{\textit{h}}(\bv,\bw) &:=& a^E(\pi^E\bv,\pi^E\bw) + 
  h_E^{-1}\langle(\bv-\pi^E\bv)_B,(\bw-\pi^E\bw)_B\rangle
\end{IEEEeqnarray}  
where $h_E$ is the diameter of $E$.
The discrete global form $a_h$ is defined by the following equation.
\begin{IEEEeqnarray}{rCl} \label{discreteGlobal_a}
  a_h(\bv,\bw) & = & \sum_{E\in \Th} a_h^E(\bv|_E,\bw|_E),
    \quad\mbox{ for } \bv \mbox{ and } \bw \in V_h.
\end{IEEEeqnarray}
\end{defi}
One fact to note
about the projection $\pi^E$ in~(\ref{projection}) is that, by
Proposition~\ref{vem_equal_fem},
it coincides with the identity $I:W(E)\to W(E)$ whenever the element $E\in\Th$
is a Prism or a Tetrahedron so that, immediately, we have the following
property.
\begin{remark}\label{ah_equal_a} If $E$ is a tetrahedron or a prism, then
  $a_h^E(\bv,\bw)=a^E(\bv,\bw)$ for all $\bv,\bw \in V_h(E)$.
\end{remark}
Now we state a key result concerning the stability of the
discrete form $a^E_{\textit{h}}$ and which explains the form of the
term $h_E^{-1}\langle\,\cdot\,,\cdot\,\rangle$ in~(\ref{discreteLocal_a}).
In what follows we analyse this term in expression~(\ref{discreteLocal_a})
when $E$ is a pyramid.
Let $B_E=\{\bv_i\,:\,\,1\leqslant i\leqslant 5\}$  denote the dual basis of the degrees of freedom~\eqref{dofs}
%, that is,
%\[
%\bv_i\in V_h(E), \qquad \mbox{and}\qquad \int_{f_j}\bv_i\cdot\bn =\delta_{i,j}, \quad 1\leqslant i,j\leqslant 5,
%\]
and let $f_j$, $1\leqslant j\leqslant 5$, denote the faces of $E$. Then 
$\bv\in V_h(E)$ has a unique expression as
\begin{IEEEeqnarray}{rCcCl}\label{auxlabel2}
\bv & = & \sum_{i=1}^5 a_i \bv_i
& = &
\sum_{i=1}^5\left\{\iint_{f_i}\bv\cdot\bn\,dS\right\}\bv_i.
\end{IEEEeqnarray}
Let $\vertiii{\cdot}_E$ denote the following norm on $V_h(E)$
\begin{IEEEeqnarray}{rCl}\label{auxlabel}
  \vertiii{\bv}_E^2 & := & \frac1{h_E}\sum_{i=1}^5 a_i^2.
\end{IEEEeqnarray}
Since the space is finite dimensional, for certain constants $c_E$ and $C_E$ we have
\begin{equation}\label{equiv0}
c_E\|\bv\|_{L^2(E)}\leqslant \vertiii{\bv}_E\leqslant C_E\|\bv\|_{L^2(E)}\mbox{,} 
\end{equation}
for all $\bv\in V_h(E)$. Now the purpose of the next Proposition is to prove that $C_E$ and $c_E$ can be
taken depending only on the aspect ratio of $E$.% \textcolor{red}{% {\bf Assumption (A).} We assume that if $\mbox{diam\,}E=1$, constants $c$ and $C$ above depends only on the shape-regularity of $E$.% }
\begin{proposition}
\label{stabilizing_term}
Let $E$ be a pyramid, and consider the basis $B_E$ of
$V_h(E)$ in~(\ref{auxlabel2}), and the associated discrete norm $\vertiii{\cdot}_E$ introduced
in~(\ref{auxlabel}). Then there exist constants $C_E$ and $c_E$ depending only on the aspect
ratio of $E$ such that~\eqref{equiv0} holds true for all $\bv\in V_h$.
\end{proposition}
\begin{proof} First we note that if $\bv\in V_h(E)$ is given by $\bv = \sum_{i=1}^5 a_i\bv_i$
then there is a function $\phi$ of zero mean over $E$ for which $\bv = \nabla\phi$ and also
$\Delta\phi = d$ on $E$, $\frac{\partial\phi}{\partial\bn}=g$ on $\partial E$
for
\begin{equation}\label{ai}
g|_{f_i}=\frac{a_i}{|f_i|},\quad 1\leqslant i\leqslant 5, \qquad |E|\,d=\sum_{i=1}^5a_i.
\end{equation}
Given $q\in H^1(E)$, integrating by parts $\Delta\phi$ times $q$ gives
\[
\int_E\nabla\phi\cdot\nabla q\,d\bx = 
-\int_Edq\,d\bx + 
\int_{\partial E}gq\,dS.
\]
Let $\overline{q}$ denote the mean of $q$ over $E$. By the last expression,
as $d$ is constant, we have
\begin{IEEEeqnarray*}{rCcCl}
\|\bv\|_{L^2(E)} & = & \|\nabla\phi\|_{L^2(E)}  & = &  
\sup_{\stackrel{q\in H^1(E),\|\nabla q\|_{0,E}=1\,}{\overline{q}=0}}
\int_{E}\nabla\phi\cdot\nabla q\,d\bx \\[5pt]
\yesnumber\label{equi0}
&&&=&\sup_{\stackrel{q\in H^1(E),\|\nabla q\|_{0,E}=1}{\overline{q}=0}}
\int_{\partial E}gq\,dS.\quad
\end{IEEEeqnarray*}
As a consequence of the definition of $d$ and $g$ in~(\ref{ai}), we have 
obtained an
expression of the norm $\|\bv\|_{L^2(E)}$ in terms of the coefficients of $\bv$
with respecto to basis $B_E$.
Now let $\hat E$ be a fixed reference pyramid, and let $F:\hat E\to E$ be an
affine mapping from $\hat E$ onto $E$ (remember that only pyramids with
parallelogram basis are considered), which can be written as
\[
{\bx}=F(\hat{\bx})=M_E\,\hat{\bx}+{\bf c}.
\] 
Given $q\in H^1(E)$ let $\hat q := q \circ F$. There 
exists constants $c_0$ and $c_1$ depending only on the
aspect ratio of $E$ such that
\begin{equation}\label{a1}
\frac{c_0}{h_E}\|\nabla q\|_{L^2(E)}^2\leqslant \|\nabla\hat q\|_{L^2(\hat E)}^2\leqslant
\frac{c_1}{h_E}\|\nabla q\|_{L^2(E)}^2,
\end{equation}
and, on the other hand, it holds
\[
\int_{\hat E}\hat q\,d\hat\bx = 0 \quad \Longleftrightarrow \quad \int_Eq\,d\bx=0.
\]
Now setting $\left|J(\hat{\bx})\right|_{f_i}|= |f_i|/|\hat f_i|\sim h_E^2$
we have
\[
\int_{\partial E}gq\,dS = \|\nabla \hat q\|_{0,E}
\left(\int_{\partial \hat E}\frac{\hat g\,\hat q\,|J|}{\|\nabla\hat q\|_{0,\hat E}}\,d\hat S
\right).
\]
It follows that
\begin{multline*}
\|\bv\|_{L^2(E)}=\sup\Bigg\{\|\nabla \hat q\|_{L^2(\hat E)}
\left(\int_{\partial \hat E}\frac{g|J|\hat q}{\|\nabla\hat q\|_{L^2(\hat E)}} 
\right):\\ \quad q\in H^1(E),\,\|\nabla q\|_{L^2(E)}=1,\,\overline{q}=0\Bigg\},
\end{multline*}
and taking~\eqref{a1} into account we obtain
% \begin{multline*}
% \|\bv\|_{L^2(E)}\leqslant \sup\Bigg\{\|\hat \nabla q\|_{L^2(\hat E)}\left(-\int_{\hat E}d|\det M_E|\frac{\hat q}{\|\nabla\hat q\|_{L^2(\hat E)}} + \int_{\partial \hat E}g|J|\frac{\hat q}{\|\nabla \hat q\|_{L^2(\hat E)}} \right):\\ \quad \hat q\in H^1(\hat E),\frac{c_0}{h_E^\frac12}\le\|\nabla \hat q\|_{L^2(\hat E)}\le\frac{c_1}{h_E^\frac12},\int_{\hat E}\hat q=0\quad\Bigg\}
% \end{multline*}
\begin{IEEEeqnarray*}{rCCl}
\|\bv\|_{L^2(E)} & \leqslant & c_1^{\nicefrac12}h_E^{-\nicefrac12}\sup&\Bigg\{
\int_{\partial \hat E}g|J|\hat q\,d\hat S:\\ 
\yesnumber\label{mult1}
&&&\quad\quad\hat q\in H^1(\hat E),\|\nabla \hat q\|_{L^2(\hat E)}=1,\overline{q}=0\Bigg\}\quad
\end{IEEEeqnarray*}
and 
\begin{IEEEeqnarray*}{rCCl}
\|\bv\|_{L^2(E)} & \geqslant & c_0^{\nicefrac12}h_E^{-\nicefrac12} \sup&\Bigg\{
\int_{\partial \hat E}g|J|\hat q \,d\hat S:\\
\yesnumber\label{mult2}
&&&\,\hat q\in H^1(\hat E),\|\nabla \hat q\|_{L^2(\hat E)}=1,\int_{\hat E} \hat q\,d\hat\bx=0\Bigg\}.\quad
\end{IEEEeqnarray*}
% and so
% \begin{multline*}
% \|\bv\|_{L^2(E)}\leqslant \frac{c_1}{h_E^\frac12}\sup\Bigg\{-\int_{\hat E}d|\det M_E|\hat q + \int_{\partial \hat E}g|J|\hat q:\\ \quad \hat q\in H^1(\hat E),\|\nabla \hat q\|_{L^2(\hat E)}=1,\int_{\hat E}\hat q=0\quad\Bigg\}.
% \end{multline*}
We remark that
\[
\int_{\hat E}d|\det M_E|\,d\hat\bx=\int_{\partial\hat E}g|J|\,d\hat S.
\]
Now, let $\{\hat\bv_i\}$ be the basis of $V_h(\hat E)$ which is dual to the degrees of
freedom over the faces of $\hat{E}$, and let
\begin{equation}\label{hatai}
\hat a_i=g|_{f_i}\,|J|_{f_i}|\,|\hat f_i|, \qquad 1\leqslant i\leqslant 5.
\end{equation}
Take $\hat\bv = \sum_{i=1}^5 \hat a_i\hat\bv_i$.
From~\eqref{equiv0} applied to the pyramid $\hat E$ we know that
\begin{equation}\label{equiv1}
c_{\hat E}^2\|\hat\bv\|_{L^2(\hat E)}^2\leqslant \frac1{h_{\hat E}}\sum_{i=1}^5
\hat a_i^2\leqslant C_{\hat E}^2\|\hat\bv\|_{L^2(\hat E)}^2\mbox{,}
\end{equation}
and using~\eqref{equi0} for $\hat E$ instead of $E$ we have %(with constants in the equivalence depending only on $\hat E$) %from our previous reasoning %(with $h_{\hat E}\sim 1$)
\begin{IEEEeqnarray*}{rCl}
\|\hat\bv\|_{L^2(\hat E)} & = & 
\sup_{\stackrel{\hat q\in H^1(\hat E),\|\nabla \hat q\|_{L^2(\hat E)}=1}{\int_{\hat E}\hat q\,=\,0}}
\int_{\partial \hat E}g|J|\hat q\,d\hat S.
\end{IEEEeqnarray*}
It follows from~\eqref{equiv1} that
\begin{IEEEeqnarray}{rCl}
  \nonumber
  \left(\frac1{h_{\hat E}}\sum_{i=1}^5 \hat a_i^2\right)^\frac12 & \sim &
  \sup_{\stackrel{\hat q\in H^1(\hat E),\|\nabla \hat q\|_{L^2(\hat E)}=1}
  {\int_{\hat E}\hat q\,=\,0}} 
  \int_{\partial \hat E}g|J|\hat q\,d\hat S
  \mbox{,}
  \\
  &&
\label{auxlabel3}
\end{IEEEeqnarray}
where the constants in this equivalence depend only on the aspect ratio of $E$
and so, since $\hat E$ can be considered with $h_{\hat E}\sim 1$, equation~\eqref{auxlabel3} together
with~\eqref{mult1} and~\eqref{mult2} give
\[
\frac1{h_E}\sum_{i=1}^5\hat a_i^2\sim \|\bv\|_{L^2(E)}^2.
\]
Using~\eqref{ai} and~\eqref{hatai},
\[
\hat a_i = a_i\frac{|J|_{f_i}|}{|f_i|}|\hat f_i|\sim a_i, \qquad 1\leqslant i\leqslant 5,
\]
and then
\[
\frac1{h_E}\sum_{i=1}^5a_i^2\sim \|\bv\|_{L^2(E)}^2
\]
as we wanted, since the constants in this equivalence also depend only on the aspect ratio of $E$.
\end{proof}
Proposition~(\ref{stabilizing_term}) yields the next corollary.
\begin{corollary}\label{equivalence} For all $\bv\in V_h(E)$ and all pyramidal
$E\in\Th$ it holds
\begin{IEEEeqnarray*}{rCcCl}
  c_E\,a^E(\bv,\bv) & \leqslant & h_E^{-1}\langle(\bv)_B,(\bv)_B\rangle & \leqslant
  & C_E\,a^E(\bv,\bv)
\end{IEEEeqnarray*}
where $c_E$ and $C_E$ depend only on the shape regularity of $E$.
\end{corollary}
\section{Discrete Problem}
The discrete problem in the Finite--Virtual Element scheme is stated as follows.
\begin{problem}\label{mixedDiscrete}
To find $\bu_{\Th}\in V_{\Th}$ and $p_{\Th}\in Q_{\Th}$ such that
$\forall \bv\in V_{\Th} \forall q\in Q_{\Th}$
  \begin{IEEEeqnarray*}{rCCCl}
    a_{\scriptscriptstyle\Th}(\bu_{\scriptscriptstyle\Th},\bv) & + &
      b_{\scriptscriptstyle\Th}(\bv,p_{\scriptscriptstyle\Th}) & = & 0 \\[5pt]
                    & - & b_{\scriptscriptstyle\Th}(\bu_{\scriptscriptstyle\Th},q) & = & (f,q).
  \end{IEEEeqnarray*}
\end{problem}
In Section~\ref{sec:well_posedness} we will be concerned in proving existence
and uniqueness for Problem~\ref{mixedDiscrete}. It will be done
showing coercivity for one of the bilinear forms and a
discrete version of the $\inf$--$\sup$ condition for the other.
\begin{lemma}\label{lemma_for_coercivity} For all $E\in\Th$, all $\bw\in W(E)$ and all $\bv\in V_h(E)$
\begin{IEEEeqnarray}{rCcCl} 
a_h^E(\bw,\bv) & = & a^E(\bw,\bv)       & &\label{L1}\\
c_E\,a^E(\bv,\bv)      & \leqslant & a_h^E(\bv,\bv) & \leqslant & C_E\,a^E(\bv,\bv)\label{L2}
\end{IEEEeqnarray}
with $c_E = C_E = 1$ when $E$ is tetrahedral or prismatic and $c_E,C_E$ depending
only on the shape regularity of $E$ when $E$ is pyramidal.
\end{lemma}
\begin{proof} The relation $\bw = \pi^E\bw$ for all $\bw\in W(E)$,
condition~\eqref{projection}
and the symmetry of $a^E$ imply
\[
  a_h^E(\bw,\bv)=a^E(\pi^E\bw,\pi^E\bv)=a^E(\pi^E\bw,\bv)=a^E(\bu,\bv)
\]
which proves~\eqref{L1}.
To prove~(\ref{L2}), if $E$ is not a Pyramid this property is already a consequence
of Remark~\ref{ah_equal_a}. In the case of a Pyramid, firstly a simple computation yields
\begin{IEEEeqnarray}{rCcCl}
  \label{comput}
  a^E(\bv,\bv)&=&a^E(\bv-\pi^E\bv,\bv-\pi^E\bv)&+&a^E(\pi^E\bv,\pi^E\bv).
\end{IEEEeqnarray}
Corollary~\ref{equivalence} and~(\ref{comput}) give
\begin{IEEEeqnarray*}{rCl}
a_h^E(\bv,\bv) &\leqslant& a^E(\pi^E\bv,\pi^E\bv) + C_E\,a^E(\bv-\pi^E\bv,\bv-\pi^E\bv) \\[5pt]
               &\leqslant& \max\{1,C_E\}\,a^E(\bv,\bv).
\end{IEEEeqnarray*}
In a similar manner, it holds
\[
  a_h^E(\bv,\bv)\geqslant \min\{1,c_E\}\,a^E(\bv,\bv).
\]
\end{proof}
\section{A Projection Space $W(E)$}
Since it holds clearly that $W(E)=V_h(E)$ when $E$ is a tetrahedron or a prism,
the purpose of this Section is to characterize $W(E)$ only when $E$ is a pyramid.
The Section finishes with some computational insights. 
\begin{lemma}\label{L3}
Let $\hat P$ be the reference pyramid 
of Definition~\ref{defi_of_ref_pyr}.
Then if $\bv \in\mathcal P_1(\hat P)^3$ verifies $\bv \cdot \bn =0$ on $\hat f_1$, $\hat f_2$, $\hat f_3$ and
$\hat f_5$, then $\bv({\bf x})=(0,cx_2,0)$ with $c$ constant.
\end{lemma}
\begin{proof}
Using that $\bv\cdot\bn$ is constant on $\hat f_1$, $\hat f_2$ and
$\hat f_5$ we obtain that $\bv$ is of the form
\[
\bv({\bf x}) = (a_1+b_1x_1, a_2+c_2x_2,a_3+d_3x_3),
\]
and since $\bv\cdot\bn=0$ on those faces it result
$a_1=a_2=a_3=0$. Now, using that $\bv\cdot\bn=0$ on
$\hat f_3$  we have that $b_1x_1+d_3x_3|_{\hat f_3}=d_3+(b_1-d_3)x_1|_{\hat f_3}=0$,
so $b_1=d_3=0$. Then $\bv=(0,c_2x_2,0)$ as we wanted to prove.
\end{proof}

\begin{lemma}\label{L4}
Let $P$ be a pyramid. Then $\mbox{dim\,}W(P)\le4$.
\end{lemma}
\begin{proof}
We have $W(P)\subseteq V(P)$ and $\mbox{dim\,}V(P)=5$. In order to prove that $W(P)\ne
V(P)$ we will show that there exists no field
$\bv=\nabla q_2$ with $q_2\in \mathcal P_2(P)$ with normal
component vanishing on four faces of $P$ and being constant different from $0$ on
the another face.

Let $\hat P$ be the reference pyramid of Lemma \ref{L3} and
use the same notation for the faces. Let $F(\hat {\bf x})=B\hat
{\bf x} + {\bf b}$ be an affine map from $\hat P$ onto $P$ and we denote
$f_i=F(\hat f_i)$. Suppose
that $\bv=\nabla q_2\in \mathcal P_2(P)$ is such that
$\bv\cdot\bn=0$ on $f_1$, $f_2$, $f_3$ and $f_5$, while
$\bv\cdot\bn=1$ on $f_4$. Now we consider $\hat\bv$
obtained via the Piola trasform from $\bv$, that is
\begin{equation}\label{piola}
\bv({\bf x}) = \frac1{|B|}B\hat\bv(\hat{\bf x}), \qquad {\bf x}=F(\hat {\bf x}),
\end{equation}
which is in $\mathcal P_1(\hat P)^3$. Using properties of the Piola transform
\cite[pages 12--14]{ricardoMixed} we have for $i=1,2,3,5$,
\[
\int_{\hat f_i}\hat \bv\cdot {\bf n}\hat\phi = \int_{f_i}\bv\cdot {\bf n}\phi =0\qquad \forall \phi\in\mathcal P_1(f_i).
\]
with $\hat \phi = \phi\circ F$. Since $\hat\bv|_{\hat f_i}\cdot {\bf n}\in\mathcal P_1(\hat f_i)$, this implies $\hat\bv|_{\hat f_i}\cdot {\bf n}=0$ for $i=1,2,3,5$. From the previous Lemma we obtain that
\[
\hat\bv(\hat{\bf x})=(0,c\hat x_2,0).
\]
Then
\[
\bv({\bf x})=\frac1{|B|}B(0,c\hat x_2,0)^t = \frac c{|B|}{\bf b}_2 \hat x_2
\]
if $B=[{\bf b}_1\,\,{\bf b}_2\,\,{\bf b}_3]$ (that s, ${\bf b}_i$, $i=1,2,3$,  are the columns of $B$). Then, on $f_4$ we have 
\[
\bv({\bf x})\cdot {\bf n}=\frac c{|B|}\hat x_2{\bf b}_2\cdot {\bf n} 
\]
and we note that $\hat x_2$ is not constant on $f_4$, it varies from $0$ to $1$,
and ${\bf b}_2\cdot {\bf n}\ne 0$, since ${\bf b}_2$ is a 
{\color{Orange}***************** transversal} 
vector to
the face $f_4$. Then, $\bv({\bf x})\cdot {\bf n}$ is not constant on $f_4$,
which contradicts our definition of $\bf v$.  
\end{proof}

\begin{proposition}
Let $P$ be a pyramid. Then $W(P)=\mathcal
P_0^3(P)+{\bf x}\mathcal P_0(P)$.
\end{proposition}
\begin{proof}
We have
\[
\mathcal P_0^3(P)+{\bf x}\mathcal P_0(P)\subseteq \hat
V(P).
\]
Since
\[
\mbox{dim\,}\left(\mathcal P_0^3(P)+{\bf x}\mathcal
P_0(P)\right)=4
\]
and that from Lemma \ref{L4}, $\mbox{dim\,}\hat
V(P)\le4$, we conclude the assertion.
\end{proof}

Given a field $\bv \in V_h(E)$, we can construct $\Pi_w^E\bv$
as follows. We choose a basis $\{\bw_i\}$ of $W(E)$, for
example,
\[
\left\{ (1,0,0), (0,1,0), (0,0,1), (x,y,z) \right\} =: \{
\bw_i: 1\leqslant i\leqslant 4\}
\]
with $\bw_i=\nabla q_i$, $i=1,2,3,4$. Then $a^E(\bv,\bw_i)$ is calculable from $\bv$'s degrees of freedom
\[
a^E(\bv,\bw_i) = \int_E\bv\cdot\nabla q_i =
-\int_E\mbox{div\,}\bv\,q_i + \int_{\partial E}\bv\cdot\bn\,q_i.
\]
Then if $\Pi_w^E\bv=\sum_{j=1}^4\alpha_j\bw_j$ we can
compute the coefficients $\alpha_j$ by solving the linear system
\[
\sum_{j=1}^4\alpha_j a^E(\bw_j,\bw_i) =
a^E(\bv,\bw_i), \qquad i=1,2,3,4.
\]
In order to compute the stabilization part of the discrete
bilinear form we need to write $\Pi_w^E\bv$ as a linear
combination of the basis $\{\bv_i\}$ of $V_h(E)$ associated with
the degrees of freedom, this is (always in the pyramidal case)
\[
\int_{f_i}\bv_j\cdot\bn =\delta_{ij}, \qquad 1\leqslant i,j\leqslant 5.
\]
In this case, we have $\Pi_w^E\bv = \sum_{i=1}^5\beta_i\bv_i$,
with
\[
\beta_i=\sum_{j=1}^4\alpha_j\int_{f_i}\bw_j\cdot\bn.
\]
\end{chapter}

\chapter{Local Interpolation}
We prove anisotropic local interpolation error estimates for two different operators
and use this result to estimate the global approximation error. Please recall 
the notation for the polynomial spaces given in Notation~\ref{auxlabel200}. We 
will alse stick to the notation and indices of Tables~\ref{prismNotationTableFaces}
and~\ref{prismNotationTableEdges}.


\section{Prismatic Finite Elements} % (fold)
\label{sec:prismatic_finite_elements}
\subsection{Stability of curl--conforming Finite Element on Prisms}
\label{stab_edge_prism}
\section{Stability of the Edge Element in $\hat{K}$}
\tikz \fill [orange] circle (2pt); {\color{red} CONTINUE HERE !!! }
\tikz \fill [orange] circle (2pt);
\begin{lemma}\label{lema_PIu3_k_cualquiera}
If $\hat\bu\in P_{\hat E}$ is such that $\hat{u}_3 = 0$,
then $(\hat{\boldsymbol{w}}_k\hat{\bu})_3 = 0$.
\end{lemma}
\begin{proof} Por definici\'on $(\pi\textbf{u})_3 =: v$ pertenece a $P_k(\hat{T})\otimes P_{k-1}(\hat{I})$, que tiene dimensi\'on 
$\frac{k(k+1)(k+2)}{2}$. Por otro lado, los grados de
libertad en los que no des\-a\-pa\-re\-ce $(\pi\textbf{u})_3$ son \'unicamente:
para cada arista $e_j$, $j = 3, 6, 7$,
\begin{IEEEeqnarray}{lClc}
	\label{aristas} \int\limits_{e_j} (\pi\textbf{u})_3 q \, ds 
	& = & \int\limits_{e_j} \textbf{u}_3 q \, ds, &\quad q\in P_{k-1}(e_j).
\end{IEEEeqnarray}
Para cada cara vertical $f=f_1$, $f_2$, $f_4$,
\begin{IEEEeqnarray}{lClc}
	\label{caras} \int\limits_{f} (\pi\textbf{u})_3 q \, 
	d\gamma & = & \int\limits_{f} \textbf{u}_3 q \, d\gamma, 
	&\quad q\in Q_{k-2,k-1}(f_{ijkl}).
\end{IEEEeqnarray}
En $\hat{K}$
\begin{IEEEeqnarray}{lClc}
	\label{enK} \int\limits_{\hat{K}} (\pi\textbf{u})_3 
	q \, d\textbf{x} & = & \int\limits_{\hat{K}} \textbf{u}_3 q \, d\textbf{x}, 
	&\quad q\in P_{k-3}(f_z) \otimes P_{k-1}(e_{xy}).
\end{IEEEeqnarray}
El n\'umero de grados de libertad es $3k$ en aristas m\'as $3k(k-1)$ en caras, m\'as $\frac{k(k-1)(k-2)}{2} $ en volumen, que es igual 
a $\frac{k(k+1)(k+2)}{2}$ (la cantidad correcta). 
Ahora supongamos que estos grados de libertad se anulan todos y veamos que de esta suposici\'on
se deduce que $(\pi\textbf{u})_3 = 0$, es decir que las ecuaciones li\-nea\-les mencionadas determinan un\'ivocamente a un elemento
$(\pi\textbf{u})_3$ pertenenciente a 
$P_k(\hat{T}) \otimes P_{k-1}(\hat{I})$. 

Empezamos por ver que $(\pi\textbf{u})_3$ se anula en ciertos subconjuntos de 
$\partial \hat{K} $. La restricci\'on de $(\pi\textbf{u})_3$ a $e_j, (j=3,6,7)$
es un elemento de $P_{k-1}(e_j)$. Entonces si usamos los degrees of freedom~(\ref{aristas}) obtenemos
\begin{IEEEeqnarray}{rCl}
	\label{restriccAristas}\int\limits_{e_j} [(\pi\textbf{u})_3]^2\, ds & = & 0,
\end{IEEEeqnarray}
es decir que $(\pi\textbf{u})_3$ es id\'enticamente cero en esas aristas. Con esto la restricci\'on $(\pi\textbf{u})_3|_{f_2}$ 
que pertenece a $P_k(x)\otimes P_{k-1}(z)$ se factoriza como 
\begin{IEEEeqnarray*}{rCl}
	v(x,0,z) 	& = 	& x\,(x-1)\,w_0(x,z),\\
	w_0 		& \in 	& P_{k-2}(x)\otimes P_{k-1}(z),
\end{IEEEeqnarray*}
pero este \'ultimo es precisamente el espacio de los degrees of freedom~(\ref{caras}) con los 
cuales llegamos a
\begin{IEEEeqnarray}{lClc}
	\int\limits_{f_2} x(x-1)[w_0(x,z)]^2\,d\gamma & = & 0.
\end{IEEEeqnarray}
Entonces en $f_2$ es $w_0 \equiv 0$ y $(\pi\textbf{u})_3|_{f_2} \equiv 0$. Por simetr\'ia, el argumento para probar que
$(\pi\textbf{u})_3|_{f_1} \equiv 0$ es exactamente igual.

Si volvemos a usar las igualdades~(\ref{restriccAristas}) obtenemos que la restricc\'on de $(\pi\textbf{u})_3$ a
$f_{4} = \{ (x,y,z) \;:\; 0\leqslant x\leqslant 1,\, y = 1 - x, \, 0\leqslant z\leqslant 1 \}$ se anula si $x=1-y=0$ o bien si $x=1-y=1$;
\begin{IEEEeqnarray}{rCl}
	v(x,1-x,z) 	& = 	& x\,(1-x)\,w_1(x,z)\\
	w_1 		& \in 	& P_{k-2}(x)\otimes P_{k-1}(z).
\end{IEEEeqnarray}
Si aplicamos los degrees of freedom~(\ref{caras}) llegamos 
a que en ${f_4}$ es $w_1$ id\'enticamente cero y as\'i tambi\'en $(\pi\textbf{u})_3$.\\
Recapitulando, 
\begin{IEEEeqnarray*}{rCl}
	(\pi\textbf{u})_3|_{f_j} & \equiv & 0\quad\text{ para } j = 1,2,4.
\end{IEEEeqnarray*}
Entonces en todo $\hat{K} $ tenemos la factorizaci\'on 
\begin{IEEEeqnarray*}{rCl}
	(\pi\textbf{u})_3(x,y,z) 	& = 	& x\,y\,(1-x-y)\,w_3(x,y,z),\\
						w_3		& \in 	& P_{k-3}(x)\otimes P_{k-1}(z).
\end{IEEEeqnarray*}
Simplemente aplicando los degrees of freedom~(\ref{enK}) nos queda que $w_3 \equiv 0$, y finalmente
$(\pi \textbf{u})_3 \equiv 0$.
%	debe anularse en todo $\hat{K}^{\textrm{o}}$, y por continuidad tambi\'en en 
%$\partial\hat{K}$. Si volvemos a la expresi\'on~(\ref{piu3}) nos queda que
\end{proof}
\begin{lemma}\label{lemma_PIu2_k_in_N}
\begin{IEEEeqnarray*}{rCl}
\label{piu2_k_in_N}
	\yesnumber\pi(0, u_2(y,z), 0) & = 	& (0, \xi_2(y,z) ,0)\textrm{,}\\
\label{piu1_k_in_N}	
	\yesnumber\pi(u_1(x,z), 0, 0) & = 	& (\xi_1(x,z), 0 ,0)\textrm{,}\\
	\xi_1 				& \in 	& P_{k-1}(x) \otimes P_k(z)\textrm{,}\\
	\xi_2 				& \in 	& P_{k-1}(y) \otimes P_k(z).
\end{IEEEeqnarray*}
\end{lemma}
\begin{proof} Demostramos s\'olo la primera igualdad, porque la otra es
an\'aloga. Lo que hay que ver es que, en la expresi\'on encontrada en
~(\ref{sub:elemento_P_k}), es $h \equiv 0$, $\xi_1 \equiv 0$ y que $\xi_2$ 
no depende de $x$. Gracias al lema~\ref{lema_PIu3_k_cualquiera} ya sabemos
que $\xi_3 \equiv 0$.
As\'{\i} que veamos primero $h \equiv 0$.
Observaci\'on: si $f$ es $f_3$ o $f_4$, entonces
\begin{IEEEeqnarray}{rCl}
	(\textbf{curl}\,\pi\textbf{u})_3 |_{_{f}} & \in & P_{k-1}(x,y).	
\end{IEEEeqnarray}
Por el Lema~\ref{lema_pi_star_rot_u}, si usamos los degrees of freedom~(\ref{momentos1hdiv})
obtenemos, para todo $q \in P_{k-1}(x,y)$,
\begin{IEEEeqnarray}{rCl}
	\int\limits_{f} (\textbf{curl}\,\pi\textbf{u})_3\,q \,d\gamma & = & 
		\int\limits_{f} \textrm{rot}(\textbf{u})_3\,q \,d\gamma\\
		& = & 0.	
\end{IEEEeqnarray}
Si tomamos $q = (\textbf{curl}\,\pi\textbf{u})_3 |_{_{f}}$ tenemos
$(\textbf{curl}\,\pi\textbf{u})_3 |_{_{f}} \equiv 0$, o, de otra manera,
$z\,(z-1)$ divide a $(\textbf{curl}\,\pi\textbf{u})_3$. Escribamos
\begin{IEEEeqnarray*}{rCl}
	(\textbf{curl}\,\pi\textbf{u})_3 & = 	& z\, (z-1)\, \psi\\[6pt]
	\psi						& \in 	& P_{k-1}(x,y) \otimes P_{k-2}(z).
\end{IEEEeqnarray*}
A continuaci\'on usamos los degrees of freedom~(\ref{momentos4hdiv}) en la misma 
definici\'on de antes. Para todo $q\in P_{k-1}(x,y) \otimes P_{k-2}(z)$
\begin{IEEEeqnarray*}{rCl}
	\int\limits_{\hat{K}} z\,(z-1)\,\psi \, q\,d\textbf{x} & = & 0,
\end{IEEEeqnarray*}
as\'{\i} que tomando $q = \psi$ probamos que $\psi \equiv 0$ en $\hat{K}$,
es decir que
\begin{IEEEeqnarray}{rCl}
	\label{rot_3_es_0} (\textbf{curl}\,\pi\textbf{u})_3 &\equiv& 0.
\end{IEEEeqnarray}
Ahora veamos c\'omo es $(\textbf{curl}\,\pi\textbf{u})_3$ en t\'erminos
de la expresi\'on encontrada en~\ref{sub:elemento_P_k}.
\begin{IEEEeqnarray*}{rCl}
	(\textbf{curl}\,\pi\textbf{u})_3 & = & 
	\partial_x\,\pi(\textbf{u})_2 -	\partial_y\,\pi(\textbf{u})_1\\
	\label{expre_h} \yesnumber & = & -(2\,h + y\,\partial_y\,h + 
				x\,\partial_x\,h) +	\partial_x\,\xi_2 - \partial_y\,\xi_1,
\end{IEEEeqnarray*}
en donde, observando los grados de cada t\'ermino,
\begin{IEEEeqnarray*}{rCl}
	2\,h + y\,\partial_y\,h + x\,\partial_x\,h & \in & \tilde{P}_{k-1}(x,y)
\otimes P_k(z)\textrm{ y }\\
\partial_x\,\xi_2 - \partial_y\,\xi_1 & \in & P_{k-2}(x,y) \otimes P_k(z).
\end{IEEEeqnarray*}
De esto necesariamente sigue que 
$g := 2\,h + y\,\partial_y\,h + x\,\partial_x\,h = 0$. Ahora exploramos los 
t\'erminos de $g$. Pongamos
\[
	h(x,y,z) = \sum\limits_{i+j=k-1, l\leqslant k} \alpha_{_{i,j,l}} x^i y^j z^l.
\]
Entonces, para todo $(x,y,z)$ en $\hat{K}$
\begin{IEEEeqnarray*}{rCl}
	g(x,y,z) & = & \sum\limits_{i+j=k-1, l\leqslant k} 
(2\alpha_{_{i,j,l}} + j \alpha_{_{i,j,l}} + i \alpha_{_{i,j,l}}) x^i y^j z^l\\
	& = &(k+1)\,h(x,y,z)\\
  \yesnumber\label{h_is_zero}	& = & 0,
\end{IEEEeqnarray*}
o sea, $h \equiv 0$.
Hasta ac\'a tenemos probado que $\pi(0, u_2(y,z), 0) = 
(\xi_1(x,y,z), \xi_2(x,y,z), 0)$. Let's see that $\xi_1$ vanishes identically. Empecemos con 
los degrees of freedom sobre las aristas. Si $e$ es $e_1$ o $e_4$ entonces
\[
	\xi_1|_{e} \in P_{k-1}(x)
\]
y adem\'as, para todo $q\in P_{k-1}(x)$
\[
	\int\limits_{e} \xi_1|_{e}\,q\,ds = 0\textrm{,}
\]
con lo que llegamos a que $z\,(z-1)$ divide tambi\'en a $\xi_1|_{f_2}$. Pongamos
\begin{IEEEeqnarray}{rCl}
	\xi_1|_{f_2}(x,z) &=&z\,(z-1)\phi(x,z)\\[6pt]
	\phi &\in& P_{k-1}(x)\otimes P_{k-2}(z).
\end{IEEEeqnarray}
A continuaci\'on, si usamos los degrees of freedom en $f_2$, tenemos, para todo 
$q\in P_{k-1}(x)\otimes P_{k-2}(z)$,
\begin{IEEEeqnarray}{rCl}
	\int\limits_{f_2} z\,(z-1)\phi\,q\,d\gamma &=&0.
\end{IEEEeqnarray}
Tomando $q=\phi$ sigue que $\xi_1|_{f_2}\equiv 0$, con lo cual, para $(x,y,z)
\in \hat{K}$
\begin{IEEEeqnarray}{rCl}
	\xi_1(x,y,z) & = & y\,\rho(x,y,z)\\[6pt]
	\rho &\in&P_{k-2}(x,y)\otimes P_k(z).	
\end{IEEEeqnarray}
Ahora miramos los degrees of freedom en $f = f_3, f_4$. Vale que $\xi_1|_{f}$ pertenece 
a $P_{k-2}(x,y)$, y que para todo $q\in P_{k-2}(x,y)$
\[
	\int\limits_{f} \xi_1\,q\,d\gamma = 0\textrm{,}
\]
de donde, por tomar $q = \xi_1|_{f}$, sigue que $\xi_1|_{f} \equiv 0$. Con todo
esto tenemos que $z\,(z-1)$ divide a $\xi_1$, as\'{\i} que
\begin{IEEEeqnarray*}{rCl}
	\xi_1(x,y,x) &=&y\,z\,(z-1)\,\theta(x,y,z)\\[6pt]
	\theta &\in& P_{k-2}(x,y)\otimes P_{k-2}(z).
\end{IEEEeqnarray*}
Resta mirar los degrees of freedom de volumen aplicados a $\pi(\textbf{u})$. Para 
todo $q \in  P_{k-2}(x,y)\otimes P_{k-2}(z)$ debe ser
\begin{IEEEeqnarray*}{rCl}
\int\limits_{\hat{K}} y\,z\,(z-1)\,\theta(x,y,z)\,q\,d\textbf{x} &=&0
\textrm{,} 
\end{IEEEeqnarray*}
de donde, al tomar $q = \theta$, sigue inmediatamente que, en todo $\hat{K}$,
\begin{IEEEeqnarray}{rCl}
\label{xi1_es_0}\xi_1 & \equiv & 0.
\end{IEEEeqnarray}
Hasta ahora probamos que  
\[
	\pi(0, u_2(y,z), 0)  = 	 (0, \xi_2(x,y,z) ,0).
\]
Pero si recordamos lo que probamos en~(\ref{rot_3_es_0}) y lo combinamos con
~(\ref{xi1_es_0}), inmediatamente llegamos a 
\begin{IEEEeqnarray*}{rCCCl}
	\partial_x \pi(\textbf{u})_2 &=& \partial_x \xi_2 &\equiv& 0\textrm{,}
\end{IEEEeqnarray*}
que implica que $\xi_2$ no depende de $x$, y que $\pi(0, u_2(y,z), 0)$ tiene 
la forma que quer\'{\i}amos.
\end{proof}
\begin{lemma}\label{pi00u3} Si $\pi$ es el interpolador determinado por el elemento de la
\emph{Definici\'on}~(\ref{edgeelement}), entonces
\begin{IEEEeqnarray}{rCl}
	\pi(0,0, u_3)& = & (0,0,\xi_3)\textrm{,}\\
	\nonumber		\xi_3 & \in & \p{k}{k-1}.
\end{IEEEeqnarray}
\end{lemma}
\begin{proof} La demostraci\'on ser\'a muy parecida a la del Lema~(\ref{lemma_PIu2_k_in_N}). Recordemos
la expresi\'on que encontramos en la Subsecci\'on~\ref{sub:elemento_P_k}; tenemos
\begin{IEEEeqnarray}{rCl}
	\label{expre_pi00u3} \pi(0,0,u_3) &=& (\xi_1 + y\,h,\xi_2 - x\,h, \xi_3).	
\end{IEEEeqnarray}
Empezamos por ver que $h$ se anula. Sea $f$ cualquiera de las dos caras horizontales del 
prisma de referencia. Usamos la expresi\'on~(\ref{expre_h}) para 
$(\textbf{curl}\,\pi\textbf{u})_3 \in \p{k-1}{k}$. Gracias al
Lema~\ref{lema_pi_star_rot_u} y a las igualdades~(\ref{momentos1hdiv}) vale
\begin{IEEEeqnarray}{rCl}
	(\textbf{curl}\,\pi\textbf{u})_3 |_{f}&\equiv&0.
\end{IEEEeqnarray}
Entonces los polinomios $z$ y $z-1$ dividen a $(\textbf{curl}\,\pi\textbf{u})_3=
z\,(z-1)\,\psi$, ($\psi\in P_{k-1}(x,y)\otimes P_{k-2}(z)$). Ahora vamos a los
degrees of freedom~(\ref{momentos4hdiv}) para obtener finalmente que $\psi$ es id\'enticamente cero. Es decir
que, en todo $\hat{K}$, $(\textbf{curl}\,\pi\textbf{u})_3 \equiv 0$ y, siguiendo
el argumento en la demostraci\'on del Lema~\ref{lemma_PIu2_k_in_N} probamos que $h$ es id\'enticamente
cero. As\'{\i} que podemos reescribir la expresi\'on~(\ref{expre_pi00u3}) y poner
\begin{IEEEeqnarray}{rCl}
	\label{expre_pi00u3_} \pi(0,0,u_3) &=& (\xi_1,\xi_2, \xi_3)\\
	\nonumber\xi_1, \xi_2&\in& \p{k-1}{k}.
\end{IEEEeqnarray}
Resta ver que $\xi_1$ y $\xi_2$ se anulan. Con la misma idea, si consideramos
las caras $f_1$ con normal $\boldsymbol{\nu}=(-1,0,0) $ y $f_2$ con normal 
$\boldsymbol{\nu}=(0,-1,0)$, entonces las igualdades~(\ref{momentos1hcurl}) 
para las aristas $e_1, e_4$ junto con las igualdades~(\ref{momentos4hcurl}) por un lado,
y las igualdades~(\ref{momentos1hcurl}) para las aristas $e_2, e_5$ y~(\ref{momentos3hcurl})
por el otro, implican, respectivamente, que $y$ divide a $\xi_1$ y $x$ divide a $\xi_2$. Si 
continuamos con los degrees of freedom sobre las dos caras horizontales~(\ref{momentos2hcurl})
obtenemos tambi\'en que $z\,(z-1)$ los divide a ambos. Finalmente, si aplicamos las
igualdades~(\ref{momentos6hcurl}) probamos que $\xi_1 = \xi_2 \equiv 0$ en todo $\hat{K}$.
En conclusi\'on, $\pi(0,0, u_3)$ tiene la forma deseada.
\end{proof}
\begin{theorem}\label{thm_stab_edge}
Dados $p > 2$, $\hat{\emph{\textbf{u}}} \in \wpcurl{\hat{K}}$, si $\hat{\pi}$ es 
el operador de interpolaci\'on determinado por el elemento de la Definici\'on~(\ref{edgeelement}),
entonces
\begin{IEEEeqnarray}{rCl}
\label{teorema_1} \norm{\hat{\pi}(\hat{\emph{\textbf{u}}})_1}_{L^{\infty}(\hat{K})} & 
	\lesssim & \|\hat{u}_1\|_{W^{1,p}(\hat{K})} + 
	\|\emph{\textbf{curl}}(\hat{\emph{\textbf{u}}})_3\|_{{\color{red} W^{1,1}(\hat{K})}} \\	
\label{teorema_2} \norm{\hat{\pi}(\hat{\emph{\textbf{u}}})_2}_{L^{\infty}(\hat{K})} & 
	\lesssim & \|\hat{u}_2\|_{W^{1,p}(\hat{K})} + 
	\|\emph{\textbf{curl}}(\hat{\emph{\textbf{u}}})_3\|_{{\color{red} W^{1,1}(\hat{K})}} \\	
\label{teorema_3} \norm{\hat{\pi}(\hat{\emph{\textbf{u}}})_3}_{L^{\infty}(\hat{K})} & 
	\lesssim & \|\hat{u}_3\|_{W^{1,p}}(\hat{K}).
\end{IEEEeqnarray}
\end{theorem}
\begin{proof}
The proof will rely on the three previous Lemmas, 
the triangular inequality applied on each component of 
expression~(\ref{edge_interp_explicit}) and traces inequalities.
Tomamos un campo suave $\hat{\textbf{u}}$ definido en $\hat{K}$ y, por el Lema~(\ref{lemaDensidad}),
la demostración concluirá con un argumento de densidad.\\[5pt]


La tercera desigualdad sigue del lema~(\ref{lema_PIu3_k_cualquiera}) que demuestra la un\'{\i}voca
determinaci\'on de $\hat{\pi}(\hat{\textbf{u}})_3$ por parte de $\hat{u}_3$.

One thing to note is that, of course, inequality~(\ref{teorema_3}) implies conversely 
the statement of Lemma~(\ref{lema_PIu3_k_cualquiera}). We wrote this in the present order
because we wanted all the previuos Lemmas~(\ref{lema_PIu3_k_cualquiera})--(\ref{pi00u3})
before Theorem~(\ref{thm_stab_edge}).

Las dos primeras se 
demuestran de forma parecida; probamos~(\ref{teorema_1}). La idea es tomar una funci\'on otra,
$\hat{\textbf{w}}$, tal que su interpolada tenga igual primera coordenada que la de 
$\hat{\textbf{u}}$,
pero que tenga degrees of freedom m\'as c\'omodos de acotar en t\'erminos exclusivamente de 
$\hat{u}_1$ y $\textbf{curl}(\hat{\textbf{u}})_3$.
Definimos, dada $\hat{\textbf{u}} \in W^{1,p}(\hat{K})^3$,
\begin{IEEEeqnarray*}{rCl}
	\hat{\textbf{w}} & = & (\hat{u}_1, \hat{u}_2 - \hat{u}_2(0,y,z), 0).
\end{IEEEeqnarray*}
Como dec\'{\i}amos, observemos primero que gracias a los lemas~(\ref{lemma_PIu2_k_in_N}) 
y~(\ref{pi00u3}) es
\begin{IEEEeqnarray*}{rCl}
	\hat{\pi}(\hat{\textbf{w}})_1 & = & (\hat{\pi}\hat{\textbf{u}})_1 - 
	\hat{\pi}(0, \hat{u}_2(0,y,z), 0)_1 -
	\hat{\pi}(0, 0, \hat{u}_3)_1\\
						& = & (\hat{\pi}\hat{\textbf{u}})_1.
\end{IEEEeqnarray*}
Ahora exploremos uno por uno los degrees of freedom que definen a $\pi(\hat{\textbf{w}})$. Los \'unicos
degrees of freedom sobre aristas que no se anulan o que no dependen expl\'{\i}citamente s\'olo de $\textrm{u}_1$
son
\begin{IEEEeqnarray*}{rCl}
	\int\limits_{e_8} \hat{\textbf{w}} \cdot \boldsymbol{\tau}\,\phi\,ds & = &
	\tfrac{1}{\sqrt{2}} \int\limits_{e_8} (w_1 - w_2)\,\phi\,ds\\
	\int\limits_{e_9} \hat{\textbf{w}} \cdot \boldsymbol{\tau}\,\phi\,ds & = &
	\tfrac{1}{\sqrt{2}} \int\limits_{e_9} (w_1 - w_2)\,\phi\,ds
\end{IEEEeqnarray*}
para $q\in \mathcal{P}_{k-1}$. Para el momento sobre $e_8$ hacemos partes en la cara $f_4
\subseteq \{ z=1 \}$. Tomemos un polinomio $\phi \in \mathcal{P}_{k-1}$ sobre $e_8$, que puede
verse como $\phi(y)$, dado que en $e_8$ es $x = 1 - y$.
\begin{IEEEeqnarray*}{rCl}
	\int\limits_{f_4} \textbf{curl}(\hat{\textbf{u}})_3\,\phi\,d\gamma & = &
	\int\limits_{f_4} \textbf{curl}(\hat{\textbf{w}})_3\,\phi\,d\gamma\\
	& = & -\int\limits_{f_4} \left(w_2\,\partial_x\phi - w_1\,\partial_y\phi\right)\,d\gamma
		+ \int\limits_{\partial f_4} \left(w_2\,\nu_x - w_1\,\nu_y\right)\,\phi\,ds\\
	& = &  \int\limits_{f_4} w_1\,\partial_y\phi\,d\gamma
		+ \int\limits_{e_8} \left(w_2 - w_1\right)\,\phi\,ds + 
			\int\limits_{e_4} w_1\,\phi\,ds.
\end{IEEEeqnarray*}
Es decir
\begin{IEEEeqnarray}{rCl}\label{momentosWaristas}
	\tfrac{1}{\sqrt{2}} \int\limits_{e_8} (w_1 - w_2)\,\phi\,ds & = &
		 \tfrac{1}{\sqrt{2}} \int\limits_{e_4} u_1\,\phi\,ds - 
		 \tfrac{1}{\sqrt{2}} \int\limits_{f_4} \textbf{curl}(\hat{\textbf{u}})_3\,\phi\,d\gamma
		 + \int\limits_{f_3} u_1\,\partial_y\phi\,d\gamma.
\end{IEEEeqnarray}
De igual manera hacemos partes en $f_3 \subseteq \{ z=0 \}$ para obtener
\begin{IEEEeqnarray}{rCl}\label{momentosWaristas2}
	\tfrac{1}{\sqrt{2}} \int\limits_{e_9} (w_1 - w_2)\,\phi\,ds & = &
		 \tfrac{1}{\sqrt{2}} \int\limits_{e_1} u_1\,\phi\,ds - 
		 \tfrac{1}{\sqrt{2}} \int\limits_{f_3} \textbf{curl}(\hat{\textbf{u}})_3\,\phi\,d\gamma
		 + \int\limits_{f_3} u_1\,\partial_y\phi\,d\gamma.
\end{IEEEeqnarray}
Ahora miramos los degrees of freedom sobre las caras. Tambi\'en aqu\'{\i} hacemos notar que s\'olo hace falta
trabajar con los degrees of freedom sobre las dos caras horizontales $f_3$ y $f_4$ y sobre la cara $f_5$.
Tomemos $\phi_1$, $\phi_2 \in \mathcal{P}_{k-2}(x,y)$, $\boldsymbol{\phi} = (\phi_1, \phi_2, 0)$.
\begin{IEEEeqnarray}{rCl}
 	\label{cotaf3}\int\limits_{f_3} \hat{\textbf{w}} \times \boldsymbol{\nu} \cdot \boldsymbol{\phi}\,d\gamma
 		& = & \int\limits_{f_3} u_1\,\phi_2\,d\gamma + \int\limits_{f_3} w_2\,\phi_1\,d\gamma.
\end{IEEEeqnarray}
Ahora sea un polinomio $\varphi_1 \in \mathcal{P}_{k-1}(x,y) $ tal que 
\begin{IEEEeqnarray*}{rCl}
	\partial_x \varphi_1 & = & \phi_1\textrm{,}\\
	\varphi_1 |_{e_9} 	 & \equiv & 0\textrm{;}
\end{IEEEeqnarray*}
tomemos por ejemplo $\varphi_1 = -\int_{x}^{1-y} \phi_1(t,y)\,dt$. Entonces
\begin{IEEEeqnarray*}{rCl}
	\int\limits_{f_3} \textbf{curl}(\hat{\textbf{u}})_3\,\varphi_1\,d\gamma & = & -\int\limits_{f_3} \left(w_2\,\phi_1 - w_1\,\partial_y\varphi_1\right)\,d\gamma
		+ \int\limits_{e_1} w_1\,\nu_y\,\varphi_1\,ds,
\end{IEEEeqnarray*}
y de esto con junto con~(\ref{cotaf3}) sigue que
\begin{IEEEeqnarray}{rCl}\label{momentosWcaras}
 	\nonumber\int\limits_{f_3} \hat{\textbf{w}} \times \boldsymbol{\nu} \cdot \boldsymbol{\phi}\,d\gamma
 		& = & \int\limits_{f_3} u_1\,\phi_2\,d\gamma - \int\limits_{f_3} \textbf{curl}(\hat{\textbf{u}})_3\,\varphi_1\,d\gamma\\
 		& 	& + \int\limits_{f_3} u_1\,\partial_y\varphi_1\,d\gamma	+ \int\limits_{e_1} u_1\,\nu_y\,\varphi_1\,ds.
\end{IEEEeqnarray}
Si repetimos lo anterior para el momento en $f_4$, obtenemos 
\begin{IEEEeqnarray}{rCl}\label{momentosWcaras2}
 	\nonumber\int\limits_{f_4} \hat{\textbf{w}} \times \boldsymbol{\nu} \cdot \boldsymbol{\phi}\,d\gamma
 		& = & \int\limits_{f_4} u_1\,\phi_2\,d\gamma - \int\limits_{f_4} \textbf{curl}(\hat{\textbf{u}})_3\,\varphi_1\,d\gamma\\
 		& 	& + \int\limits_{f_4} u_1\,\partial_y\varphi_1\,d\gamma	+ \int\limits_{e_4} u_1\,\nu_y\,\varphi_1\,ds\textrm{,}
\end{IEEEeqnarray}
y con la misma técnica para $f_5$, si dado $\textbf{q} = (0,q_3,q_1) \in \{ 0 \} \times Q_{k-2,k-1} \times Q_{k-1,k-2}$ 
tomamos $\varphi_1(x,z)=-\int\limits_{x}^{1} q_1(t,z)\,dt$, entonces obtenemos
\begin{IEEEeqnarray}{rCl}\label{momentosWcaras3}
 	\nonumber\int\limits_{f_4} \hat{\textbf{w}} \times \boldsymbol{\nu} \cdot \textbf{q}\,d\gamma
 		& = & \int\limits_{f_5} u_1\,q_1\,d\gamma + \int\limits_{f_5} \textbf{curl}(\hat{\textbf{u}})_3\,\varphi_1\,d\gamma\\
 		& 	& - \int\limits_{f_5} u_1\,\partial_y\varphi_1\,d\gamma	+ \int\limits_{e_4} u_1\,\varphi_1\,ds.
\end{IEEEeqnarray}
Finalmente, estudiamos los degrees of freedom de volumen. Tomemos
$\boldsymbol{\phi} = (\phi_1, \phi_2, \phi_3) \in (P_{k-2}(x,y) \otimes P_{k-2}(z))^{\color{red}2}
\times P_{k-3}(x,y) \otimes
P_{k-1}(z)$ (cfr.~(\ref{momentos6hcurl})) y sea $\varphi_2 = - \int\limits_{x}^{1-y} \phi_2(t,y,z)\,dt$, para el cual vale 
$\partial_x\varphi_2 = \phi_2 $ y $\varphi_2|_{f_5} \equiv 0$. Entonces, por partes,
\begin{IEEEeqnarray}{rCl}\label{momentosWvolumen}
 	\nonumber\int\limits_{\hat{K}} \hat{\textbf{w}} \cdot \boldsymbol{\phi}\,d\textbf{x}
 		& = & \int\limits_{\hat{K}} u_1\,\phi_1\,d\textbf{x} -\int\limits_{\hat{K}} \textbf{curl}(\hat{\textbf{u}})_3\,
 		\varphi_2,d\textbf{x}\\
 		& 	& + \int\limits_{\hat{K}} u_1\,\partial_y\varphi_2\,d\textbf{x}	+ 
 		\int\limits_{f_2} u_1\,\varphi_2\,d\gamma.
\end{IEEEeqnarray} %,~(\ref{momentosWcaras2}),~(\ref{momentosWcaras3})
Ahora recolectamos lo que dicen las igualdades~(\ref{momentosWaristas}),~(\ref{momentosWaristas2}),
y~(\ref{momentosWcaras})--(\ref{momentosWvolumen}), para hacer simplemente
%\begin{IEEEeqnarray*}{rCl}
%	\|\hat{\pi}(\hat{\textbf{w}})\|_{L^\infty(\hat{K})} & = 
%	 & 
%	 & \leqslant & C \left(\|\hat{u}_1\|_{W^{1,p}(\hat{K})} +
%		\|{\textbf{curl}}({\hat{\textbf{u}}})_3\|_{W^{1,1}(\hat{K})}\right)
%\end{IEEEeqnarray*}
\begin{IEEEeqnarray*}{rCl}
	\|(\hat{\pi}\hat{\textbf{u}})_1\|_{L^\infty(\hat{K})} & = 
	&\|(\hat{\pi}\hat{\textbf{w}})_1\|_{L^\infty(\hat{K})}\\
	& \leqslant & C \left\|\sum F_{i}(\hat{\textbf{w}}, \textbf{q}_i)\,\hat{\textbf{v}}_i\right\|_{L^\infty(\hat{K})}\\
	& \leqslant & C \sum \left|F_{i}(\hat{\textbf{w}}, \textbf{q}_i)\right|\,
		\left\|\hat{\textbf{v}}_i\right\|_{L^\infty(\hat{K})}\\
	& \leqslant & C \left(\|\hat{u}_1\|_{W^{1,p}(\hat{K})} +
		\|{\textbf{curl}}({\hat{\textbf{u}}})_3\|_{{\color{red} W^{1,1}(\hat{K})}}\right),
\end{IEEEeqnarray*}
que es la desigualdad a la que quer\'iamos llegar. \tikz \fill[orange] circle (2pt);
\end{proof}
\noindent the next step is to estimate the stability in 
an anisotropically rescaled prism. Given three positive numbers
$h_1$, $h_2$ and $h_3$ we denote
\begin{IEEEeqnarray*}{CCl}
    \tilde{E}   &   =   &   \tilde{T} \times \tilde{I}\\
    \tilde{T}   &   =   &   \{ 0 < \nicefrac{\tilde{x}_1}{h_1} + \nicefrac{\tilde{x}_2  }{h_2} < 1 \}\\
    \tilde{I}   &   =   &   \{ 0 < \nicefrac{\tilde{x}_3}{h_3} < 1 \}.
\end{IEEEeqnarray*}
\rescaledPrismTikz
Of course $\tilde{E} = F(\hat{E})$ where $F$ is the linear
$\mathbb{R}^3 \rightarrow \mathbb{R}^3$ transformation such as
\begin{IEEEeqnarray*}{rClCl}
  F\hat{\bx} & = & \diag{h_1}{h_2}{h_3} \hat{\bx} & = & \tilde{\bx}
\end{IEEEeqnarray*}
Denote with $\tilde{\bw}_k$ the $k$--th order curl--conforming interpolation
operator over $\tilde{E}$ defined as in~(\ref{push-forward}) via the \emph{push--forward}
$F^*$. for the rest of the subsection $\tilde\bu$ will be an element
with a well defined curl--conforming interpolate.
%, namely of
%$H({\bf curl},\tilde{E})\cap H^{1/2+\delta}(\tilde{E})^3$ for 
%a positive $\delta$ with 
%${\bf curl}\,\tilde{\bu}\in L^p(\tilde{E})^3$
%for some
%$p>2$.
Write the diameter of $\tilde{E}$ as $\textit{h}$ and as
$\tilde{x}_i,\,1\leqslant i\leqslant 3$, the coordinates along the axis
in $\mathbb{R}^3$.
\begin{lemma}\label{estabLinf} There exists a positive $C$, independent
of $h_i,\,1\leqslant i\leqslant 3$ such that for all $p > 2$ and 
$\tilde{\bu}\in\wpcurl{\tilde{E}}$
\begin{IEEEeqnarray*}{rCl}
    \left\| \tilde{\bw}_k\tilde{\bu} \right\|_{L^\infty(\tilde{E})^3}
    & \leqslant & C \left[ |\tilde{E}|^{-\nicefrac{1}{p}} \left( \left\| \tilde{\bu} 
    \right\|_{L^p(\tilde{E})^3} +
        \sum_{i=1}^3 h_i \left\| \partial_{\tilde{x}_i}\tilde{\bu} 
        \right\|_{L^p(\tilde{E})^3} \right)\right.\\
    &   & \left.\:+\; (h_1+h_2)\, |\tilde{E}|^{-1} \left( \left\|(\emph{\textbf{curl}}\,\tilde{\bu})_3 
    \right\|_{L^1(\tilde{E})} + 
    \sum_{i=1}^3 h_i \left\| \partial_{\tilde{x}_i}(\emph{\textbf{curl}}\,\tilde{\bu})_3 
    \right\|_{L^1(\tilde{E})}\right)
    \right].
\end{IEEEeqnarray*}
{\color{BrickRed} ver las cuentas donde dice $h_1 + h_2$}
\end{lemma}
\begin{proof}
The proof of this estimate will be made componentwise
using the inequalities of 
Theorem~(\ref{thm_stab_edge}). 
Probamos la cota para la primera y tercera componentes; la cota para la segunda es análoga a la de la 
primera. De estas seguirá inmediatamente la cota en norma vectorial.
%Recordamos $|\tilde{E}| = \frac{1}{2}h_1h_2h_3$.\\[5pt]
{\color{blue}\#\#\#\#\#\#\#\# continue here: put piola transform, or something
like that.}
Starting with~(\ref{teorema_1}), y de aplicar los cambios de variables del caso, tenemos
    \begin{IEEEeqnarray*}{rCl}
        \left\| (\tilde{\bw}_k\tilde{{\textbf{u}}})_1 \right\|_{L^\infty(\tilde{E})} & = &
            \frac{1}{h_1} \left\| (\hat{\bw}_k\hat{{\textbf{u}}})_1 \right\|_{L^\infty(\hat{E})}\\
            & \leqslant & C \frac{1}{h_1} \left[\|\hat{u}_1\|_{W^{1,p}(\hat{E})} + 
                \|({\textbf{curl}}\,\hat{{\textbf{u}}})_3\|_{W^{1,1}(\hat{E})}\right] \\
            & \leqslant & C
        \left[
            \frac{1}{|\tilde{E}|^\frac{1}{p}}
            \left(
            \|\tilde{u}_1\|_{L^p(\tilde{E})} + \sum_{i=1}^3 h_i \|\partial_{\tilde{x}_i}\tilde{u}_1\|_{L^p(\tilde{E})}
            \right)
        \right.\\
            & & \:\:+
        \left.
            \frac{h_2}{|\tilde{E}|}
            \left(
            \|(\textbf{curl}\,\tilde{\textbf{u}})_3\|_{L^1(\tilde{E})} + 
                \sum_{i=1}^3 h_i \|\partial_{\tilde{x}_i}(\textbf{curl}\,\tilde{\textbf{u}})_3\|_{L^1(\tilde{E})}
            \right)
        \right].
    \end{IEEEeqnarray*}
    
    
    
    En lo que respecta a la tercera componente, procediendo de la misma manera, obtenemos
    \begin{IEEEeqnarray*}{rCl}
        \left\| (\tilde{\bw}_k\tilde{{\textbf{u}}})_3 \right\|_{L^\infty(\tilde{E})}
        & \leqslant & \frac{C}{|\tilde{E}|^\frac{1}{p}}
        \left(
            \|\tilde{u}_3\|_{L^p(\tilde{E})} + \sum_{i=1}^3 h_i \|\partial_{\tilde{x}_i}\tilde{u}_3\|_{L^p(\tilde{E})}
        \right).
    \end{IEEEeqnarray*}
\end{proof}
\noindent De la anterior cota en norma infinito se deduce el siguiente
\begin{theorem} Existe $C > 0$, que es independiente de $h_1$, $h_2$ y de $h_3$, tal que para todo
$\tilde{\bu}\in\wpcurl{\tilde{E}}$
    \begin{IEEEeqnarray*}{rCl}
        \left\| \tilde{\bw}_k\tilde{\bu} \right\|_{L^p(\tilde{E})}
        & \leqslant & C \left[ \left\| \tilde{\bu} \right\|_{L^p(\tilde{E})}
        + \sum_{i=1}^3 h_i \left\| \partial_{\tilde{x}_i}\tilde{\bu} \right\|_{L^p(\tilde{E})}\right.\\
        & & \left.
        \:+\;h\left\|(\emph{\textbf{curl}}\,\tilde{\bu})_3 \right\|_{L^p(\tilde{E})}
        + h\sum_{i=1}^3 h_i
        \left\| \partial_{\tilde{x}_i}(\emph{\textbf{curl}}\,\tilde{\bu})_3 \right\|_{L^p(\tilde{E})}
    \right].
    \end{IEEEeqnarray*}
\end{theorem}
\noindent Esta es una desigualdad con anisotropía para el elemento transformado $\tilde{E}$.
\begin{proof}
    {

    \color{BrickRed}

    \noindent From Lemma~(\ref{estabLinf}), since $|\tilde{E}|$ is finite measured,
    the H\"older inequality tells us that, for any real $p \geqslant 1$,
    \begin{IEEEeqnarray*}{rCl}
        \|(\textbf{curl}\,\tilde{\textbf{u}})_3\|_{L^1(\tilde{E})} &\leqslant&
         |\tilde{E}|^{1-\frac{1}{p}}\,\|(\textbf{curl}\,\tilde{\textbf{u}})_3\|_{L^p(\tilde{E})}\\
        \|\partial_{\tilde{x}_i}(\textbf{curl}\,\tilde{\textbf{u}})_3\|_{L^1(\tilde{E})} &\leqslant&
         |\tilde{E}|^{1-\frac{1}{p}}\,\|\partial_{\tilde{x}_i}(\textbf{curl}\,\tilde{\textbf{u}})_3\|_{L^p(\tilde{E})}.
    \end{IEEEeqnarray*}
    So we get to
    \begin{IEEEeqnarray*}{rCl}
    \left\| (\tilde{\bw}_k\tilde{{\textbf{u}}})_1 \right\|_{L^p(\tilde{E})}
        & \leqslant & |\tilde{E}|^\frac{1}{p}\left\| (\tilde{\bw}_k\tilde{{\textbf{u}}})_1 \right\|_{L^\infty(\tilde{E})}\\
        & \leqslant & C
        \left[
            \|\tilde{u}_1\|_{L^p(\tilde{E})} + \sum_{i=1}^3 h_i \|\partial_{\tilde{x}_i}\tilde{u}_1\|_{L^p(\tilde{E})}
        \right.\\
            & & \:\:+
        \left.
            h_2
            \left(
            \|(\textbf{curl}\,\tilde{\textbf{u}})_3\|_{L^p(\tilde{E})} + 
                \sum_{i=1}^3 h_i \|\partial_{\tilde{x}_i}(\textbf{curl}\,\tilde{\textbf{u}})_3\|_{L^p(\tilde{E})}
            \right)
        \right].
    \end{IEEEeqnarray*}
    Now combine this with an entirely analogous argument for component $2$ and the estimate for component
    $3$ already obtained in Lemma~(\ref{estabLinf}).

    }
\end{proof}
\subsection{Stability of divergence--conforming Finite Elements on Prisms} % (fold)
The exposition in this section will be the $H(\mbox{div})$--conforming analogue
of the exposition in the previous section.\\
\noindent In present subsection $\hat\bu$ will be an element
of $W^{1,1}(\hat E)$ in which elements have well defined
normal traces over the faces of $\hat{E}$, another possibility
being, as mentioned in~\cite{monk}, Lemma $5.15$, page $120$, to assume
there is
a positive $\delta$ such that $\hat\bu$ belongs to
$H^{1/2+\delta}(\hat{E})^3$.
$\hat\br_k$ will be, for the whole subsection, the $k$--th order face 
interpolation operator on the reference
Prism determined by the element of
\emph{Definition}~\ref{defi_h_div_conforme}.
\label{stability_of_rt_element_in_hat_k}
\begin{lemma}\label{lemmaRT3zero}
$(\rku)_3$ is linearly and univocally determined by $\hat{u}_3$.
\end{lemma}
\begin{proof}
By the unisolvence of the Finite Element~(\ref{defi_h_div_conforme})
$(\rku)_3$ is determined by the following linear equations.
\begin{IEEEeqnarray}{rCrl}
\label{rku_3_1}
\rho_{f_j,q}\,(\rku) & = & \rho_{f_j,q}\,(\hat{\bu})
  &\quad\mbox{for $j$ = 3, 4 and }q\in\mathcal{P}_{\hat{f}_j} \\
\label{rku_3_2}
\rho_{\br}\,(\rku) & = & \rho_{\br}\,(\hat{\bu})
  &\quad\mbox{for }\br\in\mathcal{P}_{\hat{E}}\mbox{ with }r_1 = r_2 = 0.
\end{IEEEeqnarray}
We have $\nicefrac{k(k+1)^2}{2}$ equations, which is the 
number of independent coefficients in $(\rku)_3$.
Now set $u_3 = 0$, which makes the right hand side of all the equations in~(\ref{rku_3_1})
and~(\ref{rku_3_2}) equal to zero.
Take $\hat f$ to be either $\hat{f}_3$ or $\hat{f}_4$. Put $q = (\rku)_3|_{\hat{f}}$ in~(\ref{rku_3_1}).
It yields
\begin{IEEEeqnarray*}{rCl}
  \int\limits_{\hat{f}} ((\rku)_3)^2\,d\hat{\gamma} & = & 0
\end{IEEEeqnarray*}
so there is a polinomial $\hat{q}_3\in(\mathcal{P}_{\hat{E}})_3$ such that
$(\rku)_3 \xyz = \hat{x}_3(\hat{x}_3-1)\,\hat{q}_3\xyz$
and the Lemma will follow once we apply the degrees of freedom~(\ref{rku_3_2})
with $\br$ set equal to $(0,0,\hat{q}_3)^t$. 
\end{proof}
\begin{lemma}\label{lemma_u1_u2} If $\hat\bu\xyz = (0,0, \hat{u}_3\xyz)^t$,
then $\rku\xyz = (0,0,\hat{\xi}_3\xyz)$ for some $\hat{\xi}_3\in
P_{k-1}(\hat{f}_3)\otimes P_k(\hat{x}_3)$.
\end{lemma}
\begin{proof}
In a completely analogous way as was done in Section~\ref{sub:defEdgeElement}
we can derive the following expressions for 
$\rku = (\hat{p}_1, \hat{p}_2, \hat{p}_3)^t \in  P_{\hat{E}}$: 
\begin{IEEEeqnarray*}{rCl}
  \hat{p}_1\xyz & = & \hat{q}_1\xyz + \hat{x}_1\,\hat{h}\xyz\\
  \label{exprPrt}\yesnumber
  \hat{p}_2\xyz & = & \hat{q}_2\xyz + \hat{x}_2\,\hat{h}\xyz\\
  \hat{p}_3\xyz & = & \hat{q}_3\xyz
\end{IEEEeqnarray*}
for unique $\hat{q}_1, \hat{q}_2 \in \mathcal{P}_{k-1}(\hat{f}_3)
\otimes\mathcal{P}_{k-1}(\hat{x}_3)$,
$\hat{q}_3 \in \mathcal{P}_{k-1}(\hat{f}_3)\otimes\mathcal{P}_{k}(\hat{x}_3)$,
and
$\hat{h} \in \tilde{\mathcal{P}}_{k-1}(\hat{f}_3)\otimes\mathcal{P}_{k-1}(\hat{x}_3)$.
Take an arbitrary $\hat{q}\in\mathcal{P}_{k-1}(\hat f_1)\otimes\mathcal P_{k-1}(\hat z)$.
Continuing with the technique developed for Section~\ref{stab_edge_prism}, 
the trick will be to apply Green's Theorem to the field
$(\hat{p}_1, \hat{p}_2, 0)^t$ and the scalar $\hat{q}$. By the surface degrees of
freedom~(\ref{momentos2hdiv}) and the volume degrees of freedom~(\ref{momentos3hdiv}) we have
  \begin{IEEEeqnarray*}{rClCl}
    \int\limits_{\hat{E}} \mbox{div}(\hat{p}_1, \hat{p}_2, 0)^t\,\hat{q}\,d\hat{\bx}&=&
    \int\limits_{\partial\hat{E}} (\hat{p}_1, \hat{p}_2, 0)^t\cdot\boldsymbol{\hat\nu}\,\hat{q}\,d\hat{\gamma}
    - \int\limits_{\hat{E}} (\hat{p}_1, \hat{p}_2, 0)^t\cdot\nabla \hat{q}\,d\hat{\bx}&&\\[5pt]
    & = &
    \int\limits_{\hat{f}_5} (\hat{u}_1 + \hat{u}_2)
    \hat{q}_{|_{\hat{f}_5 = 1}}\,d\hat{\gamma}
    - \int\limits_{\hat{f}_1} \hat{u}_1 \hat{q}_{|_{\hat{f}_1}}\,d\hat{\gamma}&&\\[5pt]
    &&\,- \int\limits_{\hat{f}_2} \hat{u}_2 \hat{q}_{|_{\hat{f}_2}}\,d\hat{\gamma}
    - \int\limits_{\hat{E}} \hat{u}_1
    \frac{\partial\hat{q}}{\partial{\hat{x}_1}} 
      + \hat{u}_2\frac{\partial\hat{q}}{\partial{\hat{x}_2}}\,d\hat{\bx} & = & 0.
  \end{IEEEeqnarray*}
  And since $\mbox{div}(\hat{p}_1, \hat{p}_2, 0)^t$ also belongs to
  $\pp{k-1}{k-1}$, we just established it vanishes on all $\hat{E}$.\\[3pt]
  In a similar way as done in Lemma~\ref{lemma_PIu2_k_in_N}, eq.~(\ref{h_is_zero}),
  we get to prove $\hat{h} \equiv 0$ in expression~(\ref{exprPrt}). It means we may
  assume that $\hat{p}_1$ and $\hat{p}_2$ belong to 
  $\mathcal{P}_{k-1}(\hat{f}_3)\otimes\mathcal{P}_{k-1}(\hat{x}_3)$.
  Now it is convenient to use again the degrees of freedom on the
  faces normal to $(-1, 0, 0)^t$ and $(0, -1, 0)^t$.
  The conditions
  \begin{IEEEeqnarray*}{rCCCl}
    \rho_{\hat f_1\,\hat q}(\rku) & = & \int\limits_{\hat f_1} \hat{p}_1\hat q\,d\hat\gamma
    & = & 0\qquad\mbox{for all }\hat{q}\in\mathcal{P}_{k-1}(\hat f_1)\\
    \rho_{\hat f_2\,\hat q}(\rku) & = & \int\limits_{\hat f_2} \hat{p}_2\hat q\,d\hat\gamma
    & = & 0\qquad\mbox{for all }\hat{q}\in\mathcal{P}_{k-1}(\hat f_2)
  \end{IEEEeqnarray*}
  ensure that $\hat{x}_i$ divides $\hat{p}_i$ for both $i=1,2$.
%  In other words, there are $\phi$ and $\psi$ in $\pp{k-2}{k-1}$ such that
%  \begin{IEEEeqnarray*}{rCl}
%    \hat{p}_1 & = & \hat{x}\,\phi\\
%    \hat{p}_2 & = & \hat{y}\,\psi.
%  \end{IEEEeqnarray*}
But finally, if we evaluate the degrees of freedom~(\ref{momentos3hdiv}),
we see that  $\hat{p}_1 = (\rku)_1$ and 
$\hat{p}_2 = (\rku)_2$ can be no other that
constantly null over all $\hat{E}$. 
\end{proof}
\begin{lemma}
\begin{itemize}
  \item []
  \item [(a)]\label{piu2_k_in_N} If $\hat\bu(\hat x_1,\hat x_2,\hat x_3) =
  (0, \hat{u}_2(\hat x_1,\hat x_3), 0)^t$,
  then $\rku\xyz = (0, \hat\xi_2(\hat x_1, \hat x_3) ,0)^t$ for some 
  $\hat\xi_2 \in P_{k-1}(\hat{x}_2) \otimes P_k(\hat{x}_3)$.
  \item [(b)]\label{piu1_k_in_N} If $\hat\bu(\hat x_1,\hat x_2,\hat x_3) = 
  (\hat{u}_1(\hat x_2,\hat x_3), 0, 0)^t$
  then $\rku\xyz = (\hat\xi_1(\hat x_2,\hat x_3), 0 ,0)^t$ for some
    $\hat\xi_1\in P_{k-1}(\hat{x}_1) \otimes P_k(\hat{x}_3)$.
\end{itemize}
\end{lemma}
\begin{proof} Let us prove the first one of the two claims. The second one 
  has an entirely analogous proof. By Lemma~\ref{lemmaRT3zero} we get
  $(\rku)_3 = 0$.
  The nullity of $\dv\hat{\bu}$ and the commutative
  diagram property~(\ref{lema_pi_star_rot_u}) give us
  $\dv\rku = 0$.
  This last fact, together with the result of the evaluation of the 
  degrees of freedom~(\ref{momentos2hdiv})
  on the face $\hat f_1 \subseteq \{x_1=0\}$
  and~(\ref{momentos3hdiv}) over $\hat E$, implies, as we have seen in
  Lemma~\ref{lemma_u1_u2}, that $(\rku)_1 = 0$.
  And if look again at 
  $\dv\rku = {\partial\rku}/{\partial\hat x_2} = 0$
  we have that $(\rku)_2$ does not depend on $\hat x_2$.
\end{proof}
\noindent It is time to state and prove the Theorem that was the purpose of this section.
\begin{theorem}\label{thm_stab_div} Given $\hat{\bu} \in W^{1,1}(\hat{E})$
\begin{IEEEeqnarray}{rCl}
\label{teoremaDiv_1} \norm{(\rku)_1}_{L^{\infty}(\hat{E})} & 
    \lesssim & \|\hat{u}_1\|_{W^{1,1}(\hat{E})} + 
    \|\emph{div}(\hat{u}_1, \hat{u}_2, 0)\|_{L^{1}(\hat{E})} \\ 
\label{teoremaDiv_2} \norm{(\rku)_2}_{L^{\infty}(\hat{E})} & 
    \lesssim & \|\hat{u}_2\|_{W^{1,1}(\hat{E})} + 
    \|\emph{div}(\hat{u}_1, \hat{u}_2, 0)\|_{L^{1}(\hat{E})} \\ 
\label{teoremaDiv_3} \norm{(\rku)_3}_{L^{\infty}(\hat{E})} & 
    \lesssim & \|\hat{u}_3\|_{W^{1,1}(\hat{E})}
\end{IEEEeqnarray}
where the constants in the inequalities depend olny on $\hat{E}$.
\end{theorem}
\begin{proof}[Proof of Theorem~\ref{thm_stab_div}] The proof is based on the
last three Lemmas. By Remark~\ref{density_wpcurl} we can state the estimate for
a smooth field $\hat{\bu}$ and then finish the proof with a density argument.\\[4pt]
Take again $\omega$ as the closure of $\hat{E}$ and 
$\hat\bu\in\mathcal{C}^\infty(\omega)^3$. Let
us prove the first inequality. Set
$\hat{\bv} = (\hat{u}_1, \hat{u}_2 - \hat{u}_2(\hat{x}_1,0,\hat{x}_3), 0)$.
Then $(\hat{\br}_k\bv)_1 = (\rku)_1$.
By evaluating the degrees of freedom to $\hat{\bv}$ we observe
that some of them vanish or depend exclusively on $\hat{u}_1$ in the way
we need them to depend on $\hat{u}_1$. As for the others,
pick first 
$\hat{q}_0 \in \mathcal{P}_{\hat{f}_5} = 
P_{k-1}(\hat{x}_1)\otimes P_{k-1}(\hat{x}_3)$
and extend it as the same polynomial  
$\hat{q}$ to $Q_{k-1,k-1,k-1}$.
\begin{IEEEeqnarray*}{rCl}
  \rho_{\hat{f}_5,\,\hat{q}_0} (\hat{\bv})
  & = & \int\limits_{\hat{f}_5} \hat{u}_1\,\hat{q}_0\,d\hat{\gamma} + 
  \sqrt{2}\iint\limits_{[0,1]^2} \hat{q}_0(\hat{x}_1,\hat{x}_3)
  \int_0^{1-\hat{x}_1}\frac{\partial \hat{v}_2}{\partial\hat{x}_2}
    (\hat{x}_1,\hat{t},\hat{x}_3)\,d\hat{t}d\hat{x}_1d\hat{x}_3\\[5pt]    
  & = & \int\limits_{\hat{f}_5} \hat{u}_1\,\hat{q}_0\,d\hat{\gamma} + 
  \sqrt{2}\int\limits_{\hat{E}} \hat{q}\frac{\partial \hat{v}_2}{\partial\hat{x}_2}\,d\hat{\bx}\\[5pt]    
  & = & \int\limits_{\hat{f}_5} \hat{u}_1\,\hat{q}_0\,d\hat{\gamma} + 
  \sqrt{2}\int\limits_{\hat{E}} \hat{q}\,\{\,\mbox{div} (\hat{u}_1,\hat{u}_2,0)^t -
    \frac{\partial \hat{u}_1}{\partial\hat{x}_1}\,\}\,d\hat{\bx}.
\end{IEEEeqnarray*}
For the volume degrees of freedom~(\ref{momentos3hdiv}) take $\hat{q}_2
\in P_{k-2}(\hat{f}_3) \otimes P_{k-1}(\hat{x}_3)$. Write 
$\hat{v}_2\xyz = \int_0^{\hat{x}_2} 
\nicefrac{\partial\hat{u}_2}{\partial\hat{x}_2}(\hat{x}_1,\hat{t},\hat{x}_3)\,d\hat{t}$ and do
{\color{brown}\#\#\#\#\#\#\#\# partir este array para pasar de p'agina.}
\begin{IEEEeqnarray*}{rCl}
  \int\limits_{\hat{E}} \hat{v}_2\,\hat{q}_2 
  & = &\int\limits_0^1\int\limits_0^1\int\limits_0^{1-\hat{x}_1}
  \int\limits_0^{\hat{x}_2}
    \frac{\partial \hat{u}_2}{\partial \hat{x}_2} 
    (\hat{x}_1,\hat{t},\hat{x}_3)\,d\hat{t}\,\hat{q}_2\xyz\,d\hat{x}_2\,d\hat{x}_1\,d\hat{x}_3\\
  & = &\int\limits_0^1\int\limits_0^1\int\limits_0^{1-\hat{x}_1}\int\limits_0^{\hat{x}_2}
        \frac{\partial\hat{u}_2}{\partial\hat{x}_2}(\hat{x}_1,\hat{t},\hat{x}_3)
        \,\hat{q}_2 \xyz\,d\hat{t}\,d\hat{x}_2\,d\hat{x}_1\,d\hat{x}_3\\
  & = &\int\limits_0^1\int\limits_0^1\int\limits_0^{1-\hat{x}_1}\int\limits_{\hat{t}}^{1-\hat{x}_1}
        \frac{\partial\hat{u}_2}{\partial\hat{x}_2}(\hat{x}_1,\hat{t},\hat{x}_3)\,\hat{q}_2\xyz\,
        d\hat{x}_2\,d\hat{t}\,d\hat{x}_1\,d\hat{x}_3\\
  & = &\int\limits_0^1\int\limits_0^1\int\limits_0^{1-\hat{x}_1}
        \frac{\partial \hat{u}_2}{\partial \hat{x}_2}(\hat{x}_1,\hat{t},\hat{x}_3)
        \int\limits_{\hat{t}}^{1-\hat{x}_1}\,\hat{q}_2\xyz\,d\hat{x}_2\,d\hat{t}\,d\hat{x}_1\,d\hat{x}_3\\
  & = &\int\limits_0^1\int\limits_0^1\int\limits_0^{1-\hat{x}_1}
  \frac{\partial\hat{u}_2}{\partial\hat{x}_2}(\hat{x}_1,\hat{t},\hat{x}_3)\,
       \hat{\phi} (\hat{x}_1,\hat{t},\hat{x}_3)\,d\hat{t}\,d\hat{x}_1\,d\hat{x}_3\\
& = &\int\limits_{\hat{E}} \mbox{div} (\hat{u}_1,\hat{u}_2,0)^t\,\hat{\phi}\,d\hat{\bx}
    - \int\limits_{\hat{E}}\frac{\partial\hat{u}_1}{\partial\hat{x}_1}\,\hat{\phi}\,d\hat{\bx}
\end{IEEEeqnarray*}
(for some $\hat{\phi} \in  P_{k-1}(\hat{f}_3)\otimes P_{k-1}(\hat{x}_3)$),
which is what we needed. The inequality~(\ref{teoremaDiv_2}) is proved in the same way.
For inequality~(\ref{teoremaDiv_3})
\begin{IEEEeqnarray*}{rCl}
  (\rku)_3 & = & (\hat{\br}_k(0,0,\hat{u}_3)^t)_3\\
  & = & \sum_{i=3,4\,\hat{\bq}\,\in\,{\color{red}\mathcal{B}_{\hat f_i}}}
  \int\limits_{\hat f_i} \hat{u}_3 \hat{q}_3\,d\hat{\gamma} \,(\hat{\bv}_{\hat{f}_i,\hat{\bq}})_3
    +\sum_{\hat{\br}\,\in\,{\color{red}\mathcal{B}_{\hat E}}}
  \int\limits_{\hat E} \hat{u}_3 \hat{r}_3\,d\hat{\bx}\,(\hat{\bv}_{\hat{\br}})_3.
\end{IEEEeqnarray*}
Then, by standard result for traces in Sobolev spaces,
\begin{IEEEeqnarray*}{rCl}
  \|(\rku)_3\|_{L^\infty(\hat{E})} 
  & \leqslant & c(\hat{E}) \left\{
   \sum_{i=3,4}
     \int\limits_{\hat f_i} |\hat{u}_3|\,d\hat{\gamma}
   + \int\limits_{\hat E} |\hat{u}_3|\,d\hat{\bx}
  \right\}\\
  &\leqslant& c(\hat{E}) (\|\hat{u}_3|_{\partial\hat{E}}\|_{L^1(\partial\hat{E})} + 
    \|\hat{u}_3\|_{L^1(\hat{E})})\\
  &\leqslant& c(\hat{E}) \|\hat{u}_3\|_{W^{1,1}(\hat{E})}
\end{IEEEeqnarray*}
\end{proof}
Theorems~\ref{thm_stab_edge} and~\ref{thm_stab_div} show that the interpolations
determined by the Finite Elements~(\ref{edgeelement}) and~(\ref{defi_h_div_conforme})
are anisotropically stable, in the sense that the image of a field $\hat\bu$ under
the linear operator depends not only continously on $\hat\bu$, but also with a
\emph{componentwise} bound, with perhaps and additional
curl or divergence term, respectively.\\

\noindent Consider again the element $\tilde{E}$ defined in~(\ref{tilde_prism}).
\begin{theorem} \label{thmStabilityKtildeRT}
There is $C > 0$, independent of $h_1$, $h_2$ and $h_3$, s.t. for all $p \geqslant 1$ and 
  $\tilde{\bu}\in W^{1,p}(\tilde{E})$
  \begin{IEEEeqnarray*}{rCl}
    \left\| \rkutilde \right\|_{L^p(\tilde{E})}
    & \leqslant & C \left( \left\| \tilde{\bu} \right\|_{L^p(\tilde{E})}
    + \sum_{i=1}^3 h_i \left\| \frac{\partial\tilde{\bu}}{\partial\tilde{x}_i} \right\|_{L^p(\tilde{E})}
    + \max\{h_1,h_2\}\left\|{\dv}(\tilde{u}_1, \tilde{u}_2, 0) \,\right\|_{L^p(\tilde{E})}\right).
  \end{IEEEeqnarray*}
\end{theorem}

\noindent{\color{BrickRed}\#\#\#\#\#\#\# continue here. Ver si en aadl hacen
estabilidad fisica (creo que si). ver c'omo est'a en lombardi hcurl tetra.} 

\begin{remark}
{\color{red}esto va en el cap. de approx tal vez, decidir, porque esta dos veces
esto mismo.}  When it comes to estimate in terms of the data $f$ we
  will assume $h_3 \geqslant C\max\{h_1,h_2\}$ to be able to do
  \begin{IEEEeqnarray*}{rCl}
    \left\| \rkutilde \right\|_{L^p(\tilde{E})}
    & \leqslant & C \left( \left\| \tilde{\bu} \right\|_{L^p(\tilde{E})}
    + \sum_{i=1}^3 h_i \left\| \frac{\partial\tilde{\bu}}{\partial\tilde{x}_i} \right\|_{L^p(\tilde{E})}
    + \max\{h_1,h_2\}\left\|{\dv}\hat\bu\right\|_{L^p(\tilde{E})}
    + \max\{h_1,h_2\}\left\|\frac{\partial\hat u_3}{\partial\hat x_3}\right\|_{L^p(\tilde{E})}\right).\\[5pt]
    &\leqslant& C \left( \left\| \tilde{\bu} \right\|_{L^p(\tilde{E})}
    + \sum_{i=1}^3 h_i \left\| \frac{\partial\tilde{\bu}}{\partial\tilde{x}_i} \right\|_{L^p(\tilde{E})}
    + \max\{h_1,h_2\}\left\|{\dv}\hat\bu\right\|_{L^p(\tilde{E})}\right).
  \end{IEEEeqnarray*}
  this is not a restriction since we needed the prisms to be 
  elongated exactly along the direction paralell to the cuadrilateral faces.
\end{remark}
\begin{proof}
Pick $p\geqslant 1$. If we pull 
$\tilde{\bu}$ back to $\hat{E}$ we get the relation
\begin{IEEEeqnarray}{rCl}\label{pull1}
  \hat{\bu}(\hat{\bx}) & = & (det\,DF)DF^{-1}\tilde{\bu}(F\hat{\bx})
\end{IEEEeqnarray}
\begin{IEEEeqnarray}{rCl}
  D\hat{\bu}(\hat{\bx}) & = & \diag{h_2\,h_3}{h_1\,h_3}{h_1\,h_2}\cdot
  D\tilde{\bu}(F\hat{\bx})\cdot\diag{h_1}{h_2}{h_3}
\end{IEEEeqnarray}
and by~(\ref{push-forward})
\begin{IEEEeqnarray}{rCl}\label{pull2}
  (\det DF)DF^{-1}\rkutilde(F(\hat{\bx})) & = & \rku(\hat{\bx}).
\end{IEEEeqnarray}
With~(\ref{pull1})~(\ref{pull2}) and 
stability inequality~(\ref{teoremaDiv_1}) 
plus H\"older's inequality we obtain 
\begin{IEEEeqnarray*}{rCl}
  \|(\rkutilde)_1\|_{L^{\infty}(\tilde{E})} & = &
  (h_2\,h_3)^{-1}
  \|(\rku)_1\|_{L^{\infty}(\hat{E})}\\[7pt]
  &\leqslant&(h_2\,h_3)^{-1}\left(\|\hat{u}_1\|_{W^{1,1}(\hat{E})}
  +\|\dvg(\hat{u}_1,\hat{u}_2,0)^t\|_{L^{1}(\hat{E})}\right)\\[7pt]
  &=& 
  (h_2\,h_3)^{-1}
  \left(
    \int\limits_{\hat{E}}\left|\hat{u}_1\right|\,d\hat{\bx}
    +\sum_{i=1}^3\int\limits_{\hat{E}}\left|\frac{\partial\hat{u}_1}{\partial\hat{x}_i}\right|\,d\hat{\bx}
    +\int\limits_{\hat{E}}\left|\frac{\partial\hat{u}_1}{\partial\hat{x}_1} + \frac{\partial\hat{u}_2}{\partial\hat{x}_2}\right|
    \,d\hat{\bx}
  \right)\\[7pt]
  &=&(\det DF)^{-1}\left(
  \int\limits_{\tilde{E}}|\tilde{u}_1|\,d\tilde{\bx}
  +\sum_{i=1}^3h_i\int\limits_{\tilde{E}}|\frac{\partial\tilde{u}_1}{\partial\tilde{x}_i}|\,d\tilde{\textbf{x}}
  +h_1\int\limits_{\tilde{E}}|\frac{\partial\tilde{u}_1}{\partial\tilde{x}_1} + \frac{\partial\tilde{u}_2}{\partial\tilde{x}_2}|
  \,d\tilde{\textbf{x}}\right)\\[7pt]
  &=&(2|\tilde{E}|)^{-1}\left(
  \|\tilde{u}_1\|_{L^1(\tilde{E})}
  +\sum_{i=1}^3h_i\|\frac{\partial\tilde{u}_1}{\partial\tilde{x}_i}\|_{L^1(\tilde{E})}
  +h_1\|\dvg(\tilde{u}_1,\tilde{u}_2,0)\|_{L^1(\tilde{E})}\right)\\[7pt]
  &\leqslant&(2|\tilde{E}|^{\nicefrac{1}{p}})^{-1}\left(
  \|\tilde{u}_1\|_{L^p(\tilde{E})}
  +\sum_{i=1}^3h_i\|\frac{\partial\tilde{u}_1}{\partial\tilde{x}_i}\|_{L^p(\tilde{E})}
  +h_1\|\dvg(\tilde{u}_1,\tilde{u}_2,0)\|_{L^p(\tilde{E})}\right).
\end{IEEEeqnarray*}
Now
\begin{IEEEeqnarray*}{rCl}
  \|(\rkutilde)_1\|_{L^{p}(\tilde{E})}
  &\leqslant&
  |\tilde{E}|^{1/p}\|(\tilde{\br}_k\tilde{\boldsymbol{u}})_1\|_{L^{\infty}(\tilde{E})}\\
  &\lesssim&\|\tilde{u}_1\|_{L^p(\tilde{E})}
  +\sum_{i=1}^3h_i\|\frac{\partial\tilde{u}_1}{\partial\tilde{x}_i}\|_{L^p(\tilde{E})}
  +h_1\|\dvg(\tilde{u}_1,\tilde{u}_2,0)\|_{L^p(\tilde{E})},
\end{IEEEeqnarray*}
and again, the symmetric inequality holds for component $2$. For component $3$,
stability inequality~(\ref{teoremaDiv_3}) gives us
\begin{IEEEeqnarray*}{rCl}
  \|(\rkutilde)_3\|_{L^{\infty}(\tilde{E})} & = &({h_1h_2})^{-1}
  \|(\rku)_3\|_{L^{\infty}(\hat{E})}\\[6pt]
  &\leqslant&{C}({h_1h_2})^{-1}\,\|\hat{u}_3\|_{W^{1,1}(\hat{E})}\\[6pt]
  &=&C\,|\tilde{E}|^{-1}\,\left[\|\tilde{u}_3\|_{L^1(\tilde{E})} +
    \sum_{i=1,2,3} h_i\,\left\|\frac{\partial\tilde{u}_3}{\partial\tilde{x}_i}\right\|_{L^1(\tilde{E})}\right]\\[6pt]
  &\leqslant& {C}\,|\tilde{E}|^{-\nicefrac{1}{p}}\,\left[\|\tilde{u}_3\|_{L^p(\tilde{E})} +
    \sum_{i=1,2,3} h_i\,\left\|\frac{\partial\tilde{u}_3}{\partial\tilde{x}_i}\right\|_{L^p(\tilde{E})}\right]
\end{IEEEeqnarray*}
so, immediately,
\begin{IEEEeqnarray}{rCl} \label{aux_label18}
  \|(\rkutilde)_3\|_{L^{p}(\tilde{E})}
  &\leqslant& C\,\left(
  \|\tilde{u}_3\|_{L^p(\tilde{E})} +
    \sum_{i=1,2,3} h_i\,\left\|\frac{\partial\tilde{u}_3}{\partial\tilde{x}_i}\right\|_{L^p(\tilde{E})}
  \right)
\end{IEEEeqnarray}
and the sum of the three estimates yields the theorem.
\end{proof}
%===============================================================================
% \begin{lemma}\label{L6} Let $P$ be a right prism. There exists a constant $C$ depending only on $\alpha_P$ such that for all $\bu$ in $W^{1,1}(P)$ we have
% \begin{multline}\label{estabL1}
% \|\bu_I\|_{L^1(P)} \leqslant C\Bigg(\|\bu\|_{L^1(P)} + \sum_{i=1}^3 h_{i,P}\|\partial_{\xi_{P,i}}\bu\|_{L^1(P)}\\ + \max\{h_{P,1},h_{P,2}\}\|\mbox{div\,}(u_1,u_2,0)\|_{L^1(P)}\Bigg).
% \end{multline}
% \end{lemma}
% \begin{proof} Using the notation introduced above for the vertices of $P$, suppose that $v_0$ is the vertex with the maximum angle of the triangle $v_0v_1v_2$. Let $\tilde P$ be a prism with vertices at $(0,0,0)$, $(h_{P,1},0,0)$, $(0,h_{P,2},0)$, $(0,0,h_{P,3})$, $(h_{P,1},0,h_{P,3})$ and $(0,h_{P,2},h_{P,3})$. Then by standard rescaling arguments using the Piola Transform we can prove from Lemma \eqref{L5} that there exists a constant $C$ such that for all $\bu\in W^{1,1}(\tilde P)$ we have
% \begin{eqnarray*}
% \|\tilde\bu_I\|_{L^1(\tilde P)}&\le& C\Bigg(\|\tilde\bu\|_{L^1(\tilde P)} + \sum_{i=1}^3h_{P,i}\|\partial_{x_i}\tilde\bu\|_{L^1(\tilde P)}\\&&\qquad + 
% \max\{h_{P,1},h_{P,2}\}\|\dv(\tilde u_1,\tilde u_2,0)\|_{L^1(\tilde P)}\Bigg).
% \end{eqnarray*}
% Let $B$ be the matrix with columns $\xi_{P,1}$, $\xi_{P,2}$ and $\xi_{P,3}$ (note $B$ has the form \eqref{matrix} and $\xi_{P,3}=(0,0,1)$). Then the map $F(\tilde{\bf x})=B\tilde{\bf x}+v_0$ sends $\tilde P$ onto $P$. Then, again by a change of variables, we obtain from the previous estimate, that for all $\bu\in W^{1,1}(P)$ it holds
% \begin{eqnarray*}
% \|\bu_I\|_{L^1(P)}&\le& C\|B\|\|B^{-1}\|\bigg(\|\bu\|_{L^1(P)} + \sum_{i=1}^3h_{P,i}\|\partial_{\xi_{P,i}}\bu\|_{L^1(P)}\\ &&  \qquad +\max\{h_{P,1},h_{P,2}\}\frac1{\|B^{-1}\|}\|\dv(u_1,u_2,0)\|_{L^1(P)}\bigg). 
% \end{eqnarray*}
% Then the proof concludes by noting that $\|B\|\leqslant C$ and $\|B^{-1}\|\sim \sin\alpha_P$.
% \end{proof}
% 
% \begin{remark} Stability estimates in $L^p$-norm, $p>1$, can by proved analogously.
% In particular, from  \eqref{estabL1}, using an inverse inequality on the left hand side, and Cauchy-Schwarz inequality on the right hand side, we obtain under assumptions of Lemma \ref{L6}
% \begin{multline}\label{estabL2}
% \|\bu_I\|_{L^2(P)} \leqslant C\Bigg(\|\bu\|_{L^2(P)} + \sum_{i=1}^3 h_{i,P}\|\partial_{\xi_{P,i}}\bu\|_{L^2(P)}\\ + \max\{h_{P,1},h_{P,2}\}\|\mbox{div\,}(u_1,u_2,0)\|_{L^2(P)}\Bigg)
% \end{multline}
% \end{remark}
%===============================================================================

\begin{corollary} Under assumptions of Lemma \ref{L6}, and if $h_{P,3}\ge \max\{h_{P,1},h_{P,2}\}$ we obtain 
\begin{equation}\label{estabLp}
\|\br_E \bu\|_{L^p(E)} \leqslant \left(\|\bu\|_{L^p(E)} + \sum_{i=1}^3 h_{E,i}\|\partial_{\xi_i} \bu\|_{L^p(E)}+ h_E\|\mbox{div\,}\bu\|_{L^p(E)}\right)
\end{equation}
\end{corollary}
\begin{remark}
{\color{red}esto es del corolario aca arriba}  When it comes to estimate in terms of the data $f$ we
  will assume $h_3 \geqslant C\max\{h_1,h_2\}$ to be able to do
  \begin{IEEEeqnarray*}{rCl}
    \left\| \rkutilde \right\|_{L^p(\tilde{E})}
    & \leqslant & C \left( \left\| \tilde{\bu} \right\|_{L^p(\tilde{E})}
    + \sum_{i=1}^3 h_i \left\| \frac{\partial\tilde{\bu}}{\partial\tilde{x}_i} \right\|_{L^p(\tilde{E})}
    + \max\{h_1,h_2\}\left\|{\dv}\hat\bu\right\|_{L^p(\tilde{E})}
    + \max\{h_1,h_2\}\left\|\frac{\partial\hat u_3}{\partial\hat x_3}\right\|_{L^p(\tilde{E})}\right).\\[5pt]
    &\leqslant& C \left( \left\| \tilde{\bu} \right\|_{L^p(\tilde{E})}
    + \sum_{i=1}^3 h_i \left\| \frac{\partial\tilde{\bu}}{\partial\tilde{x}_i} \right\|_{L^p(\tilde{E})}
    + \max\{h_1,h_2\}\left\|{\dv}\hat\bu\right\|_{L^p(\tilde{E})}\right).
  \end{IEEEeqnarray*}
  this is not a restriction since we needed the prisms to be 
  elongated exactly along the direction paralell to the cuadrilateral faces.
\end{remark}



\subsection{Local Interpolation Estimates for Prismatic Elements} % (fold)
\label{sub:local_interpolation_estimates_for_prismatic_elements}

For the following paragraph we refer to the exposition in pages $25$--$26$
of~\cite{ariel}, in which the author
state the forthcoming facts for tetrahedra. We will state scaling consequences
of inequalities~(\ref{aux_label19}). For a non--negative integer
$m$ let $\partial^m f$ denote
the sum of the absolute values of all the derivatives of order $m$ of $f$.
\begin{lemma}\label{aux_label20}
Let $E$ be a prism such that (\noindent{\color{BrickRed}hip. prisma}).
Let $\xi_i$, $1\leqslant i \leqslant 3$ be unitary vectors with the directions
of the three edges sharing a vertex $\bx_E$ of $E$, whose lengths are
$h_i$, $1\leqslant i\leqslant 3$. Given $p\geqslant 1$, $m\geqslant 0$ and
$\bu \in W_{m+1,p}$ there exists a vector polynomial $\bq\in P_m(E)^3$ such that 
\begin{IEEEeqnarray}{rCl}\label{aux_label22}
  \left\|\frac{\partial}{\partial \xi_i}(\bu-\bq)\right\|_{L^p(E)}
  &\leqslant&C\sum_{|\alpha| = m} \boldsymbol{h}^\alpha 
  \left\|\frac{\partial}{\partial\xi_i} (\partial^{\alpha}\bu) \right\|_{L^p(E)}
\end{IEEEeqnarray} 
and 
\begin{IEEEeqnarray}{rCl}\label{aux_label23}
  \|\dv(\bu-\bq)\|_{L^p(E)}&\leqslant&Ch_E^m\|\partial^m\dv\bu\|_{L^p(E)}
\end{IEEEeqnarray}
where $C$ depends only on $m$ and \noindent{\color{BrickRed} ????}.

If $m \geqslant 1$ and $p \geqslant 1$
\begin{IEEEeqnarray}{rCl}
  \label{aux_label24}
  \|\curl(\tilde\bu-\tilde\bq)\|_{L^1(\tilde E)}&\leqslant&
  C|E|^{1-\nicefrac1p}\,h^m
  \|\partial^m\curl\tilde\bu\|_{L^p(\tilde E)}\\[5pt]
  \label{aux_label25}
  \|\frac{\partial}{\partial_{\tilde x_i}}\curl(\tilde\bu-\tilde\bq)\|_{L^1(\tilde E)}&\leqslant&
  C|E|^{1-\nicefrac1p}\,h^{m-1}
  \|\partial^m\curl\tilde\bu\|_{L^p(\tilde E)}
\end{IEEEeqnarray}
where C depends only on $m$, $\sigma$ (cfr. Theorem~\ref{aux_label21})
and the reference element.
\end{lemma}




one of the main results in this thesis
Sean un prisma obl\'{\i}cuo irregular $K$, $k\in\mathbb{N}$, el operador de interpolaci\'on 
$\bw_E$ de grado $k$ determinado por el elemento en la Definición~(\ref{edgeelement}), y $p>2$.

\begin{theorem}
Existe $C > 0$ independiente de $E$ tal que para todo campo $\bu\in W^{m + 1,p}(E)$
con $\bcurl \bu\in W^{m,p}(E)$ y $m\leqslant k-1$, 
\begin{IEEEeqnarray*}{rCl}
  \|\bu-\bw_E \bu\|_{L^p(E)} & \leqslant & C
  \left(
    \sum_{|\alpha|=m+1}\boldsymbol{h}^\alpha \|\partial^\alpha \bu\|_{L^p(E)} +
    h_E^{m+1}\|\partial^m \bcurl \bu\|_{L^p(E)}
  \right).
\end{IEEEeqnarray*} 
\end{theorem}

\begin{proof}

{\color{blue}\#\#\#\#\#\#\#\# continue here.} 
  ~(\ref{aux_label24})

  ~(\ref{aux_label25})








\end{proof}


{\color{BrickRed}
Take a kernel $\hat{\phi}\in\mathcal{C}_0^\infty(\mathbb{R}^n)$ such that
\begin{IEEEeqnarray*}{rCl}
  \supp{\hat{\phi}}&\subseteq&\hat{B}\\[5pt]
  \int\limits_{\hat{B}} \hat{\phi}\,d\textbf{x} & = & 1
\end{IEEEeqnarray*} 
Now define the kernel on B
\begin{IEEEeqnarray*}{rCl}
  \phi_B&=&\frac{|\hat{B}|}{|B|}\,\hat{\phi}\circ F^{-1}\\[5pt]
  &=&\frac{1}{\det DF}\,\hat{\phi}\circ F^{-1}
\end{IEEEeqnarray*}
This way we have
\begin{IEEEeqnarray*}{rClCr}
  \int\limits_{B}\phi\,d\textbf{x}&=&
  \int\limits_{\hat{B}}\hat{\phi}\,\,d\hat{\textbf{x}}&=&1\\[5pt]
\end{IEEEeqnarray*}
and also
\begin{IEEEeqnarray*}{rClClCr}
  \supp{\phi_B}&=&\supp{\hat{\phi}\circ F^{-1}}
  &=&F(\supp{\hat{\phi}})&=&B.
\end{IEEEeqnarray*}
Then we build an averaged Taylor polynomial over $B$ with the kernel
$\phi_B$ and it holds that
\begin{IEEEeqnarray*}{rCl}
  \hat{Q}^m(\hat{w})&=&(Q^mw)^{\hat{}}\\[5pt]
\end{IEEEeqnarray*}
where 
\begin{IEEEeqnarray*}{rCl}
  \hat{Q}&=&\text{avg. Taylor polyn. over }\hat{B}\text{ with kernel }\hat{\phi}\\[5pt]
  {Q}&=&\text{avg. Taylor polyn. over }{B}\text{ with kernel }{\phi_B}
\end{IEEEeqnarray*}
}











%%==============================================================================
% \begin{theorem}\label{thmErrorInterpolacionPrismas}
% Let $P$ be a right prism, and consider a local system of coordinates $x_1x_2x_3$
% such that the triangular basis of $P$ are parallel to the $x_1x_2$-coordinate
% plane. Denote by $\xi_{P,1}$ and $\xi_{P,2}$ the versors parallel to the edges
% of the triangular basis of $P$ adjacent to its maximum angle
% $\alpha_P$, $\xi_{P,3}=(0,0,1)$ and $h_{P,i}$ are the lengths of the edges of
% $P$ parallel to $\xi_{P,i}$. We assume that $h_{P,3}>ch_{P,1}$ and
% $h_{P,3}>ch_{P,2}$. Then, there exists a constant $C$ depending only on $c$
% and $\alpha_P$, such that for all $\bu\in H^1(P)$ we have
% \begin{equation}\label{interp}
% \|\bu-\boldsymbol{r}_E\bu\|_{L^2(E)} \leqslant C\left(\sum_{i=1}^3 h_{E,i}
% \|\partial_{\xi_{E,i}}\bu\|_{L^2(E)} + h_T\|\dv\bu\|_{L^2(E)}\right).
% \end{equation}
% \end{theorem}
%%==============================================================================

% subsection local_interpolation_estimates_for_prismatic_elements (end)
% section prismatic_finite_elements (end)
\section{Pyramidal Finite Elements} % (fold)
\label{sec:pyramidal_finite_elements}
$\hat{E}$ will be the reference pyramid  in Figure~\ref{reference_pyramid}.
Anisotropic interpolation error estimates for pyramidal $\bcurl$--conforming
and div--conforming finite elements of least order will we established.
\subsection{Anisotropic Stability Estimates for $H(\bcurl)$--Conforming 
Elements on Pyramids} % (fold)
\label{sub:edge_elements}
The proof is based on mere calculation.
Recall the shape funtions in Table~(\ref{shape_edge_table}).
Start with $\bu$ of the form $(u_1,0,0)'$. After computations we have
\begin{IEEEeqnarray*}{rCl}
	\nabla\times\bu &=& (0, \frac{{\s\partial} u_1}{{\s\partial} x_3},-\frac{{\s\partial} u_1}{{\s\partial} x_2})\\[5pt]
	\wku	&=& [{\s\int_{\hat{\be}_1}\bu\cdot\btau\, ds}]\bgamma_1 +
				[{\s\int_{\hat{\be}_3}\bu\cdot\btau\, ds}]\bgamma_3 + 
				[{\s\int_{\hat{\be}_6}\bu\cdot\btau\, ds}]\bgamma_6 + 
				[{\s\int_{\hat{\be}_8}\bu\cdot\btau\, ds}]\bgamma_8\\[5pt]
			&=& \alpha_1(\hat\bu)\hat\bgamma_1 + 
				\alpha_3(\hat\bu)\hat\bgamma_3 + 
				\alpha_6(\hat\bu)\hat\bgamma_6 + 
				\alpha_8(\hat\bu)\hat\bgamma_8
\end{IEEEeqnarray*}
Now, $\hat\bu$ has zero tangential component along $\hat\be_2$, so then
\begin{IEEEeqnarray*}{rCl}
	(\wku)_2 		  & = &\left(\alpha_6(\hat\bu)-\alpha_8(\hat\bu)\right)\frac{xz}{1-z} \\[5pt]
					  & = &\left(\int_{\hat\be_6}\hat\bu\cdot\hat\btau_6\,ds-
					  			 \int_{\hat\be_8}\hat\bu\cdot\hat\btau_6\,ds-
					  			 \int_{\hat\be_2}\hat\bu\cdot\hat\btau_6\,ds\right)\frac{xz}{1-z} \\[5pt]
					  & = &-\int_{\partial\hat{f}_3}\hat\bu\cdot\hat\btau\,ds\,\frac{xz}{1-z} \\[5pt]
					  & = &-\int_{\hat{f}_3}\nabla\times\hat\bu\cdot\bn\,d\gamma\,\frac{xz}{1-z} \\[5pt]
					  & = &\int_{\hat{f}_3}\frac{{\s\partial} \hat u_1}{{\s\partial} x_2}\,d\gamma\,\frac{xz}{1-z}
\end{IEEEeqnarray*}
by Stoke's theorem. Next,
\begin{IEEEeqnarray*}{rCl}
	(\wku)_3 & = &     \alpha_1(\hat\bu)\left(x-\frac{xy}{1-z}\right) + \alpha_3(\hat\bu)\frac{xy}{1-z}\\[6pt]
			 &   &\,+\,\alpha_6(\hat\bu)\left(x-\frac{xy}{1-z}+\frac{xyz}{(1-z)^2}\right)
		            +  \alpha_8(\hat\bu)\left(\frac{xy}{1-z}-\frac{xyz}{(1-z)^2}\right)\\[6pt]
			 & = &  \left(\alpha_1(\hat\bu) + \alpha_6(\hat\bu)\right)x +
			 		\left(\alpha_3(\hat\bu)-\alpha_1(\hat\bu)+\alpha_8(\hat\bu)-\alpha_6(\hat\bu)\right)\frac{xy}{1-z}\\[6pt]
			 &   &\,+ \left(\alpha_6(\hat\bu)-\alpha_8(\hat\bu)\right) \frac{xyz}{(1-z)^2}.
\end{IEEEeqnarray*}
Now we explore the new coefficients separately. Because $\hat\bu\cdot\hat\be_5$ equals zero,
\begin{IEEEeqnarray*}{rCl}
  \alpha_1(\hat\bu)+\alpha_6(\hat\bu) & = & \int_{\hat{\be}_1}\hat\bu\cdot\hat\btau_1\, ds +
  											\int_{\hat{\be}_6}\hat\bu\cdot\hat\btau_6\, ds -
  											\int_{\hat{\be}_5}\hat\bu\cdot\hat\btau_5\, ds \\[5pt]
  									  & = & \int_{\hat{f}_1} \nabla\times\hat\bu\cdot\hat\bn\,d\gamma \\[5pt]
  									  & = & -\int_{\hat{f}_1} \frac{{\s\partial} u_1}{{\s\partial} x_3}\,d\gamma.
\end{IEEEeqnarray*}
{\color{blue}\#\#\#\#\#\#\#\# continue here.}
\begin{IEEEeqnarray*}{rCl}
	\alpha_1 &=& \int\limits_{0}^{1}u_1(t,0,0)\,dt\\
	\alpha_3-\alpha_1+\alpha_8-\alpha_6&=&
	\int\limits_0^1u_1(t,1,0)\,dt-\int\limits_0^1u_1(t,0,0)\,dt-
	\int\limits_0^1u_1(1-t,1-t,t)\,dt+\int\limits_0^1u_1(1-t,0,t)\,dt\\
	&=&\int\limits_0^1\int\limits_0^1\frac{{\s\partial} u_1}{{\s\partial} x_2}(t,s,0)\,dsdt-
	\int\limits_0^1\int\limits_0^{1-t}\frac{{\s\partial} u_1}{{\s\partial} x_2}(1-t,s,t)\,dsdt\\
	&=&\int\limits_0^1\int\limits_0^1\frac{{\s\partial} u_1}{{\s\partial} x_2}(t,s,0)\,dsdt-
	\int\limits_0^1\int\limits_0^{t}\frac{{\s\partial} u_1}{{\s\partial} x_2}(t,s,1-t)\,dsdt\\
	&=&-\int\limits_0^1\int\limits_0^t\int\limits_0^{1-t}
		\partial^2_{x_3,x_2}u_1(t,s,r)\,drdsdt + 
	\int\limits_0^1\int\limits_t^{1}\frac{{\s\partial} u_1}{{\s\partial} x_2}(t,s,0)\,dsdt
\end{IEEEeqnarray*}

\begin{IEEEeqnarray*}{rCl}
	(\pi\bu)_1 & = & \alpha_1(1-z-y)+ 
			  		\alpha_3y+ 
			  		\alpha_6(-z+\frac{yz}{1-z})+ 
			  		\alpha_8(-\frac{yz}{1-z})\\
			  & = & \alpha_1 - (\alpha_1 + \alpha_6)z+ 
			  		(\alpha_3 - \alpha_1)y + (\alpha_6-\alpha_8)\frac{yz}{1-z}.
\end{IEEEeqnarray*}
\begin{IEEEeqnarray*}{rCl}
	\alpha_3-\alpha_1&=& \int\limits_{0}^{1}\int\limits_{0}^{1}\frac{{\s\partial} u_1}{{\s\partial} x_2}(t,s,0)\,dsdt
\end{IEEEeqnarray*}
\centerline{\rule{100pt}{0.5pt}}\newpage
\begin{IEEEeqnarray*}{rCl}
	\bu & = & (0,u_2,0)\\
	\nabla\times\bu &=& (-\frac{{\s\partial} u_2}{{\s\partial} x_3},0,\frac{{\s\partial} u_2}{{\s\partial} x_1})\\
	\pi\bu &=& \alpha_2\gamma_2 + \alpha_4\gamma_4+ \alpha_7\gamma_7+\alpha_8\gamma_8\\
	(\pi\bu)_1 & = &(\alpha_7-\alpha_8)\frac{yz}{1-z}\\
	\alpha_7-\alpha_8 & = & -\int\limits_0^1 u_2(0,1-t,t)\,dt
							+\int\limits_0^1 u_2(1-t,1-t,t)\,dt\\
					& = & \int\limits_0^1\int\limits_0^{1-t}\frac{{\s\partial} u_2}{{\s\partial} x_1}(s,1-t,t)\,dsdt.\\
	(\pi\bu)_2 & = &\alpha_4 + (\alpha_2-\alpha_4)x -
	(\alpha_4+\alpha_7)z + (\alpha_7-\alpha_8)\frac{xz}{1-z}.\\
	\alpha_4 & = & \int\limits_{0}^{1}u_2(0,t,0)\,dt\\
	\alpha_2-\alpha_4 & = & \int\limits_{0}^{1} u_2(1,t,0)-u_2(0,t,0)\,dt\\
		&=&\int\limits_{0}^{1}\int\limits_{0}^{1}\frac{{\s\partial} u_2}{{\s\partial} x_1}(s,t,0)\,dsdt\\
	\alpha_4+\alpha_7 & = & \int\limits_0^1 u_2(0,t,0)-u_2(0,t,1-t)\,dt\\
		& = & \int\limits_0^1\int\limits_0^{1-t}\frac{{\s\partial} u_2}{{\s\partial} x_3}(0,t,s)\,dsdt.\\
	\alpha_7-\alpha_8&=&\int\limits_{0}^{1} u_2(1-t,1-t,t)-u_2(0,1-t,t)\,dt\\
		&=&\int\limits_{0}^{1}\int\limits_{0}^{1-t}\frac{{\s\partial} u_2}{{\s\partial} x_1}(s,1-t,t)\,dsdt.
\end{IEEEeqnarray*}
\begin{IEEEeqnarray*}{rCl}
	(\pi\bu)_3&=&(\alpha_4+\alpha_7)y + (\alpha_2-\alpha_4-\alpha_7+\alpha_8)\frac{xy}{1-z}
	+(\alpha_7-\alpha_8)\frac{xyz}{(1-z)^2}.\\
	\alpha_4+\alpha_7 &=&-\int\limits_{0}^{1}\int\limits_{0}^{1-t}\frac{{\s\partial} u_2}{{\s\partial} x_3}(0,t,s)\,dsdt\\
	\alpha_7-\alpha_8 &=&\int\limits_{0}^{1}\int\limits_{0}^{1-t}\frac{{\s\partial} u_2}{{\s\partial} x_1}(s,1-t,t)\,dsdt.
\end{IEEEeqnarray*}
\begin{IEEEeqnarray*}{rCl}
	\alpha_2+\alpha_8&=&\int\limits_{e_2}\bu\cdot\btau+\int\limits_{e_8}\bu\cdot\btau
	+\int\limits_{e_6}\bu\cdot\btau\\
	&=&\iint\limits_{[a_2,a_4,a_5]}\nabla\times\bu\cdot\bn\,dS\\
	&=&\int\limits_{0}^{1}\int\limits_{0}^{t}-\frac{{\s\partial} u_2}{{\s\partial} x_3}(t,s,1-t) + \frac{{\s\partial} u_2}{{\s\partial} x_1}(t,s,1-t)\,dsdt.
\end{IEEEeqnarray*}
On the other hand, since $\bu$ is orthogonal to the tangent to $e_5$:
\begin{IEEEeqnarray*}{rCl}
	\alpha_4+\alpha_7&=&
		\int\limits_{e_4}\bu\cdot\btau+
		\int\limits_{e_7}\bu\cdot\btau+
		\int\limits_{e_5}\bu\cdot\btau\\[5pt]
		&=&\iint\limits_{[a_1,a_3,a_5]}\nabla\times\bu\cdot\bn\,dS\\[5pt]
		&=&-\int\limits_{0}^{1}\int\limits_{0}^{1-t}\frac{{\s\partial} u_2}{{\s\partial} x_3}(0,t,s)\,dsdt\\[5pt]
		&=&-\int\limits_{0}^{1}\int\limits_{t}^{1}\frac{{\s\partial} u_2}{{\s\partial} x_3}(0,t,1-s)\,dsdt\\[5pt]
		&=&-\int\limits_{0}^{1}\int\limits_{0}^{s}\frac{{\s\partial} u_2}{{\s\partial} x_3}(0,t,1-s)\,dtds\\[5pt]
		&=&-\int\limits_{0}^{1}\int\limits_{0}^{t}\frac{{\s\partial} u_2}{{\s\partial} x_3}(0,s,1-t)\,dsdt
\end{IEEEeqnarray*}
So
\begin{IEEEeqnarray*}{rCl}
	\alpha_2-\alpha_4-\alpha_7+\alpha_8 & = &
	\int\limits_{0}^{1}\int\limits_{0}^{t}
	-\frac{{\s\partial} u_2}{{\s\partial} x_3}(t,s,1-t) + \frac{{\s\partial} u_2}{{\s\partial} x_3}(0,s,1-t)\,dsdt\\
	&&\,+\int\limits_{0}^{1}\int\limits_{0}^{t}\frac{{\s\partial} u_2}{{\s\partial} x_1}(t,s,1-t)\,dsdt\\
	&=&\int\limits_{0}^{1}
	\int\limits_{0}^{t}
	\int\limits_{0}^{t}\frac{\partial^2u_2}{\partial x_1\partial x_3}(r,s,1-t)
	\,dr\,ds\,dt+
	\int\limits_{0}^{1}\int\limits_{0}^{t}\frac{{\s\partial} u_2}{{\s\partial} x_1}(t,s,1-t)\,ds\,dt.
\end{IEEEeqnarray*}
\centerline{\rule{100pt}{0.5pt}}
\begin{IEEEeqnarray*}{rCl}
	\bu & = & (0,0,u_3)\\
	\nabla\times\bu &=& (\frac{{\s\partial} u_3}{{\s\partial} x_2},-\frac{{\s\partial} u_3}{{\s\partial} x_1},0)\\
	\pi\bu &=& \alpha_5\gamma_5 + \alpha_6\gamma_6+ \alpha_7\gamma_7+\alpha_8\gamma_8\\
	(\pi\bu)_1 & = &(\alpha_5-\alpha_6)z+
		(-\alpha_5+\alpha_6+\alpha_7-\alpha_8)\frac{yz}{1-z}.\\
	\alpha_5-\alpha_6 & = &-\int\limits_{\mathcal{C}_{125}}\bu\cdot\btau\,d\sigma\\
	&=&-\iint\limits_{[a_1,a_2,a_5]}\nabla\times\bu\cdot\bn\,dS
\end{IEEEeqnarray*}
($\bn$ is the outer normal $(0,-1,0)$)
Similarly:
\begin{IEEEeqnarray*}{rCl} 	
	\alpha_7-\alpha_8 & = & 
	\iint\limits_{[a_3,a_5,a_4]}\nabla\times\bu\cdot\bn\,dS
\end{IEEEeqnarray*}
so {\color{red}($\nabla\times\bu\cdot(0,1,1)/\sqrt{2} = -\frac{{\s\partial} u_3}{{\s\partial} x_1}$)}
\begin{IEEEeqnarray*}{rCl}
	-\alpha_5+\alpha_6+\alpha_7-\alpha_8 & = & 
	\iint\limits_{[a_1,a_2,a_5]}\nabla\times\bu\cdot\bn\,dS
	+\iint\limits_{[a_3,a_5,a_4]}\nabla\times\bu\cdot\bn\,dS\\
	&=&\int\limits_{0}^{1}
	\int\limits_{0}^{1-t}\frac{{\s\partial} u_3}{{\s\partial} x_1} (s,0,t) \,ds\,dt
	-\int\limits_{0}^{1}
	 \int\limits_{0}^{1-t}\frac{{\s\partial} u_3}{{\s\partial} x_1} (s,1-t,t) \,ds\,dt\\
	&=&-\int\limits_{0}^{1}
	\int\limits_{0}^{1-t}
	\int\limits_{0}^{1-t}\frac{{\s\partial}^2u_3}{{\s\partial} x_2{\s\partial} x_1}(s,r,t)\,dr\,ds\,dt\\
	&=&-\iiint\limits_{\hat{P}}\frac{{\s\partial}^2u_3}{{\s\partial} x_2{\s\partial} x_1}\,dV.
\end{IEEEeqnarray*}
For now
\begin{IEEEeqnarray*}{rCl}
	(\pi\bu)_1 & = & -z\int\limits_{0}^{1-t}\frac{{\s\partial} u_3}{{\s\partial} x_1}\,ds\,dt
	-\frac{yz}{1-z}\iiint\limits_{\hat{P}}
		\frac{{\s\partial}^2u_3}{{\s\partial} x_2{\s\partial} x_1}\,dV.
\end{IEEEeqnarray*}
\begin{IEEEeqnarray*}{rCl}
	(\pi\bu)_2& = &(\alpha_5-\alpha_7)z
				+(-\alpha_5+\alpha_6+\alpha_7-\alpha_8)\frac{xz}{1-z}\\[5pt]
\end{IEEEeqnarray*}
\begin{IEEEeqnarray*}{rCl}
	\alpha_5-\alpha_7& = & \iint\limits_{[a_1,a_5,a_3]}\nabla\times\bu\cdot\bn\,dS\\
	& = &-\int\limits_{0}^{1}\int\limits_{0}^{1-t} \frac{{\s\partial} u_3}{{\s\partial} x_2}(0,s,t)\,ds\,dt
\end{IEEEeqnarray*}
\begin{IEEEeqnarray*}{rCl}
	(\pi\bu)_2& = &
	-z\int\limits_{0}^{1}\int\limits_{0}^{1-t}\frac{{\s\partial} u_3}{{\s\partial} x_2}(0,s,t)\,ds\,dt
	+\frac{xz}{1-z}\iiint\limits_{\hat{P}}
	\frac{{\s\partial}^2u_3}{{\s\partial} x_2{\s\partial} x_1}\,dV.
\end{IEEEeqnarray*}
\begin{IEEEeqnarray*}{rCl}
	(\pi\bu)_3&=& \alpha_5 + x(-\alpha_5+\alpha_6)+y(-\alpha_5+\alpha_7)\\[5pt]
	&&\,+(\alpha_5-\alpha_6-\alpha_7+\alpha_8)\frac{xy}{1-z}
	+(-\alpha_5+\alpha_6+\alpha_7-\alpha_8)\frac{xyz}{(1-z)^2}.
\end{IEEEeqnarray*}
\begin{IEEEeqnarray*}{rCl}
	\alpha_5&=&\int\limits_{0}^1u_3(0,0,t)\,dt.
\end{IEEEeqnarray*}
Joinig everything
\begin{IEEEeqnarray*}{rCl}
	(\pi\bu)_3 & = & \int\limits_{0}^1u_3(0,0,t)\,dt
				+ x \int\limits_{0}^{1}\int\limits_{0}^{1-t}
						\frac{{\s\partial} u_3}{{\s\partial} x_1} (s,0,t) \,ds\,dt
				+ y \int\limits_{0}^{1}\int\limits_{0}^{1-t}
						\frac{{\s\partial} u_3}{{\s\partial} x_2}(0,s,t) \,ds\,dt\\
				&&\,+ \frac{xyz}{(1-z)^2}\iiint\limits_{\hat{P}}
				\frac{{\s\partial}^2u_3}{{\s\partial} x_2{\s\partial} x_1}\,dV.
				-\frac{xz}{1-z}\iiint\limits_{\hat{P}}
				\frac{{\s\partial}^2u_3}{{\s\partial} x_2{\s\partial} x_1}\,dV.
\end{IEEEeqnarray*}
All together, for an $\bu=(u_1,u_2,u_3)$.
\begin{IEEEeqnarray*}{rCl}
	(\pi\bu)_1 & = & \int\limits_{0}^{1}u_1(t,0,0)\,dt + 
	z \int\limits_0^1\int\limits_0^{1-t}
	\frac{{\s\partial} u_1}{{\s\partial} x_3}(t,0,s)\,dsdt +
	y \int\limits_0^1\int\limits_0^{1}
	\frac{{\s\partial} u_1}{{\s\partial} x_2}(t,s,0)\,dsdt\\
	&&\,+\frac{yz}{1-z} \int\limits_0^1\int\limits_0^{1-t}
	\frac{{\s\partial} u_1}{{\s\partial} x_2}(1-t,s,t)\,dsdt +
	\frac{yz}{1-z} \int\limits_0^1\int\limits_0^{1-t}
	\frac{{\s\partial} u_2}{{\s\partial} x_1}(s,1-t,t)\,dsdt\\
	&&\,-z\int\limits_0^1\int\limits_0^{1-t}
	\frac{{\s\partial} u_3}{{\s\partial} x_1}(s,0,t)\,dsdt +
	\frac{yz}{1-z} \iiint\limits_{\hat{P}}
	\frac{{\s\partial}^2u_3}{{\s\partial} x_2{\s\partial} x_1}\,dV.
\end{IEEEeqnarray*}
\begin{IEEEeqnarray*}{rCl}
	(\pi\bu)_2 & = & \int\limits_{0}^{1}u_2(0,t,0)\,dt + 
	z \int\limits_0^1\int\limits_0^{1-t}
	\frac{{\s\partial} u_2}{{\s\partial} x_3}(0,t,s)\,dsdt +
	x \int\limits_0^1\int\limits_0^{1}
	\frac{{\s\partial} u_2}{{\s\partial} x_1}(s,t,0)\,dsdt\\
	&&\,+\frac{xz}{1-z} \int\limits_0^1\int\limits_0^{1-t}
	\frac{{\s\partial} u_1}{{\s\partial} x_2}(1-t,s,t)\,dsdt +
	\frac{xz}{1-z} \int\limits_0^1\int\limits_0^{1-t}
	\frac{{\s\partial} u_2}{{\s\partial} x_1}(s,1-t,t)\,dsdt\\
	&&\,-z\int\limits_0^1\int\limits_0^{1-t}
	\frac{{\s\partial} u_3}{{\s\partial} x_2}(0,s,t)\,dsdt +
	\frac{xz}{1-z} \iiint\limits_{\hat{P}}
	\frac{{\s\partial}^2u_3}{{\s\partial} x_2{\s\partial} x_1}\,dV.
\end{IEEEeqnarray*}
\begin{IEEEeqnarray*}{rCl}
	(\pi\bu)_3 & = & \int\limits_{0}^{1}u_3(0,0,t)\,dt + 
	x \int\limits_0^1\int\limits_0^{1-t}
	\frac{{\s\partial} u_3}{{\s\partial} x_1}(s,0,t)\,dsdt -
	x \int\limits_0^1\int\limits_0^{1-t}
	\frac{{\s\partial} u_1}{{\s\partial} x_3}(t,0,s)\,dsdt\\
	&&\,+y \int\limits_0^1\int\limits_0^{1-t}
	\frac{{\s\partial} u_3}{{\s\partial} x_2}(0,s,t)\,dsdt -
	y \int\limits_0^1\int\limits_0^{1-t}
	\frac{{\s\partial} u_2}{{\s\partial} x_3}(0,t,s)\,dsdt\\
	&&\,+\frac{xy}{1-z} \int\limits_0^1\int\limits_t^{1}
	\frac{{\s\partial} u_1}{{\s\partial} x_2}(t,s,0)\,dsdt +
	\frac{xy}{1-z} \int\limits_0^1\int\limits_0^{t}
	\frac{{\s\partial} u_2}{{\s\partial} x_1}(t,s,1-t)\,dsdt\\
	&&\,+\frac{xyz}{(1-z)^2} \int\limits_0^1\int\limits_0^{1-t}
	\frac{{\s\partial} u_2}{{\s\partial} x_1}(s,1-t,t)\,dsdt
	+\frac{xyz}{(1-z)^2} \int\limits_0^1\int\limits_0^{1-t}
	\frac{{\s\partial} u_1}{{\s\partial} x_2}(1-t,s,t)\,dsdt\\
	&&\,-\frac{xy}{1-z}
	\int\limits_{0}^{1}
	\int\limits_{0}^{t}
	\int\limits_{0}^{1-t}
	\frac{{\s\partial}^2u_1}{{\s\partial}x_2{\s\partial}x_3}(t,s,r)\,dr\,ds\,dt
	-\frac{xy}{1-z}
	\int\limits_{0}^{1}
	\int\limits_{0}^{t}
	\int\limits_{0}^{t}
	\frac{{\s\partial}^2u_2}{{\s\partial}x_1{\s\partial}x_3}(r,s,1-t)\,dr\,ds\,dt\\
	&&\,
	+\frac{xyz}{(1-z)^2} \iiint\limits_{\hat{P}}
	\frac{{\s\partial}^2 u_3}{{\s\partial} x_1{\s\partial} x_2}\,dV
	-\frac{xz}{1-z} \iiint\limits_{\hat{P}}
	\frac{{\s\partial}^2 u_3}{{\s\partial} x_1{\s\partial} x_2}\,dV
\end{IEEEeqnarray*}
{\color{red}controlar contra el papel y contra los signos
según las normales exteriores}

	% &=&-\int\limits_0^1\int\limits_0^{1-t}\frac{{\s\partial} u_3}{{\s\partial} x_1}(s,0,t)\,dsdt.


%	(\pi\bu)_2 & = &\alpha_4 + (\alpha_2-\alpha_4)x -
%	(\alpha_4+\alpha_7)z + (\alpha_7-\alpha_8)\frac{xz}{1-z}.\\
%	\alpha_4 = \int\limits_{0}^{1}u_2(0,t,0)\,dt\\
%	\alpha_2-\alpha_4 & = & \int\limits_{0}^{1} u_2(1,t,0)-u_2(0,t,0)\,dt\\
%		&=&\int\limits_{0}^{1}\int\limits_{0}^{1}\frac{{\s\partial} u_2}{{\s\partial} x_1}(s,t,0)\,dsdt\\
%	\alpha_4+\alpha_7 & = & \int\limits_0^1 u_2(0,t,0)-u_2(0,t,1-t)\,dt\\
%		& = & \int\limits_0^1\int\limits_0^{1-t}\frac{{\s\partial} u_2}{{\s\partial} x_3}(0,t,s)\,dsdt.\\
%	\alpha_7-\alpha_8&=&\int\limits_{0}^{1} u_2(1-t,1-t,t)-u_2(0,1-t,t)\,dt\\
%		&=&\int\limits_{0}^{1}\int\limits_{0}^{1-t}\frac{{\s\partial} u_2}{{\s\partial} x_1}(s,1-t,t)\,dsdt.

% subsection edge_elements (end)
\subsection{Anisotropic Stability Estimates for $H(\Div)$--Conforming 
Elements on Pyramids} % (fold)
\label{sub:face_elements}

\paragraph{Stability $\hat{E}$} 
\label{par:stability_hat}
\begin{theorem} \label{aux_label54}
\noindent{\color{Orange}\#\#\#\#\#\#\# cambiar H1 por W1p a la derecha.
corregir en las estimaciones de la demostración porque
está todo con H1} 
\begin{IEEEeqnarray*}{rCl}
  \|(\rku)_1\|_{\scriptscriptstyle{L^p(\hat{E})}}
  &\lesssim& \|\hat u_1\|_{\scriptscriptstyle{W^{1,p}(\hat{E})}} +
    \|\dv \bu\|_{\scriptscriptstyle{L^p}(\hat{E})} + 
    \left\|\hat{u}_3\right\|_{\scriptscriptstyle{H^1}(\hat{E})}\\[12pt]
  \|(\rku)_2\|_{\scriptscriptstyle{L^p(\hat{E})}}
  &\lesssim& \|\hat u_2\|_{\scriptscriptstyle{W^{1,p}(\hat{E})}} +
    \|\dv \bu\|_{\scriptscriptstyle{L^p}(\hat{E})} + 
    \left\|\hat{u}_3\right\|_{\scriptscriptstyle{H^1}(\hat{E})}\\[12pt]
  \|(\rku)_3\|_{\scriptscriptstyle{L^p(\hat{E})}} & \lesssim & 
    \|u_3\|_{\scriptscriptstyle{W^{1,p}(\hat{E})}} +
    \|\dv \bu\|_{\scriptscriptstyle{L^p}(\hat{E})}.
\end{IEEEeqnarray*}
\end{theorem}

\begin{proof}
We will use the notation of Table~\ref{shape_edge_table} for the 
shape functions and Tables~\ref{pyramidNotationTableFaces} and
~\ref{pyramidNotationTableEdges} for the border of the pyramid. The variables 
in the local coordinate system of $\hat E$ are $x,y$ and $z$.\\[5pt]
Consider the case $\hat{\bu} = (\hat{u}_1,0,0)'$ and compute it's interpolate. 
\begin{IEEEeqnarray*}{rCl}
  \rku & = & ({\scriptstyle\iint_{\hat{f}_2} \bu \cdot \hat\bn_2\,d\gamma})\,\bz_2 + 
         \iint_{\hat{f}_3} \bu \cdot \hat\bn_3\,d\gamma\,\bz_3\\[4pt]
       & = & \alpha_2(\hat\bu)\,\bz_2 + \alpha_3(\hat\bu)\,\bz_3.
\end{IEEEeqnarray*}
Then 
\begin{IEEEeqnarray*}{rCl}
  (\rku)_1\xyz & = & -2\alpha_2(\hat\bu) + 
    (\alpha_2(\hat\bu)+\alpha_3(\hat\bu))\,\frac{2x-xz}{1-z}\\[4pt]
    & = & -2{\iint_{\hat{f}_2} \hat{\bu} \cdot \hat\bn_2\,d\gamma} + 
          ({\iint_{\hat{f}_2} \hat{\bu} \cdot \hat\bn_2\,d\gamma}+
                  {\iint_{\hat{f}_3} \hat{\bu} \cdot \hat\bn_3\,d\gamma})\frac{2x-xz}{1-z}\\[4pt]
    & = & -2{\iint_{\hat{f}_2} \hat{\bu} \cdot \hat\bn_2\,d\gamma} + 
          {\iint_{\partial\hat{E}} \hat{\bu} \cdot \hat\bn\,d\gamma}\frac{2x-xz}{1-z}\\[4pt]
    & = & -2{\iint_{\hat{f}_2} \hat{\bu} \cdot \hat\bn_2\,d\gamma} + 
            {\int_{\hat{E}} \dv\hat{\bu} \,d\hat{\boldsymbol{x}}}\frac{2x-xz}{1-z}.\\[8pt]
  (\rku)_2\xyz & = & -(\alpha_2(\hat\bu)+\alpha_3(\hat\bu))\,\frac{yz}{1-z}\\[4pt]
    & = & -{\int_{\hat{E}} \dv\hat{\bu} \,d\hat{\boldsymbol{x}}}\frac{yz}{1-z}.
\end{IEEEeqnarray*}
Switch to $\hat{\bu} = (0,\hat{u}_2,0)'$.
\begin{IEEEeqnarray*}{rCl}
  \rku & = & ({\scriptstyle\iint_{\hat{f}_1} \bu \cdot \hat\bn_1\,d\gamma})\,\bz_1 + 
         \iint_{\hat{f}_4} \bu \cdot \hat\bn_4\,d\gamma\,\bz_4\\[4pt]
       & = & \alpha_1(\hat\bu)\,\bz_1 + \alpha_4(\hat\bu)\,\bz_4.
\end{IEEEeqnarray*}
Then
\begin{IEEEeqnarray*}{rCl}
  (\rku)_1\xyz & = & -(\alpha_1(\hat\bu)+\alpha_4(\hat\bu))\,\frac{xz}{1-z}\\[4pt]
    & = & -{\int_{\hat{E}} \dv\hat{\bu} \,d\hat{\boldsymbol{x}}}\frac{xz}{1-z}.\\[8pt]
  (\rku)_2\xyz & = & -2\alpha_1(\hat\bu) + 
  (\alpha_1(\hat\bu)+\alpha_4(\hat\bu))\,\frac{2y-yz}{1-z}\\[4pt]
    & = & -2{\iint_{\hat{f}_1} \hat{\bu} \cdot \hat\bn_1\,d\gamma} + 
            {\iint_{\partial\hat{E}} \hat{\bu} \cdot \hat\bn\,d\gamma}\frac{2y-yz}{1-z}\\[4pt]
    & = & -2{\iint_{\hat{f}_1} \hat{\bu} \cdot \hat\bn_1\,d\gamma} + 
            {\int_{\hat{E}} \dv\hat{\bu} \,d\hat{\boldsymbol{x}}}\frac{2y-yz}{1-z}.\\[8pt]
\end{IEEEeqnarray*}
Switch to $\hat{\bu} = (0,0,\hat{u}_3)'$.
\begin{IEEEeqnarray*}{rCl}
  \rku & = & ({\scriptstyle\iint_{\hat{f}_3} \bu \cdot \hat\bn_3\,d\gamma})\,\bz_3 + 
         \iint_{\hat{f}_4} \bu \cdot \hat\bn_4\,d\gamma\,\bz_4 + 
         \iint_{\hat{f}_5} \bu \cdot \hat\bn_5\,d\gamma\,\bz_5\\[4pt]
       & = & \alpha_3(\hat\bu)\,\bz_3 + \alpha_4(\hat\bu)\,\bz_4
       + \alpha_5(\hat\bu)\,\bz_5.
\end{IEEEeqnarray*}
Then
\begin{IEEEeqnarray*}{rCl}
  (\rku)_1\xyz & = & (\alpha_3(\hat\bu)+\alpha_5(\hat\bu))x
  + \alpha_3(\hat\bu) \frac{x}{1-z} - \alpha_4\frac{xz}{1-z}.
\end{IEEEeqnarray*}
Now observe
\begin{IEEEeqnarray*}{rCl}
  (\alpha_3(\hat\bu)+\alpha_5(\hat\bu)) & = & 
    {\iint_{\partial\hat{E}} \hat{\bu} \cdot \hat\bn\,d\gamma} - 
      {\iint_{\hat{f}_4} \hat{\bu} \cdot \hat\bn_4\,d\gamma} \\[4pt]
  & = & {\int_{\hat{E}} \dv\hat{\bu}\,d\hat{\boldsymbol{x}}} - 
        \alpha_4(\hat{\bu})
\end{IEEEeqnarray*}
and
\begin{IEEEeqnarray*}{rCl}
  \alpha_3(\hat\bu)-\alpha_4(\hat\bu) & = & 
  {\iint_{\hat{f}_3} \bu \cdot \hat\bn_3\,d\gamma} - 
  {\iint_{\hat{f}_4} \bu \cdot \hat\bn_4\,d\gamma} \\[4pt]
  & = & \int_{0}^{1}\int_{0}^{x} \hat{u}_3(x,y,1-x)\,dydx - 
        \int_{0}^{1}\int_{0}^{y} \hat{u}_3(x,y,1-y)\,dxdy\mbox{,}
\end{IEEEeqnarray*}
so
\begin{IEEEeqnarray*}{rCl}
  (\rku)_1\xyz & = & {\int_{\hat{E}} \dv\hat{\bu}\,d\hat{\boldsymbol{x}}}\,x +\\[4pt]
  \IEEEeqnarraymulticol{3}{r}{\qquad(\int_{0}^{1}\int_{0}^{x} \hat{u}_3(x,y,1-x)\,dydx - 
                          \int_{0}^{1}\int_{0}^{y} \hat{u}_3(x,y,1-y)\,dxdy)
                        \frac{x}{1-z}.}
\end{IEEEeqnarray*}
In a completely similar fashion
\begin{IEEEeqnarray*}{rCl}
  (\rku)_2\xyz & = & {\int_{\hat{E}} \dv\hat{\bu}\,d\hat{\boldsymbol{x}}}\,y\,+
  ({\iint_{\hat{f}_3} \bu \cdot \hat\bn_3\,d\gamma} - 
   {\iint_{\hat{f}_4} \bu \cdot \hat\bn_4\,d\gamma})\frac{y}{1-z}.\\[4pt]
               & = & {\int_{\hat{E}} \dv\hat{\bu}\,d\hat{\boldsymbol{x}}}\,y\,+\\[4pt]
  \IEEEeqnarraymulticol{3}{r}{\qquad(\int_{0}^{1}\int_{0}^{x} \hat{u}_3(x,y,1-x)\,dydx - 
                          \int_{0}^{1}\int_{0}^{y} \hat{u}_3(x,y,1-y)\,dxdy)
                        \frac{y}{1-z}.}
\end{IEEEeqnarray*}
We collect every term obtained so far for the first and second components in
Table~\ref{terms_table}.
\begin{table}[!h]
    \centering  
    \caption{Terms\\[4pt]$q(s,t) = \frac{2s-st}{1-t},\,r(s,t) = \frac{st}{1-t}$}
    \label{terms_table}
    \begin{IEEEeqnarraybox*}
    [\IEEEeqnarraystrutmode
    \IEEEeqnarraystrutsizeadd{2pt}{12pt}]{v/c/v/c/v/c/v/}
        \IEEEeqnarrayrulerow\\
        \IEEEeqnarrayseprow[5pt]\\
        & & & (\rku)_1 & & (\rku)_2 & \\
        \IEEEeqnarrayrulerow\\
        \IEEEeqnarrayseprow[5pt]\\
        & (\hat{u}_1,0,0)' & &
          \begin{IEEEeqnarraybox*}{l}
            -2{\iint_{\hat{f}_2} \hat{\bu} \cdot \hat\bn_2\,d\gamma}\\ + 
            {q(x,z)\int_{\hat{E}} \dv\hat{\bu} \,d\hat{\boldsymbol{x}}}
          \end{IEEEeqnarraybox*}
        & &
          -r(y,z){\int_{\hat{E}} \dv\hat{\bu} \,d\hat{\boldsymbol{x}}} &\\
        \IEEEeqnarrayrulerow\\
        \IEEEeqnarrayseprow[5pt]\\
        & (0,\hat{u}_2,0)' & & 
          -r(x,z){\int_{\hat{E}} \dv\hat{\bu} \,d\hat{\boldsymbol{x}}} 
        & & 
          \begin{IEEEeqnarraybox*}{l}
            -2{\iint_{\hat{f}_1} \hat{\bu} \cdot \hat\bn_1\,d\gamma}\\ + 
            {q(x,z)\int_{\hat{E}} \dv\hat{\bu} \,d\hat{\boldsymbol{x}}}
          \end{IEEEeqnarraybox*}
        &\\
        \IEEEeqnarrayrulerow\\
        \IEEEeqnarrayseprow[5pt]\\
        & (0,0,\hat{u}_3)' & & 
          \begin{IEEEeqnarraybox*}{l}
            x\int_{\hat{E}} \dv\hat{\bu}\,d\hat{\boldsymbol{x}} \\[5pt] +\, 
            ({\iint_{\hat{f}_3} \bu \cdot \hat\bn_3\,d\gamma}
             \\[5pt] 
             -{\iint_{\hat{f}_4} \bu \cdot \hat\bn_4\,d\gamma})r(x,z)
          \end{IEEEeqnarraybox*}
         & & 
          \begin{IEEEeqnarraybox*}{l}
            y\int_{\hat{E}} \dv\hat{\bu}\,d\hat{\boldsymbol{x}}\\[5pt] +\, 
              ({\iint_{\hat{f}_4} \bu \cdot \hat\bn_4\,d\gamma}
             \\[5pt] 
             -{\iint_{\hat{f}_3} \bu \cdot \hat\bn_3\,d\gamma})r(y,z)
          \end{IEEEeqnarraybox*}
        &\\\IEEEeqnarrayrulerow
    \end{IEEEeqnarraybox*}
\end{table}
Lastly, for any $\hat\bu$
\begin{IEEEeqnarray*}{rCl}
    (\rku)_3\xyz & = &   z\sum_{i=1}^4\iint_{\hat{f}_i} \hat{\bu}\cdot\hat{\bn}_i\,d\gamma
                      + (z-1) \iint_{\hat{f}_5}\hat{\bu}\cdot\hat{\bn}_5\,d\gamma\\[5pt]
                 & = & z\iint_{\partial\hat{E}} \hat{\bu}\cdot\hat{\bn} - \iint_{f_5}
                 \hat{\bu}\cdot\hat{\bn}_5\,d\gamma\\[5pt]
   \yesnumber\label{term_rk3}
                 & = &\hat{x}_3\int\limits_{\hat{E}} \mbox{div}\,\hat{\bu}\,d\hat{\bx} 
     + \iint\limits_{\hat{f}_5} \hat{u}_3\,d\hat{\gamma}
\end{IEEEeqnarray*}
Now we bound each term 
\begin{IEEEeqnarray*}{rCl}
  (\rku)_1 & = & (\br_{\hat{E}}(\hat{u}_1,0,0)')_1 + 
                 (\br_{\hat{E}}(0,\hat{u}_2,0)')_1 + 
                 (\br_{\hat{E}}(0,0,\hat{u}_3)')_1\\[5pt]
  & = & -2\iint_{\hat{f}_2}\hat{u}_1\,d\gamma +
  \frac{2x}{1-z}\int_{\hat{E}} \frac{\partial\hat{u}_1}{\partial\hat{x}_1}\,d\hat{\bx} -
  \frac{xz}{1-z}\int_{\hat{E}} \frac{\partial\hat{u}_1}{\partial\hat{x}_1}\,d\hat{\bx} \\[5pt]
  & & \,- \frac{xz}{1-z}\int_{\hat{E}} \frac{\partial\hat{u}_2}{\partial\hat{x}_2}\,d\hat{\bx}
  + \left( x + \frac{xz}{1-z}-\frac{xz}{1-z} \right)
  \int_{\hat{E}} \frac{\partial\hat{u}_3}{\partial\hat{x}_3}\,d\hat{\bx}\\[5pt]
  &   &\, + \left({\iint_{\hat{f}_3} \hat{u}_3\,d\gamma}
        - {\iint_{\hat{f}_4} \hat{u}_3\,d\gamma}\right)\frac{xz}{1-z}\\[5pt]
  & = & -2\iint_{\hat{f}_2}\hat{u}_1\,d\gamma +
  \frac{2x}{1-z}\int_{\hat{E}}\frac{\partial\hat{u}_1}{\partial\hat{x}_1}\,d\hat{\bx} -
  \frac{xz}{1-z}\int_{\hat{E}}\dv\hat{\bu}\,d\hat{\bx}\\[5pt]
  \yesnumber\label{face_integrals}
  &  & \,+\frac{x}{1-z}\int_{\hat{E}}\frac{\partial\hat{u}_3}{\partial\hat{x}_3}\,d\hat{\bx}
  + \left({\iint_{\hat{f}_3} \hat{u}_3\,d\gamma}
        - {\iint_{\hat{f}_4} \hat{u}_3\,d\gamma}\right)\frac{xz}{1-z}
\end{IEEEeqnarray*}
For the surface integrals in~(\ref{face_integrals}), by Theorem 3.9 in~\cite{monk}, page 43,
\begin{IEEEeqnarray*}{rCl}
  \left|\iint_{\hat{f}_2} \hat{u}_1\,d\gamma\right| 
  & \leqslant & 2^{-1/4}\,\|\mbox{Tr}_{\hat{f}_2}\hat{u}_1\|_{L^2(\hat{f}_3)} \\[5pt]
  & \leqslant & C\,\|\mbox{Tr\,}\hat{u}_1\|_{H^{\delta}(\partial\hat{E})} \\[5pt]
  & \leqslant & C\,\|\hat{u}_1\|_{H^{1/2+\delta}(\hat{E})}. \\[5pt]
  & \leqslant & C\,\|\hat{u}_1\|_{H^{1}(\hat{E})}.
\end{IEEEeqnarray*}
and similarly
\begin{IEEEeqnarray*}{rCl}
  \left|\iint_{\hat{f}_3} \hat{u}_3\,d\gamma - \iint_{\hat{f}_4} \hat{u}_3\,d\gamma\right| 
  & \leqslant & C\,\|\hat{u}_3\|_{H^{1}(\hat{E})}
\end{IEEEeqnarray*}
all of which leads to
\begin{IEEEeqnarray*}{rCl}
  \|(\rku)_1\|_{L^{\infty}(\hat{E})} & \leqslant & C_{\hat{E}} 
  \left[ 
    \|\hat{u}_1\|_{H^{1}(\hat{E})} + 
    \|\dv\hat{\bu}\|_{L^{2}(\hat{E})} + 
    \|\hat{u}_3\|_{H^{1}(\hat{E})}
  \right].
\end{IEEEeqnarray*}
Copying the argument for the second component
\begin{IEEEeqnarray*}{rCl}
  \|(\rku)_1\|_{L^{\infty}(\hat{E})} & \leqslant & C_{\hat{E}} 
  \left[ 
    \|\hat{u}_1\|_{H^{1}(\hat{E})} + 
    \|\dv\hat{\bu}\|_{L^{2}(\hat{E})} + 
    \|\hat{u}_3\|_{H^{1}(\hat{E})}
  \right].
\end{IEEEeqnarray*}
From~(\ref{term_rk3}) we deduce
\begin{IEEEeqnarray*}{rCl}
  \|(\rku)_3\|_{\scriptscriptstyle{L^\infty(\hat{E})}} & \leqslant & C_{\hat{E}}
    \left[\|u_3\|_{\scriptscriptstyle{H^{1}(\hat{E})}} +
    \|\dv \bu\|_{\scriptscriptstyle{L^2}(\hat{E})}\right].
\end{IEEEeqnarray*}
The quantity $C_{\hat{E}}$ depends only on the supremum of the (fixed)
basis shape functions of Table~\ref{shape_face_table} over the pyramid.
\end{proof}
% subsection face_elements (end)

%% ============================================================================
%% TODO: ver si esto finalmente va
%% \subsection{Local Interpolation Estimates for Pyramidal Finite Elements} % (fold)
%% \label{sub:local_interpolation_estimates_for_pyramidal_elements}
%% decir que permitimos pirámides elongadas perpendicularmente a la base
%% $h_3\geqslant C\min\{h_1,h_2\}$
%% Verificar si es esto o $h3 >= max (h1, h2)$\\
%% poner tres dibujos con casos $h1=h2<h3$; $h1<h2<h3$; $h2<h1<h3$
% subsection local_interpolation_estimates_for_pyramidal_elements (end)
%% ============================================================================


% section pyramidal_finite_elements (end)
\section{Local estimates for Pyramidal Virtual Elements}

\begin{theorem}
If $E$ is an isotropic tetrahedron or pyramid, then  
\begin{equation}\label{estab2}
\|\bu_I\|_{L^p(E)}\leqslant C\left(\|\bu\|_{L^p(E)}+ h_T|\bu|_{W^{1,p}(E)}\right), \qquad \forall \bu\in W^{1,p}(E),
\end{equation}
with the constant $C$ depending on the shape regularity of $E$, and $1\leqslant p$ if $E$ is a tetrahedron and $1\leqslant p\leqslant 2$ if $E$ is a pyramid.
\end{theorem}
\begin{proof}
{\color{blue}\#\#\#\#\#\#\#\# esto no ponerlo, al momento de usar tetra para
malla, citar del articulo\\.}
When $E$ is a tetrahedron, this result is contained in \cite{aadl}. So we assume that $E$ is a pyramid.\\ 
{\color{blue}\#\#\#\#\#\#\#\#\\}
We note that 
\begin{equation}\label{estab2:eq1}
\bu_I=\sum_{i=1}^5 \left(\int_{f_i}\bu\cdot \bn\right)\bv_i
\end{equation}
where $\{\bv_i\}_{i=1}^5$ is the basis of $V_h(E)$ dual to the degrees of freedom
\ref{dofs}. Denote by $f_j$, $j=1,\ldots,5$ the faces of $E$. First of all we need to estimate the $L^2$-norm of the basis functions $\bv_i$. Fixed $1\leqslant i\leqslant 5$, it follows from the proof of Lemma \ref{existenciaInterpolante} that $\bv_i=\nabla \psi$ where $\psi$ is the solution of
\begin{eqnarray*}
\Delta\psi&=&d\qquad\mbox{in }\Omega\\ \frac{\partial\psi}{\partial\bn}&=&g\qquad\mbox{on }\partial E\\ \int_E\psi&=&0
\end{eqnarray*}
with
\[
g|_{f_j}=\left\{\begin{array}{cl}\frac1{|f_i|}&\mbox{if }i=j\\0&\mbox{if }i\ne j\end{array}\right., \qquad d=\frac1{|E|}.
\]
Multiplying the first equation defining $\psi$ by $\psi$, and integrating by parts, we obtain
\begin{eqnarray*}
\|\nabla\psi\|_{L^2(E)}^2 &=& -\int_Ed\psi + \int_{\partial E}g\psi\\ &\leqslant & 
\|d\|_{L^2(E)}\|\psi\|_{L^2(E)} + \|g\|_{L^2(\partial E)}\|\psi\|_{L^2(\partial E)}.
\end{eqnarray*}
Using Poincare's and trace inequalities we have for, a constant $C$ depending on the aspect ratio of $E$ 
\begin{equation}\label{estab2:eq2}
\|\nabla\psi\|_{L^2(E)}^2\leqslant  C\left(h_E\|d\|_{L^2(E)} + h_E^\frac12 \|g\|_{L^2(\partial E)}\right) \|\nabla\psi\|_{L^2(E)}.
\end{equation}
Taking into account the definitions of $d$ and $g$ we have
\[
\|d\|_{L^2(E)}\leqslant Ch_E^{-\frac32}, \qquad \|g\|_{L^2(\partial E)}\leqslant Ch_E^{-1}
\]
and so from \eqref{estab2:eq2} we obtain
\begin{equation}\label{estab2:eq3}
\|\bv_i\|_{L^2(E)}=\|\nabla\psi\|_{L^2(E)}\leqslant Ch_E^{-\frac12}.
\end{equation}

Now, for $1\leqslant p\leqslant 2$ using H\"older's inequality and the expression \eqref{estab2:eq1} we have
\begin{eqnarray*}
\|\bu_I\|_{L^p(E)}&\le& |E|^{\frac1p-\frac12}\|\bu_I\|_{L^2(T)}\\&\le&|E|^{\frac1p-\frac12} \sum_{i=1}^5  \left|\int_{f_i}\bu\cdot\bn\right|\|\bv_i\|_{L^2(E)}.
\end{eqnarray*}
By using \eqref{estab2:eq3}, H\"older's inequality, trace inequalities and taking into account the shape-regularity of $E$ we obtain
\begin{eqnarray*}
\|\bu_I\|_{L^p(E)}&\le& C |E|^{\frac1p-\frac12} h_E^{-\frac1p}\left(\|\bu\|_{L^p(\Omega)}+h_E\|\nabla\bu\|_{L^p(E)}\right)|\partial E|^{1-\frac1p} \|\bv_i\|_{L^2(E)}\\&\le& C h_E^{3\left(\frac1p-\frac12\right)}h_E^{-\frac1p}h_E^{2\left(1-\frac1p\right)} h_E^{-\frac12}\left(\|\bu\|_{L^p(\Omega)}+h_E\|\nabla\bu\|_{L^p(E)}\right)\\&=& C \left(\|\bu\|_{L^p(\Omega)}+h_E\|\nabla\bu\|_{L^p(E)}\right)
\end{eqnarray*}


%\sum_{i=1}^5 \left|\int_{f_i}\bu\cdot\bn\right|\|\bv_i\|_{L^2(E)}\\&\le& \sum_{i=1}^5 |f_i|^\frac12\|\bu\|_{L^2(f_i)}\|\bv_i\|_{L^2(E)}\\ &\le& C|f_i|^\frac12\left(h_e^{-\frac12}\|\bu\|_{L^2(E)}+h_E^\frac12\|\nabla\bu\|_{L^2(E)}\right)\|\bv_i\|_{L^2(E)}\\ &\le& C\left(\|\bu\|_{L^2(E)}+h_E\|\nabla\bu\|_{L^2(E)}\right)

where $C$ depends on the shape regularity of $E$.
\end{proof}

\begin{proposition}\label{propErrorInterpolacionPiramidesTetraedros}
Let $E$ be a tetrahedron or a pyramid satisfying the shape-regularity property with constant $\sigma$. Then there exists a constant $C$ depending only on $\sigma$ such that 
\[
\|\bu-\bu_I\|_{L^2(E)}\leqslant C h_E|\bu|_{H^1(E)} \qquad \forall \bu\in H^1(E).
\]
\end{proposition}
\begin{proof} Let $Q\bu$ be the $L^2(E)$-projection of $u$ onto the constant fields. Then we have
\[
\bu-\bu_I = (\bu - Q\bu) + (Q\bu-\bu)_I
\]
and using the previous Lemma and a clasical estimate for the $L^2(E)$-projection error we have
\begin{eqnarray*}
\|\bu-\bu_I\|_{L^2(E)} &\le& \|\bu - Q\bu\|_{L^2(E)} + \|(\bu-Q\bu)_I\|_{L^2(E)}\\ &\le& \|\bu - Q\bu\|_{L^2(E)} + C \left(\|\bu-Q\bu\|_{L^2(E)} + h_E\|\nabla(\bu-Q\bu)\|_{L^2(E)}\right)\\ &=& C\left(\|\bu-Q\bu\|_{L^2(E)} + h_E\|\nabla\bu\|_{L^2(E)}\right)\\&\le& Ch_E\|\nabla\bu\|_{L^2(E)}
\end{eqnarray*}
as we wnated to prove.
\end{proof}


\begin{proposition}\label{propupi}
Let $E$ be a pyramid satisfying a shape regularity property with constant $\sigma$, and $\bu\in H^1(E)$. 
Then there exists a field $\bu_\pi\in W(E)$ such that
\[
\|\bu-\bu_\pi\|_{L^2(E)}\leqslant C h_E|\bu|_{H^1(E)}.
\]
(Poner directo) Then
\[
\|\bu-P^{\perp}\bu\|_{L^2(E)}\leqslant C h_E|\bu|_{H^1(E)}.
\]
\end{proposition}
\begin{proof}
We can define $\bu_\pi$ on $E$ as the $L^2(E)$-projection of $\bu$ on the space
of constant fields $P_0(E)^3\subset W(E)$. The error estimate follows from 
Lemma~\ref{aux_label40}	
by noticing that, at low order, the averaged Taylor is the orthogonal projection
onto de constants.
\end{proof}
\begin{remark}
  we could replicate the results por prisms, at low order. We are not
  applying pyramidal FE interpolation, though.
\end{remark}

\chapter{Approximation}
\section{regularity} % (fold)
\label{sec:regularity}
\begin{defi}
  If $\Omega$ is a non--convex polihedron $\lambda_e$ is ... and $\lambda_v$ is ---	
\end{defi}
\noindent Assuming a decomposition of $\Omega=\cup_{\ell=1}^N \Lambda_\ell$ in tetrahedral macroelements having at most a singular edge and a singular vertex, we have the following regularity result. First we introduce the space $V^{1,2}_{\beta,\delta}(\Lambda)$ for a macroelement $\Lambda$ as
\[
V^{1,2}_{\beta,\delta} = \left\{v\in \mathcal D'(\Lambda): R^{\beta-1+|\alpha|}\theta^{\delta-1+|\alpha|}D^\alpha v\in L^2(\Lambda), \alpha\in \mathbb N^3, |\alpha|\le1\right\}
\]
where $R({\bf x})$ is the distance of ${\bf x}$ to the vertices of $\Lambda$, $r({\bf x})$ is the distance from ${\bf x}$ to the edges of $\Lambda$ and finally $\theta({\bf x})$ is the angular distance $\theta({\bf x})=\frac{r({\bf x})}{R({\bf x})}$.
\begin{theorem}
The solutions $\bu$ and $p$ of problem \eqref{mixedContinuous} satisfy
\[
p\in H^1(\Omega)
\] 
and for each $\ell$
\[
\bu=\bu_r + \bu_s
\]
with $\bu_r\in H^1(\Omega)$ and
\[
\bu_s\cdot \xi_i\in V^{1,2}_{\beta,\delta}(\Lambda_\ell), \quad i=1,2, \qquad \bu_s\cdot\xi_3\in V^{1,2}_{\beta,0}(\Lambda_\ell)
\]
where $\xi_i$, $i=1,2,3$, are the directions of three concurrent edges of $\Lambda_\ell$ with $\xi_3$ being the direction of the singular edge if it exists in $\Omega_\ell$, and $\beta,\delta\ge0$ satisfying $\beta>\frac12-\lambda_v^{(\ell)}$ and $\delta>1-\lambda_e^{(\ell)}$, $v$ and $e$ being the singular vertex and edge, respectively, if they exist.
\end{theorem} 

\begin{remark}\label{sobreBetaYDelta}
\textcolor{red}{Note that it is always possible in the previous Theorem to take $0<\beta=\delta<1$.} 
\end{remark}
% section regularity (end)
\chapter{Further Results on Finite Elements}
\label{auxlabel202}
\section{An\-iso\-tropic Stability Estimates for $H(\bcurl)$ Conforming Finite
Elements on Prisms}
\label{stab_edge_prism}
We are writing the following three Lemmas which state some special
behavior of the interpolation operator, mostly related to the preservation of
null components of the fields, and whose proofs consist in a smart use of the
degrees of freedom and the very definition of the operator. These Lemmas, 
although just with technical purpose, exhibit
nice properties of the interpolators.

In the present subsection $\hat\bu$ is an element
in $W^{1,p}(\hat{E})$ for $p>2$ which is a space whose elements have well
defined tangential traces on each edge of the prism $\hat{E}$.
Another possibility, as stated in Lemma $5.38$ in the page $134$ of~\cite{monk},
is to assume there are 
a positive $\delta$ and a $p>2$ such that 
$\hat\bu$ belongs to $H^{1/2+\delta}(\hat{E})^3$ and
${\bf curl}\,\bu$ belongs to $L^p(\hat{E})^3$.
For the whole section, $\hat\bw_k$ will be the $k$--th order edge 
interpolation operator on the reference
Prism determined by the element of
\emph{Definition}~\ref{edgeelement}.

\begin{lemma}\label{lema_PIu3_k_cualquiera} 
$(\wku)_3$ is linearly and univocally 
determined by $\hat{u}_3$.
\end{lemma}
\begin{proof} If we pay attention to the directions of the unit
tangents and normals to the edges and faces, respectively, of $\hat E$,
we realize that
the degrees of freedom which involve $(\wku)_3$ give rise only to the 
following linear equations
\begin{IEEEeqnarray}{rCrc}
\varphi_{\hat{\be}_i,p}\,(\wku) & = & \varphi_{\hat{\be}_i,p}\,(\hat{\bu}) &\quad\mbox{as in~(\ref{momentos1hcurl}) for $i$ = 3, 6, 7,}\\
\varphi_{f_1,\bq}\,(\wku) & = & \varphi_{f_1,\boldsymbol{q}}\,(\hat{\bu})
  &\quad\mbox{as in~(\ref{momentos3hcurl})}\\
\varphi_{f_2,\bq}\,(\wku) & = & \varphi_{f_2,\boldsymbol{q}}\,(\hat{\bu})
  &\quad\mbox{as in~(\ref{momentos4hcurl})}  \\
\varphi_{f_5,\bq}\,(\wku) & = & \varphi_{f_5,\boldsymbol{q}}\,(\hat{\bu})
  &\quad\mbox{as in~(\ref{momentos5hcurl})}  \\
\varphi_{\boldsymbol{r}}\,(\wku) & = & \varphi_{\boldsymbol{r}}\,(\hat{\bu})
  &\quad\mbox{as in~(\ref{momentos6hcurl})}.
\end{IEEEeqnarray}
These are 
$3k$+$3k(k-1)$+$k(k-1)(k-2)/2 = k(k+1)(k+2)/2$ equations,
just the dimension of $P_k(\hat{T})\otimes P_{k-1}(\hat{I})$, 
which is the space $(\wku)_3$ belongs to by definition.
%$\frac{k(k+1)(k+2)}{2}$. 
%libertad en los que no des\-a\-pa\-re\-ce $(\wku)_3$ son \'unicamente:
Now set all those equations equal to zero (that is, pick $u_3 = 0$) and see that
the unique solution is $(\wku)_3 = 0$.
A little more explicitly, we have:
\begin{IEEEeqnarray}{lCll}
  \label{aristas} \int_{\hat\be_i} (\wku)_3\,\hat q \, d\alpha 
  & = & 0 &\qquad \mbox{for $i$ = 3, 6 and 7, }q\in P_{k-1}(\hat\be_i)\\[5pt]
  \label{caras} \iint_{\hat f_j} (\wku)_3\,\hat q \,d\hat S
  & = & 0 &\qquad \mbox{for $j$ = 1, 2, and 5, } \hat q\in Q_{k-2,k-1}(\hat f_j)\\[5pt]
  \label{enK} \int_{\hat{E}} (\wku)_3\,\hat q_3 \, d\bx 
  & = & 0 &\qquad \mbox{for }\hat q_3\in P_{k-3,k-1}.
\end{IEEEeqnarray}
Start considering the face $\hat f_2$.
The restriction of $(\wku)_3$ to $\be_3$
is itself 
an element in $P_{k-1}({\be_3})$, 
and the same holds for $\be_6$,  
so equations~(\ref{aristas}), for $i = 3$, $6$, say that $(\wku)_3$
is identically null on those edges by which
the restriction 
$(\wku)_3|_{f_2}$, which is an element of $P_k(\hat x_1)\otimes P_{k-1}(\hat x_3)$
may be se factorized
as
$(\wku)_3|_{f_2}(\hat x_1,0,\hat x_3) = \hat x_1\,(1-\hat x_1)\,w_0(\hat x_1,\hat x_3)$,
with $w_0$ equal to some polynomial in $P_{k-2}(\hat x_1) \otimes P_{k-1}(\hat x_3)$.
Now choose $\hat q = w_0$ in the degrees of freedom~(\ref{caras}) for the face
$\hat f_2$ and it holds
\begin{IEEEeqnarray}{lClc}
	\iint_{\hat f_2} \hat{x}_1(1-\hat{x}_1)w_0(\hat{x}_1,\hat{x}_3)^2\,d\hat S & = & 0.
\end{IEEEeqnarray}
But $\hat{x}_1\,(1-\hat{x}_1)$ is almost everywhere positive over the closure
of $\hat f_2$, so 
$(\wku)_3|_{\hat f_2}$ vanishes identically.
By a completely symmetric observation we can prove 
$(\wku)_3|_{\hat f_1} \equiv 0$.\\
One more time, if we use~(\ref{aristas}) for $i =$ 6, 7,
$\hat x_1\,(1-\hat x_1)$ divides the restriction $(\wku)_3|_{\hat f_5}$, so that
there is a 
$w_1 \in P_{k-2}(\hat x_1)\otimes P_{k-1}(\hat x_3)$ for which 
$(\wku)_3|_{\hat f_5}(\hat x_1, \hat x_2, \hat x_3) = \hat{x}_1\,\displaystyle{(1-\hat{x}_1)}\,w_1(\hat{x}_1,\hat{x}_3)$.
Now equality~(\ref{caras}) for $j = 5$ implies
$(\wku)_3|_{f_5} \equiv 0$.\\
Next, since $(\wku)_3$ vanishes when restricted $\hat f_1$, $\hat f_2$ and $\hat f_5$
we get to factorize it on $\hat E$ as 
\begin{IEEEeqnarray*}{rCl}
	(\wku)_3(\hat x_1, \hat x_2, \hat x_3) 	& = 	& \hat{x}_1\,\hat{x}_2\,(1-\hat{x}_1-\hat{x}_2)\,w_3(\hat{x}_1,\hat{x}_2,\hat{x}_3),\\
									w_3		& \in 	& P_{k-3}(\hat x_1)\otimes P_{k-1}(\hat x_3).
\end{IEEEeqnarray*}
And now we evaluate the degrees of freedom~(\ref{enK}) choosing
$\hat q_3 \equiv w_3$ and conclude immediately
$(\wku)_3 \equiv 0$ on $\hat E$. 
\end{proof}
\label{auxlabel500}
\begin{lemma}\label{auxlabel501}
\begin{itemize}
	\item []
	\item [(a)] If $\hat\bu(\hat x_1,\hat x_2,\hat x_3) = (0, \hat u_2(\hat x_2,\hat x_3), 0)'$,
	then 
  \[
  \wku\xyz = (0, \hat\xi_2(\hat x_2,\hat x_3) ,0)'
  \]
  for some 
	$\hat\xi_2 \in P_{k-1}(\hat{x}_2) \otimes P_k(\hat{x}_3)$.
	\item [(b)]\label{piu1_k_in_N} If $\hat\bu(\hat x_1,\hat x_2,\hat x_3) = (\hat u_1(\hat x_1,\hat x_3), 0, 0)'$
	then
  \[
  \wku\xyz = (\hat\xi_1(\hat x_1,\hat x_3), 0 ,0)'
  \]
  for some
    $\hat\xi_1\in P_{k-1}(\hat{x}_1) \otimes P_k(\hat{x}_3)$.
\end{itemize}
\end{lemma}
\begin{proof} We will prove the first inequality, as the second follows
with the same ideas. In Subsection~\ref{sub:defEdgeElement} we found 
expression~(\ref{elemento_P_k}) which states
\begin{IEEEeqnarray*}{rCl}
  \wku\xyz  & = & (p_1\xyz, p_2\xyz, p_3\xyz)^t\\[4pt]
  			    & = & \begin{pmatrix}
  					        \xi_1\xyz + \hat{x}_2\,h\xyz\\
                    \yesnumber\label{expr_wku}\xi_2\xyz - \hat{x}_1\,h\xyz \\
  					        \xi_3\xyz
  				        \end{pmatrix}
\end{IEEEeqnarray*}
for
$\xi_1$ and $\xi_2$ in $P_{k-1}(\hat{f}_3) \otimes P_k(\hat{x}_3)$,
$\xi_3$ in $P_{k}(\hat{f}_3) \otimes P_{k-1}(\hat{x}_3)$,
and $h$ in $\tilde{P}_{k-1}(\hat{f}_3) \otimes P_k(\hat{x}_3)$.
Thanks to Lemma~\ref{lema_PIu3_k_cualquiera} we already know that $\xi_3 \equiv 0$,
so we are going to show
that $h \equiv 0$, $\xi_1 \equiv 0$ and that $\xi_2$ does not depend 
on $\hat{x}_1$. First, if $\hat{f}$ is either $\hat{f}_3$ o $\hat{f}_4$, then
with a direct calculation we see that $({\bf curl}\,\wku)_3 |_{\hat f}$ belongs to
$P_{k-1}(\hat{f})$. By the commutative diagram property expressed
in~(\ref{curl_commutativity}), the definition of the degrees of freedom~(\ref{momentos1hdiv})
and the interpolation operator $\br_{\hat E}$
in Definition~\ref{defi_face_element}, it holds that, if $\hat{f}$ is 
either $\hat{f}_3$ or $\hat{f}_4$, then for every $q \in P_{k-1}(\hat{f})$,
\begin{IEEEeqnarray*}{rCCCl}
  \hat\rho_{\hat{f},q}\,({\bf curl\,}\wku)
  & = & \hat\rho_{\hat{f},q} (\br_{\hat E}\,{\bf curl\,}\hat\bu) &&\\[5pt]
  & = & \iint_{\hat f} ({\bf curl\,}\hat \bu)_3\,q \,d\hat S & = & 0.	
\end{IEEEeqnarray*}
As is for time being expected, the choice $q = ({\bf curl}\,\wku)_3 |_{\hat{f}}$
yields 
\[
  ({\bf curl}\,\wku)_3 |_{\hat{f}} \equiv 0
\]
so we may write again $(\curl \wku)_3 = \hat{x}_3\,(\hat{x}_3-1)\,\hat\psi$ for
a $\hat\psi\in P_{k-1}(\hat{f}) \otimes P_{k-2}(\hat{x}_3)$.
We choose now $q=\hat\psi$ in the degrees of freedom~(\ref{momentos4hdiv}).
By the commutative diagram property
and the definition of $\br_{\hat E}$, we have 
\begin{IEEEeqnarray*}{rCCCl}
	\int_{\hat{E}} \hat{x}_3\,(\hat{x}_3-1)\,\hat\psi^2\,d\hat{\bx}
  & = &\int_{\hat{E}} (\curl\wku)_3\,\hat\psi\,d\hat{\bx}&&\\
  & = &\int_{\hat{E}} (\br_{\hat E}\curl\hat{\bu})_3\,\hat{\psi}\,d\hat{\bx}&&\\
  & = &\int_{\hat{E}} (\curl\hat{\bu})_3\,\hat{\psi}\,d\hat{\bx} & = & 0\\
\end{IEEEeqnarray*}
and it follows that
\begin{IEEEeqnarray}{rCl}
	\label{rot_3_es_0} (\curl\wku)_3 &\equiv& 0.
\end{IEEEeqnarray}
Now if we explore $({\bf curl}\,\wku)_3$ taking derivatives in  
expression~(\ref{expr_wku}) we get 
\begin{IEEEeqnarray*}{rCl}
  (\curl\wku)_3 & = & 
  \dfrac{\partial}{\partial \hat x_1}(\wku)_2 - \dfrac{\partial}{\partial \hat x_2}(\wku)_1\\[5pt]
  \label{expre_h} \yesnumber & = & -(2\,h + \hat{x}_2\,\dfrac{\partial h}{\partial \hat{x}_2} + 
	\hat{x}_1\,\dfrac{\partial h}{\partial \hat{x}_1}) + 
	\dfrac{\partial\hat\xi_2}{\partial \hat{x}_1} - \dfrac{\partial\hat\xi_1}{\partial \hat{x}_2}.
\end{IEEEeqnarray*}
Observing the degrees in each term, there hold
\begin{enumerate}
  \item 
  $g\,:=\,2\,h + \hat{x}_2\,\dfrac{\partial h}{\partial \hat{x}_2} + 
  \hat{x}_1\,\dfrac{\partial h}{\partial \hat{x}_1}
  \mbox{ belongs to } \tilde{P}_{k-1}(\hat{f}_3) \otimes P_k(\hat x_3)$
  \item 
  $\dfrac{\partial\hat\xi_2}{\partial \hat{x}_1} -
  \dfrac{\partial\hat\xi_1}{\partial \hat{x}_2}
  \mbox{ belongs to } P_{k-2}(\hat{f}_3) \otimes P_k(\hat x_3)\mbox{,}$
\end{enumerate}
but from this it follows necessarily that $g \equiv 0$. Now, how do the terms
of $g$ look like? Let us put
\begin{IEEEeqnarray*}{rCl}
	h\xyz &=& \sum_{\stackrel{i+j\,=\,k-1}{l\,\leqslant\,k}} \alpha_{_{i,j,l}}\,\hat{x}_1^i \hat{x}_2^j \hat{x}_3^l.
\end{IEEEeqnarray*}
Then
\begin{IEEEeqnarray*}{rCl}
  g\xyz & = & \sum_{\stackrel{i+j\,=\,k-1}{l\,\leqslant\,k}} 
  (2\alpha_{_{i,j,l}} + j\,\alpha_{_{i,j,l}} + i\,\alpha_{_{i,j,l}}) \hat{x}_1^i \hat{x}_2^j \hat{x}_3^l\\
  \yesnumber\label{h_is_zero} & = &(k+1)\,h\xyz\,=\,0,
\end{IEEEeqnarray*}
so $h \equiv 0$ too and, for now, 
\begin{IEEEeqnarray*}{rCl}
\wku\xyz &=& 
(\hat\xi_1\xyz, \hat\xi_2\xyz, 0)'. 
\end{IEEEeqnarray*}
The second--to--last task is to see that $\hat\xi_1$ vanishes identically.
We turn back to de edge degrees of freedom.
Set $\be$ equal to $\hat\be_1$ or $\hat\be_4$. Then the restriction
$\hat\xi_1|_{\be}$ belongs to $P_{k-1}(\be)$, so letting $\hat q = \hat\xi_1|_{\be}$ in
~(\ref{momentos1hcurl}) we obtain
\begin{IEEEeqnarray*}{rCCCCCl}
	0 &=& \hat\varphi_{\be,\,\hat\xi_1}\,(\hat\bu) &=&
	\hat\varphi_{\be,\,\hat\xi_1}\,(\wku) &=& \int_{\be} (\hat\xi_1)^2\,d\alpha\textrm{,}
\end{IEEEeqnarray*}
so, for some $\hat{p} \in P_{k-1}(\hat x_1)\otimes P_{k-2}(\hat x_3)$ we have
\[
  \hat\xi_1|_{\hat f_2}(\hat x_1,\hat x_3) = \hat{x}_3\,(\hat{x}_3-1)\,\hat{p}(\hat x_1,\hat x_3).
\]
Next choose $\hat{f} = \hat{f}_2$ and $\boldsymbol{q}=(0,0,\hat{p})'$ in~(\ref{momentos4hcurl}).
\begin{IEEEeqnarray*}{rCCCCCl} 
  0 & = & \hat\varphi_{\hat{f}_2,\bq}\,(\hat\bu) 
    & = & \hat\varphi_{\hat{f}_2,\bq}\,(\wku) 
    & = & \iint_{\hat{f}_2} \hat{x}_3\,(\hat{x}_3-1)\hat{p}^2\,d\hat S.
\end{IEEEeqnarray*}
It follows that $\hat\xi_1|_{\hat f_2}\equiv 0$, by which we know it exists
certain $\zeta \in P_{k-2}(\hat{f}_3)\otimes P_k(\hat{x}_3)$ satisfying
\[
\hat\xi_1\xyz = \hat{x}_2\,\zeta\xyz.
\]
Now we switch to the faces $\hat{f} = \hat{f}_3$ or $\hat{f}_4$. 
Take $\hat{\boldsymbol{q}} = (\zeta|_{\hat f},0,0)'$ in~(\ref{momentos2hcurl})
\begin{IEEEeqnarray*}{rCCCCCl}
  0 & = & \varphi_{\hat f,\hat{\boldsymbol{q}}}\,(\hat\bu) 
    & = & \varphi_{\hat f,\hat{\boldsymbol{q}}}\,(\wku) 
    & = & \iint_{\hat f} \hat{x}_2\zeta^2\,d\hat S\textrm{,}
\end{IEEEeqnarray*}
and it follows that
$\hat{x}_3\,(\hat{x}_3-1)$ divides $\zeta$. So putting this together
with the previous factorization, there is some
$r \in P_{k-2}(\hat{f}_3)\otimes P_{k-2}(\hat{x}_3)$ which satisfies.
\begin{IEEEeqnarray*}{rCl}
    \hat\xi_1\xyz &=& \hat{x}_2\,\hat{x}_3\,(\hat{x}_3-1)\,r\xyz.
\end{IEEEeqnarray*}
It remains to use the volume degrees of freedom. We could choose
 $\hat\br := (r,0,0)'$ in degree of freedom~(\ref{momentos6hcurl})
to get
\begin{IEEEeqnarray*}{rCCCCCl}
  0 & = & \hat\varphi_{\hat\br}\,(\hat{\bu}) & = & \hat\varphi_{\hat\br}\,(\wku)
    & = & \int_{\hat{E}} \hat{x}_2\,\hat{x}_3\,(\hat{x}_3-1)\,r\xyz^2\,d\hat\bx\textrm{,} 
\end{IEEEeqnarray*}
which yields, over all $\hat{E}$, $\hat\xi_1  \equiv  0$. Finally, if we
combine this last property with~(\ref{rot_3_es_0}) we prove that $\hat{\xi}_2$
does not depend on $\hat{x}_2$.
\end{proof}
\begin{lemma}\label{pi00u3} 
If $\hat{\bu}\xyz=(0,0, \hat{u}_3\xyz)^t$, then
$\wku\xyz = (0,0,\hat\xi_3\xyz)^t$ for some
$\hat\xi_3 \in {P}_k(\hat{f}_3)\otimes
{P}_{k-1}(\hat{x}_3)$.
\end{lemma}
\begin{proof} We will work again with
expression~(\ref{expr_wku}).
%but we will write with less details than in the previous two Lemmas
By expression~(\ref{expre_h}) for $(\curl\wku)_3$
and the commutativity in equation~(\ref{curl_commutativity}),
if we apply degrees of freedom~(\ref{momentos1hdiv})
to $\curl\hat\bu$ we obtain that
$(\curl\wku)_3$ vanishes on any of the horizontal faces $\hat{f}_3$ or $\hat{f}_4$ in
Table~\ref{prismNotationTableFaces}.
In other words, $(\curl\wku)_3
= \hat{x}_3\,(\hat{x}_3-1)\,\hat\psi$, 
($\hat\psi\in P_{k-1}(\hat{f}_3)\otimes P_{k-2}(\hat{x}_3)$)
and if we set $\hat{\br} := (0,0,\hat\psi)'$ in the
$H(\mbox{div})$ degrees of freedom~(\ref{momentos4hdiv})
we have
\begin{IEEEeqnarray*}{rCCCCCCCl}
0 & = & \int_{\hat{E}} (\curl\hat{\bu})_3\,\hat{\psi}
  & = & \hat\rho_{\br} (\curl\hat{\bu})
  & = & \hat\rho_{\br} (\hat{\br}_k\curl\hat{\bu})
  & = & \hat\rho_{\br} (\curl\wku)\\[4pt]
  &&&&&&& = & \int_{\hat{E}} \hat{x}_3(1-\hat{x}_3)\hat{\psi}^2\,d\bx
\end{IEEEeqnarray*}
yielding that $\hat\psi$ is identically zero, and also
$(\curl\wku)_3$ is identically zero.

From this point, if
we copy the argument in the proof of
Lemma~\ref{auxlabel500} starting with equation~(\ref{expre_h}) we arrive at
$h\equiv 0$, so we may rewrite~(\ref{expr_wku}) for the present case as
\begin{IEEEeqnarray}{rCl}
  \label{expre_pi00u3_} \wku &=&
  (\hat\xi_1,\hat\xi_2,\hat\xi_3)^t.
\end{IEEEeqnarray}
We claim that $\hat{\xi}_1\equiv\hat{\xi}_2\equiv0$.
To see this, first observe that the evaluation of the degrees of freedom
for the edges $\hat\be_1$ and $\hat\be_2$ yields
$\hat\xi_1|_{\hat\be_1} \equiv \hat\xi_2|_{\hat\be_2} \equiv 0$,
hence, evaluating the degree of freedom~(\ref{momentos2hcurl})
tangent to the face $\hat{f}_3$ two times we have
$\hat\xi_1|_{\hat{f}_3}  \equiv  \hat\xi_2|_{\hat{f}_3}  \equiv  0$.
In equal manner, if we pick $\hat\be_4$ and $\hat\be_5$, and then the 
degree of freedom tangent to $\hat{f}_4$ we obtain
$\hat\xi_1|_{\hat{f}_3} \equiv \hat\xi_2|_{\hat{f}_3} \equiv  0$.
So we proved there are polynomials $p_1$ and $p_2$ in
$P_{k-1}(\hat{f}_3)\otimes P_{k-2}(\hat{x}_3)$ which allow us to write
\begin{IEEEeqnarray*}{rCl}
  \hat\xi_1\xyz & = & \hat{x}_3(1-\hat{x}_3)p_1\xyz\\[4pt]
  \hat\xi_2\xyz & = & \hat{x}_3(1-\hat{x}_3)p_2\xyz.
\end{IEEEeqnarray*}
Take $\hat\bq := (0,0, \hat{q}_2|_{\hat{f}_2})'$ and 
evaluate the degree of freedom~(\ref{momentos3hcurl}). We have
\begin{IEEEeqnarray*}{rCCCCCl}
  0 & = & \hat\varphi_{\hat{f}_1,\hat{\bq}}\,(\hat\bu) 
    & = & \hat\varphi_{\hat{f}_1,\hat{\bq}}\,(\wku) 
    & = & \iint_{\hat{f}_1} \hat{x}_3(1-\hat{x}_3)\hat{q}_2^2\,d\hat S.
\end{IEEEeqnarray*}
Hence, there is some $\hat{r}_2\in P_{k-2}(\hat{f_3})\otimes P_{k-2}(\hat{x}_3)$
such that $\hat\xi_2 = \hat{x}_1\hat{x}_3(1-\hat{x}_3)\hat{r}_2$.
Now choose $\br = (0,\hat{r}_2,0)'$ and use degree of freedom~(\ref{momentos6hcurl})
to obtain $\int_{\hat{E}}\hat{x}_1\hat{x}_3(1-\hat{x}_3)\hat{r}_2^2\,d\hat\bx=0.$
Since 
$\hat{x}_1\hat{x}_3(1-\hat{x}_3)\hat{r}_2^2$ is almost everywhere greater than zero,
this implies
$\hat{\xi}_2 = 0$.
With the simmetric procedure starting with face $\hat f_2$ we get to prove
$\hat{\xi}_1 = 0$.
\end{proof}
Now here is our first important result.
\begin{theorem}\label{thm_stab_edge}
Given $p > 2$, $\hat{\bu} \in \wpcurl{\hat{E}}$,
\begin{IEEEeqnarray}{rCl}
\label{teorema_1} \norm{(\wku)_1}_{L^{\infty}(\hat{E})} & 
	\lesssim & \|\hat{u}_1\|_{W^{1,p}(\hat{E})} + 
	\|(\curl\hat{\bu})_3\|_{W^{1,1}(\hat{E})} \\	
\label{teorema_2} \norm{(\wku)_2}_{L^{\infty}(\hat{E})} & 
	\lesssim & \|\hat{u}_2\|_{W^{1,p}(\hat{E})} + 
	\|(\curl\hat{\bu})_3\|_{W^{1,1}(\hat{E})} \\	
\label{teorema_3} \norm{(\wku)_3}_{L^{\infty}(\hat{E})} & 
	\lesssim & \|\hat{u}_3\|_{W^{1,p}(\hat{E})}
\end{IEEEeqnarray}
where the constants in the inequalities depend only on $\hat{E}$.
\end{theorem}
\begin{proof}
The proof will rely on the three previous Lemmas, 
the triangular inequality applied on each component of 
expression~(\ref{edge_interp_explicit}) and traces inequalities or,
more precisely, the proof
of Lemma $5.38$ in the page $134$ of~\cite{monk}
and Theorem $3.9$ (\emph{Trace Theorem})
in page $43$ of
the same book.
First we will take a smooth field $\hat{\bu}$ defined on $\hat{E}$
and, by Proposition~\ref{density_wpcurl}, we will conclude the Theorem 
with a density argumentation.\\[4pt]
To prove~(\ref{teorema_1}) the idea will be to take another function
$\hat{\bw}$ such that its interpolate has the same first component
as the one of $\hat\bu$ and such that its degrees of freedom are
more easily bounded in terms of $\hat{u}_1$ and $\curl(\hat{\bu})_3$.

Let us define, for a given $\hat{\bu} \in C^\infty(\bar{\hat{E}})^3$,
$\hat{\bv}\,:\,\hat{E}\to\mathbb{R}^3$ with
\begin{IEEEeqnarray}{rCl} \label{auxlabel201}
  \hat{\bv}\xyz &=& (\hat{u}_1\xyz, \hat{u}_2\xyz - \hat{u}_2(0,\hat{x}_2,\hat{x}_3), 0)'.
\end{IEEEeqnarray}
Thanks to the Lemmas~\ref{auxlabel500} and~\ref{pi00u3} it holds
\begin{IEEEeqnarray*}{rCl}
	(\hat{\bw}_{\hat E}\hat{\bv})_1 & = & (\wku)_1 - 
	\hat{\bw}_{\hat E}(0, \hat{u}_2(0,\hat{x}_2,\hat{x}_3), 0)_1 -
	\hat{\bw}_{\hat E}(0, 0, \hat{u}_3)_1\\
						& = & (\wku)_1\mbox{,}
\end{IEEEeqnarray*}
and we also have $(\curl\hat{\bu})_3 = (\curl\hat{\bv})_3$.
Now let us explore one by one the degrees of freedom that define
$\hat{\bw}_{\hat E}\hat{\bv}$. The only edge degrees
that do not vanish directly or depend explicitly just on 
$\hat{u}_1$ are
\begin{IEEEeqnarray*}{rCl}
	\int_{\hat{\be}_8} q\,\hat{\bv}\cdot d\hat\balpha & = &
	\tfrac{1}{\sqrt{2}} \int_{\hat{\be}_8} (\hat{v}_1 - \hat{v}_2)\,q\,d\alpha\\
	\int_{\hat{\be}_9} q\,\hat{\bv}\cdot d\hat\balpha & = &
	\tfrac{1}{\sqrt{2}} \int_{\hat{\be}_9} (\hat{v}_1 - \hat{v}_2)\,q\,d\alpha
\end{IEEEeqnarray*}
for $q$ in $\pazocal{P}_{k-1}(\hat{\be}_8)$ or $\pazocal{P}_{k-1}(\hat{\be}_9)$ 
respectively. 
Pick a polynomial $q \in P_{k-1}(\hat{\be}_8)$. Since on
$\hat{\be}_8$ it is $\hat{x}_1 = 1 - \hat{x}_2$, we evaluate $q$ as
$q(\hat{x}_2)$, with $0\leqslant\hat{x}_2 \leqslant 1$. Integration
by parts over the face $\hat{f}_4$ yields
\begin{IEEEeqnarray*}{rCl}
  \iint_{\hat{f}_4} (\curl\hat{\bv})_3\,q\,d\hat S
	& = & -\iint_{\hat{f}_4} \left(\hat{v}_2\,\partial_{\hat{x}_1}q - \hat{v}_1\,
  \partial_{\hat{x}_2}q\right)\,d\hat{S}
		+ \int_{\partial \hat{f}_4} \left(\hat{v}_2\,\hat{\nu}_1 
    - \hat{v}_1\,\hat{\nu}_2\right)\,q\,d\hat\alpha\\
	& = & \iint_{\hat{f}_4} \hat{v}_1\,\partial_{\hat{x}_2}q\,d\hat S
		+ \int_{\hat{\be}_8} \left(\hat{v}_2 - \hat{v}_1\right)\,q\,d\hat\alpha + 
			\int_{\hat{\be}_4} \hat{v}_1\,q\,d\hat\alpha\mbox{,}
\end{IEEEeqnarray*}
hence
\begin{IEEEeqnarray*}{rCl}
	\hat\varphi_{\hat{\be}_8,\,q}(\hat\bv) & = &
  \dfrac{1}{\sqrt{2}} \int_{\hat{\be}_8} (\hat{v}_1 - \hat{v}_2)\,q\,d\hat\alpha\\
    & = &\tfrac{1}{\sqrt{2}} \int_{\hat{\be}_4} \hat{u}_1\,q\,d\hat\alpha - 
    \tfrac{1}{\sqrt{2}} \iint_{\hat{f}_4} (\curl\hat{\bu})_3\,q\,d\hat{S}
    + \iint_{\hat{f}_3} \hat{u}_1\,\partial_{\hat{x}_2}q\,d\hat{S}.\\
	\yesnumber\label{momentosWaristas}
    &&
\end{IEEEeqnarray*}
In a similar manner if we integrate over $\hat{f}_3 \subseteq \{ \hat{x}_3 = 0 \}$
we get
\begin{IEEEeqnarray*}{rCl}
	\hat\varphi_{\hat{\be}_9,\,q}(\hat\bv) & = & \tfrac{1}{\sqrt{2}} 
  \int_{\hat{\be}_9} (\hat{v}_1 - \hat{v}_2)\,q\,d\hat\alpha \\
     &=&\tfrac{1}{\sqrt{2}} \int_{\hat{\be}_1} \hat u_1\,q\,d\hat\alpha -
     \tfrac{1}{\sqrt{2}} \iint_{\hat{f}_3} (\curl\hat{\bu})_3\,q\,d\hat{S}
     + \iint_{\hat{f}_3} \hat{u}_1\,\partial_{\hat{x}_2}q\,d\hat{S}.\\
	\yesnumber\label{momentosWaristas2}
     &&
\end{IEEEeqnarray*}
If we evaluate now the face degrees of freedom, we only have to bound
those corresponding to $\hat{f}_3$, $\hat{f}_4$ and $\hat{f}_5$.
Take $\hat{q}_1$, $\hat{q}_2 \in P_{k-2}(\hat{f}_3)$ and consider $\hat{\bq} := (\hat{q}_1, \hat{q}_2, 0)$.
\begin{IEEEeqnarray}{rCl}
 	\label{cotaf3}\iint_{\hat{f}_3} \hat{\bv} \times \hat\bn \cdot \hat\bq\,d\hat{S}
 		& = & \iint_{\hat{f}_3} \hat u_1\,\hat q_2\,d\hat{S} -
    \iint_{\hat f_3} \hat{v}_2\,\hat q_1\,d\hat{S}.
\end{IEEEeqnarray}
Observe that $\hat{v}_2$ vanishes over the face $\hat{f}_1\subseteq\{\hat{x}_1=0\}$.
Now we need a polynomial $\hat\zeta \in P_{k-1}(\hat{f}_3) $ such that 
$\partial_{\hat{x}_1} \hat\zeta = \hat{q}_1$ and
$\hat\zeta |_{\hat{\be}_9} = 0$; take for instance
$\hat\zeta(\hat{x}_1,\hat{x}_2) = -\int_{\hat{x}_1}^{1-\hat{x}_2} \hat q_1(t,\hat{x}_2)\,dt$. Then
\begin{IEEEeqnarray*}{rCl}
	\iint_{\hat{f}_3} (\curl\hat{\bv})_3\,\hat\zeta\,d\hat{S} & = & 
 -\iint_{\hat{f}_3} \left(\hat{v}_2\,\hat q_1 - 
  \hat{v}_1\,\partial_{\hat{x}_2}\hat\zeta\right)\,d\hat{S}
		-\int_{\hat{\be}_1} \hat{v}_1\,\hat n_2\,\hat\zeta\,d\hat\alpha,
\end{IEEEeqnarray*}
which, together with~(\ref{cotaf3}) implies
\begin{IEEEeqnarray}{rCl}
  \nonumber  
  \hat\varphi_{\hat{f}_3,\,\hat{\bq}}(\hat{\bv})
    & = & \iint_{\hat{f}_3} \hat{u}_1\,\hat q_2\,d\hat{S} +
    \iint_{\hat{f}_3} (\curl\hat{\bu})_3\,\hat\zeta\,d\hat{S}\\[4pt]
\label{momentosWcaras}
  &&\,- \iint_{\hat{f}_3} \hat u_1\,\partial_{\hat{x}_2}\hat\zeta\,d\hat{S} +
        \int_{\hat{\be}_1} \hat{u}_1\,\hat n_2\,\hat\zeta\,d\hat\alpha.
\end{IEEEeqnarray}
If we repeated the procedure for the degree of freedom on $\hat{f}_4$, for a given 
$\hat\bp = (\hat p_1, \hat p_2, 0) \in P_{k-2}(\hat{f}_4)^2\times \{0\}$ we would set 
$\hat\psi (\hat{x}_1,\hat{x}_2) = \int_{1-\hat{x}_2}^{\hat{x}_1} 
\hat p_1 (t,\hat{x}_2)\,dt$
and had
\begin{IEEEeqnarray}{rCl}
\nonumber
  \hat\varphi_{\hat{f}_4,\,\hat{\bp}}(\hat{\bv})
    & = & - \iint_{\hat{f}_4} \hat{u}_1\,\hat{p}_2\,d\hat{S} -
    \iint_{\hat{f}_4} (\curl\hat{\bu})_3\,\hat\psi\,d\hat{S} \\[4pt]
\label{momentosWcaras2} && \,+ 
    \iint_{\hat{f}_4} \hat{u}_1\,\partial_{\hat{x}_2}\hat\psi\,d\hat{S}	-
    \int_{\hat{\be}_4} \hat{u}_1\,\hat{n}_2\,\hat\psi\,d\hat{\alpha}.
\end{IEEEeqnarray}
For the degree of freedom~(\ref{momentos5hcurl}) corresponding to $\hat{f}_5$, given
$\hat\bq = (0, \hat q_3, \hat q_1) \in \{ 0 \} \times Q_{k-2,k-1} \times Q_{k-1,k-2}$
observe
\begin{IEEEeqnarray}{rCl}\label{momentosWcaras3}
  \iint_{\hat{f}_5} \hat{\bv} \times \bn \cdot \hat{\bq}\,d\hat{S}
    & = & \iint_{\hat{f}_5} (\hat{v}_1 - \hat{v}_2)\,\hat{q}_1\,d\hat{S}
\end{IEEEeqnarray}
Now, if $\hat{q}$ is the extension of $\hat{q}_1$ to the whole prism, then
\begin{IEEEeqnarray*}{rCl}
  \iint_{\hat{f}_5} \hat{v}_2\,\hat{q}_1\,d\hat{S} & = &
  \sqrt{2} \iint\limits_{[0,1]^2}\hat{v}_2(1-\hat{x}_2,\hat{x}_2,\hat{x}_3)\hat{q}_1(\hat{x}_2,\hat{x}_3)\,d\hat{x}_2d\hat{x}_3\\[5pt]
  &=&\sqrt{2} \iint\limits_{[0,1]^2}\int_{0}^{1-\hat{x}_2}\tfrac{\partial\hat{v}_2}{\partial{\hat{x}_1}}
  (\hat{t},\hat{x}_2,\hat{x}_3)\hat{q}(\hat{t}, \hat{x}_2,\hat{x}_3)\,d\hat{t}d\hat{x}_2d\hat{x}_3\\[5pt]
  \yesnumber\label{momentosWcaras3_}
  &=&\sqrt{2}\int_{\hat{E}} (\curl\hat\bv)_3\hat{q}\,d\hat{\bx} + 
  \sqrt{2}\int_{\hat{E}} \tfrac{\partial\hat{v}_1}{\partial{\hat{x}_2}}\hat{q}\,d\hat{\bx}.
\end{IEEEeqnarray*}
Joining~(\ref{momentosWcaras3}) and~(\ref{momentosWcaras3_}) and using the
expression for $\hat{\bv}$ in~(\ref{auxlabel201}) we get 
\begin{IEEEeqnarray}{rCl}
  \nonumber
  \varphi_{\hat{f}_5,\,\hat{\bq}}(\hat{\bv})
  & = & \iint_{\hat{f}_5} \hat{u}_1\,\hat{q}_1\,d\hat{S}
  -\sqrt{2}\int_{\hat{E}} \tfrac{\partial\hat{u}_1}{\partial{\hat{x}_2}}\hat{q}\,d\hat{\bx}
  -\sqrt{2}\int_{\hat{E}} (\curl\hat\bu)_3\,\hat{q}\,d\hat{\bx}. \\
  & & \label{momentosWcaras3__}
\end{IEEEeqnarray}
At last, we study the volume degrees of freedom. Pick
$\hat\br = (\hat r_1, \hat r_2, \hat r_3)'$ belonging to the space
\[
 (P_{k-2}(\hat{f}_3) \otimes P_{k-2}(\hat{x}_3))^{2}
\times P_{k-3}(\hat{f}_3) \otimes
P_{k-1}(\hat{x}_3)
\]
(cfr. degree of freedom~(\ref{momentos6hcurl}))
and let $\hat\varphi_2$ be defined in such a way that
$\varphi_2\xyz = \int_{1-\hat{x}_2}^{\hat{x}_1} 
\hat{r}_2(\hat{t},\hat{x}_2,\hat{x}_3)\,d\hat{t}$.
%, para el cual
%vale 
%$\partial_x\varphi_2 = \hat{r}_2 $
%y $\varphi_2|_{\hat{f}_5} \equiv 0$.
Green's Theorem and the fact that $\varphi_2|_{\hat{f}_5} \equiv 0$
give 
\begin{IEEEeqnarray}{rCl}
  \nonumber\int_{\hat{E}} \hat{\bv} \cdot \hat\br\,d\hat\bx
  & = & \int_{\hat{E}} \hat{u}_1\,\hat{r}_1\,d\hat\bx 
  - \int_{\hat{E}} (\curl\hat{\bu})_3\,
  \hat\varphi_2\,d\hat\bx
  - \int_{\hat{E}}
  \tfrac{\partial\hat{u}_1}{\partial\hat{x}_2}\,
  \hat\varphi_2\,d\hat\bx.\\
\label{momentosWvolumen}
  &&
\end{IEEEeqnarray}
%,~(\ref{momentosWcaras2}),~(\ref{momentosWcaras3})
Now we collect what has been said so far.
%%%%%%%%%%%%%%%%%%%%%%%%% equalities,,~(\ref{momentosWcaras}),~(\ref{momentosWcaras2}),~(\ref{momentosWcaras3}) and~(\ref{momentosWvolumen}).
For the edge degrees of freedom we use an inequality in page $135$ of~\cite{monk},
in the proof of Lemma 5.38, which states, for $\hat\bu$ in the present conditions,
\begin{IEEEeqnarray}{rCl}\label{edgeTrace}
  \left|\int_{\hat\be} \hat\bu\cdot\hat\btau\,q\,d\hat\alpha\right| 
  & \leqslant & C(q) \,\{\, \|\curl\hat\bu\|_{L^p(\hat{E})^3}
    + \|\mbox{Tr}\,\hat\bu\|_{L^p(\partial\hat{E})^3} \}.
\end{IEEEeqnarray}
The details needed to the proof of inequality~(\ref{edgeTrace}) can be completed
from Theorem 3.14 of~\cite{A-2001}.

Now if we put the field $(\hat{u}_1,0,0)'$ in inequality~(\ref{edgeTrace}) then
by, H\"older's Inequality and standard traces inequalities, equation~(\ref{momentosWaristas})
and~(\ref{momentosWaristas2}) yield, for $i=8$ and $9$,
\begin{IEEEeqnarray*}{rCl}
  \left|\varphi_{\hat{\be}_i,\,\hat{q}}(\hat\bv)\right| & \leqslant & c(\hat{q})\,
  \{\,\|\hat{u}_1\|_{W^{1,p}(\hat{E})} + \|\mbox{Tr}\,(\curl{\hat{\bu}})_3\|_{L^1(\partial\hat{E})}
  +\|\mbox{Tr}\,\hat{u}_1\|_{L^p(\partial\hat{E})}\,\}\\[5pt]
  \yesnumber\label{traceE8}
  & \leqslant & c(\hat{q})\,\{\,\|\hat{u}_1\|_{W^{1,p}(\hat{E})} + 
  \|(\curl\hat{\bu})_3\|_{W^{1,1}(\hat{E})}\,\}.
\end{IEEEeqnarray*}
If we repeat the argument for the line integral
terms in~(\ref{momentosWcaras}) and~(\ref{momentosWcaras2}) we get, for $j=3$ and $4$,
\begin{IEEEeqnarray}{rCl}
\nonumber
  \left|\varphi_{\hat{f}_j,\,\hat{\bq}}(\hat\bv)\right| & \leqslant &
  c(\hat{\bq})\,\{\,
    \|\mbox{Tr}\,\hat{u}_1\|_{L^p(\partial\hat{E})} +
    \|\mbox{Tr}\,(\curl{\hat{\bu}})_3\|_{L^1(\partial\hat{E})} +
    |\hat{u}_1|_{W^{1,p}(\hat{E})}\,\}.\\
\label{traceF3}&&
\end{IEEEeqnarray}
And finally, by estimates~(\ref{momentosWcaras3})--(\ref{traceF3})
and one more time H\"older's and traces inequalities,
\begin{IEEEeqnarray*}{rCCCl}
	\|(\wku)_1\|_{L^\infty(\hat{E})} & = & \|(\hat{\bw}_{\hat E}\hat{\bv})_1\|_{L^\infty(\hat{E})}
  &\lesssim&
  \sum_{i=8,9,\,\hat p} |\hat\varphi_{\hat{\be}_i,\hat p}(\hat{\bv})|\,\|(\hat{\bv}_{\hat{\be}_i,\hat p})_1\|_{L^\infty(\hat{E})}
  \\[4pt]\IEEEeqnarraymulticol{5}{C}{\,+
  \sum_{j=3,4,\,\hat\bq} |\hat\varphi_{\hat{f}_j,\hat\bq}(\hat{\bv})|\,
  \|(\hat{\bv}_{\hat f_j,\hat\bq})_1\|_{L^\infty(\hat{E})} +
  \sum_{\hat\br} |\hat\varphi_{\hat\br}(\hat{\bv})|\,\|(\hat{\bv}_{\hat\br})_1\|_{L^\infty(\hat{E})}
  }\\[4pt]
	& \leqslant & c(\hat{E})\,\{\,\|\hat{u}_1\|_{W^{1,p}(\hat{E})} & + &
		\|(\curl\hat{\bu})_3\|_{W^{1,1}(\hat{E})}\,\}
\end{IEEEeqnarray*}
which is the bound we wanted to prove. The summation indices with polynomials
$\hat p$, $\hat\bq$ and $\hat\br$ mean that we use the way of writing
the interpolator stated in~(\ref{edge_interp_explicit}). The same proving procedure applies to 
inequality~(\ref{teorema_2}).\\[7pt]
For inequality~(\ref{teorema_3}), given $\hat{\bu} \in W^{1,p}(\hat{E})^3$, define
$\hat{\bv}  =  (0,0, \hat{u}_3)'.$
Thanks to Lemma~\ref{lema_PIu3_k_cualquiera} we have 
$(\hat{\bw}_{\hat E}\hat{\bv})_3 = (\hat{\bw}_{\hat E}\hat{\bu})_3 - (\hat{\bw}_{\hat E}(\hat{u}_1, \hat{u}_2, 0))_3 = (\hat{\bw}_{\hat E}\hat{\bu})_3.$
By expression~(\ref{edge_interp_explicit}), taking another look at 
the unit tangent vector of the edges and unit normal vectors to the
faces, we have
\begin{IEEEeqnarray*}{rCl}
  (\hat{\bw}_{\hat E}\hat{\bv})_3 & = &
  \sum_{j=3,6,7;\,\hat{\bp}}
  \int_{\hat e_j} \hat{u}_3 \hat{p}_3\,d\alpha\,(\hat{\bv}_{\hat{\be}_j,\hat{\bp}})_3 +
  \sum_{i=1,2,4;\,\hat q}
  \int_{\hat f_i} \hat{u}_3 \hat q\,d\hat S \,(\hat{\bv}_{\hat{f}_i,\hat q})_3\\
  &&\,+\sum_{\hat \br}
  \int_{\hat E} \hat{u}_3 \hat r_3\,d\hat\bx\,(\hat{\bv}_{\hat\br})_3.
\end{IEEEeqnarray*}
This implies, by traces inequalities and~(\ref{edgeTrace}), that
\begin{IEEEeqnarray*}{rCl}
  \norm{(\hat{\bw}_{\hat E}\hat{\bu})_3}_{L^{\infty}(\hat{E})}
  & \leqslant & C(\hat{E}) \big\{
  \sum_{j=3,6,7; \hat{\bp}}
  \left|\int_{\hat e_j} \hat{u}_3\,\hat{p}_3\,d\alpha\right| \\[4pt]
  &&\,+
  \sum_{i=1,2,4}
  \iint_{\hat f_i} |\hat{u}_3|^p\,d\hat S
  + \int_{\hat E} |\hat{u}_3|^p\,d\hat\bx  \big\}\\[4pt]
  &\leqslant& C(\hat{E}) \|\hat{u}_3\|_{W^{1,p}(\hat{E})}.
\end{IEEEeqnarray*}
The constants in the three inequalities of this Theorem depend only
on the choice of the bases of the test polynomials for the degrees of freedom.
\end{proof}
As in the div--conforming case, the next step is to estimate the stability in 
an anisotropically rescaled prism. Consider again the element $\tilde{E}$ defined in~(\ref{tilde_prism}).
Given a natural number $k$, denote with ${\bw}_{\tilde{E}}$ the $k$--th order 
$\bcurl$--conforming interpolation
operator over $\tilde{E}$ defined as in Corollary~\ref{aux_label26}. 
For the rest of the Subsection, $\tilde\bu$ will be an element
with a well defined $\bcurl$--conforming interpolate.
%, namely of
%$H({\bf curl},\tilde{E})\cap H^{1/2+\delta}(\tilde{E})^3$ for 
%a positive $\delta$ with 
%${\bf curl}\,\tilde{\bu}\in L^p(\tilde{E})^3$
%for some
%$p>2$.
Write the diameter of $\tilde{E}$ as $h_{\tilde{E}}$ and as
$\tilde{x}_i,\,1\leqslant i\leqslant 3$, the coordinates along the axes
in $\mathbb{R}^3$.
\begin{lemma}\label{estabLinf} There exists a positive $C$, independent
of $h_i,\,1\leqslant i\leqslant 3$ such that for all $p > 2$ and 
$\tilde{\bu}\in\wpcurl{\tilde{E}}$
\begin{IEEEeqnarray*}{rCl}
    \left\| \wkutilde \right\|_{L^\infty(\tilde{E})^3}
    & \leqslant & C \left[ |\tilde{E}|^{-\nicefrac{1}{p}} \left( \left\| \tilde{\bu} 
    \right\|_{L^p(\tilde{E})^3} +
        \sum_{i=1}^3 h_i \left\| \partial_{\tilde{x}_i}\tilde{\bu} 
        \right\|_{L^p(\tilde{E})^3} \right)\right.\\
\IEEEeqnarraymulticol{3}{c}
{\left.\:+\; (h_1+h_2)\, |\tilde{E}|^{-1} \left( \left\|(\curl\,\tilde{\bu})_3 
    \right\|_{L^1(\tilde{E})} + 
    \sum_{i=1}^3 h_i \left\| \partial_{\tilde{x}_i}(\curl\,\tilde{\bu})_3 
    \right\|_{L^1(\tilde{E})}\right)
    \right].}
\end{IEEEeqnarray*}
%============
%{\color{BrickRed} ver las cuentas donde dice $h_1 + h_2$}
%============
% tal vez esto no haga falta
%para transformar $\hat{E} $ en $\tilde{E} $ v\'ia
%
%in this particular case ...
%\begin{IEEEeqnarray*}{rCl}
%    \hat{\pi}_i & = & h_i\tilde{\pi}_i \\
%    (\textbf{curl}\,\hat{\bu})_3 & = & h_1h_2(\textbf{curl}\,\tilde{\bu})_3.
%\end{IEEEeqnarray*}
%============
\end{lemma}
\begin{proof}
The proof of this estimate will be made componentwise
using the inequalities of 
Theorem~\ref{thm_stab_edge} and the vectorial bound will hold immediately.
Bounds for $(\wkutilde)_1$ and $(\wkutilde)_3$
will be established, as the bounding for $(\wkutilde)_2$ is the same as the first one.
Pulling $\wkutilde$ back to $\hat{E}$ we get by~(\ref{piTransformado}) 
that $(\wkutilde)_i = 
\nicefrac{1}{h_i} (\wku)_i,\,1\leqslant i\leqslant 3$. By inequality~(\ref{teorema_1}) and a suitable, though elementary,
change of variables dictated by~(\ref{change_var}) we do
\begin{IEEEeqnarray*}{rCl}
  \left\| (\wkutilde)_1 \right\|_{L^\infty(\tilde{E})} & = &
    \frac{1}{h_1} \left\| (\wku)_1 \right\|_{L^\infty(\hat{E})}\\
    & \leqslant & \frac{c(\hat{E})}{h_1} \left[\|\hat{u}_1\|_{W^{1,p}(\hat{E})} + 
        \|(\curl\,\hat{\bu})_3\|_{W^{1,1}(\hat{E})}\right] \\
    & \leqslant & c(\hat{E})
  \left[
    |\tilde{E}|^{\nicefrac{-1}{p}}
    \big\{
    \|\tilde{u}_1\|_{L^p(\tilde{E})} + \sum_{i=1}^3 h_i
    \|\tfrac{\partial\tilde{u}_1}{\partial\tilde{x}_i}\|_{L^p(\tilde{E})}
    \big\}
  \right.\\
\IEEEeqnarraymulticol{3}{r}{+
  \left.
    h_2|\tilde{E}|^{-1}
    \big\{
    \|(\curl\,\tilde{\bu})_3\|_{L^1(\tilde{E})} + 
        \sum_{i=1}^3 h_i \|\partial_{\tilde{x}_i}(\curl\,\tilde{\bu})_3\|_{L^1(\tilde{E})}
    \big\}
  \right].}
  \\&&\yesnumber\label{number1}
\end{IEEEeqnarray*}
With respect to component number three, from~(\ref{teorema_3}) we write
\begin{IEEEeqnarray}{rCl}\label{number2}
  \left\| (\wkutilde)_3 \right\|_{L^\infty(\tilde{E})}
  & \leqslant & C|\tilde{E}|^{-\nicefrac{1}{p}}
  \left(
    \|\tilde{u}_3\|_{L^p(\tilde{E})} + \sum_{i=1}^3 h_i \|\partial_{\tilde{x}_i}\tilde{u}_3\|_{L^p(\tilde{E})}
  \right).
\end{IEEEeqnarray}
\end{proof}
\noindent With the previous bound we deduce the following
anisotropic stability estimate for the rescaled prismatic element $\tilde{E}$.
\begin{theorem} \label{aux_label27}
There is a $C > 0$ independent of $h_i$ such that for all
$\tilde{\bu}\in\wpcurl{\tilde{E}}$ and $p>2$.
\begin{IEEEeqnarray*}{rCl}
    \|\wkutilde\|_{L^p(\tilde{E})}
    & \leqslant & C \left[ \left\| \tilde{\bu} \right\|_{L^p(\tilde{E})}
    + \sum_{i=1}^3 h_i \left\| \partial_{\tilde{x}_i}\tilde{\bu} \right\|_{L^p(\tilde{E})}\right.
\\\IEEEeqnarraymulticol{3}{r}{
\left.
    \:+\;(h_1+h_2)\left(\left\|(\curl\,\tilde{\bu})_3 \right\|_{L^p(\tilde{E})}
     + \sum_{i=1}^3 h_i
     \left\| \partial_{\tilde{x}_i}(\curl\,\tilde{\bu})_3 \right\|_{L^p(\tilde{E})}\right)
  \right].
}
\end{IEEEeqnarray*}
\end{theorem}
\begin{proof}
    \noindent From Lemma~\ref{estabLinf}, since $|\tilde{E}|$ is finite measured,
    the H\"older inequality tells us that, for any real $q \geqslant 1$,
    \begin{IEEEeqnarray*}{rCl}
        \|(\curl\tilde{\bu})_3\|_{L^1(\tilde{E})} &\leqslant&
         |\tilde{E}|^{1-\frac{1}{q}}\,\|(\curl\,\tilde{\bu})_3\|_{L^q(\tilde{E})}\\
        \|\partial_{\tilde{x}_i}(\curl\,\tilde{\bu})_3\|_{L^1(\tilde{E})} &\leqslant&
         |\tilde{E}|^{1-\frac{1}{q}}\,\|\partial_{\tilde{x}_i}(\curl\,\tilde{\bu})_3\|_{L^q(\tilde{E})}.
    \end{IEEEeqnarray*}
    So we get to
    \begin{IEEEeqnarray*}{rCl}
    \left\| (\tilde{\bw}_{\tilde E}\tilde{{\bu}})_1 \right\|_{L^p(\tilde{E})}
      & \leqslant & |\tilde{E}|^{\nicefrac{1}{p}}\left\| (\tilde{\bw}_{\tilde E}\tilde{{\bu}})_1 \right\|_{L^\infty(\tilde{E})}\\
      \mbox{(by~(\ref{number1}))\hspace{.6cm}}   & \leqslant & C
      \left[
        \|\tilde{u}_1\|_{L^p(\tilde{E})} + \sum_{i=1}^3 h_i \|\tfrac{\partial\tilde{u}_1}{\partial\tilde{x}_i}\|_{L^p(\tilde{E})}
        \right.\\
          & & \:\:+
        \left.
            h_2
            \left(
            \|(\curl\tilde{\bu})_3\|_{L^p(\tilde{E})} + 
                \sum_{i=1}^3 h_i \|\partial_{\tilde{x}_i}(\curl\tilde{\bu})_3\|_{L^p(\tilde{E})}
            \right)
        \right].
    \end{IEEEeqnarray*}
    Now combine this with an entirely analogous argument for component two and with~(\ref{number2}) and
    the Theorem follows.
\end{proof}
\section{Local Interpolation Estimates for $H(\bcurl)$ Conforming Prismatic Elements}
\begin{theorem} \label{aux_label32} Let $k\in\mathbb{N}$ and $p>2$ and let $E$ be a prism whose triangular
faces have greatest angle less than $c_0$.
There exists $C > 0$ and three edges $\be_i$ of $E$ incident to a common vertex
$\bx_E$ such that for all $\bu\in W^{m + 1,p}(E)^3$
and $m\leqslant k-1$, %with $\bcurl \bu\in W^{m,p}(E)^3$
\begin{IEEEeqnarray*}{rCl}\label{aux_label55}
  \|\bu-\bw_E \bu\|_{L^p(E)} & \leqslant & C
  \left\{
    \sum_{|{\balpha}|=m+1}\bh^{\balpha} \|\partial^{\balpha} \bu\|_{L^p(E)} +\right.\\[4pt]
  \yesnumber\label{auxlabel5}
   &&\qquad\left. h_E\sum_{|{\balpha}|=m}\bh^{\balpha}\|\partial^{\balpha} 
    (\curl \bu)_3 \|_{L^p(E)}
  \right\}.
\end{IEEEeqnarray*} 
$C$ depends only on $c_0$.
$C$ can be chosen so that, if $M_E$ is the matrix made with
$\xi_i$ as columns, then $\|M\|_\infty\leqslant C$ and $\|M^{-1}\|_\infty\leqslant C$ 
and $\det M_E \geqslant C$
\end{theorem}
Notice the an\-iso\-tropic character of the inequality in~\eqref{auxlabel5}. Notice only the component
of the $\curl$ corresponding to the direction that is orthogonal to the 
triangular faces.
%[Proof of Theorem~\ref{aux_label32}]
\begin{proof}[Proof of Theorem~\ref{aux_label32}] %%% TODO: {\color{BrickRed}\#\#\#\#\#\#\#\# Ariel, por favor mirar si es correcta la manera en que lo digo.}\\\\
Since $W^{m+1,p}(E)\hookrightarrow W^{1,p}(E)$ and $p$ is greater than $2$,
the edge interpolator $\bw_E$ is well--defined via Corollary~\ref{aux_label26}.
Consider the prism $\tilde E$ as in~(\ref{tilde_prism}). There is an affine map
$\tilde \bx \mapsto \bx = M_E\,\tilde\bx+\bx_E = F_E\,\tilde\bx$ from $\tilde E$ onto $E$, such that 
$\|M_E\|$, $\|M_E\|^{-1}\leqslant C$. The matrix $M_E$ is made up of vectors 
$\xi_i$, $i = 1$, $2$, $3$ as its columns. First we take $\bq := \Qbb_{m,E}\,\bu$ and
do%where $\xi_i$ are the unitary vectors in the directions of three edges $\ell_i$  of E of lengths $h_i$ sharing the vertex $\bx_E$.
\begin{IEEEeqnarray*}{rCl}
  \|\bu-\bw_E\bu\|_{L^p(E)} & \leqslant & \|\bu-\bq\|_{L^p(E)}
    +\|\bw_E(\bu-\bq)\|_{L^p(E)}
\end{IEEEeqnarray*}
For the first term we may simply do, by Remark~\ref{aux_label28} and the
transformation~(\ref{transfHcurl}),
\begin{IEEEeqnarray}{rCl}
\nonumber
  \|\bu-\bq\|_{L^p(E)} & = & \|M_E^{-t}(\tilde{\bu}-\tilde{\bq})\circ F_E^{-1}\|_{L^p(E)} \\[5pt]
\label{aux_label37}
  &\leqslant& \|M^{-1}\||\det M_E|^{\nicefrac1p}\|\tilde{\bu}-\tilde{\bq}\|_{L^p(\tilde E)}.
\end{IEEEeqnarray}
With regard to the second term, by the commutativity property~(\ref{piTransformado})
and again the coordinate transformation,
\begin{IEEEeqnarray*}{rCl}
  \|\bw_E(\bu-\bq)\|_{L^p(E)}&\leqslant&
    |M|^{\nicefrac1p}\|M_E^{-1}\|  
      \|\tilde{\bw}_{\tilde E}(\tilde\bu-\tilde\bq)\|_{L^p(\tilde E)}.
\end{IEEEeqnarray*}
Theorem~\ref{aux_label27} implies
\begin{IEEEeqnarray*}{rCl}
    \|\bw_E(\bu-\bq)\|_{L^p(E)}
    & \leqslant & \\
\IEEEeqnarraymulticol{3}{r}{
\begin{IEEEeqnarraybox*}{rl}
  \qquad & C\|M^{-1}\||\det M_E|^{\nicefrac1p}
\left[ \| \tilde\bu-\tilde\bq \|_{L^p(\tilde{E})}
    + \sum_{i=1}^3 h_i \| \partial_{\tilde{x}_i}(\tilde\bu-\tilde\bq) \|_{L^p(\tilde{E})}\right.\\
&
    \left.
    \:+\;h\left(\left\|(\curl(\tilde\bu-\tilde\bq))_3 \right\|_{L^p(\tilde{E})}
     + \sum_{i=1}^3 h_i
     \left\| \partial_{\tilde{x}_i}(\curl(\tilde\bu-\tilde\bq))_3 \right\|_{L^p(\tilde{E})}\right)
  \right].
\end{IEEEeqnarraybox*}
}\\[4pt]
&&\yesnumber\label{aux_label34}
\end{IEEEeqnarray*}
By expressions~(\ref{aux_label30}),~(\ref{aux_label24}),~(\ref{aux_label25})
the last expression is bounded by a constant times
$\|M_E^{-1}\||\det M_E|^{\nicefrac1p}$
times the following sum
\begin{IEEEeqnarray}{rCl}
\nonumber
\sum_{i+j+k=m+1} h_1^ih_2^jh_3^k \left\| \frac{\partial^{m+1}\tilde\bu}
    {\partial\tilde x_1^i\partial\tilde x_2^j\partial\tilde x_3^k}
    \right\|_{0,\tilde E} &+&
h \sum_{j+k+l=m}  h_1^jh_2^kh_3^l
  \left\|\frac{\tilde\partial^m(\curl\tilde\bu)_3}{\partial\tilde x_1^j\partial\tilde x_2^k\partial\tilde x_3^l}
  \right\|_{0,\tilde E}
\\[7pt]
\IEEEeqnarraymulticol{3}{r}{
\nonumber
+\,h \sum_{i=1}^3 h_i\sum_{j+k+l=m-1}  h_1^jh_2^kh_3^l
        \left\|\frac{\tilde\partial^{m-1}\tilde\partial(\tilde\curl\tilde\bu)_3}
               {\partial\tilde x_1^j\partial\tilde x_2^k\partial\tilde x_3^l\partial\tilde x_j}
       \right\|_{0,\tilde E}
}\\[7pt]
\IEEEeqnarraymulticol{3}{r}{\nonumber
\lesssim
\sum_{i+j+k=m+1} h_1^ih_2^jh_3^k \left\| \frac{\partial^{m+1}\tilde\bu}
    {\partial\tilde x_1^i\partial\tilde x_2^j\partial\tilde x_3^k}
    \right\|_{0,\tilde E}
+  h \sum_{j+k+l=m}  h_1^jh_2^kh_3^l
  \left\|\frac{\tilde\partial^m(\curl\tilde\bu)_3}{\partial\tilde x_1^j\partial\tilde x_2^k\partial\tilde x_3^l}
  \right\|_{0,\tilde E}.}\\[4pt]&&
\label{aux_label33}
\end{IEEEeqnarray}
From equality~(\ref{aux_label29}), for every $\balpha$ of order
$m+1$ it holds
\begin{IEEEeqnarray}{rCl}\label{aux_label36}
  \|\tilde{\partial}^{\balpha}\tilde\bu\|_{L^p(\tilde{E})} & \leqslant & 
  \|M_E\|\,|\det M_E|^{-\nicefrac1p} \|\partial^{\balpha}\bu\|_{L^p(E)}.
\end{IEEEeqnarray} %%{\color{BrickRed}\#\#\#\#\#\#\#\# esto de la matriz realmente hace falta?.}
Lastly, adapting Lemma 3.57 in page 77 of~\cite{monk}, we observe
\begin{IEEEeqnarray*}{rCl}
  \begin{pmatrix}
    0 & -(\tilde\curl\tilde\bu)_3 & 0 \\
    (\tilde\curl\tilde\bu)_3 & 0 & 0 \\
    0 & 0 & 0 
  \end{pmatrix}M_E^{-1}
  & = & M_E^{t}
  \begin{pmatrix}
    0 & -(\curl\bu)_3 & 0 \\
    (\curl\bu)_3 & 0 & 0 \\
    0 & 0 & 0 
  \end{pmatrix}\circ F_E
\end{IEEEeqnarray*}
which implies, for every $\balpha$ of order $m$,
\begin{IEEEeqnarray}{rCl} \label{aux_label35}
  \|\tilde{\partial}^{\balpha}(\tilde{\curl}\tilde\bu)_3\|_{L^p(\tilde E)}
  & \leqslant & C |\det M_E|^{-\nicefrac1p}\,\|M_E\|^{2} 
  \|\partial^{\balpha}(\curl\bu)_3\|_{L^p(E)}.
\end{IEEEeqnarray}   %%{\color{blue}\#\#\#\#\#\#\#\# De donde sale el $(2+m)$ de (6.15) en~\cite{ariel}?.}
Now combine~(\ref{aux_label33}),~(\ref{aux_label36}) and~(\ref{aux_label35}) 
with~(\ref{aux_label34}) and~(\ref{aux_label37}) to obtain the
Theorem.
\end{proof}
\section{Pyramidal Finite Elements}\label{label999}
Here we state and prove least order an\-is\-otrop\-ic stability inequalities
and an\-is\-otrop\-ic local interpolation
inequalities for the finite elements on pyramids defined in~\cite{gh99} 
and~\cite{Nigam-2012}. This estimates could be used to build a variant
of our method presented in this Thesis, using finite elements in all the types
of elements. As we said in the introduction of the Thesis, one of the ideas of 
our method was the combination of finite elements in prisms and tetrahedra with 
virtual elements on pyramids.
\facesOfPyramid
\edgesOfPyramid
\begin{figure}[!h]
\centering
  \unitTangentsPyramid
  \caption{Directions of the positive unit tangents (cfr. Table~\ref{pyramidNotationTableEdges}).}
  \label{reference_pyramid}
\end{figure}

\subsection{$H(\bcurl)$--Conforming Element on Pyramids} % (fold)
\label{sub:edge}
\begin{defi}\label{aux_label50}
  The following items define a least order $\bcurl$--conforming finite element
  on the reference Pyramid.
  \begin{enumerate}
    \item $\hat E$ is the reference Pyramid in Definition~\ref{defi_of_ref_pyr}. 
    \item The rational space $P_{\hat E}$ is the span of
    $\{\hat{\bgamma}_1,\,\ldots,\,\hat{\bgamma}_8\}$ with $\hat{\bgamma}_i$
    as in Table~\ref{shape_edge_table}.
    \item The degrees of freedom are the line integrals
      \begin{IEEEeqnarray*}{c}
        \int_{\hat\be_j}\hat\bu\cdot\,d\hat\balpha
      \end{IEEEeqnarray*}
      for every edge $\hat\be_j$ of $\hat E$, $1\leqslant j\leqslant 8$.
  \end{enumerate}
\end{defi}
\edgeShapeTable
A direct computation yields the following Lemma.
\begin{lemma}
  For $1\leqslant i,j\leqslant 8$,
  $\int_{\hat\be_j}\hat\bgamma_i\cdot d\hat\balpha = \delta_{ij}$ which
  implies immediately that the finite element in Definition~\ref{aux_label50}
  is unisolvent in $\hat E$. %$H(\bcurl)$--conforming and 
\end{lemma}
\begin{lemma}
  $P_0(\hat E)^3 \subseteq P_{\hat E}$.  
\end{lemma}
\begin{proof}
  Cfr. Lemma 7.3 of~\cite{Nigam-2012}.
\end{proof}
% subsection edge (end)
\subsection{$H(\Div)$--Conforming Element on Pyramids} % (fold)
\label{sub:face}
\begin{defi}\label{aux_label71}
The following items define a least order $\Div$--conforming finite element
  on the reference Pyramid.
\begin{enumerate}
  \item $\hat{E}$ is the reference pyramid of Figure~\ref{reference_pyramid}.
  \item The space $P_{\hat{E}}$ is the span of 
  $\{\hat{\bz}_1,\,\ldots,\,\hat{\bz}_5\}$ with $\hat{\bz}_i$
    as in Table~\ref{shape_face_table}.
  \item The degrees of freedom are the surface integrals
  \begin{IEEEeqnarray*}{c}
    \label{dofsdivpyramid} \iint_{\hat{f}_j} \hat\bv\cdot\hat\bn\,d\hat S
  \end{IEEEeqnarray*}
  for every face $\hat{f}_j$ of $\hat E$, $1\leqslant j\leqslant 5$.
\end{enumerate}
\end{defi}
\faceShapeTable
A direct computation yields the following Lemma.
\begin{lemma}
  For $1\leqslant i,j\leqslant 5$,
  $\iint_{f_j}\hat\bz_i\cdot\hat\bn\,d\hat S = \delta_{ij}$ which
  implies immediately that the finite element in Definition~\ref{aux_label71}
  is unisolvent in $\hat E$. %$H(\bcurl)$--conforming and 
\end{lemma}
\begin{lemma}
  $P_0(\hat E)^3 \subseteq P_{\hat E}$.
\end{lemma}
\begin{proof}
  Cfr. Lemma 7.2 of~\cite{Nigam-2012}.
\end{proof}
% subsection face (end)
\section{Stability Estimates for Pyramidal Finite Elements} % (fold)
\label{sec:pyramidal_finite_elements}
$\hat{E}$ will be the reference pyramid  in Figure~\ref{reference_pyramid}.
Anisotropic interpolation error estimates for pyramidal $\bcurl$--conforming
and div--conforming finite elements of least order will we established.
\subsection{Anisotropic Stability Estimates for $H(\bcurl)$--Conforming 
Elements on Pyramids} % (fold)
\label{sub:edge_elements}
Here we state an anisotropic stability bound for
the least order $\bcurl$ -- conforming operator on pyramids.
As always, we write it componentwise to make the anisotropy of the 
estimate clearer.
\begin{theorem} \label{aux_label53}
Let $\hat E$ be the reference pyramid and let $p>2$. Let $\bw_{\hat E}(\cdot)$ denote
the interpolation operator determined by the degrees of freedom in Definition~\ref{aux_label50}.
There is $C>0$ such that,
for all $\hat\bu\in W^{1,p}(\hat E)$ with first derivatives in $W^{1,1}(\hat E)$, there hold
\begin{IEEEeqnarray}{rCl}
  \nonumber\|(\wku)_1\|_{L^\infty(\hat E)}&\lesssim&
  \|\hat u_1\|_{\scriptscriptstyle W^{1,p}(\hat E)} +
  \|(\nabla\times\hat\bu)_2\|_{\scriptscriptstyle W^{1,1}(\hat E)} +  
  \|(\nabla\times\hat\bu)_3\|_{\scriptscriptstyle W^{1,1}(\hat E)}\\[4pt]
  \label{auxlabel203}
  &&\, +  \left\|\tfrac{\partial \hat u_1}{\partial\hat x_2}\right\|_{\scriptscriptstyle W^{1,1}(\hat E)} +
          \left\|\tfrac{\partial^2 \hat u_3}{\partial\hat x_2\partial\hat x_1}\right\|_{\scriptscriptstyle L^{1}(\hat E)}.\\[12pt]
  \nonumber\|(\wku)_2\|_{L^\infty(\hat E)}&\lesssim&
\|\hat u_2\|_{\scriptscriptstyle W^{1,p}(\hat E)} +
  \|(\nabla\times\hat\bu)_1\|_{\scriptscriptstyle W^{1,1}(\hat E)} +  
  \|(\nabla\times\hat\bu)_3\|_{\scriptscriptstyle W^{1,1}(\hat E)}\\[4pt]
  &&\, +  \left\|\tfrac{\partial \hat u_2}{\partial\hat x_1}\right\|_{\scriptscriptstyle W^{1,1}(\hat E)} +
          \left\|\tfrac{\partial^2 \hat u_3}{\partial\hat x_2\partial\hat x_1}\right\|_{\scriptscriptstyle L^{1}(\hat E)}.\\[12pt]
  \nonumber\|(\wku)_3\|_{L^\infty(\hat E)}&\lesssim&
  \|\hat u_3\|_{\scriptscriptstyle W^{1,p}(\hat E)} +
  \|(\nabla\times\hat\bu)_2\|_{\scriptscriptstyle W^{1,1}(\hat E)} +  
  \|(\nabla\times\hat\bu)_1\|_{\scriptscriptstyle W^{1,1}(\hat E)}\\[5pt]
  \IEEEeqnarraymulticol{3}{r}{\label{auxlabel209}
  +\left\|\tfrac{\partial \hat u_3}{\partial\hat x_1}\right\|_{\scriptscriptstyle W^{1,1}(\hat E)} +
        \left\|\tfrac{\partial \hat u_2}{\partial\hat x_1}\right\|_{\scriptscriptstyle W^{1,1}(\hat E)} +
        \left\|\tfrac{\partial \hat u_1}{\partial\hat x_2}\right\|_{\scriptscriptstyle W^{1,1}(\hat E)} +
        \left\|\tfrac{\partial^2 \hat u_3}{\partial\hat x_2\partial\hat x_1}\right\|_{\scriptscriptstyle L^{1}(\hat E)}.}
\end{IEEEeqnarray}
\end{theorem}
\begin{proof}
Take an element $\hat\bu$ of $W^{1,p}(\hat{E})$ for a $p > 2$.
Let us recall the shape funtions in Table~\ref{shape_edge_table}.
For the variables of the shape functions in upcoming computations we will write $x$,$y$ and $z$ instead of
$\hat x_i$ to get a cleaner reading. Start with $\hat\bu$ of the form $(\hat u_1,0,0)'$. After calculating we have
\begin{IEEEeqnarray*}{rCl} %%\nabla\times\hat\bu &=& (0, \tfrac{{\s\partial} \hat u_1}{{\s\partial} \hat x_3},-\tfrac{{\s\partial} \hat u_1}{{\s\partial} \hat x_2})'.\\[6pt]
	\wku	&=& [{\s\int_{\hat{\be}_1}\hat\bu\cdot d\hat\balpha_1}]\hat\bgamma_1 +
				    [{\s\int_{\hat{\be}_3}\hat\bu\cdot d\hat\balpha_3}]\hat\bgamma_3 + 
				    [{\s\int_{\hat{\be}_6}\hat\bu\cdot d\hat\balpha_6}]\hat\bgamma_6 + 
				    [{\s\int_{\hat{\be}_8}\hat\bu\cdot d\hat\balpha_8}]\hat\bgamma_8\\[5pt]
			&=:& \varphi_1(\hat\bu)\hat\bgamma_1 + 
				   \varphi_3(\hat\bu)\hat\bgamma_3 + 
				   \varphi_6(\hat\bu)\hat\bgamma_6 + 
				   \varphi_8(\hat\bu)\hat\bgamma_8.
\end{IEEEeqnarray*}
\begin{IEEEeqnarray*}{rCl}
  (\wku)_1(x,y,z) 
    &  = & \varphi_1(\hat\bu)(1-z-y)+ 
	  \varphi_3(\hat\bu)y+ 
	  \varphi_6(\hat\bu)(-z+\frac{yz}{1-z})\\[4pt]
    &&\,+\,  \varphi_8(\hat\bu)(-\frac{yz}{1-z})\\[4pt]
	& = & \varphi_1(\hat\bu) - (\varphi_1 + \varphi_6)(\hat\bu)\,z+ 
	  (\varphi_3 - \varphi_1)(\hat\bu)\,y\\[4pt]
  &&\, +\, (\varphi_6-\varphi_8)(\hat\bu)\,\frac{yz}{1-z}.
\end{IEEEeqnarray*}
Now we explore the new coefficients separately. As the tangential component of $\hat\bu$
along $\hat\be_5$ equals zero, and this is an argument we are using repeatedly in the forthcoming
computations, we may write, by Stokes' Theorem,
\begin{IEEEeqnarray*}{rCl}
  (\varphi_1+\varphi_6)(\hat\bu)
  	& = & \int_{\hat{\be}_1}\hat\bu\cdot d\hat{\balpha}_1  
        +	\int_{\hat{\be}_6}\hat\bu\cdot d\hat\balpha_6 -
  		  	\int_{\hat{\be}_5}\hat\bu\cdot d\hat\balpha_5 \\[5pt]
  	& = & \iint_{\hat{f}_1} \nabla\times\hat\bu\cdot\hat\bn\,d\hat S \\[5pt]
  	& = & -\iint_{\hat{f}_1} \tfrac{{\partial} \hat u_1}{{\partial}\hat x_3}\,d\hat S.
\end{IEEEeqnarray*}
Next,
\begin{IEEEeqnarray*}{rCl}
	(\varphi_3-\varphi_1)(\hat\bu) & = & (\varphi_3-\varphi_2-\varphi_1+\varphi_4)(\hat\bu)\\
	& = & - \int_{\partial\hat{f}_5}\hat{\bu}\cdot d\hat{\balpha}\\[4pt]
	&=& -\iint_{\hat{f}_5}\nabla\times\hat{\bu}\cdot\hat{\bn}_5\,d\hat S\\
	&=&	 \iint_{\hat{f}_5}\tfrac{{\partial} \hat{u}_1}{{\partial} \hat{x}_2}\,d\hat S.
\end{IEEEeqnarray*}
And
\begin{IEEEeqnarray*}{rCl}
  (\varphi_6-\varphi_8)(\hat\bu) & = & \int_{\hat\be_6}\hat\bu\cdot d\hat\balpha_6 -
    \int_{\hat\be_8}\hat\bu\cdot\hat\btau_6\,d\hat s-
	\int_{\hat\be_2}\hat\bu\cdot\hat\btau_6\,d\hat s\\[5pt]
	& = &-\int_{\partial\hat{f}_3}\hat\bu\cdot\hat\btau\,d\hat s  
	  =  -\iint_{\hat{f}_3}\nabla\times\hat\bu\cdot\hat\bn_3\,d\hat S
	  =   \tfrac{1}{\sqrt{2}}\iint_{\hat{f}_3}\tfrac{{\s\partial} \hat u_1}{{\s\partial} \hat x_2}\,d\hat S.
\end{IEEEeqnarray*}
So in this case in which $\hat\bu$ has null first and second components, it holds
\begin{IEEEeqnarray}{rCl}\label{first_a}
	\nonumber
  (\wku)_1 & = & \int_{\hat\be_1}\hat u_1\,d\hat\alpha_1 + 
                z\iint_{\hat f_1} \tfrac{{\s\partial}\hat u_1}{{\s\partial} \hat x_3}\,d\hat S +
                y\iint_{\hat f_5} \tfrac{{\s\partial}\hat u_1}{{\s\partial} \hat x_2}\,d\hat S\\
           &&\,+\frac{yz}{1-z}\,2^{-\nicefrac12}\iint_{\hat f_3} \tfrac{{\s\partial}\hat u_1}{{\s\partial} \hat x_2}\,d\hat S.
\end{IEEEeqnarray}
By exactly the last computation,
\begin{IEEEeqnarray}{rCl}\label{second_a}
  (\wku)_2 & = & (\varphi_6-\varphi_8)(\hat\bu)\frac{xz}{1-z}
  = \iint_{\hat{f}_3}\tfrac{{\s\partial}\hat u_1}{{\s\partial} \hat x_2}\,d\hat S\,\frac{xz}{1-z}.
\end{IEEEeqnarray}
Next,
\begin{IEEEeqnarray*}{rCl}
	(\wku)_3 & = &     \varphi_1(\hat\bu)\left(x-\frac{xy}{1-z}\right) + \varphi_3(\hat\bu)\frac{xy}{1-z}\\[6pt]
			 &   &\,+\,\varphi_6(\hat\bu)\left(x-\frac{xy}{1-z}+\frac{xyz}{(1-z)^2}\right)
		            +  \varphi_8(\hat\bu)\left(\frac{xy}{1-z}-\frac{xyz}{(1-z)^2}\right)\\[6pt]
			 & = &  (\varphi_1 + \varphi_6)(\hat\bu)\,x +
			 		(\varphi_3-\varphi_1+\varphi_8-\varphi_6)(\hat\bu)\,\frac{xy}{1-z}\\[6pt]
			 &   &\,+ (\varphi_6-\varphi_8)(\hat\bu)\,\frac{xyz}{(1-z)^2}.
\end{IEEEeqnarray*}
As $\hat\bu$ has zero tangential component along $\hat\be_5$ and $\hat\be_7$,
\begin{IEEEeqnarray*}{rCl}
  (\varphi_3-\varphi_1+\varphi_8-\varphi_6)(\hat\bu)&=&
  (\varphi_3-\varphi_7+\varphi_8)(\hat\bu)+(\varphi_5-\varphi_6-\varphi_1)(\hat\bu)\\[8pt]
  &=&-\int_{\partial\hat{f}_4}\hat\bu\cdot d\hat\balpha
   -\int_{\partial\hat{f}_1}\hat\bu\cdot d\hat\balpha\\[8pt]
  &=&-\iint_{\hat{f}_1}\nabla\times\hat\bu\cdot\hat\bn\,d\hat S
   -\iint_{\hat{f}_4}\nabla\times\hat\bu\cdot\hat\bn\,d\hat S\\[8pt]
%\yesnumber\label{pyr_edge_one}
  &=&\iint_{\hat{f}_1}(\nabla\times\hat\bu)_2\,d\hat S\\[8pt]
  & &\quad-2^{-\nicefrac{1}{2}}\iint_{\hat{f}_4}[(\nabla\times\hat\bu)_2 + (\nabla\times\hat\bu)_3]\,d\hat S.\\[8pt]
  &=&\iint_{\hat{f}_1}\tfrac{\partial\hat{u}_1}{\partial\hat{x}_3}\,d\hat S
  -2^{-\nicefrac{1}{2}}\iint_{\hat{f}_4}[\tfrac{\partial\hat{u}_1}{\partial\hat{x}_3}
   + \tfrac{\partial\hat{u}_1}{\partial\hat{x}_2}]\,d\hat S.
\end{IEEEeqnarray*}
We write down this component:
\begin{IEEEeqnarray}{rCl}\label{third_a}
	\nonumber
  (\wku)_3 & = & 
    -x\,\iint_{\hat{f}_1} \tfrac{{\s\partial} \hat u_1}{{\s\partial} \hat x_3}\,d\hat S
    +\frac{xyz}{(1-z)^2}\,2^{-\nicefrac12}\iint_{\hat{f}_3}\tfrac{{\s\partial} \hat u_1}{{\s\partial} \hat x_2}\,d\hat S\\[8pt]
    &&\,+\frac{xy}{1-z}
     \left\{\iint_{\hat{f}_1}\tfrac{\partial\hat{u}_1}{\partial\hat{x}_3}\,d\hat S
      -2^{-\nicefrac{1}{2}}\iint_{\hat{f}_4}[\tfrac{\partial\hat{u}_1}{\partial\hat{x}_3}
     +\tfrac{\partial\hat{u}_1}{\partial\hat{x}_2}]\,d\hat S\right\}
\end{IEEEeqnarray}
\noindent Now, as expected, we switch to $\hat\bu = (0,\hat u_2,0)'$. In this case we have
\begin{IEEEeqnarray*}{rCl}
  \wku     & = & \varphi_2(\hat{\bu})\,\hat\bgamma_2 +
	\varphi_4(\hat{\bu})\,\hat\bgamma_4+ \varphi_7(\hat{\bu})\,\hat\bgamma_7+\varphi_8(\hat{\bu})\,\hat\bgamma_8.\\[4pt]
  (\wku)_1 & = &(\varphi_7-\varphi_8)(\hat{\bu})\,\frac{yz}{1-z}\\[4pt]
  		   & = &(\varphi_7-\varphi_8 - \varphi_3)(\hat{\bu})\,\frac{yz}{1-z}\\[4pt]
  		   & = &\int_{\partial\hat{f}_4}\hat{\bu}\cdot d\hat\balpha\,\frac{yz}{1-z}\\[4pt]
  		   \yesnumber\label{first_b}
  		   & = &\iint_{\hat{f}_4} \nabla\times\hat\bu\cdot\hat\bn_4\,d\hat S\,\frac{yz}{1-z}
  		  \, = \,2^{-\nicefrac12} \iint_{\hat{f}_4} \tfrac{\partial\hat{u}_2}{\partial\hat{x}_1}\,d\hat S\,\frac{yz}{1-z}.
\end{IEEEeqnarray*}
For the next component,
\begin{IEEEeqnarray*}{rCl}
	(\wku)_2 & = &\varphi_4(\hat\bu) + (\varphi_2-\varphi_4)(\hat\bu)\,x -
	(\varphi_4+\varphi_7)(\hat\bu)\,z\\[4pt]
  &&\,+\, (\varphi_7-\varphi_8)(\hat\bu)\,\frac{xz}{1-z}.\\[4pt]
	(\varphi_2-\varphi_4)(\hat\bu) & = & (\varphi_2-\varphi_3-\varphi_4+\varphi_1)(\hat\bu)\\[4pt]
  &=&-\int_{\partial\hat{f}_5}\hat\bu\cdot d\hat\balpha\\[4pt]
  &=&-\iint_{\hat{f}_5}\nabla\times\hat{\bu}\cdot\hat\bn_5\,d\hat S
   =  \iint_{\hat{f}_5}\tfrac{\partial\hat{u}_2}{\partial\hat{x}_1}\,d\hat S.\\[4pt]
  (\varphi_4+\varphi_7)(\hat\bu) & = & 
  (\varphi_4+\varphi_7-\varphi_5)(\hat\bu)\\[4pt]
  &=& - \int_{\partial\hat{f}_2} \hat\bu\cdot d\hat\balpha\\[4pt]
  &=& -\iint_{\hat{f}_2}\nabla\times\hat\bu\cdot\hat\bn\,d\hat S~=~
      -\iint_{\hat{f}_2}\tfrac{{\s\partial} \hat u_2}{{\s\partial} \hat x_3}\,d\hat S\mbox{,}
\end{IEEEeqnarray*}
and we write down this second component of the interpolate
\begin{IEEEeqnarray}{rCl}
  \nonumber
  (\wku)_2& = & \int_{\hat\be_4}\hat u_2\,d\hat\alpha_4
               +x\iint_{\hat{f}_5}\tfrac{\partial\hat{u}_2}{\partial\hat{x}_1}\,d\hat S.
               +z\iint_{\hat{f}_2}\tfrac{{\s\partial} \hat u_2}{{\s\partial} \hat x_3}\,d\hat S\\
\label{second_b}
&&\,+\frac{xz}{1-z}\,2^{-\nicefrac12} \iint_{\hat{f}_4} \tfrac{\partial\hat{u}_2}{\partial\hat{x}_1}\,d\hat S.
\end{IEEEeqnarray}
And for the third one,
\begin{IEEEeqnarray*}{rCl}
	(\wku)_3&=&(\varphi_4+\varphi_7)(\hat\bu)\,y + (\varphi_2-\varphi_4-\varphi_7+\varphi_8)(\hat\bu)\,\frac{xy}{1-z}\\[4pt]
	& &\,+\,(\varphi_7-\varphi_8)(\hat\bu)\,\frac{xyz}{(1-z)^2}.\\[8pt]
	&=&(\varphi_2 - \varphi_4)(\hat\bu)\,\frac{xy}{1-z} + (\varphi_4+\varphi_7)(\hat\bu)\,y\\[8pt]
	& &-(\varphi_7-\varphi_8)(\hat\bu)\,\frac{xy}{(1-z)^2}.
\end{IEEEeqnarray*}
But  expressions for $(\varphi_2 - \varphi_4)(\cdot)$, $(\varphi_4+\varphi_7)(\cdot)$ and
$(\varphi_7-\varphi_8)(\cdot)$ were already stated above, so we have
\begin{IEEEeqnarray}{rCl}
  \nonumber
  (\wku)_3 & = & 
\frac{xy}{1-z}\,\iint_{\hat{f}_5}\tfrac{\partial\hat{u}_2}{\partial\hat{x}_1}\,d\hat S 
-y\,\iint_{\hat{f}_2}\tfrac{{\s\partial} \hat u_2}{{\s\partial} \hat x_3}\,d\hat S\\[8pt]
  \label{third_b}
  & &-2^{-\nicefrac12} \iint_{\hat{f}_4} \tfrac{\partial\hat{u}_2}{\partial\hat{x}_1}\,d\hat S\,\frac{xy}{(1-z)^2}.
\end{IEEEeqnarray}
\noindent{Finally for $\hat\bu = (0,0,\hat u_3)'$} it is
$\wku = \varphi_5(\hat\bu)\hat\bgamma_5 + 
		\varphi_6(\hat\bu)\hat\bgamma_6 + 
		\varphi_7(\hat\bu)\hat\bgamma_7 +
		\varphi_8(\hat\bu)\hat\bgamma_8$ and $\nabla\times\hat\bu = (\partial_2\hat{u}_3,
		-\partial_1\hat{u}_3,0)'.$\\[7pt]
First component of the interpolate:
\begin{IEEEeqnarray*}{rCl}
	(\wku)_1 & = &(\varphi_5-\varphi_6)(\hat\bu)z+
		(-\varphi_5+\varphi_6+\varphi_7-\varphi_8)(\hat\bu)\,\frac{yz}{1-z}.
\end{IEEEeqnarray*}
On one hand,
\begin{IEEEeqnarray*}{rCl}
	(\varphi_5-\varphi_6)(\hat\bu) & = & (\varphi_5-\varphi_6-\varphi_1)(\hat\bu) \\
	&=&-\iint_{\hat f_1}\nabla\times\hat\bu\cdot\hat\bn\,d\hat S
\end{IEEEeqnarray*}
On the other hand and analogously
\begin{IEEEeqnarray*}{rCl} 	
	(\varphi_7-\varphi_8)(\hat\bu) & = &	\iint_{\hat{f}_4}\nabla\times\hat\bu\cdot\hat\bn\,d\hat S
\end{IEEEeqnarray*}
% &=&-\iint_{\hat f_1}\partial_1 \hat{u}_3\,d\hat S.
so it follows   %% \noindent{\color{blue} cambia lo de region tipo I y eso? estoy haciendo $0\leqslant t_2 \leqslant 1; 0\leqslant t_1\leqslant 1-t_2$} 
\begin{IEEEeqnarray*}{rCl}
  (-\varphi_5+\varphi_6+\varphi_7-\varphi_8)(\hat\bu) & = & 
  \iint_{\hat{f}_1}\nabla\times\hat\bu\cdot\hat\bn\,d\hat S
  +\iint_{\hat{f}_4}\nabla\times\hat\bu\cdot\hat\bn\,d\hat S\\[4pt]
\IEEEeqnarraymulticol{3}{R}{
\begin{IEEEeqnarraybox*}{rCl}
\qquad&=&
  \int_{\mathbb{D}_{\hat f_1}}\tfrac{{\s\partial} \hat u_3}{{\s\partial} \hat x_1}
  (\Phi_{\hat f_1}(t_1,t_2))\,dt_1dt_2
  -\int_{\mathbb{D}_{\hat f_4}}\tfrac{{\s\partial} \hat u_3}{{\s\partial} \hat x_1}
  (\Phi_{\hat f_4}(t_1,t_2))\,dt_1dt_2\\[4pt]
&=&
  \int_0^1\int_0^{1-t_2} 
  \left[\tfrac{{\s\partial} \hat u_3}{{\s\partial} \hat x_1}(t_1,0,t_2)
    - \tfrac{{\s\partial} \hat u_3}{{\s\partial} \hat x_1}(t_1,1-t_2,t_2)\right]
  \,dt_1dt_2\\[4pt]
&=&
  -\int_0^1\int_0^{1-t_2}\int_0^{1-t_2} 
  \tfrac{{\s\partial^2} \hat u_3}{{\s\partial} \hat x_2{\s\partial} \hat x_1}(t_1,s,t_2)
  \,dsdt_1dt_2\\[6pt]
&=&-\int_{\hat{E}}\tfrac{{\s\partial}^2\hat u_3}{{\s\partial} \hat x_2{\s\partial} \hat x_1}
\,d\hat\bx.
\end{IEEEeqnarraybox*}
}
\end{IEEEeqnarray*}
For now we obtained
\begin{IEEEeqnarray}{rCl}\label{first_c}
	(\wku)_1 & = & -z\iint_{\hat f_1}\tfrac{{\s\partial} \hat u_3}{{\s\partial} \hat x_1}\,d\hat S
	-\frac{yz}{1-z}\int_{\hat{E}}
		\tfrac{{\s\partial}^2\hat u_3}{{\s\partial} \hat x_2{\s\partial} \hat x_1}\,d\hat\bx.
\end{IEEEeqnarray}
Regarding the second component, it is the symmetrical case, so we write
\begin{IEEEeqnarray*}{rCl}
  (\wku)_2& = &(\varphi_5-\varphi_7)(\hat\bu)z
		+(-\varphi_5+\varphi_6+\varphi_7-\varphi_8)(\hat\bu)\frac{xz}{1-z}\\
	& = &z\iint_{\hat f_2}\nabla\times\hat\bu\cdot\hat\bn\,d\hat S
	-\frac{xz}{1-z}\int_{\hat{E}}
	\tfrac{{\s\partial}^2\hat u_3}{{\s\partial} \hat x_2{\s\partial} \hat x_1}\,d\hat\bx\\
	\yesnumber\label{second_c}
	& = &-z\iint_{\hat f_2}\tfrac{{\s\partial} \hat u_3}{{\s\partial} \hat{x_2}}\,d\hat S
		-\frac{xz}{1-z}\int_{\hat{E}}
	\tfrac{{\s\partial}^2\hat u_3}{{\s\partial} \hat x_2{\s\partial} \hat x_1}\,d\hat\bx.
\end{IEEEeqnarray*}
For the third component let us denote $\xi(x,y,z) = 
  \frac{xyz}{(1-z)^2}-\frac{xz}{1-z}$. Then
\begin{IEEEeqnarray*}{rCl}
  (\wku)_3& = & \varphi_5(\hat\bu) + x\,(\varphi_6-\varphi_5)(\hat\bu)+
  y\,(\varphi_7-\varphi_5)(\hat\bu)\\[5pt]
  && \,+\,\xi(x,y,z)\,(\varphi_6-\varphi_5+\varphi_7-\varphi_8)(\hat\bu)\\[5pt]
  &=& \int_{\hat\be_5}\hat\bu\cdot d\hat\balpha_5 + 
  y\,\iint_{\hat{f}_2}\tfrac{{\s\partial} \hat u_3}{{\s\partial} {x_2}}\,d\hat S +
  x\,\iint_{\hat{f}_1}\tfrac{{\s\partial} \hat u_3}{{\s\partial} {x_1}}\,d\hat S
  \\[5pt]&&\,-\,\xi(x,y,z)\int_{\hat{E}}
    \tfrac{{\s\partial}^2\hat u_3}{{\s\partial} \hat x_2{\s\partial} \hat x_1}\,d\hat\bx.
  \yesnumber\label{third_c}
\end{IEEEeqnarray*}
%\begin{IEEEeqnarray*}{rCl}
%	(\wku)_3 & = & \int_{0}^1u_3(0,0,t)\,dt
%				+ x \int_{0}^{1}\int_{0}^{1-t}
%						\tfrac{{\s\partial} \hat u_3}{{\s\partial} \hat x_1} (s,0,t) \,d\hat s\,dt
%				+ y \int_{0}^{1}\int_{0}^{1-t}
%						\tfrac{{\s\partial} \hat u_3}{{\s\partial} \hat x_2}(0,s,t) \,d\hat s\,dt\\
%				&&\,\xi\int_{\hat{E}}
%				\tfrac{{\s\partial}^2\hat u_3}{{\s\partial} \hat x_2{\s\partial} \hat x_1}\,d\hat\bx.
%\end{IEEEeqnarray*}
All together, for a $\hat\bu=(\hat u_1,\hat u_2,\hat u_3)'$, if we combine 
what was obtained in~(\ref{first_a})--(\ref{third_c}) 
then it holds
\begin{IEEEeqnarray*}{rCl}
  (\wku)_1 & = & 
    \int_{\hat\be_1}\hat u_1\,d\hat\alpha + 
  z \iint_{\hat f_1}(\nabla\times\hat\bu)_2\,d\hat S +
  y \iint_{{\hat f_5}}\tfrac{{\s\partial} \hat u_1}{{\s\partial} \hat x_2}\,d\hat S\\[6pt]
    &&\,
+\frac{yz}{1-z} \iint_{\hat f_3} \tfrac{{\s\partial} \hat u_1}{{\s\partial} \hat x_2}\,d\hat S +
 \frac{yz}{1-z} \iint_{\hat f_4} \tfrac{{\s\partial} \hat u_2}{{\s\partial} \hat x_1}\,d\hat S\\[6pt]
    &&\,
-\frac{yz}{1-z} \int_{\hat{E}}\tfrac{{\s\partial}^2\hat u_3}{{\s\partial} \hat x_2{\s\partial} \hat x_1}\,d\hat\bx.\\[12pt]
    (\wku)_2 & = & \int_{\hat\be_4}\hat u_2\,d\hat\alpha - 
    z \iint_{\hat f_2}(\nabla\times\hat\bu)_1\,d\hat S +
    x \iint_{\hat f_5}\tfrac{{\s\partial} \hat u_2}{{\s\partial} \hat x_1}\,d\hat S\\
    &&\,+\frac{xz}{1-z} \iint_{\hat f_3}
    \tfrac{{\s\partial} \hat u_1}{{\s\partial} \hat x_2}\,d\hat S +
    \frac{xz}{1-z} \iint_{\hat f_4}
    \tfrac{{\s\partial} \hat u_2}{{\s\partial} \hat x_1}\,d\hat S\\
    &&\,+\frac{xz}{1-z} \int_{\hat{E}}
    \tfrac{{\s\partial}^2\hat u_3}{{\s\partial} \hat x_2{\s\partial} \hat x_1}\,d\hat\bx.\\[12pt]
  (\wku)_3 & = & \int_{\hat\be_5}\hat u_3\,d\hat\alpha - 
    x \iint_{\hat{f}_1} (\nabla\times\hat\bu)_2\,d\hat S +
    y \iint_{\hat{f}_2} (\nabla\times\hat\bu)_1\,d\hat S\\[8pt]
  &&\,+\frac{xy}{1-z}
\left\{
  \iint_{\hat{f}_5}\tfrac{{\s\partial} \hat u_2}{{\s\partial} \hat x_1}\,d\hat S+
  \iint_{\hat{f}_1}\tfrac{{\s\partial} \hat u_1}{{\s\partial} \hat x_3}\,d\hat S-
  2^{-\nicefrac12}\iint_{\hat{f}_4}\tfrac{{\s\partial} \hat u_1}{{\s\partial} \hat x_3}\,d\hat S
\right.\\[8pt]
  &&\,-
\left.
  2^{-\nicefrac12}\iint_{\hat{f}_4}\tfrac{{\s\partial} \hat u_1}{{\s\partial} \hat x_2}\,d\hat S
\right\}-
\frac{xy}{(1-z)^2}
2^{-\nicefrac12}\iint_{\hat{f}_4}\tfrac{{\s\partial} \hat u_2}{{\s\partial} \hat x_1}\,d\hat S\\[8pt]
\yesnumber\label{aux_label42}
&&\,+
\frac{xyz}{(1-z)^2}
2^{-\nicefrac12}\iint_{\hat{f}_3}\tfrac{{\s\partial} \hat u_1}{{\s\partial} \hat x_2}\,d\hat S+
\xi(x,y,z)\,
\int_{\hat{E}}
  \tfrac{{\s\partial}^2 u_3}{{\s\partial} \hat x_1{\s\partial} \hat x_2}\,d\hat\bx.
\end{IEEEeqnarray*}
From here we apply Lemma~\ref{auxlabel350} and the result follows.
\end{proof}
We continue with the local interpolation error estimate.
\rescaledPyramidTikz
\begin{theorem} \label{auxlabel211}
  Let $E$ be any pyramid which is
  a non degenerate affine image 
  of the reference pyramid $\hat{E}$. We fix a positively oriented local system of 
  coordinates $(\bxi_1, \bxi_2, \bxi_3)$
  with origin in a vertex $\bx_E$ of the parallelogram basis, for which $(\bxi_1, \bxi_2)$
  correspond to the two basis edges incident to $\bx_E$ and $\bxi_3$ is parallel to the 
  edge joining $\bx_E$ with the top of the pyramid. Let $h_1, h_2, h_3$ be the corresponding 
  edge lengths. With $\partial^{\balpha}$ we denote 
  $\tfrac{\partial^{|\balpha|}}{\partial_{\bxi_1}^{\alpha_1}\partial_{\bxi_2}^{\alpha_2}\partial_{\bxi_3}^{\alpha_3}}$.
  Suppose that
  $h_3 \geqslant \min \{h_1, h_2\}$ and let  $p>2$.
  For all $\bu\in W^{2,p}(E)^3$
\begin{IEEEeqnarray*}{rCl}\label{aux_label55}
  \|\bu-\bw_E \bu\|_{L^p(E)} & \lesssim &
    \sum_{|{\balpha}|=1}\bh^{\balpha} \|\partial^{\balpha} \bu\|_{L^p(E)} +\\[4pt]
   &&\,+\,h_E\big\{\|\curl\bu\|_{\scriptscriptstyle L^p(E)} + 
   \sum_{|{\balpha}|=1}\bh^{\balpha}\|\partial^{\balpha} \curl \bu\|_{\scriptscriptstyle L^p(E)}\big\}\\
  & &\,+\,h_E^2 |\bu|_{2,p,E} \\[5pt]
  & &\,+\,\max \{h_{1}, h_2\} \big\{ \|\partial_{\tilde{x}_1}\tilde{u}_2\|_{\scriptscriptstyle L^p(\tilde E)}
   + \|\partial_{\tilde{x}_2}\tilde{u}_1\|_{\scriptscriptstyle L^p(\tilde E)}\big\}.
\end{IEEEeqnarray*} 
\end{theorem}
\begin{proof}
  Consider the matrix $M_{\tilde{E}}$ with coefficients $h_i\delta_{i,j}$, 
  $1\leqslant i,j\leqslant 3$ and take $\tilde{E}$ as the rescaled reference
  pyramid, that is, $\tilde{E} = M_{\tilde{E}}\hat{E}$. In 
  Figure~\ref{rescaled_pyramid} we have illustrated the scaling.
  Let us start with a stability estimate in $\tilde{E}$. Given a field $\tilde{\bu}$
  in $\tilde{E}$, pulling $\tilde{\bu}$ back to $\hat{E}$, using~\eqref{auxlabel203}
  and pushing forward to $\tilde{E}$ we get
  \begin{IEEEeqnarray*}{rCl}
    \|(\bw_{\tilde{E}}\tilde{\bu})_1\|_{\scriptscriptstyle L^\infty(\tilde{E})} &\lesssim&
      |\tilde{E}|^{-1/p}
      \big\{
        \|\tilde u_1\|_{\scriptscriptstyle L^p(\tilde E)} + 
          \sum_{i=1}^3 h_i \|\partial_{\tilde{x}_i}\tilde{u}_1\|_{\scriptscriptstyle L^p(\tilde{E})}
      \big\} \\[5pt]
    \IEEEeqnarraymulticol{3}{r}{+\,
      |\tilde{E}|^{\scriptscriptstyle -1} h_2
      \big\{
        \|(\nabla\times\tilde{\bu})_3\|_{\scriptscriptstyle L^1(\tilde{E})} + 
        \sum_{i=1}^3h_i(\|\partial_{\tilde{x}_i}(\nabla\times\tilde{\bu})_3\|_{\scriptscriptstyle L^1(\tilde{E})} +
                     \|\tfrac{\partial^2\tilde{u}_1}{\partial\tilde{x}_i\partial\tilde{x}_2}\|_{\scriptscriptstyle L^1(\tilde{E})})
      \big\}} \\[5pt]
    &&\,+\,
      |\tilde{E}|^{-1} h_3
      \big\{
        \|(\nabla\times\tilde{\bu})_2\|_{\scriptscriptstyle L^1(\tilde{E})} + 
             \sum_{i=1}^3h_i(\|\partial_{\tilde{x}_i}(\nabla\times\tilde{\bu})_2\|_{\scriptscriptstyle L^1(\tilde{E})}
      \big\} \\[5pt]
    &&\,+\,
      |\tilde{E}|^{-1} h_2h_3 \|\tfrac{\partial^2\tilde{u}_3}{\partial\tilde x_1\partial\tilde x_2}\|_{L^1(\tilde E)}.
  \end{IEEEeqnarray*}
  Estimate for component number two yields the analogue and now we write
  something similar to the third component. Note that in some cases we group terms
  using 
  \begin{IEEEeqnarray}{rCl}\label{auxlabel213}
  |\tilde E|^{-\tfrac{1}{q}}\|g\|_{\scriptscriptstyle L^q(\tilde E)} &\leqslant &
  |\tilde E|^{-\tfrac{1}{p}}\|g\|_{\scriptscriptstyle L^p(\tilde E)}\mbox{,}
  \end{IEEEeqnarray}
  whenever $q<p$, for scalar 
  functions
  in $L^p$. From~\eqref{auxlabel209}
  \begin{IEEEeqnarray*}{rCl}
    \|(\tilde\bw_{\tilde{E}}\tilde{\bu})_3\|_{L^\infty(\tilde{E})}&\lesssim&
    |\tilde{E}|^{-1/p}
    \big\{ 
      \|\tilde{u}_3\|_{\scriptscriptstyle L^p(\tilde{E})} + 
      \sum_{i=1}^3 h_i \|\tfrac{\partial\tilde{u}_3}{\partial\tilde{x}_i}\|_{\scriptscriptstyle L^p(\tilde{E})}
    \big\}\\[5pt]
    &&\,+\,|\tilde{E}|^{-1}h_1
    \big\{
      \|(\nabla\times\tilde{\bu})_2\|_{\scriptscriptstyle L^1(\tilde{E})}+
      \\[5pt]
    &&\,+
      \sum_{i=1}^3h_i
      (
        \|\partial_{\tilde{x}_i}(\nabla\times\tilde{\bu})_2\|_{\scriptscriptstyle L^1(\tilde{E})}+
        \|\tfrac{\partial^2\tilde{u}_3}{\partial\tilde{x}_i\partial\tilde{x}_1}\|_{\scriptscriptstyle L^1(\tilde{E})}
      )
    \big\}\\[5pt]
    \IEEEeqnarraymulticol{3}{r}{\,+\,|\tilde{E}|^{-1}h_2
        \big\{
        \|(\nabla\times\tilde{\bu})_1\|_{\scriptscriptstyle L^1(\tilde{E})} + 
                 \sum_{i=1}^3h_i(\|\partial_{\tilde{x}_i}(\nabla\times\tilde{\bu})_1\|_{\scriptscriptstyle L^1(\tilde{E})}
        \big\}}\\[5pt]
    &&  \,+\,|\tilde{E}|^{-1}\tfrac{h_1h_2}{h_3}
    \big\{
      \|\partial_{\tilde{x}_1}\tilde{u}_2\|_{\scriptscriptstyle L^1(\tilde{E})} + 
      \|\partial_{\tilde{x}_2}\tilde{u}_1\|_{\scriptscriptstyle L^1(\tilde{E})}\\[5pt]
    &&\,+ 
\sum_{i=1}^3 h_i
      (
        \|\tfrac{\partial^2\tilde{u}_1}{\partial\tilde{x}_i\partial\tilde{x}_2}\|_{\scriptscriptstyle L^1(\tilde{E})}+
        \|\tfrac{\partial^2\tilde{u}_2}{\partial\tilde{x}_i\partial\tilde{x}_1}\|_{\scriptscriptstyle L^1(\tilde{E})}
      )
    \big\}
    \\[5pt]
    &&\,+\,|\tilde{E}|^{-1}h_1h_2\|\tfrac{\partial^2\tilde{u}_3}{\partial\tilde{x}_1\partial\tilde{x}_2}\|_{\scriptscriptstyle L^1(\tilde{E})}.
  \end{IEEEeqnarray*}
Proceeding as in the proof of Theorem~\ref{aux_label27}
we obtain the following vectorial stability estimate in $\tilde E$:
\begin{IEEEeqnarray*}{rCl}
  \| \bw_{\tilde E}\tilde{\bu} \|_{\scriptscriptstyle L^p(\tilde E)}
  & \leqslant & \|\tilde{\bu}\|_{\scriptscriptstyle L^p(\tilde E)}
     + \sum_{i=1}^3 h_i\|\partial_{\tilde{x_i}}\tilde{\bu}\|_{\scriptscriptstyle L^p(\tilde E)} \\[5pt]  
  & &\,+\, \max \{h_{i}\} \left( \|\nabla\times\tilde{\bu}\|_{\scriptscriptstyle L^p(\tilde E)} + 
   \sum_{i=1}^3 h_i \|\partial_{\tilde{x}_i}\nabla\times\tilde{\bu}\|_{\scriptscriptstyle L^p(\tilde E)} \right) \\[5pt]
  & &\,+\, \max \{h_{i}\}^2 |\tilde{\bu}|_{2,p,\tilde{E}} \\[5pt]
  & &\,+\, \max \{h_{1}, h_2\} \big( \|\partial_{\tilde{x}_1}\tilde{u}_2\|_{\scriptscriptstyle L^p(\tilde E)}
   + \|\partial_{\tilde{x}_2}\tilde{u}_1\|_{\scriptscriptstyle L^p(\tilde E)} \big)
\end{IEEEeqnarray*}
And now we proceed as in the proof of Theorem~\ref{aux_label32}. First we 
transform from
a physical pyramidal element $E$ to $\tilde{E}$. In Section 5 of~\cite{gh99}
the approximation property of the finite element is stated and then we add the 
estimate~\eqref{aux_label30} for the rescaled pyramid $\tilde E$ in the case with 
multi--indices of order two,
to use in the corresponding terms of the averaged Taylor polynomial approximation
and the result follows.
\end{proof}













%%==========================================================================
%{\color{brown}
%    \begin{IEEEeqnarray*}{rCl}
%        (\wku)_1 & = & \int_{0}^{1}u_1(t,0,0)\,dt + 
%        z \int_0^1\int_0^{1-t_1}
%        \tfrac{{\s\partial} \hat u_1}{{\s\partial} \hat x_3}(t_1,0,t_2)\,dt_2dt_1 +
%        y \int_{{\hat f_5}}
%        \tfrac{{\s\partial} \hat u_1}{{\s\partial} \hat x_2}\,d\hat S\\
%        &&\,+\frac{yz}{1-z} \int_0^1\int_0^{1-t}
%        \tfrac{{\s\partial} \hat u_1}{{\s\partial} \hat x_2}(1-t,s,t)\,d\hat sdt +
%        \frac{yz}{1-z} \int_0^1\int_0^{1-t}
%        \tfrac{{\s\partial} \hat u_2}{{\s\partial} \hat x_1}(s,1-t,t)\,d\hat sdt\\
%        &&\,-z\int_0^1\int_0^{1-t_1}
%        \tfrac{{\s\partial} \hat u_3}{{\s\partial} \hat x_1}(t_1,0,t_2)\,dt_2dt_1 -
%        \frac{yz}{1-z} \int_{\hat{E}}
%        \tfrac{{\s\partial}^2\hat u_3}{{\s\partial} \hat x_2{\s\partial} \hat x_1}\,dV.
%    \end{IEEEeqnarray*}
%}
%%=========================================================================

%%===================================================================
%{\color{brown}
%\begin{IEEEeqnarray*}{rCl}
%    (\wku)_2 & = & \int_{0}^{1}u_2(0,t,0)\,dt + 
%    z \int_0^1\int_0^{1-t}
%    \tfrac{{\s\partial} \hat u_2}{{\s\partial} \hat x_3}(0,t,s)\,d\hat sdt +
%    x \int_0^1\int_0^{1}
%    \tfrac{{\s\partial} \hat u_2}{{\s\partial} \hat x_1}(s,t,0)\,d\hat sdt\\
%    &&\,+\frac{xz}{1-z} \int_0^1\int_0^{1-t}
%    \tfrac{{\s\partial} \hat u_1}{{\s\partial} \hat x_2}(1-t,s,t)\,d\hat sdt +
%    \frac{xz}{1-z} \int_0^1\int_0^{1-t}
%    \tfrac{{\s\partial} \hat u_2}{{\s\partial} \hat x_1}(s,1-t,t)\,d\hat sdt\\
%    &&\,-z\int_0^1\int_0^{1-t}
%    \tfrac{{\s\partial} \hat u_3}{{\s\partial} \hat x_2}(0,s,t)\,d\hat sdt +
%    \frac{xz}{1-z} \int_{\hat{P}}
%    \tfrac{{\s\partial}^2\hat u_3}{{\s\partial} \hat x_2{\s\partial} \hat x_1}\,dV.
%\end{IEEEeqnarray*}
%}
%%==================================================================

%%===========================================================================
%{\color{brown}
%\begin{IEEEeqnarray*}{rCl}
%        (\wku)_3 & = & \int_{0}^{1}u_3(0,0,t)\,dt + 
%        x \int_0^1\int_0^{1-t}
%        \tfrac{{\s\partial} \hat u_3}{{\s\partial} \hat x_1}(s,0,t)\,d\hat sdt -
%        x \int_0^1\int_0^{1-t}
%        \tfrac{{\s\partial} \hat u_1}{{\s\partial} \hat x_3}(t,0,s)\,d\hat sdt\\
%        &&\,+y \int_0^1\int_0^{1-t}
%        \tfrac{{\s\partial} \hat u_3}{{\s\partial} \hat x_2}(0,s,t)\,d\hat sdt -
%        y \int_0^1\int_0^{1-t}
%        \tfrac{{\s\partial} \hat u_2}{{\s\partial} \hat x_3}(0,t,s)\,d\hat sdt\\
%        &&\,+\frac{xy}{1-z} \int_0^1\int_t^{1}
%        \tfrac{{\s\partial} \hat u_1}{{\s\partial} \hat x_2}(t,s,0)\,d\hat sdt +
%        \frac{xy}{1-z} \int_0^1\int_0^{t}
%        \tfrac{{\s\partial} \hat u_2}{{\s\partial} \hat x_1}(t,s,1-t)\,d\hat sdt\\
%        &&\,+\frac{xyz}{(1-z)^2} \int_0^1\int_0^{1-t}
%        \tfrac{{\s\partial} \hat u_2}{{\s\partial} \hat x_1}(s,1-t,t)\,d\hat sdt
%        +\frac{xyz}{(1-z)^2} \int_0^1\int_0^{1-t}
%        \tfrac{{\s\partial} \hat u_1}{{\s\partial} \hat x_2}(1-t,s,t)\,d\hat sdt\\
%        &&\,-\frac{xy}{1-z}
%        \int_{0}^{1}
%        \int_{0}^{t}
%        \int_{0}^{1-t}
%        \tfrac{{\s\partial}^2u_1}{{\s\partial}x_2{\s\partial}x_3}(t,s,r)\,dr\,d\hat s\,dt
%        -\frac{xy}{1-z}
%        \int_{0}^{1}
%        \int_{0}^{t}
%        \int_{0}^{t}
%        \tfrac{{\s\partial}^2u_2}{{\s\partial}x_1{\s\partial}x_3}(r,s,1-t)\,dr\,d\hat s\,dt\\
%        &&\,
%        +\frac{xyz}{(1-z)^2} \int_{\hat{P}}
%        \tfrac{{\s\partial}^2 u_3}{{\s\partial} \hat x_1{\s\partial} \hat x_2}\,dV
%        -\frac{xz}{1-z} \int_{\hat{P}}
%        \tfrac{{\s\partial}^2 u_3}{{\s\partial} \hat x_1{\s\partial} \hat x_2}\,dV
%    \end{IEEEeqnarray*}
%    }
%%======================================================================================

% &=&-\int_0^1\int_0^{1-t}\tfrac{{\s\partial} \hat u_3}{{\s\partial} \hat x_1}(s,0,t)\,d\hat sdt.

%	(\pi\bu)_2 & = &\varphi_4 + (\varphi_2-\varphi_4)x -
%	(\varphi_4+\varphi_7)z + (\varphi_7-\varphi_8)\frac{xz}{1-z}.\\
%	\varphi_4 = \int_{0}^{1}u_2(0,t,0)\,dt\\
%	\varphi_2-\varphi_4 & = & \int_{0}^{1} u_2(1,t,0)-u_2(0,t,0)\,dt\\
%		&=&\int_{0}^{1}\int_{0}^{1}\tfrac{{\s\partial} \hat u_2}{{\s\partial} \hat x_1}(s,t,0)\,d\hat sdt\\
%	\varphi_4+\varphi_7 & = & \int_0^1 u_2(0,t,0)-u_2(0,t,1-t)\,dt\\
%		& = & \int_0^1\int_0^{1-t}\tfrac{{\s\partial} \hat u_2}{{\s\partial} \hat x_3}(0,t,s)\,d\hat sdt.\\
%	\varphi_7-\varphi_8&=&\int_{0}^{1} u_2(1-t,1-t,t)-u_2(0,1-t,t)\,dt\\
%		&=&\int_{0}^{1}\int_{0}^{1-t}\tfrac{{\s\partial} \hat u_2}{{\s\partial} \hat x_1}(s,1-t,t)\,d\hat sdt.

% subsection edge_elements (end)
\subsection{Anisotropic Stability Estimates for $H(\text{div})$--Conforming 
Elements on Pyramids} % (fold)
\label{sub:face_elements}
Here we will work on the div--conforming analogue of
Theorem~\ref{aux_label53}.
\begin{theorem} \label{aux_label54}
\begin{IEEEeqnarray*}{rCl}
  \|(\rku)_1\|_{\scriptscriptstyle{L^\infty(\hat{E})}}
  &\lesssim& \|\hat u_1\|_{\scriptscriptstyle{W^{1,p}(\hat{E})}} +
    \|\dv \hat\bu\|_{\scriptscriptstyle{L^p}(\hat{E})} + 
    \left\|\hat{u}_3\right\|_{\scriptscriptstyle{W^{1,p}}(\hat{E})}\\[12pt]
  \|(\rku)_2\|_{\scriptscriptstyle{L^\infty(\hat{E})}}
  &\lesssim& \|\hat u_2\|_{\scriptscriptstyle{W^{1,p}(\hat{E})}} +
    \|\dv \hat\bu\|_{\scriptscriptstyle{L^p}(\hat{E})} + 
    \left\|\hat{u}_3\right\|_{\scriptscriptstyle{W^{1,p}}(\hat{E})}\\[12pt]
  \|(\rku)_3\|_{\scriptscriptstyle{L^\infty(\hat{E})}} & \lesssim & 
    \|\hat u_3\|_{\scriptscriptstyle{W^{1,p}(\hat{E})}} +
    \|\dv \hat\bu\|_{\scriptscriptstyle{L^p}(\hat{E})}.
\end{IEEEeqnarray*}
\end{theorem}
\begin{proof}
We will use the notation of Table~\ref{shape_face_table} for the 
shape functions and Tables~\ref{pyramidNotationTableFaces} and~\ref{pyramidNotationTableEdges}
for the boundary of the reference pyramid. This proof is based on explicit computation as well.
The variables 
in the local coordinate system of $\hat E$ for the shape functions $\hat\bz_i$ 
are $x$, $y$ and $z$ instead
of $\hat x_1$, $\hat x_2$ and $\hat x_3$.\\[5pt]
Consider the case $\hat{\bu} = (\hat{u}_1,0,0)'$ to start with and compute it's 
interpolate. 
\begin{IEEEeqnarray*}{rCl}
  \rku & = & \{{\scriptstyle\iint_{\hat{f}_2} \hat\bu \cdot \hat\bn_2\,d\hat S}\}\,\hat\bz_2 + 
             \{{\scriptstyle\iint_{\hat{f}_3} \hat\bu \cdot \hat\bn_3\,d\hat S}\}\,\hat\bz_3\\[4pt]
       & =: & \rho_2(\hat\bu)\,\hat\bz_2 + \rho_3(\hat\bu)\,\hat\bz_3.
\end{IEEEeqnarray*}
Then for the first two components of the interpolate it holds
\begin{IEEEeqnarray*}{rCl}
  (\rku)_1(x,y,z) & = & -2\rho_2(\hat\bu) + 
    \{\rho_2(\hat\bu)+\rho_3(\hat\bu)\}\,\tfrac{2x-xz}{1-z}\\[4pt]
    & = & -2{\iint_{\hat{f}_2} \hat{\bu} \cdot \hat\bn_2\,d\hat S}\\[4pt]
    &&\, +\,\left\{
          {\iint_{\hat{f}_2} \hat{\bu} \cdot \hat\bn_2\,d\hat S}+
                  {\iint_{\hat{f}_3} \hat{\bu} \cdot \hat\bn_3\,d\hat S}\right\}
                  \tfrac{2x-xz}{1-z}\\[4pt]
    & = & -2{\iint_{\hat{f}_2} \hat{\bu} \cdot \hat\bn_2\,d\hat S} + 
          {\iint_{\partial\hat{E}} \hat{\bu} \cdot \hat\bn\,d\hat S}\,\tfrac{2x-xz}{1-z}\\[4pt]
    & = & -2{\iint_{\hat{f}_2} \hat{\bu} \cdot \hat\bn_2\,d\hat S} + 
            {\int_{\hat{E}} \dv\hat{\bu} \,d\hat{\boldsymbol{x}}}\,\tfrac{2x-xz}{1-z}
\end{IEEEeqnarray*}
and
\begin{IEEEeqnarray*}{rCl}
  (\rku)_2\xyz & = & -(\rho_2(\hat\bu)+\rho_3(\hat\bu))\,\tfrac{yz}{1-z}\\[4pt]
    & = & -{\int_{\hat{E}} \dv\hat{\bu} \,d\hat{\boldsymbol{x}}}\,\tfrac{yz}{1-z}.
\end{IEEEeqnarray*}
Switch to $\hat{\bu}$ of the form $(0,\hat{u}_2,0)'$.
\begin{IEEEeqnarray*}{rCl}
  \rku & = & ({\scriptstyle\iint_{\hat{f}_1} \hat\bu \cdot \hat\bn_1\,d\hat S})\,\hat\bz_1 + 
         ({\scriptstyle\iint_{\hat{f}_4} \hat\bu \cdot \hat\bn_4\,d\hat S})\,\hat\bz_4\\[4pt]
       & = & \rho_1(\hat\bu)\,\hat\bz_1 + \rho_4(\hat\bu)\,\hat\bz_4.
\end{IEEEeqnarray*}
Then summing up yields, for now,
\begin{IEEEeqnarray*}{rCl}
  (\rku)_1(x,y,z) & = & -(\rho_1(\hat\bu)+\rho_4(\hat\bu))\,\tfrac{xz}{1-z}\\[4pt]
    & = & -{\int_{\hat{E}} \dv\hat{\bu} \,d\hat{\boldsymbol{x}}}\,\tfrac{xz}{1-z}.\\[8pt]
  (\rku)_2(x,y,z) & = & -2\rho_1(\hat\bu) + 
  (\rho_1(\hat\bu)+\rho_4(\hat\bu))\,\tfrac{2y-yz}{1-z}\\[4pt]
    & = & -2{\iint_{\hat{f}_1} \hat{\bu} \cdot \hat\bn_1\,d\hat S} + 
            {\iint_{\partial\hat{E}} \hat{\bu} \cdot \hat\bn\,d\hat S}\,\tfrac{2y-yz}{1-z}\\[4pt]
    & = & -2{\iint_{\hat{f}_1} \hat{\bu} \cdot \hat\bn_1\,d\hat S} + 
            {\int_{\hat{E}} \dv\hat{\bu} \,d\hat{\boldsymbol{x}}}\,\tfrac{2y-yz}{1-z}.\\[8pt]
\end{IEEEeqnarray*}
Now continue with $\hat{\bu}$ of the form $(0,0,\hat{u}_3)'$.
\begin{IEEEeqnarray*}{rCl}
  \rku & = & ({\scriptstyle\iint_{\hat{f}_3} \hat\bu \cdot \hat\bn_3\,d\hat S})\,\hat\bz_3 + 
         ({\scriptstyle\iint_{\hat{f}_4} \hat\bu \cdot \hat\bn_4\,d\hat S})\,\hat\bz_4 + 
         ({\scriptstyle\iint_{\hat{f}_5} \hat\bu \cdot \hat\bn_5\,d\hat S})\,\hat\bz_5\\[4pt]
       & =: & \rho_3(\hat\bu)\,\hat\bz_3 + \rho_4(\hat\bu)\,\hat\bz_4
       + \rho_5(\hat\bu)\,\hat\bz_5.
\end{IEEEeqnarray*}
Then
\begin{IEEEeqnarray*}{rCl}
  (\rku)_1(x,y,z) & = & \{\rho_3(\hat\bu) + \rho_5(\hat\bu)\}\,x
  + \rho_3(\hat\bu) \tfrac{x}{1-z} - \rho_4\tfrac{xz}{1-z}.
\end{IEEEeqnarray*}
Now observe that
\begin{IEEEeqnarray*}{rCl}
  (\rho_3 + \rho_5)(\hat\bu) & = & 
    {\iint_{\partial\hat{E}} \hat{\bu} \cdot \hat\bn\,d\hat S} - 
      {\iint_{\hat{f}_4} \hat{\bu} \cdot \hat\bn_4\,d\hat S} \\[4pt]
  & = & {\int_{\hat{E}} \dv\hat{\bu}\,d\hat{\boldsymbol{x}}} - 
        \rho_4(\hat{\bu})
\end{IEEEeqnarray*}
and, on the other hand,
\begin{IEEEeqnarray*}{rCl}
  (\rho_3-\rho_4)(\hat\bu) & = & 
  {\iint_{\hat{f}_3} \hat\bu \cdot \hat\bn_3\,d\hat S} - 
  {\iint_{\hat{f}_4} \hat\bu \cdot \hat\bn_4\,d\hat S} \\[4pt]
  & = & \int_{0}^{1}\int_{0}^{x} \hat{u}_3(x,y,1-x)\,dydx - 
        \int_{0}^{1}\int_{0}^{y} \hat{u}_3(x,y,1-y)\,dxdy\mbox{,}
\end{IEEEeqnarray*}
so
\begin{IEEEeqnarray*}{rCl}
  (\rku)_1(x,y,z) & = & {x\int_{\hat{E}} \dv\hat{\bu}\,d\hat{\boldsymbol{x}}}\,+\\[4pt]
  \IEEEeqnarraymulticol{3}{r}{
    \qquad\left\{\int_{0}^{1}\int_{0}^{x} \hat{u}_3(x,y,1-x)\,dydx - 
    \int_{0}^{1}\int_{0}^{y} \hat{u}_3(x,y,1-y)\,dxdy\right\}\,
    \tfrac{x}{1-z}.}
\end{IEEEeqnarray*}
In a completely similar fashion we arrive at
\begin{IEEEeqnarray*}{rCl}
  (\rku)_2(x,y,z) & = & y\,{\int_{\hat{E}} \dv\hat{\bu}\,d\hat{\boldsymbol{x}}}\,+
  \left\{{\iint_{\hat{f}_3} \hat\bu \cdot \hat\bn_3\,d\hat S} - 
   {\iint_{\hat{f}_4} \hat\bu \cdot \hat\bn_4\,d\hat S}\right\}\,\tfrac{y}{1-z}.\\[4pt]
               & = & y\,{\int_{\hat{E}} \dv\hat{\bu}\,d\hat{\boldsymbol{x}}}\,+\\[4pt]
  \IEEEeqnarraymulticol{3}{r}{
    \qquad\left\{\int_{0}^{1}\int_{0}^{x} \hat{u}_3(x,y,1-x)\,dydx - 
    \int_{0}^{1}\int_{0}^{y} \hat{u}_3(x,y,1-y)\,dxdy\right\}
    \tfrac{y}{1-z}.}
\end{IEEEeqnarray*}
We collect every term obtained so far for the first and second components in
Table~\ref{terms_table}.
\begin{table}[!h]
    \centering  
    \caption{Terms\\[4pt]$q(s,t) = \tfrac{2s-st}{1-t},\,r(s,t) = \tfrac{st}{1-t}$}
    \label{terms_table}
    \begin{IEEEeqnarraybox*}
    [\IEEEeqnarraystrutmode
    \IEEEeqnarraystrutsizeadd{2pt}{12pt}]{v/c/v/c/v/c/v/}
        \IEEEeqnarrayrulerow\\
        \IEEEeqnarrayseprow[5pt]\\
        & & & (\rku)_1 & & (\rku)_2 & \\
        \IEEEeqnarrayrulerow\\
        \IEEEeqnarrayseprow[5pt]\\
        & (\hat{u}_1,0,0)' & &
          \begin{IEEEeqnarraybox*}{l}
            -2{\iint_{\hat{f}_2} \hat{\bu} \cdot \hat\bn_2\,d\hat S}\\ + 
            {q(x,z)\int_{\hat{E}} \dv\hat{\bu} \,d\hat{\boldsymbol{x}}}
          \end{IEEEeqnarraybox*}
        & &
          -r(y,z){\int_{\hat{E}} \dv\hat{\bu} \,d\hat{\boldsymbol{x}}} &\\
        \IEEEeqnarrayrulerow\\
        \IEEEeqnarrayseprow[5pt]\\
        & (0,\hat{u}_2,0)' & & 
          -r(x,z){\int_{\hat{E}} \dv\hat{\bu} \,d\hat{\boldsymbol{x}}} 
        & & 
          \begin{IEEEeqnarraybox*}{l}
            -2{\iint_{\hat{f}_1} \hat{\bu} \cdot \hat\bn_1\,d\hat S}\\ + 
            {q(x,z)\int_{\hat{E}} \dv\hat{\bu} \,d\hat{\boldsymbol{x}}}
          \end{IEEEeqnarraybox*}
        &\\
        \IEEEeqnarrayrulerow\\
        \IEEEeqnarrayseprow[5pt]\\
        & (0,0,\hat{u}_3)' & & 
          \begin{IEEEeqnarraybox*}{l}
            x\int_{\hat{E}} \dv\hat{\bu}\,d\hat{\boldsymbol{x}} \\[5pt] +\, 
            \left\{\iint_{\hat{f}_3} \hat\bu \cdot \hat\bn_3\,d\hat S\right.
             \\[5pt] 
            \left. -\iint_{\hat{f}_4} \hat\bu \cdot \hat\bn_4\,d\hat S\right\}r(x,z)
          \end{IEEEeqnarraybox*}
         & & 
          \begin{IEEEeqnarraybox*}{l}
            y\int_{\hat{E}} \dv\hat{\bu}\,d\hat{\boldsymbol{x}}\\[5pt] +\, 
              \left\{\iint_{\hat{f}_4} \hat\bu \cdot \hat\bn_4\,d\hat S\right.
             \\[5pt] 
             \left.-\iint_{\hat{f}_3} \hat\bu \cdot \hat\bn_3\,d\hat S\right\}r(y,z)
          \end{IEEEeqnarraybox*}
        &\\\IEEEeqnarrayrulerow
    \end{IEEEeqnarraybox*}
\end{table}

Lastly, the third component of $\rku$ can be treated at once for any 
field $\hat\bu$ as follows:
\begin{IEEEeqnarray*}{rCl}
    (\rku)_3(x,y,z) & = &   z\sum_{i=1}^4\iint_{\hat{f}_i} \hat{\bu}\cdot\hat{\bn}_i\,d\hat S
                      + (z-1) \iint_{\hat{f}_5}\hat{\bu}\cdot\hat{\bn}_5\,d\hat S\\[5pt]
                    & = & z\iint_{\partial\hat{E}} \hat{\bu}\cdot\hat{\bn} - \iint_{f_5}
                 \hat{\bu}\cdot\hat{\bn}_5\,d\hat S\\[5pt]
    \yesnumber\label{term_rk3}
                    & = &\hat{x}_3\int_{\hat{E}} \mbox{div}\,\hat{\bu}\,d\hat{\bx} 
     + \iint_{\hat{f}_5} \hat{u}_3\,d\hat{S}.
\end{IEEEeqnarray*}
Now we bound each term in Table~\ref{terms_table} and in expression~(\ref{term_rk3}).
\begin{IEEEeqnarray*}{rCl}
  (\rku)_1 & = & (\br_{\hat{E}}(\hat{u}_1,0,0)')_1 + 
                 (\br_{\hat{E}}(0,\hat{u}_2,0)')_1 + 
                 (\br_{\hat{E}}(0,0,\hat{u}_3)')_1\\[5pt]
           & = & -2\iint_{\hat{f}_2}\hat{u}_1\,d\hat S +
  \tfrac{2x}{1-z}\int_{\hat{E}} \tfrac{\partial\hat{u}_1}{\partial\hat{x}_1}\,d\hat{\bx} -
  \tfrac{xz}{1-z}\int_{\hat{E}} \tfrac{\partial\hat{u}_1}{\partial\hat{x}_1}\,d\hat{\bx} \\[5pt]
  & & \,- \tfrac{xz}{1-z}\int_{\hat{E}} \tfrac{\partial\hat{u}_2}{\partial\hat{x}_2}\,d\hat{\bx}
  + \left( x + \tfrac{xz}{1-z}-\tfrac{xz}{1-z} \right)
  \int_{\hat{E}} \tfrac{\partial\hat{u}_3}{\partial\hat{x}_3}\,d\hat{\bx}\\[5pt]
  &   &\, + \left({\iint_{\hat{f}_3} \hat{u}_3\,d\hat S}
        - {\iint_{\hat{f}_4} \hat{u}_3\,d\hat S}\right)\tfrac{xz}{1-z}\\[5pt]
           & = & -2\iint_{\hat{f}_2}\hat{u}_1\,d\hat S +
  \tfrac{2x}{1-z}\int_{\hat{E}}\tfrac{\partial\hat{u}_1}{\partial\hat{x}_1}\,d\hat{\bx} -
  \tfrac{xz}{1-z}\int_{\hat{E}}\dv\hat{\bu}\,d\hat{\bx}\\[5pt]
  \yesnumber\label{face_integrals}
  &  & \,+\tfrac{x}{1-z}\int_{\hat{E}}\tfrac{\partial\hat{u}_3}{\partial\hat{x}_3}\,d\hat{\bx}
  + \left({\iint_{\hat{f}_3} \hat{u}_3\,d\hat S}
        - {\iint_{\hat{f}_4} \hat{u}_3\,d\hat S}\right)\tfrac{xz}{1-z}.
\end{IEEEeqnarray*}
For the surface integrals in~(\ref{face_integrals}), by Lemma 5.15 in~\cite{monk}, page 120,
\begin{IEEEeqnarray*}{rCl}
\left|\iint_{\hat{f}_2} \hat{u}_1\,d\hat S\right| 
  & \leqslant & C\,\|\hat{u}_1\|_{H^{1}(\hat{E})}
\end{IEEEeqnarray*}
and similarly
\begin{IEEEeqnarray*}{rCl}
  \left|\iint_{\hat{f}_3} \hat{u}_3\,d\hat S - \iint_{\hat{f}_4} \hat{u}_3\,d\hat S\right| 
  & \leqslant & C\,\|\hat{u}_3\|_{H^{1}(\hat{E})}
\end{IEEEeqnarray*}
all of which leads to
\begin{IEEEeqnarray*}{rCl}
  \|(\rku)_1\|_{L^{\infty}(\hat{E})} & \leqslant & C_{\hat{E}} 
  \left[ 
    \|\hat{u}_1\|_{H^{1}(\hat{E})} + 
    \|\dv\hat{\bu}\|_{L^{2}(\hat{E})} + 
    \|\hat{u}_3\|_{H^{1}(\hat{E})}
  \right].
\end{IEEEeqnarray*}
Copying the argument for the second component
\begin{IEEEeqnarray*}{rCl}
  \|(\rku)_2\|_{L^{\infty}(\hat{E})} & \leqslant & C_{\hat{E}} 
  \left[ 
    \|\hat{u}_2\|_{H^{1}(\hat{E})} + 
    \|\dv\hat{\bu}\|_{L^{2}(\hat{E})} + 
    \|\hat{u}_3\|_{H^{1}(\hat{E})}
  \right].
\end{IEEEeqnarray*}
Finally from~(\ref{term_rk3}) we deduce
\begin{IEEEeqnarray*}{rCl}
  \|(\rku)_3\|_{\scriptscriptstyle{L^\infty(\hat{E})}} & \leqslant & C_{\hat{E}}
    \left[\|\hat u_3\|_{\scriptscriptstyle{H^{1}(\hat{E})}} +
    \|\dv \hat\bu\|_{\scriptscriptstyle{L^2}(\hat{E})}\right].
\end{IEEEeqnarray*}
The quantity $C_{\hat{E}}$ depends only on the supremum of the (fixed)
basis shape functions of Table~\ref{shape_face_table} over the pyramid.
\end{proof}
% subsection face_elements (end)

%% ============================================================================
%% TODO: ver si esto finalmente va
%% \subsection{Local Interpolation Estimates for Pyramidal Finite Elements} % (fold)
%% \label{sub:local_interpolation_estimates_for_pyramidal_elements}
%% decir que permitimos pirámides elongadas perpendicularmente a la base
%% $h_3\geqslant C\min\{h_1,h_2\}$
%% Verificar si es esto o $h3 >= max (h1, h2)$\\
%% poner tres dibujos con casos $h1=h2<h3$; $h1<h2<h3$; $h2<h1<h3$
% subsection local_interpolation_estimates_for_pyramidal_elements (end)
%% ============================================================================


% section pyramidal_finite_elements (end)
\chapter{7}
\section{examples of the meshing procedure} % (fold)
\label{sec:examples_of_the_meshing_procedure}

% section examples_of_the_meshing_procedure (end)




\bibliographystyle{acm}
\bibliography{mybib}
	
\end{document}

********************************************************************************

{\color{blue}\#\#\#\#\#\#\#\# al momento de usar
teorema~\ref{auxlabel6}  en   tetra para
malla, citar del articulo aadl\\.}


& & +\;h_1 \left( \| \partial_{1} \tilde{u}_{s,1} \|_{L^{1}(\tilde{E})} +
        \| \partial_{1} \tilde{u}_{s,2} \|_{L^{1}(\tilde{E})}\right)\\
   & & +\;h_2 \left( \| \partial_{2} \tilde{u}_{s,1} \|_{L^{1}(\tilde{E})} +
        \| \partial_{2} \tilde{u}_{s,2} \|_{L^{1}(\tilde{E})}\right)\\
   & & +\;\left.h_3 \left( \| \partial_{3} \tilde{u}_{s,1} \|_{L^{1}(\tilde{E})} +
        \| \partial_{3} \tilde{u}_{s,2} \|_{L^{1}(\tilde{E})}\right) + 

